%
% $Id: SANDExampleArticleNotstrict.tex,v 1.23 2007/12/13 21:27:14 rolf Exp $
%
% This is an example LaTeX file which uses the SANDreport class file.
% It shows how a SAND report should be formatted, what sections and
% elements it should contain, and how to use the SANDreport class.
% It uses the LaTeX article class, but not the strict option.
% It uses .eps logos and files to show how pdflatex can be used
%
% Get the latest version of the class file and more at
%    http://www.cs.sandia.gov/~rolf/SANDreport
%
% This file and the SANDreport.cls file are based on information
% contained in "Guide to Preparing {SAND} Reports", Sand98-0730, edited
% by Tamara K. Locke, and the newer "Guide to Preparing SAND Reports and
% Other Communication Products", SAND2002-2068P.
% Please send corrections and suggestions for improvements to
% Rolf Riesen, Org. 9223, MS 1110, rolf@cs.sandia.gov
%
\documentclass[pdf,ps2pdf,12pt]{INLreport}


\setcounter{tocdepth}{5}
\setcounter{secnumdepth}{5}

\makeatletter
\renewcommand\paragraph{\@startsection{paragraph}{4}{\z@}%
                                     {-3.25ex\@plus -1ex \@minus -.2ex}%
                                     {0.0001pt \@plus .2ex}%
                                     {\normalfont\normalsize\bfseries}}
\renewcommand\subparagraph{\@startsection{subparagraph}{5}{\z@}%
                                     {-3.25ex\@plus -1ex \@minus -.2ex}%
                                     {0.0001pt \@plus .2ex}%
                                     {\normalfont\normalsize\bfseries}}
\makeatother

\usepackage{pslatex}
\usepackage{mathptmx}	% Use the Postscript Times font
%\usepackage[FIGBOTCAP,normal,bf,tight]{subfigure}
\usepackage{graphicx}
\usepackage{caption}
\usepackage[table,xcdraw]{xcolor}
\usepackage{subcaption}
%
\usepackage{amsmath}
\usepackage{amssymb}
\usepackage{pifont}
\usepackage{float} %utiliser H pour forcer � mettre l'image o� on ve

%\usepackage[dvips,light,first,bottomafter]{draftcopy}
%\draftcopyName{Sample, contains no OUO}{70}
%\draftcopyName{Draft}{300}

\usepackage{xspace}
%
%=================================================================================================
% new commands
% +++++++++++++++++++++++++++++++++++++++++++++++++++++++++++++++++++++++++++++++++++++++++++++++++
\newcommand{\nc}{\newcommand}
%
% Ways of grouping things
%
\newcommand{\bracket}[1]{\left[ #1 \right]}
\newcommand{\bracet}[1]{\left\{ #1 \right\}}
\newcommand{\fn}[1]{\left( #1 \right)}
\newcommand{\ave}[1]{\left\langle #1 \right\rangle}
%
% Derivative forms
%
\newcommand{\dx}[1]{\,d#1}
\newcommand{\dxdy}[2]{\frac{\partial #1}{\partial #2}}
\newcommand{\dxdt}[1]{\frac{\partial #1}{\partial t}}
\newcommand{\dxdz}[1]{\frac{\partial #1}{\partial z}}
\newcommand{\dfdt}[1]{\frac{\partial}{\partial t} \fn{#1}}
\newcommand{\dfdz}[1]{\frac{\partial}{\partial z} \fn{#1}}
\newcommand{\ddt}[1]{\frac{\partial}{\partial t} #1}
\newcommand{\ddz}[1]{\frac{\partial}{\partial z} #1}
\newcommand{\dd}[2]{\frac{\partial}{\partial #1} #2}
\newcommand{\ddx}[1]{\frac{\partial}{\partial x} #1}
\newcommand{\ddy}[1]{\frac{\partial}{\partial y} #1}
%
% Vector forms
%
%\renewcommand{\vec}[1]{\ensuremath{\stackrel{\rightarrow}{#1}}}
%\renewcommand{\div}{\ensuremath{\vec{\nabla} \cdot}}
%\newcommand{\grad}{\ensuremath{\vec{\nabla}}}

\renewcommand{\div}{\vec{\nabla}\! \cdot \!}
\newcommand{\grad}{\vec{\nabla}}
\newcommand{\oa}[1]{\fn{\frac{1}{3}\hat{\Omega}\!\cdot\!\overrightarrow{A_{#1}}}}

%
% Equation beginnings and endings
%
\newcommand{\bea}{\begin{eqnarray}}
\newcommand{\eea}{\end{eqnarray}}
\newcommand{\be}{\begin{equation}}
\newcommand{\ee}{\end{equation}}
\newcommand{\beas}{\begin{eqnarray*}}
\newcommand{\eeas}{\end{eqnarray*}}
\newcommand{\bdm}{\begin{displaymath}}
\newcommand{\edm}{\end{displaymath}}
%
% Equation punctuation
%
\newcommand{\pec}{\hspace{0.25in},}
\newcommand{\pep}{\hspace{0.25in}.}
\newcommand{\pev}{\hspace{0.25in}}
%
% Equation labels and references, figure references, table references
%
\newcommand{\LEQ}[1]{\label{eq:#1}}
\newcommand{\EQ}[1]{Eq.~(\ref{eq:#1})}
\newcommand{\EQS}[1]{Eqs.~(\ref{eq:#1})}
\newcommand{\REQ}[1]{\ref{eq:#1}}
\newcommand{\LFI}[1]{\label{fi:#1}}
\newcommand{\FI}[1]{Fig.~\ref{fi:#1}}
\newcommand{\RFI}[1]{\ref{fi:#1}}
\newcommand{\LTA}[1]{\label{ta:#1}}
\newcommand{\TA}[1]{Table~\ref{ta:#1}}
\newcommand{\RTA}[1]{\ref{ta:#1}}

%
% List beginnings and endings
%
\newcommand{\bl}{\bss\begin{itemize}}
\newcommand{\el}{\vspace{-.5\baselineskip}\end{itemize}\ess}
\newcommand{\benu}{\bss\begin{enumerate}}
\newcommand{\eenu}{\vspace{-.5\baselineskip}\end{enumerate}\ess}
%
% Figure and table beginnings and endings
%
\newcommand{\bfg}{\begin{figure}}
\newcommand{\efg}{\end{figure}}
\newcommand{\bt}{\begin{table}}
\newcommand{\et}{\end{table}}
%
% Tabular and center beginnings and endings
%
\newcommand{\bc}{\begin{center}}
\newcommand{\ec}{\end{center}}
\newcommand{\btb}{\begin{center}\begin{tabular}}
\newcommand{\etb}{\end{tabular}\end{center}}
%
% Single space command
%
%\newcommand{\bss}{\begin{singlespace}}
%\newcommand{\ess}{\end{singlespace}}
\newcommand{\bss}{\singlespacing}
\newcommand{\ess}{\doublespacing}
%
%---New environment "arbspace". (modeled after singlespace environment
%                                in Doublespace.sty)
%   The baselinestretch only takes effect at a size change, so do one.
%
\def\arbspace#1{\def\baselinestretch{#1}\@normalsize}
\def\endarbspace{}
\newcommand{\bas}{\begin{arbspace}}
\newcommand{\eas}{\end{arbspace}}
%
% An explanation for a function
%
\newcommand{\explain}[1]{\mbox{\hspace{2em} #1}}
%
% Quick commands for symbols
%
\newcommand{\half}{\frac{1}{2}}
\newcommand{\third}{\frac{1}{3}}
\newcommand{\twothird}{\frac{2}{3}}
\newcommand{\fourth}{\frac{1}{4}}
\newcommand{\mdot}{\dot{m}}
\newcommand{\ten}[1]{\times 10^{#1}\,}
\newcommand{\cL}{{\cal L}}
\newcommand{\cD}{{\cal D}}
\newcommand{\cF}{{\cal F}}
\newcommand{\cE}{{\cal E}}
\renewcommand{\Re}{\mbox{Re}}
\newcommand{\Ma}{\mbox{Ma}}
%
% Inclusion of Graphics Data
%
%\input{psfig}
%\psfiginit
%
% More Quick Commands
%
\newcommand{\bi}{\begin{itemize}}
\newcommand{\ei}{\end{itemize}}
\newcommand{\ben}{\begin{enumerate}}
\newcommand{\een}{\end{enumerate}}
\newcommand{\dxi}{\Delta x_i}
\newcommand{\dyj}{\Delta y_j}
\newcommand{\ts}[1]{\textstyle #1}


\newcommand{\bu}{\boldsymbol{u}}
\newcommand{\ber}{\boldsymbol{e}}
\newcommand{\br}{\boldsymbol{r}}
\newcommand{\bo}{\boldsymbol{\Omega}}

\newcommand{\bn}{\boldsymbol{\nabla}}

% DGFEM commands
\newcommand{\jmp}[1]{[\![#1]\!]}                     % jump
\newcommand{\mvl}[1]{\{\!\!\{#1\}\!\!\}}             % mean value


\newcommand{\boxedeqn}[1]{%
  \[\fbox{%
      \addtolength{\linewidth}{-2\fboxsep}%
      \addtolength{\linewidth}{-2\fboxrule}%
      \begin{minipage}{\linewidth}%
      \begin{equation}#1\end{equation}%
      \end{minipage}%
    }\]%
}
\newcommand{\mboxed}[1]{\boxed{\phantom{#1}}}
\newcommand{\ud}{\,\mathrm{d}}

% keff
\newcommand{\keff}{\ensuremath{k_{\textit{eff}}}\xspace}

% margin par
\newcommand{\mt}[1]{\marginpar{ {\footnotesize #1} }}

% shortcut for aposterio in italics
\newcommand{\apost}{\textit{a posteriori\xspace}}
\newcommand{\Apost}{\textit{A posteriori}\xspace}

% shortcut for multi-group
\newcommand{\mg}{multigroup\xspace}
\newcommand{\Mg}{Multigroup\xspace}
\newcommand{\ho}{higher-order\xspace}
\newcommand{\Ho}{Higher-order\xspace}
\newcommand{\HO}{Higher-Order\xspace}
\newcommand{\HObig}{HIGHER-ORDER\xspace}
\newcommand{\Mgbig}{MULTIGROUP\xspace}
\newcommand{\sn}{$S_N$\xspace}
\newcommand{\pn}{$P_N$\xspace}

% shortcut for domain notation
\newcommand{\D}{\mathcal{D}}
\newcommand{\Sp}{\mathcal{S}}

% shortcut for xuthus
\newcommand{\psc}[1]{{\sc {#1}}}
\newcommand{\xuthus}{\psc{xuthus}\xspace}

% vector shortcuts
\newcommand{\vo}{\vec{\Omega}}
\newcommand{\vr}{\vec{r}}
\newcommand{\vn}{\vec{n}}
\newcommand{\vnk}{\vec{\mathbf{n}}}

% extra space
\newcommand{\qq}{\quad\quad}

% sign function
\DeclareMathOperator{\sgn}{sgn}


\makeatletter
\newcommand{\rmnum}[1]{\romannumeral #1}
\newcommand{\Rmnum}[1]{\expandafter\@slowromancap\romannumeral #1@}
\makeatother

\newcommand{\ensuretext}[1]{\ensuremath{\text{#1}}}
\newcommand{\Rmnumb}[1]{\ensuretext{\Rmnum{#1}}}

% common reference commands
\newcommand{\eqt}[1]{Eq.~(\ref{#1})}                     % equation
\newcommand{\fig}[1]{Fig.~\ref{#1}}                      % figure
\newcommand{\tbl}[1]{Table~\ref{#1}}                     % table
\newcommand{\app}[1]{Appendix~\ref{#1}}                  % appendix


% for mathematica notebook
\newcommand{\IndentingNewLine}{ \\ }


\newcommand{\rhs}{right-hand-side\xspace}
\newcommand{\clearemptydoublepage}{\newpage{\pagestyle{empty}\cleardoublepage}}

\newenvironment{myverbatim}%            To change the pseudocode font
{\par\noindent%
 \rule[0pt]{\linewidth}{0.2pt}
 \vspace*{-9pt}
 \linespread{0.0}\small\verbatim}%
{\rule[-5pt]{\linewidth}{0.2pt}\endverbatim}

\newenvironment{myverbatim1}%            To change the pseudocode font
{\par\noindent%
 \rule[0pt]{\linewidth}{0.2pt}
 \vspace*{-9pt}
 \linespread{1.0}\scriptsize\verbatim}%
{\rule[-5pt]{\linewidth}{0.2pt}\endverbatim}

\newcommand{\theHalgorithm}{\arabic{algorithm}} % remove the error of algorithm+hyperref

%\hypersetup{
%    bookmarks=true,         % show bookmarks bar?
%    unicode=false,          % non-Latin characters in Acrobat's bookmarks
%    pdftoolbar=true,        % show Acrobat's toolbar?
%    pdfmenubar=true,        % show Acrobat's menu?
%    pdffitwindow=false,     % window fit to page when opened
%    pdfstartview={FitH},    % fits the width of the page to the window
%    pdftitle={Dissertation},    % title
%    pdfauthor={Yaqi Wang},     % author
%    pdfsubject={Transport AMR},   % subject of the document
%    pdfcreator={Yaqi Wang},   % creator of the document
%    pdfproducer={Yaqi Wang}, % producer of the document
%    pdfkeywords={Transport, AMR}, % list of keywords
%    pdfnewwindow=true,      % links in new window
%    colorlinks=false,       % false: boxed links; true: colored links
%    linkcolor=red,          % color of internal links
%    citecolor=green,        % color of links to bibliography
%    filecolor=magenta,      % color of file links
%    urlcolor=cyan           % color of external links
%}

% prepare generating nomenclature and change default options
%\makenomenclature
%\renewcommand{\nomname}{NOMENCLATURE}
%\RequirePackage{ifthen}
%\renewcommand{\nomgroup}[1]{%
%\item[]\hspace*{-\leftmargin}%
%%\rule[2pt]{0.45\linewidth}{1pt}%
%%\hfill
%\ifthenelse{\equal{#1}{A}}{\textbf{Abbreviations}}{%
%\ifthenelse{\equal{#1}{S}}{\textbf{Symbols}}{
%\ifthenelse{\equal{#1}{U}}{\textbf{Superscripts}}{
%\ifthenelse{\equal{#1}{V}}{\textbf{Subscripts}}{}}}}
%%\hfill
%%\rule[2pt]{0.45\linewidth}{1pt}
%}

% a new environment for splitting a long algorithm
\makeatletter
\newenvironment{breakalgo}[2][alg:\thealgorithm]{%
  \def\@fs@cfont{\bfseries}%
  \let\@fs@capt\relax%
  \par\noindent%
  \medskip%
  \rule{\linewidth}{.8pt}%
  \vspace{-20pt}%
  %\par\noindent
  \captionof{algorithm}{#2}\label{#1}%
  \vspace{-1.2\baselineskip}%
%  \noindent\rule{\linewidth}{.4pt}%
  \vspace{8pt}%
  \noindent\rule{\linewidth}{.4pt}%
  \vspace{-1.3\baselineskip}%
}{%
  \vspace{-.75\baselineskip}%
  \par\noindent%
  \rule{\linewidth}{.4pt}%
  \medskip%
}
\makeatother



% If you want to relax some of the SAND98-0730 requirements, use the "relax"
% option. It adds spaces and boldface in the table of contents, and does not
% force the page layout sizes.
% e.g. \documentclass[relax,12pt]{SANDreport}
%
% You can also use the "strict" option, which applies even more of the
% SAND98-0730 guidelines. It gets rid of section numbers which are often
% useful; e.g. \documentclass[strict]{SANDreport}

%\newcommand{\bu}{{\bf u}}
\newcommand{\bx}{{\bf x}}
\newcommand{\ba}{{\bf a}}

\newcommand{\NLF}{\mathcal{F}}
\newcommand{\NLFi}{\mathbf{F}^{i}}
\newcommand{\Jac}{\mathcal{J}}
\newcommand{\JacD}{\mathcal{D}}
\newcommand{\JacC}{\mathcal{C}}
\newcommand{\PreJac}{\mathcal{P}}

\newcommand{\mta}{g_{\alpha \beta}}
\newcommand{\mtb}{g^{\alpha \beta}}


\newcommand{\MT}{\mathcal{G}}
%\newcommand{\mt}{g}

\newcommand{\mtdet}{\mt}
\newcommand{\mtcov}{\mt_{\alpha\beta}}
\newcommand{\mtcon}{\mt^{\alpha\beta}}

\newcommand{\MTc}{\widehat{\MT}}
\newcommand{\etal}{et al.}


% ---------------------------------------------------------------------------- %
%
% Set the title, author, and date
%
    \title{RAVEN: Development of the Adaptive Dynamic Event Tree Approach}

    \author{Andrea Alfonsi \\
	  Idaho National Laboratory\\
	  P.O. Box 1625\\
	  Idaho Falls, ID 83415-3870\\
	  andrea.alfonsi@inl.gov\\
	  \\
	  \and
         Cristian Rabiti \\
	  Idaho National Laboratory\\
	  P.O. Box 1625\\
	  Idaho Falls, ID 83415-3870\\
	  cristian.rabiti@inl.gov\\
	  \\
	  \and
         Diego Mandelli \\
	  Idaho National Laboratory\\
	  P.O. Box 1625\\
	  Idaho Falls, ID 83415-3850\\
	  diego.mandelli@inl.gov\\
	  \\
	  \and
         Joshua Cogliati \\
	  Idaho National Laboratory\\
	  P.O. Box 1625\\
	  Idaho Falls, ID 83415-3870\\
	  joshua.cogliati@inl.gov\\
	  \\
	  \and
         Robert Kinoshita \\
	  Idaho National Laboratory\\
	  P.O. Box 1625\\
	  Idaho Falls, ID 83415-2210\\
	  robert.kinoshita@inl.gov\\
	  \\
	 }

    % There is a "Printed" date on the title page of a SAND report, so
    % the generic \date should generally be empty.
    \date{}


% ---------------------------------------------------------------------------- %
% Set some things we need for SAND reports. These are mandatory
%
\SANDnum{INL/MIS-14-33246 - Rev.1}
\SANDprintDate{09/30/2014}
\SANDauthor{Andrea Alfonsi, Cristian Rabiti, Diego Mandelli, Joshua Cogliati, Robert Kinoshita}


% ---------------------------------------------------------------------------- %
% Include the markings required for your SAND report. The default is "Unlimited
% Release". You may have to edit the file included here, or create your own
% (see the examples provided).
%
% \include{MarkOUO} % Not needed for unlimted release reports



% ---------------------------------------------------------------------------- %

% ---------------------------------------------------------------------------- %
%
% Start the document
%
\begin{document}
    \maketitle

    % ------------------------------------------------------------------------ %
    % An Abstract is required for SAND reports
    %
%    \begin{abstract}
%      \input abstract
%    \end{abstract}


    % ------------------------------------------------------------------------ %
    % An Acknowledgement section is optional but important, if someone made
    % contributions or helped beyond the normal part of a work assignment.
    % Use \section* since we don't want it in the table of context
    %
    \clearpage
    \section*{Acknowledgment}
	This work is supported by the U.S. Department of Energy, under DOE Idaho Operations Office Contract DE-AC07-05ID14517. Accordingly, the U.S. Government retains a nonexclusive, royalty-free license to publish or reproduce the published form of this contribution, or allow others to do so, for U.S. Government purposes.
%
%	The format of this report is based on information found
%	in~\cite{Sand98-0730}.


    % ------------------------------------------------------------------------ %
    % The table of contents and list of figures and tables
    % Comment out \listoffigures and \listoftables if there are no
    % figures or tables. Make sure this starts on an odd numbered page
    %
    \cleardoublepage		% TOC needs to start on an odd page
    \tableofcontents
    \listoffigures
    %\listoftables


    % ---------------------------------------------------------------------- %
    % An optional preface or Foreword
%    \clearpage
%    \section*{Preface}
%    \addcontentsline{toc}{section}{Preface}
%	Although muggles usually have only limited experience with
%	magic, and many even dispute its existence, it is worthwhile
%	to be open minded and explore the possibilities.


    % ---------------------------------------------------------------------- %
    % An optional executive summary
%    \clearpage
%    \section*{Summary}
%    \addcontentsline{toc}{section}{Summary}
%    \input summary


    % ---------------------------------------------------------------------- %
    % An optional glossary. We don't want it to be numbered
%    \clearpage
%    \section*{Nomenclature}
%    \addcontentsline{toc}{section}{Nomenclature}
%    \begin{description}
%	\item[alohomoral]
%	    spell to open locked doors and containers
%	\item[leviosa]
%	    spell to levitate objects
%%	\item[remembrall]
%	    device to alert you that you have forgotten something
%	\item[wand]
%	    device to execute spells
%    \end{description}


    % ---------------------------------------------------------------------- %
    % This is where the body of the report begins; usually with an Introduction
    %
    \SANDmain		% Start the main part of the report

    \section{Introduction}
    \input introduction

    \section{The Dynamic Event Tree Methodology}
   \input det

    \section{Adaptive Dynamic Event Tree}
   \input adet

    \section{Dynamic PRA analysis on a simplified PWR model}
    \input results

    \section{Conclusions}
    \input conclusion

    %\nocite{*}


    % ---------------------------------------------------------------------- %
    % References
    %
    \clearpage
    % If hyperref is included, then \phantomsection is already defined.
    % If not, we need to define it.
    \providecommand*{\phantomsection}{}
    \phantomsection

    \addcontentsline{toc}{section}{References}
    \bibliographystyle{plain}
    \bibliography{bib.bib}


    % ---------------------------------------------------------------------- %
    %

    % \printindex

    %\include{distribution}

\end{document}
%
% $Id: SANDExampleArticleNotstrict.tex,v 1.23 2007/12/13 21:27:14 rolf Exp $
%
% This is an example LaTeX file which uses the SANDreport class file.
% It shows how a SAND report should be formatted, what sections and
% elements it should contain, and how to use the SANDreport class.
% It uses the LaTeX article class, but not the strict option.
% It uses .eps logos and files to show how pdflatex can be used
%
% Get the latest version of the class file and more at
%    http://www.cs.sandia.gov/~rolf/SANDreport
%
% This file and the SANDreport.cls file are based on information
% contained in "Guide to Preparing {SAND} Reports", Sand98-0730, edited
% by Tamara K. Locke, and the newer "Guide to Preparing SAND Reports and
% Other Communication Products", SAND2002-2068P.
% Please send corrections and suggestions for improvements to
% Rolf Riesen, Org. 9223, MS 1110, rolf@cs.sandia.gov
%
\documentclass[pdf,ps2pdf,12pt]{INLreport}


\setcounter{tocdepth}{5}
\setcounter{secnumdepth}{5}

\makeatletter
\renewcommand\paragraph{\@startsection{paragraph}{4}{\z@}%
                                     {-3.25ex\@plus -1ex \@minus -.2ex}%
                                     {0.0001pt \@plus .2ex}%
                                     {\normalfont\normalsize\bfseries}}
\renewcommand\subparagraph{\@startsection{subparagraph}{5}{\z@}%
                                     {-3.25ex\@plus -1ex \@minus -.2ex}%
                                     {0.0001pt \@plus .2ex}%
                                     {\normalfont\normalsize\bfseries}}
\makeatother

\usepackage{pslatex}
\usepackage{mathptmx}	% Use the Postscript Times font
%\usepackage[FIGBOTCAP,normal,bf,tight]{subfigure}
\usepackage{graphicx}
\usepackage{caption}
\usepackage[table,xcdraw]{xcolor}
\usepackage{subcaption}
%
\usepackage{amsmath}
\usepackage{amssymb}
\usepackage{pifont}
\usepackage{float} %utiliser H pour forcer � mettre l'image o� on ve

%\usepackage[dvips,light,first,bottomafter]{draftcopy}
%\draftcopyName{Sample, contains no OUO}{70}
%\draftcopyName{Draft}{300}

\usepackage{xspace}
%
%=================================================================================================
% new commands
% +++++++++++++++++++++++++++++++++++++++++++++++++++++++++++++++++++++++++++++++++++++++++++++++++
\newcommand{\nc}{\newcommand}
%
% Ways of grouping things
%
\newcommand{\bracket}[1]{\left[ #1 \right]}
\newcommand{\bracet}[1]{\left\{ #1 \right\}}
\newcommand{\fn}[1]{\left( #1 \right)}
\newcommand{\ave}[1]{\left\langle #1 \right\rangle}
%
% Derivative forms
%
\newcommand{\dx}[1]{\,d#1}
\newcommand{\dxdy}[2]{\frac{\partial #1}{\partial #2}}
\newcommand{\dxdt}[1]{\frac{\partial #1}{\partial t}}
\newcommand{\dxdz}[1]{\frac{\partial #1}{\partial z}}
\newcommand{\dfdt}[1]{\frac{\partial}{\partial t} \fn{#1}}
\newcommand{\dfdz}[1]{\frac{\partial}{\partial z} \fn{#1}}
\newcommand{\ddt}[1]{\frac{\partial}{\partial t} #1}
\newcommand{\ddz}[1]{\frac{\partial}{\partial z} #1}
\newcommand{\dd}[2]{\frac{\partial}{\partial #1} #2}
\newcommand{\ddx}[1]{\frac{\partial}{\partial x} #1}
\newcommand{\ddy}[1]{\frac{\partial}{\partial y} #1}
%
% Vector forms
%
%\renewcommand{\vec}[1]{\ensuremath{\stackrel{\rightarrow}{#1}}}
%\renewcommand{\div}{\ensuremath{\vec{\nabla} \cdot}}
%\newcommand{\grad}{\ensuremath{\vec{\nabla}}}

\renewcommand{\div}{\vec{\nabla}\! \cdot \!}
\newcommand{\grad}{\vec{\nabla}}
\newcommand{\oa}[1]{\fn{\frac{1}{3}\hat{\Omega}\!\cdot\!\overrightarrow{A_{#1}}}}

%
% Equation beginnings and endings
%
\newcommand{\bea}{\begin{eqnarray}}
\newcommand{\eea}{\end{eqnarray}}
\newcommand{\be}{\begin{equation}}
\newcommand{\ee}{\end{equation}}
\newcommand{\beas}{\begin{eqnarray*}}
\newcommand{\eeas}{\end{eqnarray*}}
\newcommand{\bdm}{\begin{displaymath}}
\newcommand{\edm}{\end{displaymath}}
%
% Equation punctuation
%
\newcommand{\pec}{\hspace{0.25in},}
\newcommand{\pep}{\hspace{0.25in}.}
\newcommand{\pev}{\hspace{0.25in}}
%
% Equation labels and references, figure references, table references
%
\newcommand{\LEQ}[1]{\label{eq:#1}}
\newcommand{\EQ}[1]{Eq.~(\ref{eq:#1})}
\newcommand{\EQS}[1]{Eqs.~(\ref{eq:#1})}
\newcommand{\REQ}[1]{\ref{eq:#1}}
\newcommand{\LFI}[1]{\label{fi:#1}}
\newcommand{\FI}[1]{Fig.~\ref{fi:#1}}
\newcommand{\RFI}[1]{\ref{fi:#1}}
\newcommand{\LTA}[1]{\label{ta:#1}}
\newcommand{\TA}[1]{Table~\ref{ta:#1}}
\newcommand{\RTA}[1]{\ref{ta:#1}}

%
% List beginnings and endings
%
\newcommand{\bl}{\bss\begin{itemize}}
\newcommand{\el}{\vspace{-.5\baselineskip}\end{itemize}\ess}
\newcommand{\benu}{\bss\begin{enumerate}}
\newcommand{\eenu}{\vspace{-.5\baselineskip}\end{enumerate}\ess}
%
% Figure and table beginnings and endings
%
\newcommand{\bfg}{\begin{figure}}
\newcommand{\efg}{\end{figure}}
\newcommand{\bt}{\begin{table}}
\newcommand{\et}{\end{table}}
%
% Tabular and center beginnings and endings
%
\newcommand{\bc}{\begin{center}}
\newcommand{\ec}{\end{center}}
\newcommand{\btb}{\begin{center}\begin{tabular}}
\newcommand{\etb}{\end{tabular}\end{center}}
%
% Single space command
%
%\newcommand{\bss}{\begin{singlespace}}
%\newcommand{\ess}{\end{singlespace}}
\newcommand{\bss}{\singlespacing}
\newcommand{\ess}{\doublespacing}
%
%---New environment "arbspace". (modeled after singlespace environment
%                                in Doublespace.sty)
%   The baselinestretch only takes effect at a size change, so do one.
%
\def\arbspace#1{\def\baselinestretch{#1}\@normalsize}
\def\endarbspace{}
\newcommand{\bas}{\begin{arbspace}}
\newcommand{\eas}{\end{arbspace}}
%
% An explanation for a function
%
\newcommand{\explain}[1]{\mbox{\hspace{2em} #1}}
%
% Quick commands for symbols
%
\newcommand{\half}{\frac{1}{2}}
\newcommand{\third}{\frac{1}{3}}
\newcommand{\twothird}{\frac{2}{3}}
\newcommand{\fourth}{\frac{1}{4}}
\newcommand{\mdot}{\dot{m}}
\newcommand{\ten}[1]{\times 10^{#1}\,}
\newcommand{\cL}{{\cal L}}
\newcommand{\cD}{{\cal D}}
\newcommand{\cF}{{\cal F}}
\newcommand{\cE}{{\cal E}}
\renewcommand{\Re}{\mbox{Re}}
\newcommand{\Ma}{\mbox{Ma}}
%
% Inclusion of Graphics Data
%
%\input{psfig}
%\psfiginit
%
% More Quick Commands
%
\newcommand{\bi}{\begin{itemize}}
\newcommand{\ei}{\end{itemize}}
\newcommand{\ben}{\begin{enumerate}}
\newcommand{\een}{\end{enumerate}}
\newcommand{\dxi}{\Delta x_i}
\newcommand{\dyj}{\Delta y_j}
\newcommand{\ts}[1]{\textstyle #1}


\newcommand{\bu}{\boldsymbol{u}}
\newcommand{\ber}{\boldsymbol{e}}
\newcommand{\br}{\boldsymbol{r}}
\newcommand{\bo}{\boldsymbol{\Omega}}

\newcommand{\bn}{\boldsymbol{\nabla}}

% DGFEM commands
\newcommand{\jmp}[1]{[\![#1]\!]}                     % jump
\newcommand{\mvl}[1]{\{\!\!\{#1\}\!\!\}}             % mean value


\newcommand{\boxedeqn}[1]{%
  \[\fbox{%
      \addtolength{\linewidth}{-2\fboxsep}%
      \addtolength{\linewidth}{-2\fboxrule}%
      \begin{minipage}{\linewidth}%
      \begin{equation}#1\end{equation}%
      \end{minipage}%
    }\]%
}
\newcommand{\mboxed}[1]{\boxed{\phantom{#1}}}
\newcommand{\ud}{\,\mathrm{d}}

% keff
\newcommand{\keff}{\ensuremath{k_{\textit{eff}}}\xspace}

% margin par
\newcommand{\mt}[1]{\marginpar{ {\footnotesize #1} }}

% shortcut for aposterio in italics
\newcommand{\apost}{\textit{a posteriori\xspace}}
\newcommand{\Apost}{\textit{A posteriori}\xspace}

% shortcut for multi-group
\newcommand{\mg}{multigroup\xspace}
\newcommand{\Mg}{Multigroup\xspace}
\newcommand{\ho}{higher-order\xspace}
\newcommand{\Ho}{Higher-order\xspace}
\newcommand{\HO}{Higher-Order\xspace}
\newcommand{\HObig}{HIGHER-ORDER\xspace}
\newcommand{\Mgbig}{MULTIGROUP\xspace}
\newcommand{\sn}{$S_N$\xspace}
\newcommand{\pn}{$P_N$\xspace}

% shortcut for domain notation
\newcommand{\D}{\mathcal{D}}
\newcommand{\Sp}{\mathcal{S}}

% shortcut for xuthus
\newcommand{\psc}[1]{{\sc {#1}}}
\newcommand{\xuthus}{\psc{xuthus}\xspace}

% vector shortcuts
\newcommand{\vo}{\vec{\Omega}}
\newcommand{\vr}{\vec{r}}
\newcommand{\vn}{\vec{n}}
\newcommand{\vnk}{\vec{\mathbf{n}}}

% extra space
\newcommand{\qq}{\quad\quad}

% sign function
\DeclareMathOperator{\sgn}{sgn}


\makeatletter
\newcommand{\rmnum}[1]{\romannumeral #1}
\newcommand{\Rmnum}[1]{\expandafter\@slowromancap\romannumeral #1@}
\makeatother

\newcommand{\ensuretext}[1]{\ensuremath{\text{#1}}}
\newcommand{\Rmnumb}[1]{\ensuretext{\Rmnum{#1}}}

% common reference commands
\newcommand{\eqt}[1]{Eq.~(\ref{#1})}                     % equation
\newcommand{\fig}[1]{Fig.~\ref{#1}}                      % figure
\newcommand{\tbl}[1]{Table~\ref{#1}}                     % table
\newcommand{\app}[1]{Appendix~\ref{#1}}                  % appendix


% for mathematica notebook
\newcommand{\IndentingNewLine}{ \\ }


\newcommand{\rhs}{right-hand-side\xspace}
\newcommand{\clearemptydoublepage}{\newpage{\pagestyle{empty}\cleardoublepage}}

\newenvironment{myverbatim}%            To change the pseudocode font
{\par\noindent%
 \rule[0pt]{\linewidth}{0.2pt}
 \vspace*{-9pt}
 \linespread{0.0}\small\verbatim}%
{\rule[-5pt]{\linewidth}{0.2pt}\endverbatim}

\newenvironment{myverbatim1}%            To change the pseudocode font
{\par\noindent%
 \rule[0pt]{\linewidth}{0.2pt}
 \vspace*{-9pt}
 \linespread{1.0}\scriptsize\verbatim}%
{\rule[-5pt]{\linewidth}{0.2pt}\endverbatim}

\newcommand{\theHalgorithm}{\arabic{algorithm}} % remove the error of algorithm+hyperref

%\hypersetup{
%    bookmarks=true,         % show bookmarks bar?
%    unicode=false,          % non-Latin characters in Acrobat's bookmarks
%    pdftoolbar=true,        % show Acrobat's toolbar?
%    pdfmenubar=true,        % show Acrobat's menu?
%    pdffitwindow=false,     % window fit to page when opened
%    pdfstartview={FitH},    % fits the width of the page to the window
%    pdftitle={Dissertation},    % title
%    pdfauthor={Yaqi Wang},     % author
%    pdfsubject={Transport AMR},   % subject of the document
%    pdfcreator={Yaqi Wang},   % creator of the document
%    pdfproducer={Yaqi Wang}, % producer of the document
%    pdfkeywords={Transport, AMR}, % list of keywords
%    pdfnewwindow=true,      % links in new window
%    colorlinks=false,       % false: boxed links; true: colored links
%    linkcolor=red,          % color of internal links
%    citecolor=green,        % color of links to bibliography
%    filecolor=magenta,      % color of file links
%    urlcolor=cyan           % color of external links
%}

% prepare generating nomenclature and change default options
%\makenomenclature
%\renewcommand{\nomname}{NOMENCLATURE}
%\RequirePackage{ifthen}
%\renewcommand{\nomgroup}[1]{%
%\item[]\hspace*{-\leftmargin}%
%%\rule[2pt]{0.45\linewidth}{1pt}%
%%\hfill
%\ifthenelse{\equal{#1}{A}}{\textbf{Abbreviations}}{%
%\ifthenelse{\equal{#1}{S}}{\textbf{Symbols}}{
%\ifthenelse{\equal{#1}{U}}{\textbf{Superscripts}}{
%\ifthenelse{\equal{#1}{V}}{\textbf{Subscripts}}{}}}}
%%\hfill
%%\rule[2pt]{0.45\linewidth}{1pt}
%}

% a new environment for splitting a long algorithm
\makeatletter
\newenvironment{breakalgo}[2][alg:\thealgorithm]{%
  \def\@fs@cfont{\bfseries}%
  \let\@fs@capt\relax%
  \par\noindent%
  \medskip%
  \rule{\linewidth}{.8pt}%
  \vspace{-20pt}%
  %\par\noindent
  \captionof{algorithm}{#2}\label{#1}%
  \vspace{-1.2\baselineskip}%
%  \noindent\rule{\linewidth}{.4pt}%
  \vspace{8pt}%
  \noindent\rule{\linewidth}{.4pt}%
  \vspace{-1.3\baselineskip}%
}{%
  \vspace{-.75\baselineskip}%
  \par\noindent%
  \rule{\linewidth}{.4pt}%
  \medskip%
}
\makeatother



% If you want to relax some of the SAND98-0730 requirements, use the "relax"
% option. It adds spaces and boldface in the table of contents, and does not
% force the page layout sizes.
% e.g. \documentclass[relax,12pt]{SANDreport}
%
% You can also use the "strict" option, which applies even more of the
% SAND98-0730 guidelines. It gets rid of section numbers which are often
% useful; e.g. \documentclass[strict]{SANDreport}

%\newcommand{\bu}{{\bf u}}
\newcommand{\bx}{{\bf x}}
\newcommand{\ba}{{\bf a}}

\newcommand{\NLF}{\mathcal{F}}
\newcommand{\NLFi}{\mathbf{F}^{i}}
\newcommand{\Jac}{\mathcal{J}}
\newcommand{\JacD}{\mathcal{D}}
\newcommand{\JacC}{\mathcal{C}}
\newcommand{\PreJac}{\mathcal{P}}

\newcommand{\mta}{g_{\alpha \beta}}
\newcommand{\mtb}{g^{\alpha \beta}}


\newcommand{\MT}{\mathcal{G}}
%\newcommand{\mt}{g}

\newcommand{\mtdet}{\mt}
\newcommand{\mtcov}{\mt_{\alpha\beta}}
\newcommand{\mtcon}{\mt^{\alpha\beta}}

\newcommand{\MTc}{\widehat{\MT}}
\newcommand{\etal}{et al.}


% ---------------------------------------------------------------------------- %
%
% Set the title, author, and date
%
    \title{RAVEN: Development of the Adaptive Dynamic Event Tree Approach}

    \author{Andrea Alfonsi \\
	  Idaho National Laboratory\\
	  P.O. Box 1625\\
	  Idaho Falls, ID 83415-3870\\
	  andrea.alfonsi@inl.gov\\
	  \\
	  \and
         Cristian Rabiti \\
	  Idaho National Laboratory\\
	  P.O. Box 1625\\
	  Idaho Falls, ID 83415-3870\\
	  cristian.rabiti@inl.gov\\
	  \\
	  \and
         Diego Mandelli \\
	  Idaho National Laboratory\\
	  P.O. Box 1625\\
	  Idaho Falls, ID 83415-3850\\
	  diego.mandelli@inl.gov\\
	  \\
	  \and
         Joshua Cogliati \\
	  Idaho National Laboratory\\
	  P.O. Box 1625\\
	  Idaho Falls, ID 83415-3870\\
	  joshua.cogliati@inl.gov\\
	  \\
	  \and
         Robert Kinoshita \\
	  Idaho National Laboratory\\
	  P.O. Box 1625\\
	  Idaho Falls, ID 83415-2210\\
	  robert.kinoshita@inl.gov\\
	  \\
	 }

    % There is a "Printed" date on the title page of a SAND report, so
    % the generic \date should generally be empty.
    \date{}


% ---------------------------------------------------------------------------- %
% Set some things we need for SAND reports. These are mandatory
%
%\SANDnum{INL/EXT-xx-xxxxx Rev 1}
%\SANDprintDate{08/30/2014}
%\SANDauthor{Andrea Alfonsi, Cristian Rabiti, Diego Mandelli, Joshua Cogliati, Robert Kinoshita}


% ---------------------------------------------------------------------------- %
% Include the markings required for your SAND report. The default is "Unlimited
% Release". You may have to edit the file included here, or create your own
% (see the examples provided).
%
% \include{MarkOUO} % Not needed for unlimted release reports



% ---------------------------------------------------------------------------- %

% ---------------------------------------------------------------------------- %
%
% Start the document
%
\begin{document}
    \maketitle

    % ------------------------------------------------------------------------ %
    % An Abstract is required for SAND reports
    %
%    \begin{abstract}
%      \input abstract
%    \end{abstract}


    % ------------------------------------------------------------------------ %
    % An Acknowledgement section is optional but important, if someone made
    % contributions or helped beyond the normal part of a work assignment.
    % Use \section* since we don't want it in the table of context
    %
    \clearpage
    \section*{Acknowledgment}
	This work is supported by the U.S. Department of Energy, under DOE Idaho Operations Office Contract DE-AC07-05ID14517. Accordingly, the U.S. Government retains a nonexclusive, royalty-free license to publish or reproduce the published form of this contribution, or allow others to do so, for U.S. Government purposes.
%
%	The format of this report is based on information found
%	in~\cite{Sand98-0730}.


    % ------------------------------------------------------------------------ %
    % The table of contents and list of figures and tables
    % Comment out \listoffigures and \listoftables if there are no
    % figures or tables. Make sure this starts on an odd numbered page
    %
    \cleardoublepage		% TOC needs to start on an odd page
    \tableofcontents
    \listoffigures
    %\listoftables


    % ---------------------------------------------------------------------- %
    % An optional preface or Foreword
%    \clearpage
%    \section*{Preface}
%    \addcontentsline{toc}{section}{Preface}
%	Although muggles usually have only limited experience with
%	magic, and many even dispute its existence, it is worthwhile
%	to be open minded and explore the possibilities.


    % ---------------------------------------------------------------------- %
    % An optional executive summary
%    \clearpage
%    \section*{Summary}
%    \addcontentsline{toc}{section}{Summary}
%    \input summary


    % ---------------------------------------------------------------------- %
    % An optional glossary. We don't want it to be numbered
%    \clearpage
%    \section*{Nomenclature}
%    \addcontentsline{toc}{section}{Nomenclature}
%    \begin{description}
%	\item[alohomoral]
%	    spell to open locked doors and containers
%	\item[leviosa]
%	    spell to levitate objects
%%	\item[remembrall]
%	    device to alert you that you have forgotten something
%	\item[wand]
%	    device to execute spells
%    \end{description}


    % ---------------------------------------------------------------------- %
    % This is where the body of the report begins; usually with an Introduction
    %
    \SANDmain		% Start the main part of the report

    \section{Introduction}
    \input introduction

    \section{The Dynamic Event Tree Methodology}
   \input det

    \section{Adaptive Dynamic Event Tree}
   \input adet

    \section{Dynamic PRA analysis on a simplified PWR model}
    \input results

    \section{Conclusions}
    \input conclusion

    %\nocite{*}


    % ---------------------------------------------------------------------- %
    % References
    %
    \clearpage
    % If hyperref is included, then \phantomsection is already defined.
    % If not, we need to define it.
    \providecommand*{\phantomsection}{}
    \phantomsection

    \addcontentsline{toc}{section}{References}
    \bibliographystyle{plain}
    \bibliography{bib.bib}


    % ---------------------------------------------------------------------- %
    %

    % \printindex

    %\include{distribution}

\end{document}
