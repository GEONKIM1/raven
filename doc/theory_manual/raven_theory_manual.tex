%
% This is an example LaTeX file which uses the SANDreport class file.
% It shows how a SAND report should be formatted, what sections and
% elements it should contain, and how to use the SANDreport class.
% It uses the LaTeX article class, but not the strict option.
% ItINLreport uses .eps logos and files to show how pdflatex can be used
%
% Get the latest version of the class file and more at
%    http://www.cs.sandia.gov/~rolf/SANDreport
%
% This file and the SANDreport.cls file are based on information
% contained in "Guide to Preparing {SAND} Reports", Sand98-0730, edited
% by Tamara K. Locke, and the newer "Guide to Preparing SAND Reports and
% Other Communication Products", SAND2002-2068P.
% Please send corrections and suggestions for improvements to
% Rolf Riesen, Org. 9223, MS 1110, rolf@cs.sandia.gov
%
\documentclass[pdf,12pt]{INLreport}
% pslatex is really old (1994).  It attempts to merge the times and mathptm packages.
% My opinion is that it produces a really bad looking math font.  So why are we using it?
% If you just want to change the text font, you should just \usepackage{times}.
% \usepackage{pslatex}
\usepackage{times}
\usepackage[FIGBOTCAP,normal,bf,tight]{subfigure}
\usepackage{amsmath}
\usepackage{amssymb}
\usepackage{soul}
\usepackage{pifont}
\usepackage{enumerate}
\usepackage{listings}
\usepackage{fullpage}
\usepackage{xcolor}          % Using xcolor for more robust color specification
\usepackage{ifthen}          % For simple checking in newcommand blocks
\usepackage{textcomp}
\usepackage{mathtools}
\usepackage{relsize}
\usepackage{lscape}
\usepackage[toc,page]{appendix}
\usepackage{RAVEN}
\usepackage{tabls}
\usepackage{multirow}
\usepackage{float}

\newtheorem{mydef}{Definition}
\newcommand{\norm}[1]{\lVert#1\rVert}
%\usepackage[table,xcdraw]{xcolor}
%\usepackage{authblk}         % For making the author list look prettier
%\renewcommand\Authsep{,~\,}

% Custom colors
\definecolor{deepblue}{rgb}{0,0,0.5}
\definecolor{deepred}{rgb}{0.6,0,0}
\definecolor{deepgreen}{rgb}{0,0.5,0}
\definecolor{forestgreen}{RGB}{34,139,34}
\definecolor{orangered}{RGB}{239,134,64}
\definecolor{darkblue}{rgb}{0.0,0.0,0.6}
\definecolor{gray}{rgb}{0.4,0.4,0.4}

\lstset {
  basicstyle=\ttfamily,
  frame=single
}


\setcounter{secnumdepth}{5}
\lstdefinestyle{XML} {
    language=XML,
    extendedchars=true,
    breaklines=true,
    breakatwhitespace=true,
%    emph={name,dim,interactive,overwrite},
    emphstyle=\color{red},
    basicstyle=\ttfamily,
%    columns=fullflexible,
    commentstyle=\color{gray}\upshape,
    morestring=[b]",
    morecomment=[s]{<?}{?>},
    morecomment=[s][\color{forestgreen}]{<!--}{-->},
    keywordstyle=\color{cyan},
    stringstyle=\ttfamily\color{black},
    tagstyle=\color{darkblue}\bf\ttfamily,
    morekeywords={name,type},
%    morekeywords={name,attribute,source,variables,version,type,release,x,z,y,xlabel,ylabel,how,text,param1,param2,color,label},
}
\lstset{language=python,upquote=true}

\usepackage{titlesec}
\newcommand{\sectionbreak}{\clearpage}
\setcounter{secnumdepth}{4}

%\titleformat{\paragraph}
%{\normalfont\normalsize\bfseries}{\theparagraph}{1em}{}
%\titlespacing*{\paragraph}
%{0pt}{3.25ex plus 1ex minus .2ex}{1.5ex plus .2ex}

%%%%%%%% Begin comands definition to input python code into document
\usepackage[utf8]{inputenc}

% Default fixed font does not support bold face
\DeclareFixedFont{\ttb}{T1}{txtt}{bx}{n}{9} % for bold
\DeclareFixedFont{\ttm}{T1}{txtt}{m}{n}{9}  % for normal

\usepackage{listings}

% Python style for highlighting
\newcommand\pythonstyle{\lstset{
language=Python,
basicstyle=\ttm,
otherkeywords={self, none, return},             % Add keywords here
keywordstyle=\ttb\color{deepblue},
emph={MyClass,__init__},          % Custom highlighting
emphstyle=\ttb\color{deepred},    % Custom highlighting style
stringstyle=\color{deepgreen},
frame=tb,                         % Any extra options here
showstringspaces=false            %
}}


% Python environment
\lstnewenvironment{python}[1][]
{
\pythonstyle
\lstset{#1}
}
{}

% Python for external files
\newcommand\pythonexternal[2][]{{
\pythonstyle
\lstinputlisting[#1]{#2}}}

\lstnewenvironment{xml}
{}
{}

% Python for inline
\newcommand\pythoninline[1]{{\pythonstyle\lstinline!#1!}}

% Named Colors for the comments below (Attempted to match git symbol colors)
\definecolor{RScolor}{HTML}{8EB361}  % Sonat (adjusted for clarity)
\definecolor{DPMcolor}{HTML}{E28B8D} % Dan
\definecolor{JCcolor}{HTML}{82A8D9}  % Josh (adjusted for clarity)
\definecolor{AAcolor}{HTML}{8D7F44}  % Andrea
\definecolor{CRcolor}{HTML}{AC39CE}  % Cristian
\definecolor{RKcolor}{HTML}{3ECC8D}  % Bob (adjusted for clarity)
\definecolor{DMcolor}{HTML}{276605}  % Diego (adjusted for clarity)
\definecolor{PTcolor}{HTML}{990000}  % Paul

\def\DRAFT{} % Uncomment this if you want to see the notes people have been adding
% Comment command for developers (Should only be used under active development)
\ifdefined\DRAFT
  \newcommand{\nameLabeler}[3]{\textcolor{#2}{[[#1: #3]]}}
\else
  \newcommand{\nameLabeler}[3]{}
\fi
\newcommand{\alfoa}[1] {\nameLabeler{Andrea}{AAcolor}{#1}}
\newcommand{\cristr}[1] {\nameLabeler{Cristian}{CRcolor}{#1}}
\newcommand{\mandd}[1] {\nameLabeler{Diego}{DMcolor}{#1}}
\newcommand{\maljdan}[1] {\nameLabeler{Dan}{DPMcolor}{#1}}
\newcommand{\cogljj}[1] {\nameLabeler{Josh}{JCcolor}{#1}}
\newcommand{\bobk}[1] {\nameLabeler{Bob}{RKcolor}{#1}}
\newcommand{\senrs}[1] {\nameLabeler{Sonat}{RScolor}{#1}}
\newcommand{\talbpaul}[1] {\nameLabeler{Paul}{PTcolor}{#1}}
% Commands for making the LaTeX a bit more uniform and cleaner
\newcommand{\TODO}[1]    {\textcolor{red}{\textit{(#1)}}}
\newcommand{\xmlAttrRequired}[1] {\textcolor{red}{\textbf{\texttt{#1}}}}
\newcommand{\xmlAttr}[1] {\textcolor{cyan}{\textbf{\texttt{#1}}}}
\newcommand{\xmlNodeRequired}[1] {\textcolor{deepblue}{\textbf{\texttt{<#1>}}}}
\newcommand{\xmlNode}[1] {\textcolor{darkblue}{\textbf{\texttt{<#1>}}}}
\newcommand{\xmlString}[1] {\textcolor{black}{\textbf{\texttt{'#1'}}}}
\newcommand{\xmlDesc}[1] {\textbf{\textit{#1}}} % Maybe a misnomer, but I am
                                                % using this to detail the data
                                                % type and necessity of an XML
                                                % node or attribute,
                                                % xmlDesc = XML description
\newcommand{\default}[1]{~\\*\textit{Default: #1}}
\newcommand{\nb} {\textcolor{deepgreen}{\textbf{~Note:}}~}


%%%%%%%% End comands definition to input python code into document

%\usepackage[dvips,light,first,bottomafter]{draftcopy}
%\draftcopyName{Sample, contains no OUO}{70}
%\draftcopyName{Draft}{300}

% The bm package provides \bm for bold math fonts.  Apparently
% \boldsymbol, which I used to always use, is now considered
% obsolete.  Also, \boldsymbol doesn't even seem to work with
% the fonts used in this particular document...
\usepackage{bm}


% Define tensors to be in bold math font.
\newcommand{\tensor}[1]{{\bm{#1}}}

% Override the formatting used by \vec.  Instead of a little arrow
% over the letter, this creates a bold character.
\renewcommand{\vec}{\bm}

% Define unit vector notation.  If you don't override the
% behavior of \vec, you probably want to use the second one.
\newcommand{\unit}[1]{\hat{\bm{#1}}}
% \newcommand{\unit}[1]{\hat{#1}}

% Use this to refer to a single component of a unit vector.
\newcommand{\scalarunit}[1]{\hat{#1}}

% \toprule, \midrule, \bottomrule for tables
\usepackage{booktabs}

% \llbracket, \rrbracket
\usepackage{stmaryrd}

\usepackage{hyperref}
\hypersetup{
    colorlinks,
    citecolor=black,
    filecolor=black,
    linkcolor=black,
    urlcolor=black
}

% Compress lists of citations like [33,34,35,36,37] to [33-37]
\usepackage{cite}

% If you want to relax some of the SAND98-0730 requirements, use the "relax"
% option. It adds spaces and boldface in the table of contents, and does not
% force the page layout sizes.
% e.g. \documentclass[relax,12pt]{SANDreport}
%
% You can also use the "strict" option, which applies even more of the
% SAND98-0730 guidelines. It gets rid of section numbers which are often
% useful; e.g. \documentclass[strict]{SANDreport}

% The INLreport class uses \flushbottom formatting by default (since
% it's intended to be two-sided document).  \flushbottom causes
% additional space to be inserted both before and after paragraphs so
% that no matter how much text is actually available, it fills up the
% page from top to bottom.  My feeling is that \raggedbottom looks much
% better, primarily because most people will view the report
% electronically and not in a two-sided printed format where some argue
% \raggedbottom looks worse.  If we really want to have the original
% behavior, we can comment out this line...
\raggedbottom
\setcounter{secnumdepth}{5} % show 5 levels of subsection
\setcounter{tocdepth}{5} % include 5 levels of subsection in table of contents

% ---------------------------------------------------------------------------- %
%
% Set the title, author, and date
%
\title{RAVEN Theory Manual}
%\author{%
%\begin{tabular}{c} Author 1 \\ University1 \\ Mail1 \\ \\
%Author 3 \\ University3 \\ Mail3 \end{tabular} \and
%\begin{tabular}{c} Author 2 \\ University2 \\ Mail2 \\ \\
%Author 4 \\ University4 \\ Mail4\\
%\end{tabular} }


\author{
\\Andrea Alfonsi
\\Cristian Rabiti
\\Diego Mandelli
\\Joshua Cogliati
\\Congjian Wang
\\Paul W. Talbot
\\Daniel P. Maljovec
\\Curtis Smith
\\Mohammad G. Abdo
}
% \\James B. Tompkins}   Just people who actually ``developed'' a significant capability in the code should be placed here. Andrea
%\author{\textbf{\textit{Main Developers:}}  \\Andrea Alfonsi}
%\affil{Idaho National Laboratory, Idaho Falls, ID 83402}
%\\\{cristian.rabiti, andrea.alfonsi, joshua.cogliati, diego.mandelli, robert.kinoshita, ramazan.sen\}@inl.gov}

% There is a "Printed" date on the title page of a SAND report, so
% the generic \date should [WorkingDir:]generally be empty.
\date{}


% ---------------------------------------------------------------------------- %
% Set some things we need for SAND reports. These are mandatory
%
\SANDnum{INL/EXT-16-38178}
\SANDprintDate{\today}
\SANDauthor{Andrea Alfonsi, Cristian Rabiti, Diego Mandelli, Joshua Cogliati, Congjian Wang, Paul W. Talbot, Daniel P. Maljovec, Curtis Smith,Mohammad G. Abdo}
\SANDreleaseType{Revision 4}


% ---------------------------------------------------------------------------- %
% Include the markings required for your SAND report. The default is "Unlimited
% Release". You may have to edit the file included here, or create your own
% (see the examples provided).
%
% \include{MarkOUO} % Not needed for unlimted release reports

\def\component#1{\texttt{#1}}

% ---------------------------------------------------------------------------- %
\newcommand{\systemtau}{\tensor{\tau}_{\!\text{SUPG}}}

% Added by Sonat
\usepackage{placeins}
\usepackage{array}

\newcolumntype{L}[1]{>{\raggedright\let\newline\\\arraybackslash\hspace{0pt}}m{#1}}
\newcolumntype{C}[1]{>{\centering\let\newline\\\arraybackslash\hspace{0pt}}m{#1}}
\newcolumntype{R}[1]{>{\raggedleft\let\newline\\\arraybackslash\hspace{0pt}}m{#1}}

% end added by Sonat
% ---------------------------------------------------------------------------- %
%
% Start the document
%

\begin{document}

    \maketitle

    % ------------------------------------------------------------------------ %
    % An Abstract is required for SAND reports
    %
%    \begin{abstract}
%    \input abstract
%    \end{abstract}


    % ------------------------------------------------------------------------ %
    % An Acknowledgement section is optional but important, if someone made
    % contributions or helped beyond the normal part of a work assignment.
    % Use \section* since we don't want it in the table of context
    %
%    \clearpage
%    \section*{Acknowledgment}



%	The format of this report is based on information found
%	in~\cite{Sand98-0730}.


    % ------------------------------------------------------------------------ %
    % The table of contents and list of figures and tables
    % Comment out \listoffigures and \listoftables if there are no
    % figures or tables. Make sure this starts on an odd numbered page
    %
    \cleardoublepage		% TOC needs to start on an odd page
    \tableofcontents
    %\listoffigures
    %\listoftables


    % ---------------------------------------------------------------------- %
    % An optional preface or Foreword
%    \clearpage
%    \section*{Preface}
%    \addcontentsline{toc}{section}{Preface}
%	Although muggles usually have only limited experience with
%	magic, and many even dispute its existence, it is worthwhile
%	to be open minded and explore the possibilities.


    % ---------------------------------------------------------------------- %
    % An optional executive summary
    %\clearpage
    %\section*{Summary}
    %\addcontentsline{toc}{section}{Summary}
    %\input{Summary.tex}
%	Once a certain level of mistrust and skepticism has
%	been overcome, magic finds many uses in todays science



%	and engineering. In this report we explain some of the
%	fundamental spells and instruments of magic and wizardry. We
%	then conclude with a few examples on how they can be used
%	in daily activities at national Laboratories.


    % ---------------------------------------------------------------------- %
    % An optional glossary. We don't want it to be numbered
%    \clearpage
%    \section*{Nomenclature}
%    \addcontentsline{toc}{section}{Nomenclature}
%    \begin{description}
%          \item[alohomoral]
%           spell to open locked doors and containers
%          \item[leviosa]
%           spell to levitate objects
%    \item[remembrall]
%           device to alert you that you have forgotten something
%    \item[wand]
%           device to execute spells
%    \end{description}


    % ---------------------------------------------------------------------- %
    % This is where the body of the report begins; usually with an Introduction
    %
    \SANDmain		% Start the main part of the report

\label{sec:introduction}
RAVEN (Risk Analysis and Virtual control ENviroment), under the support of the Nuclear Energy Advanced Modeling and Simulation (NEAMS) program ~\cite{neams}, is increasing its capabilities to perform probabilistic analysis of stochastic dynamic systems. This supports the goal of providing the tools needed by the Risk Informed Safety Margin Characterization (RISMC) path-lead ~\cite{mandelliANS_RISMC} under the Department of Energy (DOE) Light Water Reactor Sustainability program~\cite{lwrs}. In particular, the development of RAVEN in conjunction with the thermal-hydraulic code RELAP-7~\cite{relap7FY12}, will allow the deployment of advanced methodologies for nuclear power plant (NPP) safety analysis at the industrial level. The investigation of accident scenarios in a probabilistic environment for a complex system (i.e. NPPs) is not a minor task. The complexity of such systems, and a large quantity of stochastic parameters, lead to demanding computational requirements (several CPU/hour). Moreover, high consequence scenarios are usually located in low probability regions of the input space, making even more computational demands of the risk assessment process.

This extreme need for computational power leads to the necessity to investigate methodologies for the most efficient use of available computational resources, either by increasing effectiveness of the global exploration of input space, or by focusing on regions of interest (e.g. failure/success boundaries, etc.). The milestone reported in September 2013 ~\cite{DETmilestone2013} described the capability of RAVEN to perform exploration of the uncertain domain (probabilistic space) through the support of the well-known Dynamic Event Tree (DET) approach. This report will show that the Dynamic Event Tree approach can be considered intrinsically adaptive around the failure prone input zone, if one or more of the uncertain parameters is/are responsible for the transition of interest (e.g. failure or success). Leveraging on this feature is a natural choice to extend the classical DET approach to the Adaptive Dynamic Event Tree (ADET). This extension of the DET methodology to ADET and its implementation in the RAVEN code is the subject of this report. In order to show the effectiveness of this methodology, a Station Black Out (SBO) scenario for a Pressurized Water Reactor has been employed. The ADET approach will be used to focus the exploration of the input space toward the computation of the failure probability of the system (i.e. clad failure). This report is organized in four additional sections. Section 2 recalls the concept of the DET methodology. Section 3 reports how the newer developed algorithm is employed. Section 4 is focused on the analysis performed on the PWR SBO, and, section 5 draws the conclusions.

%\subsection{subsection}
%text

%figure template

%\subsubsection{subsubsection}
%more text
%\paragraph{paragraph}
%lot of text
%\subparagraph{subparagraph}
%if you arrive at this point you have issues

\input{ravenStructure.tex}
\input{forwardSampling.tex}
\input{adaptiveSampling.tex}
\input{reducedOrderModeling.tex}
\section{Statistical Analysis}
\label{sec:statisticalAnalysis}
One of the most assessed ways to investigate the impact of the intrinsic variation of the input space is through the computation of
statistical moments and linear correlation among variables/parameters/FOMs.

As shown in Section~\ref{sec:forwardSamplingStrategies}, RAVEN employs several different sampling methodologies to explore the response of a model subject to uncertainties. In order to correctly compute the statistical moments a weight-based approach is used. Each \textit{Sampler} in RAVEN associate to each ``sample'' (i.e.
realization in the input/uncertain space) a \textbf{weight}  to represent the \textit{importance} of the particular
combination of input values from a statistical point of view (e.g., reliability weights). These weights are used in subsequential
steps in order to compute the previously listed statistical moments and correlation metrics.
\\In the following subsections, the formulation of these statistical moments is reported.
\subsection{Expected Value}
The expected value represents one of the most fundamental metrics in probability theory: it represents a measurement of the center of the distribution (mean) of the random variable.
From a practical point of view, the expected value of a discrete random variable is the probability-weighted average of all possible values of the subjected variable. Formally, the expected value of a random variable $X$:
\begin{equation}
\begin{matrix}
\mathbb{E}(X) = \mu = \sum_{x \in \chi} x  pdf_{X}(x) & \text{if  $X$  discrete} \\
\\
\mathbb{E}(X) = \mu = \int_{x \in \chi} x pdf_{X}(x) & \, \text{if $X$ continuous}
\end{matrix}
\end{equation}
In RAVEN, the expected value (i.e. first central moment) is computed as follows:
\begin{equation}
\begin{matrix}
\mathbb{E}(X) = \mu \approx \overline{x} = \frac{1}{n} \sum_{i=1}^{n}  x_{i} & \text{if  random sampling} \\
\\
\mathbb{E}(X) = \mu \approx \overline{x} = \frac{1}{V_{1}} \sum_{i=1}^{n} w_{i}  x_{i}  & \, \text{otherwise}
\end{matrix}
\end{equation}
where:
\begin{itemize}
  \item $w_{i}$ is the weight associated with the sample $i$
  \item $n$ are the total number of samples
  \item $V_{1} = \sum_{i=1}^{n} w_{i}$.
\end{itemize}
\subsection{Standard Deviation and Variance}
The variance ($\sigma^{2}$) and standard deviation ($\sigma$) of $X$ are both measures of the spread of the distribution of the random variable about the
mean. Simplistically, the variance measures how far a set of realizations of a random variable are spread out.
The standard deviation is the square root of the variance. The standard deviation has the same unit of the original data, and hence is comparable to deviations from the mean.
\\Formally:
\begin{equation}
  \begin{matrix}
  \sigma^{2}(X)= \mathbb{E}\left(\left[X - \mathbb{E}(X)\right]^{2}\right) = \int_{x \in \chi} (x - \mu)^2 pdf(x) dx  & \,\text{if $X$ continuous} \\
  \sigma^{2}(X)= \mathbb{E}\left(\left[X - \mathbb{E}(X)\right]^{2}\right)  = \sum_{x \in \chi} (x - \mu)^2 pdf(x)  & \text{if  $X$ discrete}
  \\
  \\
  \sigma(X)= \mathbb{E}\left(\left[X - \mathbb{E}(X)\right]\right)  = \sqrt{\sigma^{2}(X)}
  \end{matrix}
\end{equation}
In RAVEN, variance (i.e., second central moment) and standard deviation are computed as follows:
\begin{equation}
\begin{matrix}
\mathbb{E}\left(\left[X - \mathbb{E}(X)\right]^{2}\right)  \approx  m_{2} = \frac{1}{n} \sum_{i=1}^{n}  (x_{i} - \overline{x})^{2} & \text{if  random sampling} \\
\\
\\
\mathbb{E}\left(\left[X - \mathbb{E}(X)\right]^{2}\right)  \approx m_{2}  = \frac{1}{V_{1}} \sum_{i=1}^{n} w_{i}  (x_{i} - \overline{x})^{2}  & \, \text{otherwise}
\\
\\
\mathbb{E}\left(\left[X - \mathbb{E}(X)\right]^{2}\right)  \approx s  =  \sqrt{m_{2}}
\end{matrix}
\end{equation}
where:
\begin{itemize}
  \item $w_{i}$ is the weight associated with the sample $i$
  \item $n$ are the total number of samples
  \item $V_{1} = \sum_{i=1}^{n} w_{i}$.
\end{itemize}
RAVEN performs an additional correction of variance to obtain an unbiased estimation  with respect to the sample-size~\cite{RimoldiniUnbiased}:
\begin{equation}
\begin{matrix}
\mathbb{E}\left(\left[X - \mathbb{E}(X)\right]^{2}\right)  \approx M_{2} = \displaystyle \frac{n}{n-1}m_{2} & & \text{if random sampling}
\\
\mathbb{E}\left(\left[X - \mathbb{E}(X)\right]^{2}\right)  \approx M_{2} = \frac{V_{1}^{2}}{V_{1}^{2} - V_{2}}m_{2} &  text{otherwise}
\end{matrix}
\end{equation}
\begin{equation}
S = \sqrt{M_{2}}
\end{equation}
where:
\begin{itemize}
  \item $w_{i}$ is the weight associated with the sample $i$
  \item $n$ are the total number of samples
  \item $V_{1} = \sum_{i=1}^{n} w_{i}^{1}$.
  \item $V_{2} = \sum_{i=1}^{n} w_{i}^{2}$.
\end{itemize}
It is important to notice that $S$ is not an unbiased estimator.

\subsection{Skewness}
The Skewness is a measure of the asymmetry of the distribution of a
real-valued random variable about its mean. Negative skewness
indicates that the tail on the left side of the distribution is longer or fatter
than the right side.  Positive skewness indicates that the tail on the right
side is longer or fatter than the left side. From a practical point of view, the
skewness is useful to identify distortion  of the random variable with respect to
the Normal distribution function.
\\Formally,
\begin{equation}
\gamma_{1} = \mathbb{E} \left [ \left ( \frac{X-\mu}{\sigma} \right )^{3} \right ] = \frac{ \mathbb{E}\left [ \left ( X-\mu \right )^{3} \right ]}{\left ( \mathbb{E}\left [ \left ( X-\mu \right )^{2} \right ] \right )^{3/2}}
\end{equation}
In RAVEN, the skewness is computed as follows:
\begin{equation}
\begin{matrix}
\mathbb{E} \left [ \left ( \frac{X-\mu}{\sigma} \right )^{3} \right ]  \approx \frac{m_{3}}{m_{2}^{3/2}} = \frac{  \frac{1}{n} \sum_{i=1}^{n}  (x_{i} - \overline{x})^{3} }{\left ( \frac{1}{n} \sum_{i=1}^{n}  (x_{i} - \overline{x})^{2} \right )^{3/2}} & \text{if random sampling}
\\
\\
\mathbb{E} \left [ \left ( \frac{X-\mu}{\sigma} \right )^{3} \right ]  \approx \frac{m_{3}}{m_{2}^{3/2}} = \frac{  \frac{1}{V_{1}} \sum_{i=1}^{n} w_{i} \times (x_{i} - \overline{x})^{3} }{\left ( \frac{1}{V_{1}} \sum_{i=1}^{n}  w_{i} \times (x_{i} - \overline{x})^{2} \right )^{3/2}} &  \, \text{otherwise}
\end{matrix}
\end{equation}
where:
\begin{itemize}
  \item $w_{i}$ is the weight associated with the sample $i$
  \item $n$ are the total number of samples
  \item $V_{1} = \sum_{i=1}^{n} w_{i}$.
\end{itemize}
RAVEN performs an additional correction of skewness to obtain an unbiased estimation  with respect to the sample-size~\cite{RimoldiniUnbiased}:
\begin{equation}
\begin{matrix}
\mathbb{E} \left [ \left ( \frac{X-\mu}{\sigma} \right )^{3} \right ]  \approx \frac{M_{3}}{M_{2}^{3/2}}  = \displaystyle \frac{n^{2}}{(n-1)(n-2)}m_{3}\times \frac{1}{\left ( \displaystyle \frac{n}{n-1}m_{2}  \right )^{3/2}} & \text{if random sampling}
\\
\\
\mathbb{E} \left [ \left ( \frac{X-\mu}{\sigma} \right )^{3} \right ]  \approx \frac{M_{3}}{M_{2}^{3/2}}  = \displaystyle \frac{V_{1}^{3}}{V_{1}^{3}-3V_{1}V_{2}+2V_{3}}m_{3} \times \frac{1}{\left ( \displaystyle \frac{V_{1}^{2}}{V_{1}^{2}-V_{2}}m_{2}  \right )^{3/2}} &  \,  \text{otherwise}
\end{matrix}
\end{equation}
where:
\begin{itemize}
  \item $w_{i}$ is the weight associated with the sample $i$
  \item $n$ are the total number of samples
  \item $V_{1} = \sum_{i=1}^{n} w_{i}^{1}$
  \item $V_{2} = \sum_{i=1}^{n} w_{i}^{2}$
  \item $V_{3} = \sum_{i=1}^{n} w_{i}^{3}$.
\end{itemize}

\subsection{Excess Kurtosis}
The  Kurtosis~\cite{Abramowitz}  is the degree of peakedness of a distribution of a real-valued random variable. In a similar way to the concept of skewness, kurtosis describes the shape of the distribution. The Kurtosis is defined in order to
obtain a value of $0$ for a Normal distribution. If it is greater than zero, it indicates that the distribution is high peaked; If it is smaller
that zero, it testifies that the distribution is flat-topped.
\\Formally, the Kurtosis can be expressed as follows:
\begin{equation}
\gamma_{2} = \frac{ \mathbb{E}\left [ \left ( X-\mu \right )^{4} \right ]}{\left ( \mathbb{E}\left [ \left ( X-\mu \right )^{2} \right ] \right )^{2}}
\end{equation}
In RAVEN, the kurtosis (excess) is computed as follows:
\begin{equation}
\begin{matrix}
\frac{ \mathbb{E}\left [ \left ( X-\mu \right )^{4} \right ]}{\left ( \mathbb{E}\left [ \left ( X-\mu \right )^{2} \right ] \right )^{2}}   \approx \frac{m_{4}-3m_{2}^{2}}{m_{2}^{2}} = \displaystyle  \frac{  \frac{1}{n} \sum_{i=1}^{n}  (x_{i} - \overline{x})^{4} -3\left ( \frac{1}{n} \sum_{i=1}^{n}  (x_{i} - \overline{x})^{2} \right )^{2}}{\left ( \frac{1}{n} \sum_{i=1}^{n}  (x_{i} - \overline{x})^{2} \right )^{2}} & \text{if random sampling}
\\
\\
\frac{ \mathbb{E}\left [ \left ( X-\mu \right )^{4} \right ]}{\left ( \mathbb{E}\left [ \left ( X-\mu \right )^{2} \right ] \right )^{2}}   \approx \frac{m_{4}-3m_{2}^{2}}{m_{2}^{2}} = \displaystyle  \frac{  \frac{1}{V_{1}} \sum_{i=1}^{n} w_{i} \times (x_{i} - \overline{x})^{4} -3\left ( \frac{1}{V_{1}} \sum_{i=1}^{n}  w_{i} \times (x_{i} - \overline{x})^{2} \right )^{2}}{\left ( \frac{1}{V_{1}} \sum_{i=1}^{n}  w_{i} \times (x_{i} - \overline{x})^{2} \right )^{2}} &   \text{otherwise}
\end{matrix}
\end{equation}
where:
\begin{itemize}
  \item $w_{i}$ is the weight associated with the sample $i$
  \item $n$ are the total number of samples
  \item $V_{1} = \sum_{i=1}^{n} w_{i}$.
\end{itemize}
RAVEN performs an additional correction of kurtosis (excess) to obtain an unbiased estimation  with respect to the sample-size~\cite{RimoldiniUnbiased}:
\begin{equation}
\begin{split}
\begin{matrix}
\frac{ \mathbb{E}\left [ \left ( X-\mu \right )^{4} \right ]}{\left ( \mathbb{E}\left [ \left ( X-\mu \right )^{2} \right ] \right )^{2}}   \approx \frac{M_{4}-3M_{2}^{2}}{M_{2}^{2}}  = \displaystyle \frac{n^{2}(n+1)}{(n-1)(n-2)(n-3)}m_{4}-\frac{3n^{2}}{(n-2)(n-3)}m_{2}^{2} & \text{if random sampling}
\\
\\
\frac{ \mathbb{E}\left [ \left ( X-\mu \right )^{4} \right ]}{\left ( \mathbb{E}\left [ \left ( X-\mu \right )^{2} \right ] \right )^{2}}    \approx \frac{M_{4}-3M_{2}^{2}}{M_{2}^{2}}  = \displaystyle  \frac{V_{1}^{2}(V_{1}^{4}-4V_{1}V_{3}+3V_{2}^{2})}{(V_{1}^{2}-V_{2})(V_{1}^{4}-6V_{1}^{2}V_{2}+8V_{1}V_{3}+3V_{2}^{2}-6V_{4})}m_{4}
- \\
\displaystyle \frac{3V_{1}^{2}(V_{1}^{4}-2V_{1}^{2}V_{2}+4V_{1}V_{3}-3V_{2}^{2})}{(V_{1}^{2}-V_{2})(V_{1}^{4}-6V_{1}^{2}V_{2}+8V_{1}V_{3}+3V_{2}^{2}-6V_{4})}m_{2}^{2} & \text{otherwise}
\end{matrix}
\end{split}
\end{equation}
where:
\begin{itemize}
  \item $w_{i}$ is the weight associated with the sample $i$
  \item $n$ are the total number of samples
  \item $V_{1} = \sum_{i=1}^{n} w_{i}^{1}$
  \item $V_{2} = \sum_{i=1}^{n} w_{i}^{2}$
  \item $V_{3} = \sum_{i=1}^{n} w_{i}^{3}$
  \item $V_{4} = \sum_{i=1}^{n} w_{i}^{4}$.
\end{itemize}

\subsection{Median}
The median of the distribution of a real-valued random variable is the number separating the higher half from the lower half of all
the possible values. The median of a finite list of numbers can be found by arranging all the observations from lowest value to highest value and picking the middle value.
\\Formally, the median $m$ can be cast as the number that satisfy the following relation:
\begin{equation}
  P(X\leq m) = P(X \geq m) = \int_{-\infty}^{m} pdf(x) dx=\frac{1}{2}
\end{equation}

\subsection{Percentile}
A percentile (or a centile) is a measure indicating the value below which a given percentage of observations in a group of observations fall.

\subsection{Covariance and Correlation Matrices}
Simplistically, the Covariance is a measure of how much two random variables variate together. In other words, It represents a
measurement of the correlation, in terms of variance,  among different variables. If the greater values of one variable mainly
correspond with the greater values of the other variable, and the same holds for the lesser values (i.e., the variables tend to show
similar behavior) the covariance is positive. In the opposite case, when the greater values of one variable mainly correspond to the
lesser values of the other (i.e., the variables tend to show opposite behavior) the covariance is negative.
Formally, the Covariance can be expressed as
\begin{equation}
 \boldsymbol{\Sigma}(\boldsymbol{X},\boldsymbol{Y})  = \mathbb{E} \left [ \left ( \boldsymbol{X}- \mathbb{E}\left [ \boldsymbol{X} \right ] \right ) \left ( \boldsymbol{Y}- \mathbb{E}\left [ \boldsymbol{Y} \right ] \right )^{T}\right ]
\end{equation}
Based on the previous equation, in RAVEN each entry of the Covariance matrix is computed as follows:
\begin{equation}
\begin{matrix}
 \mathbb{E} \left [ \left ( X- \mathbb{E}\left [ X \right ] \right ) \left ( Y- \mathbb{E}\left [ Y \right ] \right )\right ] \approx
 \frac{1}{n}\sum_{i=1}^{n} (x_{i} - \mu_{x})(y_{i} -  \mu_{y})  & \text{if random sampling}
\\
\\
 \mathbb{E} \left [ \left ( X- \mathbb{E}\left [ X \right ] \right ) \left ( Y- \mathbb{E}\left [ Y \right ] \right )\right ] \approx
\frac{1}{V_{1}} \sum_{i=1}^{n} w_{i} \times (x_{i} -  \mu_{x})(y_{i} -  \mu_{y}) &   \text{otherwise}
\end{matrix}
\end{equation}
where:
\begin{itemize}
  \item $w_{i}$ is the weight associated with the sample $i$
  \item $n$ are the total number of samples
  \item $V_{1} = \sum_{i=1}^{n} w_{i}$.
\end{itemize}
The correlation matrix (Pearson product-moment correlation coefficient matrix) can be obtained through the Covariance matrix, as follows:
\begin{equation}
\boldsymbol{\Gamma}(\boldsymbol{X},\boldsymbol{Y}) = \frac{\boldsymbol{\Sigma}(\boldsymbol{X},\boldsymbol{Y})}{\sigma_{x} \sigma_{y}}
\end{equation}
As it can be seen, The correlation between $X$ and $Y$ is the
covariance of the corresponding standard scores.

\subsection{Spearman's Rank Correlation Coefficient Matrix}
Spearman's rank correlation coefficient or Spearman's $\rho$ is a nonparametric measure of rank correlation
 (statistical dependence between the rankings of two variables). 
 It assesses how well the relationship between two variables can be described using a monotonic function.

The Spearman correlation coefficient between two variables is equal to the Pearson correlation coefficient 
between the rank values of those two variables; Spearman's 
correlation assesses monotonic relationships (whether linear or not). If there are no repeated data values, 
a perfect Spearman correlation of $+1$ or $-1$ occurs when each of the variables is a perfect monotone
function of the other.

The Spearman correlation between two variables will be large when observations have a similar 
(or identical for a correlation of $1$) rank between the two variables, and low when observations
 have a dissimilar (or fully opposed for a correlation of $-1$) rank between the two variables.

The Spearman correlation  matrix  can be obtained as the Pearson correlation coefficient matrix of the ranked variables, as follows:
\begin{equation}
\boldsymbol{\Gamma}(\boldsymbol{R(X)},\boldsymbol{R(Y)}) = \frac{\boldsymbol{\Sigma}(\boldsymbol{R(X)},\boldsymbol{R(Y)})}{\sigma_{R(x)} \sigma_{R(y)}}
\end{equation}

In RAVEN, the Spearman correlation  matrix  is computed in its weighted form \cite{Bailey2018}. 
Firstly, each element of the vector of realizations (each variable) is  ranked using the following weighted formulation:
\begin{equation}
    R(X)_j = \sum_{i=1}^n (w_i *\Theta(z_i, z_j)) + \frac{1+\sum_{i=1}^{n}  \Gamma(w_i, w_j)} {2} * \frac{\sum_{i=1}^{n} w_i*\Gamma(w_i, w_j)}{\sum_{i=1}^{n} \Gamma(w_i, w_j)}
\end{equation}
where: 
\begin{equation}
\Theta (z_i, z_j)=\left\{\begin{matrix}
1 & if \; w_i = w_j  \\
 0 & if \; w_i \neq w_j \\
\end{matrix}\right.
\end{equation}
and
\begin{equation}
     \Gamma (w_i, w_j) =\begin{cases}1 \rightarrow if \;  w_i = w_j \\ 0  \rightarrow if \; w_i \neq  w_j\end{cases}
\end{equation}

Once all the variables are ranked, the vector of the ranked variables are used for computing the weighted Covariance matrix:
\begin{equation}
 \boldsymbol{\Sigma}(\boldsymbol{R(X)},\boldsymbol{R(Y)})  = \mathbb{E} \left [ \left ( \boldsymbol{R(X)}- \mathbb{E}\left [ \boldsymbol{R(X)} \right ] \right ) \left ( \boldsymbol{R(Y)}- \mathbb{E}\left [ \boldsymbol{R(Y)} \right ] \right )^{T}\right ]
\end{equation}
Finally, the Spearman correlation  matrix  is then computed from the weighted Covariance matrix (see Pearson correlation matrix above) of the ranked variables:
 \begin{equation}
\boldsymbol{\mathrm{P}}(\boldsymbol{X},\boldsymbol{Y}) = \frac{\boldsymbol{\Sigma}(\boldsymbol{R(X)},\boldsymbol{R(Y)})}{\sigma_{R(x)} \sigma_{R(y)}}
\end{equation}

\subsection{Variance-Dependent Sensitivity Matrix}
The variance dependent sensitivity matrix is the matrix of the sensitivity
coefficients that show the relationship of the individual uncertainty
component to the standard deviation of the reported value for a test
item.
\\ Formally:
\begin{equation}
\boldsymbol{\Lambda}= \boldsymbol{\Sigma}(\boldsymbol{X},\boldsymbol{Y})  vc^{-1}(\boldsymbol{Y})
\end{equation}
where:
\begin{itemize}
  \item $vc^{-1}(\boldsymbol{Y})$ is the inverse of the covariance of the
  input space.
\end{itemize}

\subsection{Normalized Sensitivity Matrix}
The normalized sensitivity matrix is the matrix of the sensitivity
coefficients normalized with respect their expected values. This means that this matrix
represents, in a linear formulation, the p.u. sensitivity coefficients.
\\ Formally:
\begin{equation}
\boldsymbol{\Lambda}= \left ( \boldsymbol{\Sigma}(\boldsymbol{X},\boldsymbol{Y})  vc^{-1}(\boldsymbol{Y}) \right ) \times \frac{\mathbb{E}(\boldsymbol{Y}) }{\mathbb{E}(\boldsymbol{X}) } 
\end{equation}

where:
\begin{itemize}
  \item $vc^{-1}(\boldsymbol{Y})$ is the inverse of the covariance of the
  input space.
  \item $\mathbb{E}(\boldsymbol{Y})$ is the expected value of the input space
  \item $\mathbb{E}(\boldsymbol{X})$ is the expected value of the output space
\end{itemize}








\input{dataMining.tex}
\section{Dynamical System Scaling}
\label{sec:dssdoc}

The DSS approach to system scaling is based on transforming the typical view of processes to a special coordinate system in terms of the parameter of interest and its agents of change \cite{DSS2015}.
By parameterizing using a time term that will be introduced later in this section, data reproduced can be converted to the special three coordinate system (also called the phase space)
and form a geometry with curves along the surface containing invariant and intrinsic properties. The remainder of this section is a review of DSS theory introduced in publications
by Reyes \cite{DSS2015,Reyes2015,Martin2019} and is used in this analysis for FR scaling. The parameter of interest is defined to be a conserved quantity within a control volume:
\begin{equation}
  \label{eq_1}
  \beta(t)=\frac{1}{\Psi_{0}}\iiint_{V}{\psi\left(\vec{x},t\right)}dV
\end{equation}
$\beta$ is defined as the volume integral of the time and space dependent conserved quantity $\psi$ normalized by a time-independent value, $\Psi_{0}$, that characterizes the process. The agents of change are defined as the first derivative of the normalized parameter of interest:
\begin{equation}
  \label{eq_2}
  \omega=\frac{1}{\Psi_{0}}\frac{d}{dt}\iiint_{V}{\psi\left(\vec{x},t\right)}dV=\iiint_{V}{\left(\phi_{v}+\phi_{f}\right)}dV+\iint_{A}{\left(\vec{j}\cdot\vec{n}\right)}dA-\iint_{A}{\psi\left(\vec{v}-\vec{v}_{s}\cdot\vec{n}dA\right)}dA
\end{equation}
The change is categorized into three components; volumetric, surface, and quantity transport. The agents of change is also the sum of the individual agent of change:
\begin{equation}
  \omega=\frac{1}{\Psi_{0}}\frac{d}{dt}\iiint_{V}{\psi\left(\vec{x},t\right)}dV=\sum^{n}_{i=1}{\omega_{i}}
\end{equation}
The relation of $\omega$ and $\beta$ is the following:
\begin{equation}
  \label{eq_3}
  \omega(t)=\left.\frac{d\beta}{dt}\right|_{t}=\sum^{n}_{i=1}{\omega_{i}}
\end{equation}
Where $\omega$ is the first derivative of reference time. As defined in Einstein and Infeld, time is a value stepping in constant increments \cite{Einstein1966}. The process dependent term in DSS is called process time:
\begin{equation}
  \label{eq_4}
  \tau(t)=\frac{\beta(t)}{\omega(t)}
\end{equation}
To measure the progression difference between reference time and process time in respect to reference time, the idea of temporal displacement rate (D) is adopted:
\begin{equation}
  \label{eq_5}
  D=\frac{d\tau-dt}{dt}=-\frac{\beta}{\omega^{2}}\frac{d\omega}{dt}
\end{equation}
The interval of process time is:
\begin{equation}
  \label{eq_8}
  d\tau=\tau_{s}=\left(1+D\right)dt
\end{equation}
Applying the process action to normalize the phase space coordinates gives the following normalized terms:
\begin{equation}
  \label{eq_10}
  \tilde{\Omega}=\omega\tau_{s},\qquad \tilde{\beta}=\beta,\qquad \tilde{t}=\frac{t}{\tau_{s}},\qquad \tilde{\tau}=\frac{\tau}{\tau_{s}},\qquad
  \tilde{D}=D
\end{equation}
The scaling relation between the prototype and model can be defined both for $\beta$ and $\omega$ and represents the scaling of the parameter of interest and the corresponding agents of change (or frequency given from the units of per time):
\begin{equation}
  \label{eq_11}
  \lambda_{A}=\frac{\beta_{M}}{\beta_{P}},\qquad \lambda_{B}=\frac{\omega_{M}}{\omega_{P}}
\end{equation}
The subscripts $M$ and $P$ stand for the model and prototype. Applying these scaling ratios to equations (\ref{eq_4}), (\ref{eq_5}), and (\ref{eq_10}) provides the scaling ratios for other parameters as well:
\begin{equation}
  \label{eq_12}
  \frac{t_{M}}{t_{P}}=\frac{\lambda_{A}}{\lambda_{B}},\qquad \frac{\tau_{M}}{\tau_{P}}=\frac{\lambda_{A}}{\lambda_{B}},\qquad \frac{\tilde{\beta}_{M}}{\tilde{\beta}_{P}}=\lambda_{A},\qquad \frac{\tilde{\Omega}_{M}}{\tilde{\Omega}_{P}}=\lambda_{A},\qquad \frac{\tilde{\tau}_{M}}{\tilde{\tau}_{P}}=1,\qquad \frac{D_{M}}{D_{P}}=1
\end{equation}
Normalized agents of change is the sum in the same respect:
\begin{equation}
  \label{eq_18}
  \Omega=\sum^{k}_{i=1}{\Omega_{i}}
\end{equation}
The ratio of $\Omega$ is expressed in the following alternate form:
\begin{equation}
  \label{eq_19}
  \Omega_{R}=\frac{\Omega_{M}}{\Omega_{P}}=\frac{\sum^{k}_{i=1}{\Omega_{M,i}}}{\sum^{k}_{i=1}{\Omega_{P,i}}}=\frac{\Omega_{M,1}+\Omega_{M,2}+...+\Omega_{M,k}}{\Omega_{P,1}+\Omega_{P,2}+...+\Omega_{P,k}}
\end{equation}
By the law of scaling ratios, The following must be true:
\begin{equation}
  \label{eq_13}
  \lambda_{A}=\frac{\Omega_{M,1}}{\Omega_{P,1}},\lambda_{A}=\frac{\Omega_{M,2}}{\Omega_{P,2}},...,\lambda_{A}=\frac{\Omega_{M,k}}{\Omega_{P,k}}
\end{equation}
Depending on the scaling ratio values, From Reyes, the scaling methods and similarity criteria is subdivided into five categories; 2-2 affine, dilation, $\beta$-strain, $\omega$-strain, and identity \cite{DSS2015}.
Table \ref{DSS:table_1} summarizes the similarity criteria. Despite the five categories, in essence, all are 2-2 affine with exceptions of partial scaling ratios values being 1.
\begin{table}[H]
\centering
\begin{tabular}{c|c|c|c|c}
\hline
%\rowcolor{lightgray}
\multicolumn{5}{c}{Basis for Process Space-time Coordinate Scaling}\\
\hline
Metric & \multirow{2}{*}{$d\tilde{\tau}_{P}=d\tilde{\tau}_{P}$} & \multirow{2}{*}{And} & Covariance & \multirow{2}{*}{$\frac{1}{\omega_{P}}\frac{d\beta_{P}}{dt_{P}}=\frac{1}{\omega_{M}}\frac{d\beta_{M}}{dt_{M}}$} \\
Invariance   & & & Principle & \\
\hline
\multicolumn{5}{c}{$\beta-\omega$ Coordinate Transformations}\\
\hline
2-2 Affine  & Dilation  & $\beta$-Strain & $\omega$-Strain & Identity \\
$\beta_{R}=\lambda_{A}$ & $\beta_{R}=\lambda$ & $\beta_{R}=\lambda_{A}$ & $\beta_{R}=1=\lambda_{B}$ & $\beta_{R}=1$ \\
$\omega_{R}=\lambda_{B}$ & $\omega_{R}=\lambda$ & $\omega_{R}=1$ & $\omega_{R}=\lambda_{B}$ & $\omega_{R}=1$ \\
\hline
\multicolumn{5}{c}{Similarity Criteria}\\
\hline
$\tilde{\Omega}_{R}=\lambda_{A}$ & $\tilde{\Omega}_{R}=\lambda$ & $\tilde{\Omega}_{R}=\lambda_{A}$ & $\tilde{\Omega}_{R}=1$ & $\tilde{\Omega}_{R}=1$ \\
$\tau_{R}=t_{R}=\frac{\lambda_{A}}{\lambda_{B}}$ & $\tau_{R}=t_{R}=1$ & $\tau_{R}=t_{R}=\lambda_{A}$ & $\tau_{R}=t_{R}=\frac{1}{\lambda_{B}}$ & $\tau_{R}=t_{R}=1$ \\
\hline
\end{tabular}
\caption{Scaling Methods and Similarity Criteria Resulting from Two-Parameter Transformations \cite{DSS2015}}\label{DSS:table_1}
\end{table}
The separation between both process curves along the constant normalized process time is the local distortion \cite{Martin2019}:
\begin{equation}
  \label{eq_15}
  \tilde{\eta}_{k}=\beta_{P_{k}}\sqrt{\varepsilon D_{P_{k}}}\left[\frac{1}{\Omega_{P_{k}}}-\frac{\lambda_{A}}{\Omega_{M_{k}}}\right]
\end{equation}
Where $\epsilon$ is a sign adjuster ensuring positive values within the square root. The total distortion is:
\begin{equation}
  \label{eq_16}
  \tilde{\eta}_{T}=\sum^{N}_{k=1}{\left|\tilde{\eta}_{k}\right|}
\end{equation}
And, the equivalent standard deviation is:
\begin{equation}
  \label{eq_17}
  \sigma_{est}=\sqrt{\frac{1}{N}\sum^{N}_{k=1}{\tilde{\eta}^{2}_{k}}}
\end{equation}


\clearpage
\begin{appendices}
  \section{Document Version Information}
  \input{../version.tex}
\end{appendices}
%\appendix
\section{Appendix: Example Primer}
\label{sec:examplePrimer}
In this Appendix, a set of examples are reported. In order to be as general as possible, the \textit{Model} type ``ExternalModel'' has been used.
%%%% EXAMPLE 1
\subsection{Example 1.}
\label{subsec:ex1}
This simple example is about the construction of a ``Lorentz attractor'', sampling the relative input space. The parameters that are sampled represent the initial coordinate (x0,y0,z0) of the attractor origin.

\begin{lstlisting}[style=XML,morekeywords={debug,re,seeding,class,subType,limit}]
<?xml version="1.0" encoding="UTF-8"?>
<Simulation verbosity="debug">
<!-- RUNINFO -->
<RunInfo>
    <WorkingDir>externalModel</WorkingDir>
    <Sequence>FirstMRun</Sequence>
    <batchSize>3</batchSize>
</RunInfo>
<!-- Files -->
<Files>
    <Input name='lorentzAttractor.py' type=''>lorentzAttractor</Input>
</Files>
<!-- STEPS -->
<Steps>
    <MultiRun name='FirstMRun'  re-seeding='25061978'>
        <Input   class='Files'     type=''               >lorentzAttractor.py</Input>
        <Model   class='Models'    type='ExternalModel'  >PythonModule</Model>
        <Sampler class='Samplers'  type='MonteCarlo'     >MC_external</Sampler>
        <Output  class='DataObjects'     type='HistorySet'      >testPrintHistorySet</Output>
        <Output  class='Databases' type='HDF5'           >test_external_db</Output>
        <Output  class='OutStreams' type='Print'   >testPrintHistorySet_dump</Output>
    </MultiRun >
</Steps>
<!-- MODELS -->
<Models>
    <ExternalModel name='PythonModule' subType='' ModuleToLoad='externalModel/lorentzAttractor'>
       <variables>sigma,rho,beta,x,y,z,time,x0,y0,z0</variables>
    </ExternalModel>
</Models>
<!-- DISTRIBUTIONS -->
<Distributions>
    <Normal name='x0_distrib'>
        <mean>4</mean>
        <sigma>1</sigma>
    </Normal>
    <Normal name='y0_distrib'>
        <mean>4</mean>
        <sigma>1</sigma>
    </Normal>
    <Normal name='z0_distrib'>
        <mean>4</mean>
        <sigma>1</sigma>
    </Normal>
</Distributions>
<!-- SAMPLERS -->
<Samplers>
    <MonteCarlo name='MC_external'>
      <samplerInit>
        <limit>3</limit>
      </samplerInit>
      <variable name='x0' >
        <distribution  >x0_distrib</distribution>
      </variable>
      <variable name='y0' >
        <distribution  >y0_distrib</distribution>
      </variable>
      <variable name='z0' >
        <distribution  >z0_distrib</distribution>
      </variable>
    </MonteCarlo>
</Samplers>
<!-- DATABASES -->
<Databases>
  <HDF5 name="test_external_db"/>
</Databases>
<!-- OUTSTREAMS -->
<OutStreams>
  <Print name='testPrintHistorySet_dump'>
    <type>csv</type>
    <source>testPrintHistorySet</source>
  </Print>
</OutStreams>
<!-- DATA OBJECTS -->
<DataObjects>
    <HistorySet name='testPrintHistorySet'>
        <Input>x0,y0,z0</Input>
        <Output>time,x,y,z</Output>
   </HistorySet>
</DataObjects>
</Simulation>
\end{lstlisting}
The Python \textit{ExternalModel} is reported below:
\begin{lstlisting}[language=python]
import numpy as np

def run(self,Input):
  max_time = 0.03
  t_step = 0.01

  numberTimeSteps = int(max_time/t_step)

  self.x = np.zeros(numberTimeSteps)
  self.y = np.zeros(numberTimeSteps)
  self.z = np.zeros(numberTimeSteps)
  self.time = np.zeros(numberTimeSteps)

  self.x0 = Input['x0']
  self.y0 = Input['y0']
  self.z0 = Input['z0']

  self.x[0] = Input['x0']
  self.y[0] = Input['y0']
  self.z[0] = Input['z0']
  self.time[0]= 0

  for t in range (numberTimeSteps-1):
    self.time[t+1] = self.time[t] + t_step
    self.x[t+1]    = self.x[t] +  self.sigma*
                      (self.y[t]-self.x[t]) * t_step
    self.y[t+1]    = self.y[t] + (self.x[t]*
                      (self.rho-self.z[t])-self.y[t]) * t_step
    self.z[t+1]    = self.z[t] + (self.x[t]*
                          self.y[t]-self.beta*self.z[t]) * t_step
\end{lstlisting}
%%%% EXAMPLE 2
\subsection{Example 2.}
\label{subsec:ex1}
This example shows a slightly more complicated example, that employs the usage of:
\begin{itemize}
    \item \textit{Samplers:} Grid and Adaptive;
    \item \textit{Models:} External, Reduce Order Models and Post-Processors;
    \item \textit{OutStreams:} Prints and Plots;
    \item \textit{Data Objects:} PointSets;
    \item \textit{Functions:} ExternalFunctions.
\end{itemize}
The goal of this input is to compute the ``SafestPoint''.
It provides the coordinates of the farthest
point from the limit surface that is given as an input.
%
The safest point coordinates are expected values of the coordinates of the
farthest points from the limit surface in the space of the ``controllable''
variables based on the probability distributions of the ``non-controllable''
variables.

The term ``controllable'' identifies those variables that are under control
during the system operation, while the ``non-controllable'' variables are
stochastic parameters affecting the system behavior randomly.

The ``SafestPoint'' post-processor requires the set of points belonging to the
limit surface, which must be given as an input.

\begin{lstlisting}[style=XML,morekeywords={debug,re,seeding,class,subType,limit}]
<Simulation verbosity='debug'>

<!-- RUNINFO -->
<RunInfo>
  <WorkingDir>SafestPointPP</WorkingDir>
  <Sequence>pth1,pth2,pth3,pth4</Sequence>
  <batchSize>50</batchSize>
</RunInfo>

<!-- STEPS -->
<Steps>
  <MultiRun name = 'pth1' pauseAtEnd = 'False'>
    <Sampler  class = 'Samplers'  type = 'Grid'           >grd_vl_ql_smp_dpt</Sampler>
    <Input    class = 'DataObjects'     type = 'PointSet'   >grd_vl_ql_smp_dpt_dt</Input>
    <Model    class = 'Models'    type = 'ExternalModel'  >xtr_mdl</Model>
    <Output   class = 'DataObjects'     type = 'PointSet'   >nt_phy_dpt_dt</Output>
  </MultiRun >

  <MultiRun name = 'pth2' pauseAtEnd = 'True'>
    <Sampler          class = 'Samplers'  type = 'Adaptive'      >dpt_smp</Sampler>
    <Input            class = 'DataObjects'     type = 'PointSet'  >bln_smp_dt</Input>
    <Model            class = 'Models'    type = 'ExternalModel' >xtr_mdl</Model>
    <Output           class = 'DataObjects'     type = 'PointSet'  >nt_phy_dpt_dt</Output>
    <SolutionExport   class = 'DataObjects'     type = 'PointSet'  >lmt_srf_dt</SolutionExport>
  </MultiRun>

  <PostProcess name='pth3' pauseAtEnd = 'False'>
    <Input    class = 'DataObjects'          type = 'PointSet'       >lmt_srf_dt</Input>
    <Model    class = 'Models'         type = 'PostProcessor'  >SP</Model>
    <Output   class = 'DataObjects'          type = 'PointSet'     >sfs_pnt_dt</Output>
  </PostProcess>

  <OutStreamStep name = 'pth4' pauseAtEnd = 'True'>
  	<Input  class = 'DataObjects'            type = 'PointSet'  >lmt_srf_dt</Input>
  	<Output class = 'OutStreams' type = 'Print'         >lmt_srf_dmp</Output>
    <Input  class = 'DataObjects'            type = 'PointSet'  >sfs_pnt_dt</Input>
  	<Output class = 'OutStreams' type = 'Print'         >sfs_pnt_dmp</Output>
  </OutStreamStep>
</Steps>

<!-- DATA OBJECTS -->
<DataObjects>
  <PointSet name = 'grd_vl_ql_smp_dpt_dt'>
    <Input>x1,x2,gammay</Input>
    <Output>OutputPlaceHolder</Output>
  </PointSet>

  <PointSet name = 'nt_phy_dpt_dt'>
    <Input>x1,x2,gammay</Input>
    <Output>g</Output>
  </PointSet>

  <PointSet name = 'bln_smp_dt'>
    <Input>x1,x2,gammay</Input>
    <Output>OutputPlaceHolder</Output>
  </PointSet>

  <PointSet name = 'lmt_srf_dt'>
    <Input>x1,x2,gammay</Input>
    <Output>g_zr</Output>
  </PointSet>

  <PointSet name = 'sfs_pnt_dt'>
    <Input>x1,x2,gammay</Input>
    <Output>p</Output>
  </PointSet>
</DataObjects>

<!-- DISTRIBUTIONS -->
<Distributions>
  <Normal name = 'x1_dst'>
    <upperBound>10</upperBound>
    <lowerBound>-10</lowerBound>
  	<mean>0.5</mean>
    <sigma>0.1</sigma>
  </Normal>

  <Normal name = 'x2_dst'>
    <upperBound>10</upperBound>
    <lowerBound>-10</lowerBound>
    <mean>-0.15</mean>
    <sigma>0.05</sigma>
  </Normal>

  <Normal name = 'gammay_dst'>
    <upperBound>20</upperBound>
    <lowerBound>-20</lowerBound>
    <mean>0</mean>
    <sigma>15</sigma>
  </Normal>
</Distributions>

<!-- SAMPLERS -->
<Samplers>
  <Grid name = 'grd_vl_ql_smp_dpt'>
    <variable name = 'x1' >
      <distribution>x1_dst</distribution>
      <grid type = 'value' construction = 'equal' steps = '10' upperBound = '10'>2</grid>
    </variable>
    <variable name='x2' >
      <distribution>x2_dst</distribution>
      <grid type = 'value' construction = 'equal' steps = '10' upperBound = '10'>2</grid>
    </variable>
    <variable name='gammay' >
      <distribution>gammay_dst</distribution>
      <grid type = 'value' construction = 'equal' steps = '10' lowerBound = '-20'>4</grid>
    </variable>
  </Grid>

  <Adaptive name = 'dpt_smp' verbosity='debug'>
    <ROM              class = 'Models'    type = 'ROM'           >accelerated_ROM</ROM>
    <Function         class = 'Functions' type = 'External'      >g_zr</Function>
    <TargetEvaluation class = 'DataObjects'     type = 'PointSet'  >nt_phy_dpt_dt</TargetEvaluation>
    <Convergence limit = '3000' forceIteration = 'False' weight = 'none' persistence = '5'>1e-2</Convergence>
      <variable name = 'x1'>
        <distribution>x1_dst</distribution>
      </variable>
      <variable name = 'x2'>
        <distribution>x2_dst</distribution>
      </variable>
      <variable name = 'gammay'>
        <distribution>gammay_dst</distribution>
      </variable>
  </Adaptive>
</Samplers>

<!-- MODELS -->
<Models>
  <ExternalModel name = 'xtr_mdl' subType = '' ModuleToLoad = 'SafestPointPP/safest_point_test_xtr_mdl'>
    <variables>x1,x2,gammay,g</variables>
  </ExternalModel>

  <ROM name = 'accelerated_ROM' subType = 'SciKitLearn'>
    <Features>x1,x2,gammay</Features>
    <Target>g_zr</Target>
    <SKLtype>svm|SVC</SKLtype>
    <kernel>rbf</kernel>
    <gamma>10</gamma>
    <tol>1e-5</tol>
    <C>50</C>
  </ROM>

  <PostProcessor name='SP' subType='SafestPoint'>
    <!-- List of Objects (external with respect to this PP) needed by this post-processor -->
    <Distribution     class = 'Distributions'  type = 'Normal'>x1_dst</Distribution>
    <Distribution     class = 'Distributions'  type = 'Normal'>x2_dst</Distribution>
    <Distribution     class = 'Distributions'  type = 'Normal'>gammay_dst</Distribution>
    <!- end of the list -->
    <controllable>
    	<variable name = 'x1'>
    		<distribution>x1_dst</distribution>
    		<grid type = 'value' steps = '20'>1</grid>
    	</variable>
    	<variable name = 'x2'>
    		<distribution>x2_dst</distribution>
    		<grid type = 'value' steps = '20'>1</grid>
    	</variable>
    </controllable>
    <non-controllable>
    	<variable name = 'gammay'>
    		<distribution>gammay_dst</distribution>
    		<grid type = 'value' steps = '20'>2</grid>
    	</variable>
    </non-controllable>
  </PostProcessor>
</Models>

<!-- FUNCTIONS -->
<Functions>
  <External name='g_zr' file='SafestPointPP/safest_point_test_g_zr.py'>
    <variable>g</variable>
  </External>
</Functions>

<!-- OUT-STREAMS -->
<OutStreams>
  <Print name = 'lmt_srf_dmp'>
  	<type>csv</type>
  	<source>lmt_srf_dt</source>
  </Print>

  <Print name = 'sfs_pnt_dmp'>
  	<type>csv</type>
  	<source>sfs_pnt_dt</source>
  </Print>
</OutStreams>

</Simulation>
\end{lstlisting}
The Python \textit{ExternalModel} is reported below:
\begin{lstlisting}[language=python]
def run(self,Input):
  self.g = self.x1+4*self.x2-self.gammay
\end{lstlisting}
The ``Goal Function'',the function that defines the transitions with respect the input space coordinates, is as follows:
\begin{lstlisting}[language=python]
def __residuumSign(self):
  if self.g<0 : return  1
  else        : return -1
\end{lstlisting}

%%%%% EXAMPLE 3
%\subsection{Example3}
%\label{subsec:ex1}
%example 3



    % ---------------------------------------------------------------------- %
    % References
    %
    \clearpage
    % If hyperref is included, then \phantomsection is already defined.
    % If not, we need to define it.
    \providecommand*{\phantomsection}{}
    \phantomsection
    \addcontentsline{toc}{section}{References}
    \bibliographystyle{ieeetr}
    \bibliography{raven_theory_manual}


    % ---------------------------------------------------------------------- %
    %

    % \printindex

    %\include{distribution}

\end{document}
