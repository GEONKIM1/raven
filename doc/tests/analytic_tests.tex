%
% This is an example LaTeX file which uses the SANDreport class file.
% It shows how a SAND report should be formatted, what sections and
% elements it should contain, and how to use the SANDreport class.
% It uses the LaTeX article class, but not the strict option.
% ItINLreport uses .eps logos and files to show how pdflatex can be used
%
% Get the latest version of the class file and more at
%    http://www.cs.sandia.gov/~rolf/SANDreport
%
% This file and the SANDreport.cls file are based on information
% contained in "Guide to Preparing {SAND} Reports", Sand98-0730, edited
% by Tamara K. Locke, and the newer "Guide to Preparing SAND Reports and
% Other Communication Products", SAND2002-2068P.
% Please send corrections and suggestions for improvements to
% Rolf Riesen, Org. 9223, MS 1110, rolf@cs.sandia.gov
%
\documentclass[pdf,12pt]{INLreport}
% pslatex is really old (1994).  It attempts to merge the times and mathptm packages.
% My opinion is that it produces a really bad looking math font.  So why are we using it?
% If you just want to change the text font, you should just \usepackage{times}.
% \usepackage{pslatex}
\usepackage{times}
\usepackage[FIGBOTCAP,normal,bf,tight]{subfigure}
\usepackage{amsmath}
\usepackage{amssymb}
\usepackage{pifont}
\usepackage{enumerate}
\usepackage{listings}
\usepackage{fullpage}
\usepackage{xcolor}          % Using xcolor for more robust color specification
\usepackage{ifthen}          % For simple checking in newcommand blocks
\usepackage{textcomp}
\usepackage{graphicx}
\usepackage{float}
\usepackage[toc,page]{appendix}
%\usepackage{authblk}         % For making the author list look prettier
%\renewcommand\Authsep{,~\,}

% Custom colors
\definecolor{deepblue}{rgb}{0,0,0.5}
\definecolor{deepred}{rgb}{0.6,0,0}
\definecolor{deepgreen}{rgb}{0,0.5,0}
\definecolor{forestgreen}{RGB}{34,139,34}
\definecolor{orangered}{RGB}{239,134,64}
\definecolor{darkblue}{rgb}{0.0,0.0,0.6}
\definecolor{gray}{rgb}{0.4,0.4,0.4}

\lstset {
  basicstyle=\ttfamily,
  frame=single
}

\setcounter{secnumdepth}{5}
\lstdefinestyle{XML} {
    language=XML,
    extendedchars=true,
    breaklines=true,
    breakatwhitespace=true,
%    emph={name,dim,interactive,overwrite},
    emphstyle=\color{red},
    basicstyle=\ttfamily,
%    columns=fullflexible,
    commentstyle=\color{gray}\upshape,
    morestring=[b]",
    morecomment=[s]{<?}{?>},
    morecomment=[s][\color{forestgreen}]{<!--}{-->},
    keywordstyle=\color{cyan},
    stringstyle=\ttfamily\color{black},
    tagstyle=\color{darkblue}\bf\ttfamily,
    morekeywords={name,type},
%    morekeywords={name,attribute,source,variables,version,type,release,x,z,y,xlabel,ylabel,how,text,param1,param2,color,label},
}
\lstset{language=python,upquote=true}

\usepackage{titlesec}
\newcommand{\sectionbreak}{\clearpage}
\setcounter{secnumdepth}{4}

%\titleformat{\paragraph}
%{\normalfont\normalsize\bfseries}{\theparagraph}{1em}{}
%\titlespacing*{\paragraph}
%{0pt}{3.25ex plus 1ex minus .2ex}{1.5ex plus .2ex}

%%%%%%%% Begin comands definition to input python code into document
\usepackage[utf8]{inputenc}

% Default fixed font does not support bold face
\DeclareFixedFont{\ttb}{T1}{txtt}{bx}{n}{9} % for bold
\DeclareFixedFont{\ttm}{T1}{txtt}{m}{n}{9}  % for normal

\usepackage{listings}

% Python style for highlighting
\newcommand\pythonstyle{\lstset{
language=Python,
basicstyle=\ttm,
otherkeywords={self, none, return},             % Add keywords here
keywordstyle=\ttb\color{deepblue},
emph={MyClass,__init__},          % Custom highlighting
emphstyle=\ttb\color{deepred},    % Custom highlighting style
stringstyle=\color{deepgreen},
frame=tb,                         % Any extra options here
showstringspaces=false            %
}}


% Python environment
\lstnewenvironment{python}[1][]
{
\pythonstyle
\lstset{#1}
}
{}

% Python for external files
\newcommand\pythonexternal[2][]{{
\pythonstyle
\lstinputlisting[#1]{#2}}}

\lstnewenvironment{xml}
{}
{}

% Python for inline
\newcommand\pythoninline[1]{{\pythonstyle\lstinline!#1!}}

% Named Colors for the comments below (Attempted to match git symbol colors)
\definecolor{RScolor}{HTML}{8EB361}  % Sonat (adjusted for clarity)
\definecolor{DPMcolor}{HTML}{E28B8D} % Dan
\definecolor{JCcolor}{HTML}{82A8D9}  % Josh (adjusted for clarity)
\definecolor{AAcolor}{HTML}{8D7F44}  % Andrea
\definecolor{CRcolor}{HTML}{AC39CE}  % Cristian
\definecolor{RKcolor}{HTML}{3ECC8D}  % Bob (adjusted for clarity)
\definecolor{DMcolor}{HTML}{276605}  % Diego (adjusted for clarity)
\definecolor{PTcolor}{HTML}{990000}  % Paul

\def\DRAFT{} % Uncomment this if you want to see the notes people have been adding
% Comment command for developers (Should only be used under active development)
\ifdefined\DRAFT
  \newcommand{\nameLabeler}[3]{\textcolor{#2}{[[#1: #3]]}}
\else
  \newcommand{\nameLabeler}[3]{}
\fi
\newcommand{\alfoa}[1] {\nameLabeler{Andrea}{AAcolor}{#1}}
\newcommand{\cristr}[1] {\nameLabeler{Cristian}{CRcolor}{#1}}
\newcommand{\mandd}[1] {\nameLabeler{Diego}{DMcolor}{#1}}
\newcommand{\maljdan}[1] {\nameLabeler{Dan}{DPMcolor}{#1}}
\newcommand{\cogljj}[1] {\nameLabeler{Josh}{JCcolor}{#1}}
\newcommand{\bobk}[1] {\nameLabeler{Bob}{RKcolor}{#1}}
\newcommand{\senrs}[1] {\nameLabeler{Sonat}{RScolor}{#1}}
\newcommand{\talbpaul}[1] {\nameLabeler{Paul}{PTcolor}{#1}}
% Commands for making the LaTeX a bit more uniform and cleaner
\newcommand{\TODO}[1]    {\textcolor{red}{\textit{(#1)}}}
\newcommand{\xmlAttrRequired}[1] {\textcolor{red}{\textbf{\texttt{#1}}}}
\newcommand{\xmlAttr}[1] {\textcolor{cyan}{\textbf{\texttt{#1}}}}
\newcommand{\xmlNodeRequired}[1] {\textcolor{deepblue}{\textbf{\texttt{<#1>}}}}
\newcommand{\xmlNode}[1] {\textcolor{darkblue}{\textbf{\texttt{<#1>}}}}
\newcommand{\xmlString}[1] {\textcolor{black}{\textbf{\texttt{'#1'}}}}
\newcommand{\xmlDesc}[1] {\textbf{\textit{#1}}} % Maybe a misnomer, but I am
                                                % using this to detail the data
                                                % type and necessity of an XML
                                                % node or attribute,
                                                % xmlDesc = XML description
\newcommand{\default}[1]{~\\*\textit{Default: #1}}
\newcommand{\nb} {\textcolor{deepgreen}{\textbf{~Note:}}~}

%

%%%%%%%% End comands definition to input python code into document

%\usepackage[dvips,light,first,bottomafter]{draftcopy}
%\draftcopyName{Sample, contains no OUO}{70}
%\draftcopyName{Draft}{300}

% The bm package provides \bm for bold math fonts.  Apparently
% \boldsymbol, which I used to always use, is now considered
% obsolete.  Also, \boldsymbol doesn't even seem to work with
% the fonts used in this particular document...
\usepackage{bm}

% Define tensors to be in bold math font.
\newcommand{\tensor}[1]{{\bm{#1}}}

% Override the formatting used by \vec.  Instead of a little arrow
% over the letter, this creates a bold character.
\renewcommand{\vec}{\bm}

% Define unit vector notation.  If you don't override the
% behavior of \vec, you probably want to use the second one.
\newcommand{\unit}[1]{\hat{\bm{#1}}}
% \newcommand{\unit}[1]{\hat{#1}}

% Use this to refer to a single component of a unit vector.
\newcommand{\scalarunit}[1]{\hat{#1}}

% set method for expressing expected value as E[f]
\newcommand{\expv}[1]{\ensuremath{\mathbb{E}[ #1]}}

% \toprule, \midrule, \bottomrule for tables
\usepackage{booktabs}

% \llbracket, \rrbracket
\usepackage{stmaryrd}

\usepackage{hyperref}
\hypersetup{
    colorlinks,
    citecolor=black,
    filecolor=black,
    linkcolor=black,
    urlcolor=black
}

% Compress lists of citations like [33,34,35,36,37] to [33-37]
\usepackage{cite}

% If you want to relax some of the SAND98-0730 requirements, use the "relax"
% option. It adds spaces and boldface in the table of contents, and does not
% force the page layout sizes.
% e.g. \documentclass[relax,12pt]{SANDreport}
%
% You can also use the "strict" option, which applies even more of the
% SAND98-0730 guidelines. It gets rid of section numbers which are often
% useful; e.g. \documentclass[strict]{SANDreport}

% The INLreport class uses \flushbottom formatting by default (since
% it's intended to be two-sided document).  \flushbottom causes
% additional space to be inserted both before and after paragraphs so
% that no matter how much text is actually available, it fills up the
% page from top to bottom.  My feeling is that \raggedbottom looks much
% better, primarily because most people will view the report
% electronically and not in a two-sided printed format where some argue
% \raggedbottom looks worse.  If we really want to have the original
% behavior, we can comment out this line...
\raggedbottom
\setcounter{secnumdepth}{5} % show 5 levels of subsection
\setcounter{tocdepth}{5} % include 5 levels of subsection in table of contents

% ---------------------------------------------------------------------------- %
%
% Set the title, author, and date
%
\title{RAVEN Analytic Test Documentation}

\author{
\\Andrea Alfonsi
\\Paul W. Talbot
\\Congjian Wang
\\Diego Mandelli
\\Joshua Cogliati
\\Mohammad G. Abdo
\\Cristian Rabiti
}

% There is a "Printed" date on the title page of a SAND report, so
% the generic \date should [WorkingDir:]generally be empty.
\date{}


% ---------------------------------------------------------------------------- %
% Set some things we need for SAND reports. These are mandatory
%
\SANDnum{INL/EXT-19-53843}
\SANDprintDate{\today}
\SANDauthor{Andrea Alfonsi, Paul W. Talbot, Diego Mandelli, Congjian Wang, Joshua Cogliati, Mohammad G. Abdo, Cristian Rabiti}
\SANDreleaseType{Revision 0}


% ---------------------------------------------------------------------------- %
% Include the markings required for your SAND report. The default is "Unlimited
% Release". You may have to edit the file included here, or create your own
% (see the examples provided).
%
% \include{MarkOUO} % Not needed for unlimted release reports

\def\component#1{\texttt{#1}}

% ---------------------------------------------------------------------------- %
\newcommand{\systemtau}{\tensor{\tau}_{\!\text{SUPG}}}

% ---------------------------------------------------------------------------- %
%
% Start the document
%

\begin{document}
    \maketitle

    \cleardoublepage		% TOC needs to start on an odd page
    \tableofcontents

    % ---------------------------------------------------------------------- %
    % This is where the body of the report begins; usually with an Introduction
    %
    \SANDmain		% Start the main part of the report

\input{intro.tex}
\input{projectile.tex}
\input{attenuate.tex}
\input{tensor_poly.tex}
\input{gamma_scgpc.tex}
\input{sobol_sens.tex}
\input{rims.tex}
\input{parabolas.tex}
\input{fourier.tex}
\section{Optimization Functions}

The functions in this section are models with analytic optimal points.
\setcounter{tocdepth}{4}
\subsection{Continuous}
\subsubsection{Unconstrained}
\paragraph{Beale's Function}
Beale's function has a two-dimensional input space and a single global/local minimum.
See \url{https://en.wikipedia.org/wiki/Test_functions_for_optimization}.

\begin{itemize}
  \item Objective Function: $f(x,y) = (1.5-x+xy)^2+(2.25-x+xy^2)^2+(2.625-x+xy^3)^2$
  \item Domain: $-4.5 \leq x,y \leq 4.5$
  \item Global Minimum: $f(3,0.5)=0$
\end{itemize}


\paragraph{Rosenbrock Function}
The Rosenbrock function can take a varying number of inputs.  For up to three inputs, a single global minimum
exists.  For four to seven inputs, there is one local minimum and one global maximum.
See \url{https://en.wikipedia.org/wiki/Rosenbrock_function}.

\begin{itemize}
  \item Objective Function: $f(\vec x) = \sum_{i=1}^{n-1}\left[100\left(x_{i+1}-x_i^2\right)^2+\left(x_i-1)^2\right) \right]$
  \item Domain: $ -\infty \leq x_i \leq \infty \hspace{10pt} \forall \hspace{10pt} 1\leq i \leq n$
  \item Global Minimum: $f(1,1,\cdots,1,1)=0$
  \item Local minimum ($n\geq4$): near $f(-1,1,\cdots,1)$
\end{itemize}


\paragraph{Goldstein-Price Function}
The Goldstein-Price function is a two-dimensional input function with a single global minimum.
See \url{https://en.wikipedia.org/wiki/Test_functions_for_optimization}.

\begin{itemize}
  \item Objective Function:
    \begin{align}
      f(x,y) =& \left[1+(x+y+1)^2\left(19-14x+3x^2-14y+6xy+3y^2\right)\right] \\ \nonumber
        & \cdot\left[30+(2x-3y)^2(18-32x+12x^2+48y-36xy+27y^2)\right]
    \end{align}
  \item Domain: $-2 \leq x,y \leq 2$
  \item Global Minimum: $f(0,-1)=3$
\end{itemize}

\paragraph{McCormick Function}
The McCormick function is a two-dimensional input function with a single global minimum.
See \url{https://en.wikipedia.org/wiki/Test_functions_for_optimization}.

\begin{itemize}
  \item Objective Function: $f(x,y) = \sin(x+y) + (x-y)^2 - 1.5x + 2.5y + 1$
  \item Domain: $-1.5 \leq x \leq 4$, $-3 \leq y \leq 4$
  \item Global Minimum: $f(-0.54719,-1.54719) = -1.9133$
\end{itemize}

\paragraph{2D Canyon}
The two-dimensional canyon offers a low region surrounded by higher walls.

\begin{itemize}
  \item Objective Function: $f(x,y) = xy\cos(x+y)$
  \item Domain: $0 \leq x,y \leq \pi$
  \item Global Minimum: $f(1.8218,1.8218)=-2.90946$
\end{itemize}

\paragraph{Stochastic Diagonal Valley}
The stochastic three-dimensional diagonal valley offers a low region surrounded by higher walls.

\begin{itemize}
  \item Objective Function: $f(x,y,stoch) = stoch * R + (x+0.5)^2 + (y-0.5)^2 + \sqrt{(x - \cfrac{x}{x-y})^2 + (y - \cfrac{y}{y-x})*^2} * 10, while R\sim ([-0.1,0.1])$
  \item Domain: $0 \leq x,y \leq 1$
  \item Global Minimum if $stoch =0$: $f(-0.5,0.5)=0$
\end{itemize}

\paragraph{Parabolic Valley}
The parabolic valley is dominated by distance from $x=y$ but secondarily motivated by $x+y$.

\begin{itemize}
  \item Objective Function: $f(x,y) = 100 (x - y)^2 + (x + y)^2$
  \item Domain: $-1 \leq x,y \leq 1$
  \item Global Minimum: $f(0,0)=0$
\end{itemize}

\paragraph{Paraboloid}
The elliptic paraboloid is a two-dimensional function with a single global minimum.

\begin{itemize}
  \item Objective Function: $f(x,y) = 10(x+0.5)^2 + 10(y-0.5)^2 $
  \item Domain: $-2 \leq x \leq 2$, $-2 \leq y \leq 2$
  \item Global Minimum: $f(-0.5,0.5) = 0$
\end{itemize}

\paragraph{Time-Parabola}
This model features the sum of a parabola in both $x$ and $y$; however, the parabola in $y$ moves in time and
has a reduced magnitude as time increases. As such, the minimum is always found at $x=0$ and at $(t-y)=0$, with
$y$ values at low $t$ being more impactful than at later $t$, and $x$ as impactful as $y(t=0)$.

\begin{itemize}
  \item Objective Function: $f(x,y,t) = x^2 + \sum_{t} (t-y)^2 \exp{-t}$
  \item Domain: $-10 \leq x,y,t \leq 10$
  \item Global Minimum: $f(0,y=t,t) = 0$
\end{itemize}

\paragraph{Eggholder}
This is a 2D function with a plethora of local minima and maxima (i.e., non-convex). It is challenging for most optimizers to find the global minimum for such a function. Metaheuristic methods are prefered for their ability to avoid getting stuck in local optima.

\begin{itemize}
	\item Objective Function: $f(x,y,) =-(y+47) \sin(\sqrt{|\frac{x}{2}+(y+47)|}) - x\sin(\sqrt{|x - (y+47)|})$
	\item Domain: $-512 \leq x,y \leq 512$
	\item Global Minimum: $f(x=512, y=404.2319) = -959.6407$
\end{itemize}

\subsubsection{Constrained}
\paragraph{Mishra's Bird Function}
The Mishra bird function offers a constrained problem with multiple peaks, local minima, and one steep global
minimum.
See \url{https://en.wikipedia.org/wiki/Test_functions_for_optimization}.

\begin{itemize}
  \item Objective Function: $f(x,y) = \sin(y)\exp[1-\cos(x)]^2 + \cos(x)\exp[1-\sin(y)]^2 + (x-y)^2$
  \item Constraint: $(x+5)^2 + (y+5)^2 < 25$
  \item Domain: $-10 \leq x \leq 0$, $-6.5 \leq y \leq 0$
  \item Global Minimum: $f(-3.1302468, -1.5821422) = -106.7645367$
\end{itemize}

\paragraph{ND Slant}
This trivial linear problem extends to arbitrary dimension and has optimal points at infinities,
which is usually outside the domain of exploration.
\begin{itemize}
  \item Objective Function: $f(\vec x) = 1 - \frac{1}{N}\sum_{x_i\in \vec x} x_i$
  \item Domain: $0 \leq x_i \leq 1 \forall x_i \in \vec x$
  \item Global Minimum: $f(\{1\}_N) = 0$
\end{itemize}

% \paragraph{Rosenbrock with cubic and linear constraints}
% Here the Rosenbrock is another two dimensional highly nonlinear function. The global maximum is on the intersection of the two constraints which renders this problem challenging for optimizers that uses random decisions for next points such as simulated annealing and genetic algorithms.
% \begin{itemize}
% 	\item Objective Function: $f(x,y) = (1-x^2) + 100(y-x^2)^2$
% 	\item Domain: $-1.5 \leq x \leq 1.5$ and $ -0.5 \leq y \leq 2.5$
% 	\item Constraints: $(x-1)^3 - y +1 \leq 0$ and $x+y-2 \leq 0$
% 	\item Global Maximum: $f(1.0,1.0) = 0$
% \end{itemize}

\paragraph{Rosenbrock with a disk constraint}
This is the same 2D function, but constrained to a disk. This constraint makes it a little easier than the previous function.
\begin{itemize}
	\item Objective Function: $f(x,y) = (1-x^2) + 100(y-x^2)^2$
	\item Domain: $-1.5 \leq x,y \leq 1.5$
	\item Constraint: $x^2 + y^2 \leq 2$
	\item Global Maximum: $f(1.0,1.0) = 0$
\end{itemize}

\paragraph{Townsend}
It is a highly nonlinear non convex function with multiple minima and maxima. The constraint is highly nonlinear and the y-domain is asymmetric.
\begin{itemize}
	\item Objective Function: $f(x,y) = -\left[\cos((x-0.1)y)\right]^2-x\sin(3x+y)$
	\item Domain: $-2.25 \leq x \leq 2.25$ and $-2.5 \leq y \leq 1.75$
	\item Constraint: $x^2 + y^2 \leq \left[2\cos(t)-\frac{1}{2}\cos(2t)-\frac{1}{4}\cos(3t) - \frac{1}{8}cos(4t) \right]^2+\left[2\sin(t)\right]^2 $;  \[ \text{where: } t=\arctan2(x,y) \]
	\item Global Minimum: $f(2.0052938,1.1944509) = -2.0239884$
\end{itemize}

\paragraph{Simionescu}
An exteremly simple two dimensional nonlinear function formed from 45-degrees hyperbolas. Subjected to highly nonlinear constraint.
\begin{itemize}
	\item Objective Function: $f(x,y) = 0.1xy$
	\item Domain: $-1.25 \leq x,y \leq 1.25$
	\item Constraint: $x^2 + y^2 \leq \left[r_{\Gamma}+r_S\cos\left(n \arctan(\frac{x}{y})\right)\right]^2 $; \[ \text{where: } r_{\Gamma}=1, r_S = 0.2, \text{and } n=8 \]
	\item Global Minimum: $f(\pm 0.84852813,\mp0.84852813) = -0.072$
\end{itemize}
\subsection{Discrete}
\subsubsection{Unconstrained}
\paragraph{Locally weighted sum without replacement}
This function computes the sum of the input vector components weighted by the location in the vector (i.e., component/dimension number). The input vector is sampled from a disrete uniform distribution of intergers between an upper bound $ub$, and a lower bound $lb$ without replacement (i.e., if an integer is selected for one variable (a component in the $n$-th dimensional input vector it cannot be selected for the remaining components)).
\begin{itemize}
	\item Objective Function: $f(\vec{x}) = \sum_{i=1}^{G}i \times x_i$; where $G$ is the number of variables (aka dimension of search/design space or number of Genes in Genetic algorithms) 
	\item Domain: $x_i \sim \mathcal{U}^{\text{Discrete int}}_{\text{w/o replacement}} (lb,ub)$
	\item Global Minimum: $f(\vec{x}_{opt}|x_i = lb + G - i) = \sum_{i=1}^{G} (lb + G -i) \times i$
	For instance, if $lb =1$, $ub=20$ , $G=6$, $\vec{x}_{opt} = [\vec{x}]_i | x_i = lb + G - i = 1+6-i = \begin{bmatrix}
	6 & 5 & \dots & 1 \end{bmatrix}$ and the minimum will be $f_{min} = \sum_{i=1}^{G} (1 + 6 -i) \times i = \frac{G(G+1)}{6}[3lb+G-1] = 56$
	\item Global Maximum: $f(\vec{x}_{opt}|x_i = ub - G + i) = \sum_{i=1}^{G} (ub - G +i) \times i$
	For instance, if $lb =1$, $ub=20$ , $G=6$, $\vec{x}_{opt} = [\vec{x}]_i | x_i = ub - G + i = 20-6+i = \begin{bmatrix}
	15 & 16 & \dots & 20 \end{bmatrix}$ and the maximum will be $f_{max} = \sum_{i=1}^{G} (20 - 6 +i) \times i = \frac{G(G+1)}{6}[3ub-G+1] = 385$
\end{itemize}

\paragraph{Locally weighted sum with replacement}
This function computes the sum of the input vector components weighted by the location in the vector (i.e., component/dimension number). The input vector is sampled from a disrete uniform distribution of intergers between an upper bound $ub$, and a lower bound $lb$ with replacement (i.e., if an integer is selected for one variable, it can be selected again for any or all other variables).
\begin{itemize}
	\item Objective Function: $f(\vec{x}) = \sum_{i=1}^{G}i \times x_i$; where $G$ is the number of variables (aka dimension of search/design space or number of Genes in Genetic algorithms) 
	\item Domain: $x_i \sim \mathcal{U}^{\text{Discrete int}}_{\text{w/ replacement}} (lb,ub)$
	\item Global Minimum: $f(\vec{x}_{opt}|x_i = lb) = \sum_{i=1}^{G} lb \times i$
	For instance, if $lb =1$, $ub=20$ , $G=6$, $\vec{x}_{opt} = [\vec{x}]_i | x_i = lb = 1 = \begin{bmatrix}
	1 & 1 & \dots & 1 \end{bmatrix}$ and the minimum will be $f_{min} = \sum_{i=1}^{G} i = \frac{G(G+1)}{2}[lb] = 21$
	\item Global Maximum: $f(\vec{x}_{opt}|x_i = ub) = \sum_{i=1}^{G} ub \times i$
	For instance, if $lb =1$, $ub=20$ , $G=6$, $\vec{x}_{opt} = [\vec{x}]_i | x_i = ub = 20 = \begin{bmatrix}
	20 & 20 & \dots & 20 \end{bmatrix}$ and the maximum will be $f_{max} = \sum_{i=1}^{G} 20 \times i = \frac{G(G+1)}{2}[ub] = 420$
\end{itemize}

\subsubsection{Constrained}
\paragraph{Locally weighted sum with replacement linearly constrained}
It uses the same function but adds a constraint.
\begin{itemize}
	\item Objective Function: $f(\vec{x}) = \sum_{i=1}^{G}i \times x_i$; where $G$ is the number of variables (aka dimension of search/design space or number of Genes in Genetic algorithms) 
	\item Domain: $x_i \sim \mathcal{U}^{\text{Discrete int}}_{\text{w/o replacement}} (lb,ub)$
	\item constraint $x_2 \leq 8.5 - 0.5 * x_1$
	\item Global Minimum if $lb = 1, ub = 6, G =3$: $f(\vec{x}_{opt} = \begin{bmatrix}
	1 & 1 & 1 \end{bmatrix}) = 6.0$
	\item Global Maximum if $lb = 1, ub = 6, G =3$: $f(\vec{x}_{opt} = \begin{bmatrix}
	6 & 5 & 6 \end{bmatrix} = 34.0$
\end{itemize}

\paragraph{Locally weighted sum with replacement sphere constrained}
It uses the same function but adds a constraint.
\begin{itemize}
	\item Objective Function: $f(\vec{x}) = \sum_{i=1}^{G}i \times x_i$; where $G$ is the number of variables (aka dimension of search/design space or number of Genes in Genetic algorithms) 
	\item Domain: $x_i \sim \mathcal{U}^{\text{Discrete int}}_{\text{w/o replacement}} (lb,ub)$
	\item constraint $x^{2}_{1} + x^{2}_{2} \leq 72$
	\item Global Minimum if $lb = 1, ub = 6, G =3$: $f(\vec{x}_{opt} = \begin{bmatrix}
1 & 1 & 1 \end{bmatrix})= 6.0$
\item Global Maximum if $lb = 1, ub = 6, G =3$: $f(\vec{x}_{opt} = \begin{bmatrix}
6 & 5 & 6 \end{bmatrix} = 34.0$
\end{itemize}

\input{processing.tex}
\section{MCSSolver}
This test evaluates the MCSSolver model for a specific case found in literature (see pages 108-110 
of N. J. McCormick, ``Reliability and Risk Analysis'', Academic Press inc. (1981)).
Provided this set of minimal cuts sets (MCSs) for the top event TopEvent:
\begin{equation}
  TopEvent = A + BD + BE + CD + CE
\end{equation}
the probability of TopEvent can be calculated as 
\begin{equation}
  P(TopEvent) = [A + BD + BE + CD + CE]
              - [ABD + ABE + ACD + ACE + BCD + BCE + BDE + CDE + 2*BCDE]
              + [ABCD + ABCE + ABDE + ACDE + 4*BCDE + 2*ABCDE]
              - [BCDE + 4*ABCDE]
              + [ABCDE]
\end{equation}

Give these probability values for the basic events:
\begin{itemize}
  \item $P(A)=0.01$
  \item $P(B)=P(C)=P(D)=P(E)=0.1$
\end{itemize}
then the analytical values for $P(TopEvent)$ for each order are as follows:

\begin{table}
\begin{tabular}{cc}
Order &  $P(TopEvent)$\\
\hline
1 & 0.05     \\
2 & 0.0454   \\
3 & 0.045842 \\
4 & 0.045738 \\
5 & 0.045739 \\
\hline  
\end{tabular}
\end{table}


\begin{appendices}

  \section{Document Version Information}
  \input{../version.tex}
\end{appendices}


    % ---------------------------------------------------------------------- %
    % References
    %
    \clearpage
    % If hyperref is included, then \phantomsection is already defined.
    % If not, we need to define it.
    \providecommand*{\phantomsection}{}
    \phantomsection
    \addcontentsline{toc}{section}{References}
    \bibliographystyle{ieeetr}
    \bibliography{analytic_tests}

\end{document}
