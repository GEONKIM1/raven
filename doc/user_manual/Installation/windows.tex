\subsection{Microsoft Windows}
\label{sec:install windows}

The process of establishing the required environment for Windows is notably more involved than the other two
systems; however, it is straightforward.  First, RAVEN has the following prerequisites on Windows:

\begin{itemize}
    \item A system running a 64-bit version of Microsoft Windows. Installation and operation
        has been verified on Windows 7, 10, and Windows Server 2012 R2 Standard.
    \item At least 9 Gigabytes of available disk space:
    \begin{itemize}
        \item 0.5 GB for GIT SCM, including supporting tools and git source code control
        \item 1.5 GB for Python language and supporting packages
        \item 1 GB for RAVEN framework
        \item 5.0 GB for the Visual Studio compiler needed to build RAVEN
    \end{itemize}
\end{itemize}

\subsubsection{A Visual Guide}
Note: An illustrated version of this procedure may be found on the \wiki.

\subsubsection{GIT SCM for Windows}
RAVEN currently works on Windows using basic tools freely available online. 
The first software to be downloaded and installed is \textbf{Git SCM} available at \url{https://gitforwindows.org/}.
\begin{enumerate}
    \item Obtain the latest Git SCM for Windows installer from  \url{https://gitforwindows.org/} and install it. 
    Install Git Bash and have 
    the installer add Git Bash to your Windows \textit{PATH} environment variables. 
    The \textit{PATH} can be updated either automatically (allowing the Git SCM installer to update it for you) or manually 
    (Systems Properties - Environment Variables - Edit Environment Variables).
\end{enumerate}

\subsubsection{Install Python Language and Package Support}
\begin{enumerate} 
	\item Download the latest 64-bit installer for Windows Python 3 from
		\url{https://conda.io/miniconda.html} and install it.  \item The installer
		will ask whether Python should be installed for only the logged in user or
		for all users.  Either option will work for RAVEN.
	\item 	have  the installer add \textit{conda} to your Windows \textit{PATH} environment variables.   
	The \textit{PATH} can be updated either automatically (allowing the  \textit{conda} installer to update it for you) 
	or manually (Systems Properties - Environment Variables - Edit Environment Variables).
	\item Check the installation of Python and coda locating and testing the Python installation.   
	Open a Windows command prompt and enter the
		command "{\it where python}", which attempts to locate a the Python language interpreter
		in the current system path.  This looks like:

    \begin{lstlisting}[language=bash, basicstyle=\small]
    C:\Users\USERID> where python
    C:\Users\USERID\AppData\Local\Continuum\Miniconda3\python.exe
    \end{lstlisting}

\end{enumerate}


\subsubsection{Compiler Installation and Configuration}
\begin{enumerate}
	\item Download and install Visual Studio.  A C++ language compiler that supports C++11 features
		is needed to perform this step. Microsoft's Visual Studio Community Edition is free and
		available from \url{https://www.visualstudio.com/downloads/}.

		The current version (as of this writing) is 2017. The 2015 and 2017 versions have been
		successfully used to build RAVEN. Professional and Enterprise versions of these will
		also work. If one of these is already present on your system, it is not necessary to
		obtain another one. Note that because C++11 language features are required, the
		"Microsoft Visual C++ Compiler for Python 2.7 or 3.x" often used for building Python
		add-ons will {\bf not} work.

		After downloading and running the Visual Studio installer, it will ask what features
		to install. For building RAVEN, "Desktop development with C++" is needed at a minimum.
		Installation of other Visual Studio features should be fine.
\end{enumerate}

Once the compiler installation and configuration is complete, you are prepared to install the RAVEN libraries
(see section \ref{sec:install conda}).


