%%%%%%%%%%%%%%%%%%%%%%%%%%%%%%%%%%%%%%%%%%%%%%%%%
%%%%%%%%%%%%%%%%%%%%%%%%%%%%%%%%%%%%%%%%%%%%%%%%%
%%%%%%%%%%%%% PHISICS/RELAP5 INTERFACE %%%%%%%%%%
%%%%%%%%%%%%%%%%%%%%%%%%%%%%%%%%%%%%%%%%%%%%%%%%%
\subsection{PHISICS/RELAP5 Interface}
\label{subsec:PhisicsRelap5Interface}
%
\subsubsection{General Information}
%
This section covers the input specification for running PHISICS/RELAP5 through RAVEN.
This interface can be used to perturb the PHISICS and/or RELAP5 input files. This interface is strongly built around the PHISICS and RELAP5
standalone interfaces, hence this sections covers the additional cautions to take care of to run the coupled PHISICS/RELAP5 code. The
user will find additional information regarding PHISICS in section \ref{subsec:PhisicsInterface} or RELAP5 in section \ref{subsec:RELAP5Interface}.
%
\subsubsection{Files}
\label{subsec:PhisicsRelap5Files}
\xmlNode{Files} includes two attributes \xmlAttr{name} and \xmlAttr{type} entries, identically as other interfaces.
It also includes two optional attributes \xmlAttr{perturbable} and \xmlAttr{subDirectory}.

The \xmlAttr{name} attribute is a user-defined internal name for the file contained in the node.
Default: None (required entry).

For the files parsed by the PHISICS interface or the PHISICS interface's parsers, some of the \xmlAttr{type} attributes are hardcoded.
The accepted PHISICS \xmlAttr{type} attributes are given in Table \ref{TypeTable} and are not repeated here. Additional information can be found in section \ref{subsubsec:PhisicsInterfaceFiles}.
All the necessary RELAP5 input files need to have a \xmlAttr{type} attribute starting with the string 'relap'.
The necessary RELAP5 files for use in the coupled PHISICS/RELAP mode within RAVEN are given in Table~\ref{PhisicsRelap5TypeTable}. The files associated
with a  \xmlAttr{type} that does not start with the string 'relap' will be treated by the PHISICS interface.
The  \xmlAttr{type} attributes are case-insensitive.
\begin{table}[]
  \centering
  \caption{Example of RELAP5 type attributes in coupled PHISICS/RELAP5 mode}\label{PhisicsRelap5TypeTable}
    \begin{tabular}{l|l|l}
\textit{Type Attribute} & \textit{Corresponding RELAP5 input} & \textit{Perturbable} \\
\hline
relapFluid              & fluid properties                  & No  \\
relapInp                & Relap input File                  & yes \\
relapLicense            & license for the RELAP5 executable & No  \\
    \end{tabular}
\end{table}

Example of acceptable RELAP5 entries within PHISICS/RELAP5:
\begin{lstlisting}[style=XML]
<File>
  <Input name="H2O"       type="relaph2o"     perturbable="False">tph2o</Input>
  <Input name="H2"        type="relaph2"      perturbable="False">tph2</Input>
  <Input name="inputDeck" type="relapInput"   perturbable="True" >inp.i</Input>
  <Input name="lic"       type="relapLicence" perturbable="False">license.bin</Input>
</File>
\end{lstlisting}
%
\subsubsection{Models}
The user has to provide the paths to executables for the sampled variables within the \xmlNode{Models} block.

The \xmlNode{Code} block will contain attributes \xmlAttr{name} and \xmlAttr{subType}.
The \xmlAttr{name} identifies the particular \xmlNode{Code} model within RAVEN, and \xmlAttr{subType}
specifies which code interface the model will use. \xmlAttr{subType}=\xmlString{PhisicsRelap5} is the class name
currently used for PHISICS/RELAP5 coupled calculations.

The \xmlNode{executable} block contains the absolute or relative path (with respect to the current working directory) to PHISICS/RELAP5
that RAVEN will use to run the code. The additional nodes in the \xmlNode{Models} applicable to PHISICS standalone and RELAP5 standalone
are valid in coupled mode and can be consulted in section~\ref{subsubsection:PhisicsModel} and section~\ref{subsubsection:Relap5Models} respectively.
Exception: the use of MrTau in standalone mode (i.e., \xmlNode{mrtauStandAlone} set to \xmlString{True}) is not allowed in PHISICS/RELAP5 coupled
calculations.

An example of the \xmlNode{Models} block is given below:

\begin{lstlisting}[style=XML]
  <Models>
    <Code name="PHISICS_RELAP5" subType="PhisicsRelap5">
       <executable>./path/to/instant/executable</executable>
    </Code>
  </Models>
\end{lstlisting}
%%%%%%%%%%%%%%%%%%%%%%%%%%%%%%%%%%%%%%%%%%%%%%%%%%
\subsubsection{Distributions}
The \xmlNode{Distributions} block defines all distributions used to
sample variables in the current RAVEN run.

For all the possible distributions and their possible inputs please
refer to the Distributions chapter (see~\ref{sec:distributions}).
%
%%%%%%%%%%%%%%%%%%%%%%%%%%%%%%%%%%%%%%%%%%%%%%%%%%
\subsubsection{Samplers}\label{SamplerPhisicsRelap5}
The \xmlNode{Samplers} block defines the variables to be sampled.
After defining a sampling scheme, the variables to be sampled and
their distributions are identified in the \xmlNode{variable} blocks.
The \xmlAttr{name} must be formatted according to the PHISICS library which the variable belongs to.
Information relative to PHISICS distributions are in section~\ref{subsubsection:SamplersPhisics},
as well as specifications on PHISICS variable names.
An example of a \xmlNode{Samplers} block is given below:
\begin{lstlisting}[style=XML]
 <Samplers>
    <MonteCarlo name="MC_samp">
      <samplerInit>
        <limit>10</limit>
      </samplerInit>
      <variable name="DENSITY|FUEL1|U238">
        <distribution>DENSITY|FUEL1|U238_distrib</distribution>
      </variable>
      <variable name="20100154:2">
        <distribution>heat_capacity_154</distribution>
      </variable>
    </MonteCarlo>
  </Samplers>
\end{lstlisting}
In this example, the variable \xmlString{DENSITY$\vert$FUEL1$\vert$U238} is relative to PHISICS and the variable \xmlString{20100154:2} is relative to RELAP5.
%%%%%%%%%%%%%%%%%%%%%%%%%%%%%%%%%%%%%%%%%%%%%%%%%
\subsubsection{Steps}
The tasks performed by RAVEN need to be defined in the \xmlNode{Steps} block. Each task needs to be defined with a \xmlAttr{name}. This \xmlAttr{name} is later on used in the
the \xmlNode{Sequence} block. In the example, the step is called \xmlString{testDummyStep}.
%
\begin{lstlisting}[style=XML]
<Steps>
  <MultiRun name='testDummyStep' verbosity='debug'>
    <Input class='Files' type='decay'>decay.dat</Input>
    <Input class='Files' type='inp'>inp.xml</Input>
    <Input class='Files' type='XS'>xs.xml</Input>
    <Input class="Files" type="relapFluid">tpfhe</Input>
    <Input class="Files" type="relapExec">relap5Exec.x</Input>
    <Model class="Models" type="Code">PHISICS_RELAP5</Model>
    <Sampler class="Samplers" type="MonteCarlo">MC_samp</Sampler>
    <Output class="Databases" type="HDF5">DataB_REL5_1</Output>
    <Output class="DataObjects" type="PointSet">collset</Output>
    </MultiRun>
  </MultiRun>
</Steps>
\end{lstlisting}
%
%%%%%%%%%%%%%%%%%%%%%%%%%%%%%%%%%%%%%%%%%%%%%%%%%
\subsubsection{Additional Input}
\label{subsubsection:PhisicsRelap5AdditionalInput}
The PHISICS additional inputs are described in section~\ref{subsubsection:PhisicsAdditionalInput}. The RELAP5 additional inputs are described in
section ~\ref{subsubsection:Relap5Models}.
%
%%%%%%%%%%%%%%%%%%%%%%%%%%%%%%%%%%%%%%%%%%%%%%%%%
\subsubsection{Output Files Conversion}
\label{subsubsection:PhisicsRelap5OutputFileConversion}

The PHISICS output available for RAVEN post-processing are described in section~\ref{subsubsection:PhisicsOutFileConv}.
The output printed from PHISICS and RELAP5 are synchronized in the RAVEN csv output. The synchronization scheme is explained in this section.

At t = 0 seconds, the RELAP5 initialized output are printed in the csv output, while the output variables from PHISICS are taken equal to 0.
Then, the RAVEN/PHISICS/RELAP5 post-processor finds the time step number at the end of each PHISICS
burn step based on the \xmlNode{tab\_time\_step} values, and prints the RELAP5 minor edits according to the \xmlNode{TH\_between\_BURN} values.

Let's consider the following example:
\begin{enumerate}
\item [$-$]\xmlNode{tab\_time\_step} \xmlString{5 3 2} \xmlNode{/tab\_time\_step} (in the PHISICS depletion file);
\item [$-$]\xmlNode{TH\_between\_BURN} \xmlString{1.0 2.0} \xmlNode{TH\_between\_BURN}() in the PHISICS input file);
\item [$-$]\xmlNode{tabulation\_boundaries} \xmlString{5.0 35.0 45.0}\xmlNode{tabulation\_boundaries} (upper burn step boundaries in the PHISICS depletion file).
\item [$-$]in the RELAP5 input, a 3.0 seconds steady state is considered, with minor edits every 0.5 seconds.
\end{enumerate}
The first PHISICS/RELAP5 output line printed will be at t = 0 seconds. The PHISICS outputs are set to 0.0, the RELAP5 values are obtained at the end of the initialization.
The second line printed will be at the PHISICS time step \xmlString{5} (end of the first burn step, corresponding to t = 5.0 seconds),
and prints the RELAP5 minor edits as long as the time from the minor edits is lower than the first \xmlNode{TH\_between\_BURN} value \xmlString{1.0}.
The RELAP5 minor edits are printed along with the PHISICS burn step \xmlString{5} as long as time in the minor edits is smaller or equal to \xmlString{1.0}.
When the RELAP5 time in the minor edits time is greater than \xmlString{1.0}, the end of the second PHISICS burn step is targetted, from the
\xmlNode{tab\_time\_step}: \xmlString{3}. This corresponds to a PHISICS time equal to 35.0 seconds. The RELAP5 minor edits are printed along with the PHISICS values at t = 35.0 s,
as long as the minor data time is smaller than the \xmlNode{TH\_between\_BURN} equal to \xmlString{2.0}.
Finally, the last PHISICS time step at 45.0 seconds is printed along with the RELAP5 minor edits.
Overall, the output variables printed will be:

mrtauTime, n2n$\vert$gr1$\vert$reg4, httemp\_3001010\_1, time

0.000, 0.000, 1.526, 0.000

5.000, 1.111, 1.859, 0.500

5.000, 1.111, 2.369, 1.000

35.00, 7.800, 3.666, 1.500

35.00, 7.800, 4.789, 2.000

45.00, 9.330, 4.225, 3.000
