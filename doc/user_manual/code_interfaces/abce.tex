\subsection{ABCE Interface}
\label{subsec:AbceInterface}

\subsubsection{General Information}
The ABCE Interface is used to run agent-based 
capacity expansion (CE) modeling for electricity market systems. 
\url{https://github.com/abce-dev/abce}

This allows inputs to be perturbed and data to be read out.

\subsubsection{Sampler}

For perturbing inputs, the sampled variable needs to be placed inside
of \$RAVEN-( )\$ like \verb'$RAVEN-(var)$'. 

\begin{lstlisting}[style=XML]
  <Samplers>
    <Grid name="grid">
      <variable name="var">
        <distribution>dist</distribution>
        <grid construction="equal" steps="1" type="CDF">0.0 1.0</grid>
      </variable>
    </Grid>
  </Samplers>
\end{lstlisting}


\subsubsection{Files}

The \xmlNode{Files} XML node has to contain all the files required to run
the ABCE model. 
For RAVEN coupled with ABCE, the \xmlNode{Files} XML node has to contain
the following files:

\begin{itemize}
  \item \textbf{settings.yml}: contains all run-specific settings for each simulation. Data specified here supersedes data specified anywhere else.
  \item \textbf{inputs/}:
  \begin{itemize}
    \item \textbf{agent\_specifications.yml}: definitions for the agents: financial parameters, starting portfolios by unit type, and mandatory retirement dates for owned units
    \item \textbf{C2N\_project\_definitions.yml}: contains project activity cost and schedule information for coal-to-nuclear projects
    \item \textbf{demand\_data.csv}: normalized peak demand levels per simulated year (used to scale the \texttt{peak\_demand} parameter)
    \item \textbf{unit\_specs.yml}: construction and operations cost and parameter data for all possible unit types in the model
    \item \textbf{inputs/ts\_data/}:
    \begin{itemize}
      \item \textbf{timeseries\_<quantity>\_hourly.csv}: hourly timeseries data for each of the following quantities in the system:
      \begin{itemize}
        \item \textbf{load}: normalized to \texttt{peak\_demand}
        \item \textbf{wind} and \textbf{solar}: wind and solar availability, normalized to the start-of-year installed capacity of each technology, respectively
        \item \textbf{reg}, \textbf{spin}, and \textbf{nspin}: ancillary service procurement requirements, in absolute terms (not scaled)
      \end{itemize}
    \end{itemize}
  \end{itemize}
\end{itemize}



The inputs in the \xmlNode{Files} section are shown below:

\begin{lstlisting}[style=XML]
    <Files>
        <Input name="settings.yml" type="">settings.yml</Input>
        <Input name="demand_data_file" type="" subDirectory="inputs">demand_data.csv</Input>
        <Input name="agent_specifications_file" type="" subDirectory="inputs">single_agent_testing.yml</Input>
        <Input name="unit_specs_data_file" type="" subDirectory="inputs">unit_specs.yml</Input>
        <Input name="C2N_project_definitions.yml" type="" subDirectory="inputs">C2N_project_definitions.yml</Input>
        <Input name="timeseries_nspin_hourly.csv" type="" subDirectory="inputs/ts_data">timeseries_nspin_hourly.csv</Input>
        <Input name="timeseries_spin_hourly.csv" type="" subDirectory="inputs/ts_data">timeseries_spin_hourly.csv</Input>
        <Input name="timeseries_reg_hourly.csv" type="" subDirectory="inputs/ts_data">timeseries_reg_hourly.csv</Input>
        <Input name="timeseries_load_hourly.csv" type="" subDirectory="inputs/ts_data">timeseries_load_hourly.csv</Input>
        <Input name="timeseries_wind_hourly.csv" type="" subDirectory="inputs/ts_data">timeseries_wind_hourly.csv</Input>
        <Input name="timeseries_pv_hourly.csv" type="" subDirectory="inputs/ts_data">timeseries_pv_hourly.csv</Input>
    </Files>
\end{lstlisting}

\subsubsection{Models}

The \xmlNode{Code} model can be used with
\xmlAttr{subType="Abce"} to run the ABCE Code Interface. The \xmlNode{executable} needs to be ABCE's executable \textbf{run.py}.

\begin{lstlisting}[style=XML]
  <Models>
    <Code name="abce" subType="Abce">
      <executable>abce/run.py</executable>
      <clargs arg="python" type="prepend" />
      <clargs arg="--settings_file" extension=".yml" type="input" delimiter="=" />
      <clargs arg="--inputs_path=inputs --verbosity=3" type="text" />
    </Code>
  </Models>
\end{lstlisting}

\subsubsection{Output Files Conversion}

The code interface reads in the \verb'outputs\ABCE_run\outputs.xlsx'. 
It will output the assets sheets as the csv file. The \texttt{asset\_id} 
variable that can be used as the \xmlNode{pivotParameter} and is a
string with the year. 

Exactly which variables will appear will vary depending on the
ABCE input files, but typical ones include \texttt{asset\_id},
\texttt{unit\_type}, \texttt{start\_pd}, \texttt{completion\_pd},
\texttt{cancellation\_pd}, \texttt{retirement\_pd},
\texttt{total\_capex}, \texttt{cap\_pmt}, and
\texttt{C2N\_reserved}.


\begin{lstlisting}[style=XML]
    <HistorySet name="grid">
      <Input>var</Input>
      <Output> agent_id, unit_type, start_pd, completion_pd, cancellation_pd, 
      retirement_pd,total_capex, cap_pmt, C2N_reserved </Output>
      <options>
        <pivotParameter>asset_id</pivotParameter>
      </options>
    </HistorySet>
\end{lstlisting}


\subsubsection{Installation of Libraries}

Installing ABCE so that RAVEN can run it requires that RAVEN and
ABCE have a superset of the libraries that they use so that both
can run.  One way to set this up is to install RAVEN, and then 
inside of the RAVEN conda environment install ABCE dependencies.
This is shown in the following listing:

\begin{lstlisting}
  #first clone raven, into a directory
  git clone git@github.com:idaholab/raven.git
  git clone git@github.com:abce-dev/abce.git
  #Switch to raven directory
  cd raven
  #install raven libraries
  ./scripts/establish_conda_env.sh --install
  #switch to using raven libraries
  source ./scripts/establish_conda_env.sh --load
  #Switch to ABCE and install
  cd ../abce/
  bash ./install.sh
  conda install -c conda-forge mesa openpyxl pytest PyYAML julia=1.8
  cd ../abce/env/
  julia
  # in julia hit the ']' key to enter package mode
  activate .
  status
  instantiate
  # exit julia with the backspace key
  exit()
\end{lstlisting}


