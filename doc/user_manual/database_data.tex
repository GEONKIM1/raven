\section{DataObjects}
\label{sec:DataObjects}

As seen in the previous chapters, different entities in the RAVEN
code interact with each other in order to create, ideally, an infinite number of
different calculation flows.
%
These interactions are made possible through a data handling system that each
entity understands.
%
This system is called the ``DataObjects'' framework.

The \xmlNode{DataObjects} tag is a container of data objects of various types that can
be constructed during the execution of a particular calculation flow.
%
These data objects can be used as input or output for a particular
\textbf{Model} (see Roles' meaning in section \ref{sec:models}), etc.
%
Currently, RAVEN supports 3 different data types, each with a particular
conceptual meaning.
%
These data types are instantiated as sub-nodes in the \xmlNode{DataObjects} block of
an input file:
\begin{itemize}
  \item \xmlNode{PointSet} is a collection of individual objects, each
  describing the state of the system at a certain point (e.g. in time).
  %
  It can be considered a mapping between multiple sets of parameters in the
  input space and the resulting sets of outcomes in the output space at a
  particular point (e.g. in time).
  %
  \item \xmlNode{HistorySet} is a collection of individual objects each
  describing the temporal evolution of the state of the system within a certain
  input domain.
  %
  It can be considered a mapping between multiple sets of parameters in the
  input space and the resulting sets of temporal evolution in the output
  space.
  %
   \item \xmlNode{DataSet} is a generalization of the previously described DataObject,
   aimed to contain a mixture of data (scalars, arrays, etc.). The variables here stored
   can be independent (i.e. scalars) or dependent (arrays) on certain dimensions (e.g. time, coordinates, etc.).
  %
  It can be considered a mapping between multiple sets of parameters in the
  input space (both dependent and/or independent) and the resulting sets of evolution in the output
  space (both dependent and/or independent).
  %
  \nb \textcolor{red} {\textbf{The  \xmlNode{DataSet} is currently usable in the  \xmlNode{EnsembleModel} only (see \ref{subsec:models_EnsembleModel} )}}
\end{itemize}

In summary, the DataObjects accept the following data in their input/output spaces:
\begin{table}[h]
\centering
\caption{DataObjects' accepted data formats.}
\label{DataObjectDataFormatTable}
\begin{tabular}{|c|c|c|}
\hline
\textbf{DataObject}                        & \textbf{Input Space} & \textbf{Output Space} \\ \hline
{\color[HTML]{FE0000} \textit{PointSet}}   & scalars              & scalars               \\ \hline
{\color[HTML]{FE0000} \textit{HistorySet}} & scalars              & vectors               \\ \hline
{\color[HTML]{FE0000} \textit{DataSet}}    & any                  & any                   \\ \hline
\end{tabular}
\end{table}


As noted above, each data object represents a mapping between a set of
parameters and the resulting outcomes.
%
The data objects are defined within the main XML block called \xmlNode{DataObjects}:
\begin{lstlisting}[style=XML]
<Simulation>
   ...
  <DataObjects>
    <PointSet name='***'>...</PointSet>
    <HistorySet name='***'>...</HistorySet>
    <DataSet name='***'>...</DataSet>
  </DataObjects>
   ...
</Simulation>
\end{lstlisting}

Independently on the type of data, the respective XML node has the following
available attributes:
\vspace{-5mm}
\begin{itemize}
  \itemsep0em
  \item \xmlAttr{name}, \xmlDesc{required string attribute}, is a user-defined
  identifier for this data object.
    %
  \nb As with other objects, this name can be used to refer to this specific
  entity from other input blocks in the XML.
  %
%  % Regarding the time attribute, we need to take a better decision... Now it is very confusing.
%  \item \xmlAttr{time}, \xmlDesc{optional float or string attribute}, time
%    attribute.
%    %
%    Here, the user can specify either the time (value) at which the outcomes
%    need to be taken (History-like object, it represents the time from which the
%    outcomes' evolution need to be tracked) or a string  that can be either
%    ``end'', at the end of the history, or ``all'', consider.
%    %
%    \default{random seed};
%  \item \xmlAttr{inputTs}, \xmlDesc{optional integer attribute}, used to
%  specify at which ``time step'' the input space needs to be retrieved.
%  %
%  \nb If the user wants to take conditions from the end of the simulation, s/he
%  can directly input ``-1.''
%  %
%  \default{0}
%  \item \xmlAttr{operator}, \xmlDesc{optional string attribute}, is aimed at
%  performing simple operations on the data to be stored.
%  %
%  %
%  The 3 options currently available are:
%  \begin{itemize}
%    \item \xmlString{max}
%    \item \xmlString{min}
%    \item \xmlString{average}
%  \end{itemize}
%  %
%  \default{None}

  \item \xmlAttr{hierarchical}, \xmlDesc{optional boolean attribute},
  This flag is going to ``control'' the printing/plotting of the DataObject in
  case a hierarchical structure is determined (e.g.
  data coming from Dynamic Event Tree-like approaches):
  \begin{itemize}
    \item if \textbf{True} all the branches of the tree are going to be printed/plotted independently
               (i.e. all the branches are going to be exposed independently)
    \item if \textbf{False} all the branches are going to be walked back and reconstructed in order to create independent histories
  \end{itemize}
  %
  \default{False}
\end{itemize}
\vspace{-5mm}
In each XML node (e.g. \xmlNode{PointSet}, \xmlNode{HistorySet} or  \xmlNode{DataSet}), the user
specifies the following sub-nodes:
\begin{itemize}
  \item \xmlNode{Input}, \xmlDesc{comma separated string, required field}, lists
  the input parameters to which this data is connected.
  %
  \item \xmlNode{Output}, \xmlDesc{comma separated string, required field}, lists
  the output parameters to which this data is connected.
  %
  \item \xmlNode{Index}, \xmlDesc{comma separated string, required for \xmlNode{DataSet}}, lists
  the dependent variables that depend on this index (specified through the attribute  \xmlAttr{var}).
  This XML node requires the following attribute:
   \begin{itemize}
     \item \xmlAttr{var}, \xmlDesc{required string attribute}, the dimension name of this index (e.g. time)
   \end{itemize}
   This XML node allows the following attribute:
   \begin{itemize}
     \item \xmlAttr{autogenerate}, \xmlDesc{optional boolean attribute}, if this index should be generated automatically if it does not exist. This will generate integer numbers from 0 to the maximum needed. This can be used for reading CSV files that do not have an otherwise use-able index.
   \end{itemize}
 \item \xmlNode{options}, \xmlDesc{optional node}, contains additional option nodes
   for data objects.  This node contains the following subnodes:
   \begin{itemize}
     \item \xmlNode{pivotParameter}, \xmlDesc{optional, string}, specifies the \textit{pivotParameter} for a
       \xmlNode{HistorySet}. The pivot parameter is the shared index of the output variables in the data
       object.  \default time
    \item \xmlNode{inputRow}, \xmlDesc{integer, optional field}, used to
         specify  the row (in a CSV file or HDF5 table) from which the input space
        needs to be retrieved (e.g. the time-step);
    %
    \item \xmlNode{outputRow}, \xmlDesc{integer, optional field}, used to
         specify  the row (in the CSV file or HDF5 table) from which the output space
        needs to be retrieved (e.g. the time-step). If this node is inputted, the nodes
         \xmlNode{operator} and  \xmlNode{outputPivotValue} can not be inputted (mutually exclusive).
       \\\nb This XML node is available for DataObjects of type \xmlNode{PointSet} only;
    %
    \item \xmlNode{operator}, \xmlDesc{string, optional field}, is aimed to perform
         simple operations on the data to be stored.
         The 3 options currently available are:
         \begin{itemize}
            \item \xmlString{max}
            \item \xmlString{min}
            \item \xmlString{average}
         \end{itemize}
         If this node is inputted, the nodes
         \xmlNode{outputRow} and  \xmlNode{outputPivotValue} can not be inputted (mutually exclusive).
         \\\nb This XML node is available for DataObjects of type \xmlNode{PointSet} only;
   \end{itemize}

  %
\end{itemize}

The  \xmlNode{PointSet} and  \xmlNode{HistorySet} objects are a specialization of the  \xmlNode{DataSet}. In
the \xmlNode{PointSet}, the input and output space are all exclusively scalar values.  These values might be
extracted from a vector of values for each entry using the \xmlNode{options} node, but the end result is a
single scalar per input or output variable.

For the \xmlNode{HistorySet}, all inputs must be scalar, and all outputs must share an index (the
\textit{pivotParameter}.  There cannot be scalars in any of the outputs. The pivotParameter can be changed
through the corresponding node in the \xmlNode{options} node.

RAVEN automatically creates a \texttt{prefix} in the output space
that is used to generate the directory name for sampling and other
purposes. A user can also add \texttt{prefix} explicitly to the
variables in the object if it is useful to keep this information.

  %
  %%%%%%%%%%%%%%%%%%%%%%%%%%%%%%%%%%%%%%%%%%%%%%%%%%%%%%%%%%%%%%%%%%%%%%%%%%%%%%
  %%%% This feature is being disabled until the DataObjects handle data in a
  %%%% more encapsulated fashion. When the data can handle this all internally
  %%%% then we can re-add this feature. As of now, determining the rows
  %%%% associated to the outputPivotValue or inputPivotValue requires knowing
  %%%% information outside of the "value" passed into
  %%%% DataObject.updateOutputValue or DataObject.updateInputValue, thus the
  %%%% caller has to do this computation, but currently the caller occurs in ~50
  %%%% different places according to my grep of "updateOutputValue"
  %%%% -- DPM 8/29/2017
  % \item \xmlNode{pivotParameter}, \xmlDesc{string, optional field} the name of
  %   the parameter whose values need to be used as reference for the values
  %   specified in the XML nodes \xmlNode{inputPivotValue},
  %   \xmlNode{outputPivotValue}, or \xmlNode{inputPivotValue} (if inputted).
  %   This field can be used, for example, if the driven code output file uses  a
  %   different name for the variable ``time'' or to specify a different reference
  %   parameter (e.g. PRESSURE). Default value is \xmlString{time}.
  %   \\\nb The variable specified here should be monotonic; the code does not
  %   check for eventual oscillation and is going to take the first occurance for
  %   the values specified in the XML nodes \xmlNode{inputPivotValue},
  %   \xmlNode{outputPivotValue}, and  \xmlNode{inputPivotValue};
  % %
  % \item \xmlNode{inputPivotValue}, \xmlDesc{float, optional field}, the value of the \xmlNode{pivotParameter} at which the input space needs to be retrieved
  %   If this node is inputted, the node  \xmlNode{inputRow} can not be inputted (mutually exclusive).
  %   %
  % \item \xmlNode{outputPivotValue}. This node can be either a float or a list of floats, depending on the type of DataObjects:
  %  \begin{itemize}
  %     \item if \xmlNode{HistorySet},\xmlNode{outputPivotValue}, \xmlDesc{list of floats, optional field},  list of values of the
  %                         \xmlNode{pivotParameter} at which the output space needs to be retrieved;
  %     \item if \xmlNode{PointSet},\xmlNode{outputPivotValue}, \xmlDesc{float, optional field},  the value of the \xmlNode{pivotParameter}
  %        at which the output space needs to be retrieved. If this node is inputted, the node  \xmlNode{outputRow} can not be inputted (mutually exclusive);
  %  \end{itemize}
  %%%%%%%%%%%%%%%%%%%%%%%%%%%%%%%%%%%%%%%%%%%%%%%%%%%%%%%%%%%%%%%%%%%%%%%%%%%%%%
  %
Note that if the optional nodes in the block \xmlNode{options} are not inputted, the following default are applied:
\begin{itemize}
   \item the Input space (scalars) is retrieved from the first row in the CSVs files or HDF5 tables (if the parameters specified are not
      among the variables sampled by RAVEN); In case of the  \xmlNode{DataSet}, if any of the input space variables depend on an \xmlNode{Index}, they
      are going to be linked to the \xmlNode{Index} variable
   \item  the output space defaults are as follows:
   \begin{itemize}
       \item if \xmlNode{PointSet}, the output space is retrieved from the last row in the CSVs files or HDF5 tables;
       \item if \xmlNode{HistorySet}, the output space is represented by all the rows found in  the CSVs or HDF5 tables.
       \item if \xmlNode{DataSet}, the output space of the variables that do not depends on any index is retrieved from the last row in the CSVs files or HDF5 tables;
       on the contrary, the output space of the variables that depends on indexes is represented by all the rows found in  the CSVs or HDF5 tables (if they match
       with the indexes' dimension)
    \end{itemize}
\end{itemize}


\begin{lstlisting}[style=XML,morekeywords={operator,hierarchical,name,var}]
  <DataObjects>
    <PointSet name='outTPS1'>
      <options>
       <inputRow>1</inputRow>
       <outputRow>-1</outputRow>
      </options>
      <Input>pipe_Area,pipe_Dh,Dummy1</Input>
      <Output>pipe_Hw,pipe_Tw,time</Output>
    </PointSet>
    <HistorySet name='stories1'>
        <options>
            <pivotParameter>TIME</pivotParameter>
            <inputRow>1</inputRow>
            <outputRow>-1</outputRow>
        </options>
      <Input>pipe_Area,pipe_Dh</Input>
      <Output>pipe_Hw,pipe_Tw,time</Output>
    </HistorySet>
    <DataSet name='aDataSet'>
      <Input>pipe_Area,pipe_Dh</Input>
      <Output>pipe_Hw,pipe_Tw</Output>
      <Index var="time">pipe_Hw,pipe_Tw</Index>
    </DataSet>
  </DataObjects>
\end{lstlisting}

\section{Databases}
\label{sec:Databases}
The RAVEN framework provides the capability to store and retrieve data to/from
an external database.
%
Currently RAVEN has support for \textbf{netCDF4} and \textbf{HDF5} formats.
NetCDF shares native format with RAVEN's DataObjects, but HDF5 is also included for convenience.
%
This database, depending on the data format it is receiving, will organize
itself in a ``parallel'' or ``hierarchical'' fashion.
%
The user can create as many database objects as needed.
%
The Database objects are defined within the main XML block called
\xmlNode{Databases}:
\begin{lstlisting}[style=XML]
<Simulation>
  ...
  <Databases>
    ...
    <NetCDF name="aDatabaseName1" readMode="overwrite"/>
    <HDF5 name="aDatabaseName2" readMode="overwrite"/>
    ...
  </Databases>
  ...
</Simulation>
\end{lstlisting}
The specifications for these two formats are listed below.

\subsection{NetCDF}
The specifications of each Database of type NetCDF needs to be defined within the
XML block \xmlNode{NetCDF}, that recognizes the following attributes:
\vspace{-5mm}
\begin{itemize}
  \itemsep0em
  \item \xmlAttr{name}, \xmlDesc{required string attribute}, a user-defined
  identifier of this object.
  %
  \nb As with other objects, this is name can be used to reference this specific
  entity from other input blocks in the XML.
  \item \xmlAttr{readMode}, \xmlDesc{required string attribute}, defines whether an existing database should
    be read when loaded (\xmlString{read}) or overwritten (\xmlString{overwrite}).
    \nb if in \xmlString{read} mode and the database is not found, RAVEN will read in
    the data as empty and raise a warning, NOT an error.
  %
  \item \xmlAttr{directory}, \xmlDesc{optional string attribute}, this attribute
  can be used to specify a particular directory path where the database will be
  created or read from.  If an absolute path is given, RAVEN will respect it; otherwise,
  the path will be assumed to be relative to the \xmlNode{WorkingDir} from the \xmlNode{RunInfo} block.
  RAVEN recognizes path expansion tools such as tildes (\emph{user dir}), single dots (\emph{current dir}),
  and double dots (\emph{parent dir}).
  %
  \default{workingDir/DatabaseStorage}.  The \xmlNode{workingDir} is
   the one defined within the \xmlNode{RunInfo} XML block (see Section~\ref{sec:RunInfo}).
  \item \xmlAttr{filename}, \xmlDesc{optional string attribute}, specifies the
  filename of the database that will be created in the \xmlAttr{directory}.
  %
  \nb When this attribute is not specified, the newer database filename will be
  named \texttt{name.nc}, where \textit{name} corresponds to the \xmlAttr{name}
  attribute of this object.
  %
  \default{None}
\end{itemize}

Example:
\begin{lstlisting}[style=XML,morekeywords={directory,filename}]
<Databases>
  <NetCDF name="name1" directory=''path_to_a_dir'' readMode='overwrite'/>
  <HDF5 name="name2" filename=''Name2.nc'' readMode='read'/>
</Databases>
\end{lstlisting}



\subsection{HDF5}
The specifications of each Database of type HDF5 needs to be defined within the
XML block \xmlNode{HDF5}, that recognizes the following attributes:
\vspace{-5mm}
\begin{itemize}
  \itemsep0em
  \item \xmlAttr{name}, \xmlDesc{required string attribute}, a user-defined
  identifier of this object.
  %
  \nb As with other objects, this is name can be used to reference this specific
  entity from other input blocks in the XML.
  \item \xmlAttr{readMode}, \xmlDesc{required string attribute}, defines whether an existing database should
    be read when loaded (\xmlString{read}) or overwritten (\xmlString{overwrite}).
    \nb if in \xmlString{read} mode and the database is not found, RAVEN will read in
    the data as empty and raise a warning, NOT an error.
  %
  \item \xmlAttr{directory}, \xmlDesc{optional string attribute}, this attribute
  can be used to specify a particular directory path where the database will be
  created or read from.  If an absolute path is given, RAVEN will respect it; otherwise,
  the path will be assumed to be relative to the \xmlNode{WorkingDir} from the \xmlNode{RunInfo} block.
  RAVEN recognizes path expansion tools such as tildes (\emph{user dir}), single dots (\emph{current dir}),
  and double dots (\emph{parent dir}).
  %
  \default{workingDir/DatabaseStorage}.  The \xmlNode{workingDir} is
   the one defined within the \xmlNode{RunInfo} XML block (see Section~\ref{sec:RunInfo}).
  \item \xmlAttr{filename}, \xmlDesc{optional string attribute}, specifies the
  filename of the HDF5 that will be created in the \xmlAttr{directory}.
  %
  \nb When this attribute is not specified, the newer database filename will be
  named \texttt{name}.h5, where \textit{name} corresponds to the \xmlAttr{name}
  attribute of this object.
  %
  \default{None}
  \item \xmlAttr{compression}, \xmlDesc{optional string attribute}, compression
  algorithm to be used.
  %
  Available are:
  \begin{itemize}
    \item \xmlString{gzip}, best where portability is required.
    %
    Good compression, moderate speed.
    %
    \item \xmlString{lzf}, Low to moderate compression, very fast.
    %
  \end{itemize}
  \default{None}
\end{itemize}

In addition, the \xmlNode{HDF5} recognizes the following subnodes:
\begin{itemize}
  \itemsep0em
  \item \xmlNode{variables}, \xmlDesc{optional, comma-separated string}, allows only a pre-specified set of variables to be
    included in the HDF5 when it is written to.  If this node is not included, by default the HDF5 will
    include ALL of the input/output variables as a result of the step it is part of.  If included, only the
    comma-separated variable names will be included if found.

    \nb RAVEN will not error if one of the requested variables is not found; instead, it will silently pass.
    It is recommended that a small trial run is performed, loading the HDF5 back into a data object, to check
    that the correct variables are saved to the HDF5 before performing large-scale calculations.
\end{itemize}


Example:
\begin{lstlisting}[style=XML,morekeywords={directory,filename}]
<Databases>
  <HDF5 name="aDatabaseName1" directory=''path_to_a_dir'' compression=''lzf'' readMode='overwrite'/>
  <HDF5 name="aDatabaseName2" filename=''aDatabaseName2.h5'' readMode='read'/>
</Databases>
\end{lstlisting}
