\subsubsection{dataObjectLabelFilter}
\label{dataObjectLabelFilter}

This PostProcessor allows to filter the portion of a dataObject, either PointSet or HistorySet, with a given clustering label.
A clustering algorithm associates a unique cluster label to each element of the dataObject (PointSet or HistorySet).
This cluster label is a natural number ranging from $0$ (or $1$ depending on the algorithm) to $N$ where $N$ is the number of obtained clusters.
Recall that some clustering algorithms (e.g., K-Means) receive $N$ as input while others (e.g., Mean-Shift) determine $N$ after clustering has been performed.
Thus, this Post-Processor is naturally employed after a data-mining clustering techniques has been performed on a dataObject so that each clusters
can be analyzed separately.

\ppType{dataObjectLabelFilter}{dataObjectLabelFilter}

In the \xmlNode{PostProcessor} input block, the following XML sub-nodes are required,
independently of the \xmlAttr{subType} specified:

\begin{itemize}
   \item \xmlNode{label}, \xmlDesc{string, required field}, name of the clustering label
   \item \xmlNode{clusterIDs}, \xmlDesc{integers, required field}, ID of the selected clusters. Note that more than one ID can be provided as input
\end{itemize}
