\subsubsection{Metric}
\label{MetricPP}
The \textbf{Metric} PostProcessor is specifically used to calculate the distance values among points from PointSets and histories from HistorySets,
while the \textbf{Metrics} block (See Chapter \ref{sec:Metrics}) allows the user to specify the similarity/dissimilarity metrics to be used in this
post-processor. Both \textbf{PointSet} and \textbf{HistorySet} can be accepted by this post-processor.
If the name of given variable is unique, it can be used directly, otherwise the variable can be specified
with $DataObjectName|InputOrOutput|VariableName$ like other places in RAVEN.
Some of the Metrics also accept distributions to calculate the distance against.
These are specified by using the name of the distribution.
%
\ppType{Metric}{Metric}
%
\begin{itemize}
  \item \xmlNode{Features}, \xmlDesc{comma separated string, required field}, specifies the names of the features.
    This xml-node accepts the following attribute:
    \begin{itemize}
      \item \xmlAttr{type}, \xmlDesc{required string attribute}, the type of provided features. Currently only
        accept `variable'.
    \end{itemize}
  \item \xmlNode{Targets}, \xmlDesc{comma separated string, required field}, contains a comma separated list of
    the targets. \nb Each target is paired with a feature listed in xml node \xmlNode{Features}. In this case, the
    number of targets should be equal to the number of features.
    This xml-node accepts the following attribute:
    \begin{itemize}
      \item \xmlAttr{type}, \xmlDesc{required string attribute}, the type of provided features. Currently only
        accept `variable'.
    \end{itemize}
  \item \xmlNode{multiOutput}, \xmlDesc{optional string attribute}, only used when \textbf{HistorySet} is used as
    input. Defines aggregating of time-dependent metrics' calculations. Available options include:
    \textbf{mean, max, min, raw\_values} over the time. For example, when `mean' is used, the metrics' calculations
    will be averaged over the time. When `raw\_values' is used, the full set of  metrics' calculations will be dumped.
    \default{raw\_values}
  \item \xmlNode{weight}, \xmlDesc{comma separated floats, optional field}, when `mean' is provided for \xmlNode{multiOutput},
    the user can specify the weights that can be used for the average calculation of all outputs.
  \item \xmlNode{pivotParameter}, \xmlDesc{optional string attribute}, only used when \textbf{HistorySet}
    is used as input. The pivotParameter for given metrics' calculations.
    \default{time}
  \item \xmlNode{Metric}, \xmlDesc{string, required field}, specifies the \textbf{Metric} name that is defined via
    \textbf{Metrics} entity. In this xml-node, the following xml attributes need to be specified:
    \begin{itemize}
      \item \xmlAttr{class}, \xmlDesc{required string attribute}, the class of this metric (e.g. Metrics)
      \item \xmlAttr{type}, \xmlDesc{required string attribute}, the sub-type of this Metric (e.g. SKL, Minkowski)
    \end{itemize}
\end{itemize}

\textbf{Example:}

\begin{lstlisting}[style=XML]
<Simulation>
 ...
  <Models>
    ...
    <PostProcessor name="pp1" subType="Metric">
      <Features type="variable">ans</Features>
      <Targets type="variable">ans2</Targets>
      <Metric class="Metrics" type="SKL">euclidean</Metric>
      <Metric class="Metrics" type="SKL">cosine</Metric>
      <Metric class="Metrics" type="SKL">manhattan</Metric>
      <Metric class="Metrics" type="ScipyMetric">braycurtis</Metric>
      <Metric class="Metrics" type="ScipyMetric">canberra</Metric>
      <Metric class="Metrics" type="ScipyMetric">correlation</Metric>
      <Metric class="Metrics" type="ScipyMetric">minkowski</Metric>
    </PostProcessor>
    ...
  </Models>
 ...
</Simulation>
\end{lstlisting}

In order to access the results from this post-processor, RAVEN will define the variables as ``MetricName'' +
``\_'' + ``TargetVariableName'' + ``\_'' + ``FeatureVariableName'' to store the calculation results, and these
variables are also accessible by the users through RAVEN entities \textbf{DataObjects} and \textbf{OutStreams}.
\nb We will replace ``|'' in ``TargetVariableName'' and ``FeatureVariableName'' with ``\_''.
In previous example, variables such as \textit{euclidean\_ans2\_ans, cosine\_ans2\_ans, poly\_ans2\_ans} are accessible
by the users.
