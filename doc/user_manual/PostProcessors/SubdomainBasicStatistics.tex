\subsubsection{SubdomainBasicStatistics}
\label{SubdomainBasicStatistics}
The \textbf{SubdomainBasicStatistics} post-processor is aimed to allow the \textbf{BasicStatistics}
post-processor to be used on subdomains of the sampling space. This post-processor fully leverages
the \textbf{BasicStatistics} post-processor and, consequentially, computes many of the most important statistical quantities on 
a mesh/grid (subdomains). 
Similarly to the \textbf{BasicStatistics}, the post-processor can accept as input both \textit{\textbf{PointSet}} 
and \textit{\textbf{HistorySet}}.

\ppType{SubdomainBasicStatistics post-processor}{SubdomainBasicStatistics}

\begin{itemize}
  \item \xmlNode{"subdomain"}, \xmlDesc{XML node, required parameter}, definition of the subdomain grid. This node
  can specify the following XML node:
  \begin{itemize}
    \item \xmlNode{variable}, \xmlDesc{XML node, required parameter} can specify the \xmlAttr{name} attribute, which is \xmlDesc{required string attribute} and specifies the user-defined name of this variable.
    \variableChildrenIntro
    \begin{itemize}
     \item \gridDescriptionNoCDF
    \end{itemize}
  \end{itemize}
\end{itemize}

Based on the definition above, the post-processor will compute \textbf{BasicStatistics} on each ``cell'' defined
by the discretization grid. The results will be exported as function of the variables used
in the definition of the grid (see below for a detailed example).

In addition to the subdomain grid defined above, the same nodes/settings available for \textbf{BasicStatistics}
are available:

\basicStatisticsBody

\textbf{Example (Static Subdomain Statistics):}
\begin{lstlisting}[style=XML,morekeywords={name,subType,debug}]
<Simulation>
  ...
  <Models>
    ...
    <PostProcessor name='aUserDefinedName' subType='SubdomainBasicStatistics' verbosity='debug'>
      <subdomain>
        <variable name="a">
          <grid construction="equal" steps="2" type="value">10 100</grid>
        </variable>
        <variable name="b">
          <grid construction="equal" steps="1" type="value">10 50</grid>
        </variable>
      </subdomain>
      <expectedValue prefix='mean'>x01,x02</expectedValue>
      <sensitivity prefix='sen'>
        <targets>x01,x02</targets>
        <features>a,b</features>
      </sensitivity>
    </PostProcessor>
    ...
  </Models>
  ...
  <PointSet name='statsOut'>
    <Input>
      a,b
    </Input>
    <Output>
      mean_x01,mean_x02,
      sen_x01_a, sen_x01_b,
      sen_x02_a, sen_x02_b
    </Output>
  </PointSet>
</Simulation>
\end{lstlisting}

In this case, the RAVEN variables ``mean\_x01, mean\_x02, sen\_x01\_a, sen\_x02\_a, sen\_x01\_b, sen\_x02\_b''
will be computed for each cell defined above (2 cells in this case).
The results will be associated to the mid-point of each cell. For the example above:
\begin{itemize}
 \item $(a=32.5,b=30.0)$
 \item $(a=77.5,b=30.0)$
\end{itemize}
The results will be then accessible for the RAVEN entities \textbf{DataObjects} and \textbf{OutStreams}.

\textbf{Example (Dynamic Subdomain Statistics):}

\begin{lstlisting}[style=XML,morekeywords={name,subType,debug}]
<Simulation>
  ...
  <Models>
    ...
    <PostProcessor name='aUserDefinedNameForDynamicPP' subType='BasicStatistics' verbosity='debug'>
      <expectedValue prefix='mean'>x01,x02</expectedValue>
      <sensitivity prefix='sen'>
        <targets>x01,x02</targets>
        <features>a,b</features>
      </sensitivity>
      <pivotParameter>time</pivotParameter>
    </PostProcessor>
    ...
  </Models>
  ...
  <HistorySet name='basicStatHistorySet'>
    <Input>
      a,b
    </Input>
    <Output>
      mean_x01,mean_x02,
      sen_x01_a, sen_x01_b,
      sen_x02_a, sen_x02_b
    </Output>
    <options>
      <pivotParameter>time</pivotParameter>
    </options>
  </HistorySet>
</Simulation>
\end{lstlisting}

