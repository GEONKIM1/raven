\subsubsection{LimitSurface}
\label{LimitSurface}
The \textbf{LimitSurface} PostProcessor is aimed to identify the transition
zones that determine a change in the status of the system (Limit Surface).

\ppType{LimitSurface}{LimitSurface}

\begin{itemize}
  \item \xmlNode{parameters}, \xmlDesc{comma separated string, required field},
  lists the parameters that define the uncertain domain and from which the LS
  needs to be computed.
  \item \xmlNode{tolerance}, \xmlDesc{float, optional field}, sets the absolute
  value (in CDF) of the convergence tolerance.
 %
  This value defines the coarseness of the evaluation grid.
 %
 \default{1.0e-4}
  \item \xmlNode{side}, \xmlDesc{string, optional field}, in this node the user can specify
  which side of the limit surface needs to be computed. Three options are available:
  \\ \textit{negative},  Limit Surface corresponding to the goal function value of ``-1'';
  \\ \textit{positive}, Limit Surface corresponding to the goal function value of ``1'';
  \\ \textit{both}, either positive and negative Limit Surface is going to be computed.
  %
  %
\default{negative}
  % Assembler Objects
  \item \textbf{Assembler Objects} These objects are either required or optional
  depending on the functionality of the Adaptive Sampler.
  %
  The objects must be listed with a rigorous syntax that, except for the xml
  node tag, is common among all the objects.
  %
  Each of these nodes must contain 2 attributes that are used to map those
  within the simulation framework:
   \begin{itemize}
    \item \xmlAttr{class}, \xmlDesc{required string attribute}, is the main
    ``class'' of the listed object.
    %
    For example, it can be ``Models,'' ``Functions,'' etc.
    \item \xmlAttr{type}, \xmlDesc{required string attribute}, is the object
    identifier or sub-type.
    %
    For example, it can be ``ROM,'' ``External,'' etc.
    %
  \end{itemize}
  The \textbf{LimitSurface} post-processor requires or optionally accepts the
  following objects' types:
   \begin{itemize}
    \item \xmlNode{ROM}, \xmlDesc{string, optional field}, body of this xml
    node must contain the name of a ROM defined in the \xmlNode{Models} block
    (see section \ref{subsec:models_ROM}).
    \item \xmlNode{Function}, \xmlDesc{string, required field}, the body of
    this xml block needs to contain the name of an External Function defined
    within the \xmlNode{Functions} main block (see section \ref{sec:functions}).
    %
    This object represents the boolean function that defines the transition
    boundaries.
    %
    This function must implement a method called
    \textit{\_\_residuumSign(self)}, that returns either -1 or 1, depending on
    the system conditions (see section \ref{sec:functions}).
    %
    \end{itemize}
\end{itemize}

\textbf{Example:}
\begin{lstlisting}[style=XML,morekeywords={name,subType,debug,class,type}]
<Simulation>
 ...
 <Models>
  ...
    <PostProcessor name="computeLimitSurface" subType='LimitSurface' verbosity='debug'>
      <parameters>x0,y0</parameters>
      <ROM class='Models' type='ROM'>Acc</ROM>
      <!-- Here, you can add a ROM defined in Models block.
           If it is not Present, a nearest neighbor algorithm
           will be used.
       -->
      <Function class='Functions' type='External'>
        goalFunctionForLimitSurface
      </Function>
    </PostProcessor>
    ...
  </Models>
  ...
</Simulation>
\end{lstlisting}
