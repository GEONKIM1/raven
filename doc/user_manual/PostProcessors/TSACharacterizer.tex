\subsubsection{TSACharacterizer}
\label{TSACharacterizer}
The \xmlNode{TSACharacterizer} PostProcessor is a tool to characterize sets of histories using
the Time Series Analysis (TSA) module. It takes each history realization from the input data object
and returns characterizations. Note the characterizations are entirely stored in the \texttt{metadata}
portion of the output data object; no data is store in the input or output.

The \xmlNode{SampleSelector} PostProcessor can only act on \xmlNode{HistorySet} \xmlNode{DataObjects},
and generates a \xmlNode{PointSet} \xmlNode{DataObject} in return with as many realizations as the input
history set.

\ppType{TSACharacterizer}{TSACharacterizer}
%
\begin{itemize}
  \item \xmlNode{pivotParameter}, \xmlDesc{string, required field}, specifies the name of the time-like
    monotonic variable used for the signals in the input history set.
\end{itemize}

\tsaList

\textbf{TSACharacterizer Example:}

\begin{lstlisting}[style=XML]
<Simulation>
 ...
  <Models>
    ...
    <PostProcessor name="chz" subType="TSACharacterizer">
      <pivotParameter>pivot</pivotParameter>
      <fourier target='signal_f, signal_fa'>
        <periods>2, 5, 10</periods>
      </fourier>
      <arma target="signal_a, signal_fa" seed='42'>
        <SignalLag>2</SignalLag>
        <NoiseLag>3</NoiseLag>
      </arma>
    </PostProcessor>
    ...
  </Models>
 ...
</Simulation>
\end{lstlisting}
