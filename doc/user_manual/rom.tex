\newcommand{\zNormalizationPerformed}[1]
{
  \textcolor{red}{\\It is important to NOTE that RAVEN uses a Z-score normalization of the training data before
  constructing the \textit{#1} ROM:
\begin{equation}
  \mathit{\mathbf{X'}} = \frac{(\mathit{\mathbf{X}}-\mu )}{\sigma }
\end{equation}
 }
}

\newcommand{\zNormalizationNotPerformed}[1]
{
  \textcolor{red}{
  \\It is important to NOTE that RAVEN does not pre-normalize the training data before
  constructing the \textit{#1} ROM.}
}

\newcommand{\romClusterOption}[0]
{
  \item \xmlNode{Segment}, \xmlDesc{node, optional}, provides an alternative way to build the ROM. When
    this mode is enabled, the subspace of the ROM (e.g. ``time'') will be divided into segments as
    requested, then a distinct ROM will be trained on each of the segments. This is especially helpful if
    during the subspace the ROM representation of the signal changes significantly. For example, if the signal
    is different during summer and winter, then a signal can be divided and a distinct ROM trained on the
    segments. By default, no segmentation occurs.

    To futher enable clustering of the segments, the \xmlNode{Segment} has the following attributes:
    \begin{itemize}
      \item \xmlAttr{grouping}, \xmlDesc{string, optional field} enables the use of ROM subspace clustering in
        addition to segmenting if set to \xmlString{cluster}. If set to \xmlString{segment}, then performs
        segmentation without clustering. If clustering, then an additional node needs to be included in the
        \xmlNode{Segment} node, as described below.
        \default{segment}
    \end{itemize}

    This node takes the following subnodes:
    \begin{itemize}
      \item \xmlNode{subspace}, \xmlDesc{string, required field} designates the subspace to divide. This
        should be the pivot parameter (often ``time'') for the ROM. This node also requires an attribute
        to determine how the subspace is divided, as well as other attributes, described below:
        \begin{itemize}
          \item \xmlAttr{pivotLength}, \xmlDesc{float, optional field}, provides the value in the subspace
            that each segment should attempt to represent, independently of how the data is stored. For
            example, if the subspace has hourly resolution, is measured in seconds, and the desired
            segmentation is daily, the \xmlAttr{pivotLength} would be 86400.
            Either this option or \xmlAttr{divisions} must be provided.
          \item \xmlAttr{divisions}, \xmlDesc{integer, optional field}, as an alternative to
            \xmlAttr{pivotLength}, this attribute can be used to specify how many data points to include in
            each subdivision, rather than use the pivot values. The algorithm will attempt to split the data
            points as equally as possible.
            Either this option or \xmlAttr{pivotLength} must be provided.
          \item \xmlAttr{shift}, \xmlDesc{string, optional field}, governs the way in which the subspace is
            treated in each segment. By default, the subspace retains its actual values for each segment; for
            example, if each segment is 4 hours long, the first segment starts at time 0, the second at 4
            hours, the third at 8 hours, and so forth. Options to change this behavior are \xmlString{zero}
            and \xmlString{first}. In the case of \xmlString{zero}, each segment restarts the pivot with the
            subspace value as 0, shifting all other values similarly. In the example above, the first segment
            would start at 0, the second at 0, and the third at 0, with each ending at 4 hours. Note that the
            pivot values are restored when the ROM is evaluated. Using \xmlString{first}, each segment
            subspace restarts at the value of the first segment. This is useful in the event subspace 0 is not
            a desirable value.
        \end{itemize}
      \item \xmlNode{Classifier}, \xmlDesc{string, optional field} associates a \xmlNode{PostProcessor}
        defined in the \xmlNode{Models} block to this segmentation. If clustering is enabled (see
        \xmlAttr{grouping} above), then this associated Classifier will be used to cluster the segmented ROM
        subspaces. The attributes \xmlAttr{class}=\xmlString{Models} and
        \xmlAttr{type}=\xmlString{PostProcessor} must be set, and the text of this node is the \xmlAttr{name}
        of the requested Classifier. Note this Classifier must be a valid Classifier; not all PostProcessors
        are suitable. For example, see the DataMining PostProcessor subtype Clustering.
      \item \xmlNode{clusterFeatures}, \xmlDesc{string, optional field}, if clustering then delineates
        the fundamental ROM features that should be considered while clustering. The available features are
        ROM-dependent, and an exception is raised if an unrecognized request is given. See individual ROMs
        for options. \default All ROM-specific options.
      \item \xmlNode{evalMode}, \xmlDesc{string, optional field}, one of \xmlString{truncated},
        \xmlString{full}, or \xmlString{clustered}, determines how the evaluations are
        represented, as follows:
        \begin{itemize}
          \item \xmlString{full}, reproduce the full signal using representative cluster segments,
          \item \xmlString{truncated}, reproduce a history containing exactly segment from each
            cluster placed back-to-back, with the \xmlNode{pivotParameter} spanning the clustered
            dimension. Note this will almost surely not be the same length as the original signal;
            information about indexing can be found in the ROM's XML metadata.
          \item \xmlString{clustered}, reproduce a N-dimensional object with the variable
            \texttt{\_ROM\_cluster} as one of the indexes for the ROM's sampled variables. Note that
            in order to use the option, the receiving \xmlNode{DataObject} should be of type
            \xmlNode{DataSet} with one of the indices being \texttt{\_ROM\_cluster}.
        \end{itemize}
     \item \xmlNode{evaluationClusterChoice}, \xmlDesc{string, optional field}, one of \xmlString{first} or
        \xmlString{random}, determines, if \xmlAttr{grouping}$=cluster$, which
        strategy needs to be followed for the evaluation stage. If ``first'', the
        first ROM (representative segmented ROM),in each cluster, is considered to
         be representative of the full space in the cluster (i.e. the evaluation is always performed
         interrogating the first ROM in each cluster); If ``random'', a random ROM, in each cluster,
         is choosen when an evaluation is requested.
	 \nb if ``first'' is used, there is \emph{substantial} memory savings when compared to using
	 ``random''.
         %If ``centroid'', a ROM ``trained" on the centroids
         %information of each cluster is used for the evaluation (\nb ``centroid'' option is not
         %available yet).
         \default{first}
    \end{itemize}
}

\subsection{ROM}
\label{subsec:models_ROM}
A Reduced Order Model (ROM) is a mathematical model consisting of a fast
solution trained to predict a response of interest of a physical system.
%
The ``training'' process is performed by sampling the response of a physical
model with respect to variations of its parameters subject, for example, to
probabilistic behavior.
%
The results (outcomes of the physical model) of the sampling are fed into the
algorithm representing the ROM that tunes itself to replicate those results.
%
RAVEN supports several different types of ROMs, both internally developed and
imported through an external library called ``scikit-learn''~\cite{SciKitLearn}.

Currently in RAVEN, the ROMs are classified into several sub-types that, once chosen,
provide access to several different algorithms.
%
\specBlock{a}{ROM}
%
\attrsIntro
%
\vspace{-5mm}
\begin{itemize}
  \itemsep0em
  \item \nameDescription
  \item \xmlAttr{subType}, \xmlDesc{required string attribute}, defines which of
  the sub-types should be used, choosing among the previously reported
  types.
  %
  This choice conditions the subsequent the required and/or optional
  \xmlNode{ROM} sub-nodes.
  %
\end{itemize}
\vspace{-5mm}

In the \xmlNode{ROM} input block, the following XML sub-nodes are required,
independent of the \xmlAttr{subType} specified:
%
\begin{itemize}
  %
   \item \xmlNode{Features}, \xmlDesc{comma separated string, required field},
     specifies the names of the features of this ROM.
   \nb These parameters are going to be requested for the training of this object
    (see Section~\ref{subsec:stepRomTrainer});
    \item \xmlNode{Target}, \xmlDesc{comma separated string, required field},
      contains a comma separated list of the targets of this ROM. These parameters
      are the Figures of Merit (FOMs) this ROM is supposed to predict.
    \nb These parameters are going to be requested for the training of this
    object (see Section \ref{subsec:stepRomTrainer}).
\end{itemize}

If a time-dependent ROM is requested, a \textbf{HistorySet} should be provided.
The temporal vairable specified in the \textbf{HistorySet} should be also listed
as sub-nodes inside \xmlNode{ROM}
%
\begin{itemize}
  \item \xmlNode{pivotParameter}, \xmlDesc{string, optional parameter}, specifies the pivot
    variable (e.g. time, etc) used in the input HistorySet.
    \default{time}
\end{itemize}
%
In addition, if the user wants to use the alias system, the following XML block can be inputted:
\begin{itemize}
  \item \aliasSystemDescription{ROM}
\end{itemize}


The types and meaning of the remaining sub-nodes depend on the sub-type
specified in the attribute \xmlAttr{subType}.

%
Note that if an HistorySet is provided in the training step then a temporal ROM is created, i.e. a ROM that generates not a single value prediction of each element indicated in the  \xmlNode{Target} block but its full temporal profile.
\\
\textcolor{red}{\\\textbf{It is important to NOTE that RAVEN uses a Z-score normalization of the training data before constructing most of the
Reduced Order Models (e.g. most of the SciKitLearn-based ROMs):}}
\begin{equation}
  \mathit{\mathbf{X'}} = \frac{(\mathit{\mathbf{X}}-\mu )}{\sigma }
\end{equation}
\\In the following sections the specifications of each ROM type are reported, highlighting when a \textbf{Z-score normalization} is performed by RAVEN before constructing the ROM or when it is not performed.




\subsubsection{NDspline}
  \xmlNode{NDspline} is a ROM based on an $N$-dimensional                             spline
  interpolation/extrapolation scheme.                             In spline interpolation, the
  regressor is a special type of piece-wise                             polynomial called tensor
  spline.                             The interpolation error can be made small even when using low
  degree polynomials                             for the spline.                             Spline
  interpolation avoids the problem of Runge's phenomenon, in which
  oscillation can occur between points when interpolating using higher degree
  polynomials.                             In order to use this ROM, the \xmlNode{ROM} attribute
  \xmlAttr{subType} needs to                             be \xmlString{NDspline}
  No further XML sub-nodes are required.                             \nb This ROM type must be
  trained from a regular Cartesian grid.                             Thus, it can only be trained
  from the outcomes of a grid sampling strategy.
  \zNormalizationPerformed{NDspline}

  The \xmlNode{NDspline} node recognizes the following parameters:
    \begin{itemize}
      \item \xmlAttr{name}: \xmlDesc{string, required}, 
        User-defined name to designate this entity in the RAVEN input file.
      \item \xmlAttr{verbosity}: \xmlDesc{[silent, quiet, all, debug], optional}, 
        Desired verbosity of messages coming from this entity
      \item \xmlAttr{subType}: \xmlDesc{string, required}, 
        specify the type of ROM that will be used
  \end{itemize}

  The \xmlNode{NDspline} node recognizes the following subnodes:
  \begin{itemize}
    \item \xmlNode{Features}: \xmlDesc{comma-separated strings}, 
      specifies the names of the features of this ROM.         \nb These parameters are going to be
      requested for the training of this object         (see Section~\ref{subsec:stepRomTrainer})

    \item \xmlNode{Target}: \xmlDesc{comma-separated strings}, 
      contains a comma separated list of the targets of this ROM. These parameters         are the
      Figures of Merit (FOMs) this ROM is supposed to predict.         \nb These parameters are
      going to be requested for the training of this         object (see Section
      \ref{subsec:stepRomTrainer}).

    \item \xmlNode{pivotParameter}: \xmlDesc{string}, 
      If a time-dependent ROM is requested, please specifies the pivot         variable (e.g. time,
      etc) used in the input HistorySet.
  \default{time}

    \item \xmlNode{featureSelection}:
      Apply feature selection algorithm

      The \xmlNode{featureSelection} node recognizes the following subnodes:
      \begin{itemize}
        \item \xmlNode{RFE}:
          The \xmlString{RFE} (Recursive Feature Elimination) is a feature selection algorithm.
          Feature selection refers to techniques that select a subset of the most relevant features
          for a model (ROM).         Fewer features can allow ROMs to run more efficiently (less
          space or time complexity) and be more effective.         Indeed, some ROMs (machine
          learning algorithms) can be misled by irrelevant input features, resulting in worse
          predictive performance.         RFE is a wrapper-type feature selection algorithm. This
          means that a different ROM is given and used in the core of the         method,         is
          wrapped by RFE, and used to help select features.         \\RFE works by searching for a
          subset of features by starting with all features in the training dataset and successfully
          removing         features until the desired number remains.         This is achieved by
          fitting the given ROM used in the core of the model, ranking features by importance,
          discarding the least important features, and re-fitting the model. This process is
          repeated until a specified number of         features remains.         When the full model
          is created, a measure of variable importance is computed that ranks the predictors from
          most         important to least.         At each stage of the search, the least important
          predictors are iteratively eliminated prior to rebuilding the model.         Features are
          scored either using the ROM model (if the model provides a mean to compute feature
          importances) or by         using a statistical method.         \\In RAVEN the
          \xmlString{RFE} class refers to an augmentation of the basic algorithm, since it allows,
          optionally,         to perform the search on multiple groups of targets (separately) and
          then combine the results of the search in a         single set. In addition, when the RFE
          search is concluded, the user can request to identify the set of features         that
          bring to a minimization of the score (i.e. maximimization of the accuracy).         In
          addition, using the ``applyClusteringFiltering'' option, the algorithm can, using an
          hierarchal clustering algorithm,         identify highly correlated features to speed up
          the subsequential search.
          The \xmlNode{RFE} node recognizes the following parameters:
            \begin{itemize}
              \item \xmlAttr{name}: \xmlDesc{string, required}, 
                User-defined name to designate this entity in the RAVEN input file.
              \item \xmlAttr{verbosity}: \xmlDesc{[silent, quiet, all, debug], optional}, 
                Desired verbosity of messages coming from this entity
          \end{itemize}

          The \xmlNode{RFE} node recognizes the following subnodes:
          \begin{itemize}
            \item \xmlNode{parametersToInclude}: \xmlDesc{comma-separated strings}, 
              List of IDs of features/variables to include in the search.
  \default{None}

            \item \xmlNode{whichSpace}: \xmlDesc{[Feature, feature, Target, target]}, 
              Which space to search? Target or Feature (this is temporary till DataSet training is
              implemented)
  \default{feature}

            \item \xmlNode{nFeaturesToSelect}: \xmlDesc{integer}, 
              Exact Number of features to select. If not inputted, ``nFeaturesToSelect'' will be set
              to $1/2$ of the features in the training dataset.
  \default{None}

            \item \xmlNode{maxNumberFeatures}: \xmlDesc{integer}, 
              Maximum Number of features to select, the algorithm will automatically determine the
              feature list to minimize a total score.
  \default{None}

            \item \xmlNode{onlyOutputScore}: \xmlDesc{[True, Yes, 1, False, No, 0, t, y, 1, f, n, 0]}, 
              If maxNumberFeatures is on, only output score should beconsidered? Or, in case of
              particular models (e.g. DMDC), state variable space score should be considered as
              well.
  \default{False}

            \item \xmlNode{applyClusteringFiltering}: \xmlDesc{[True, Yes, 1, False, No, 0, t, y, 1, f, n, 0]}, 
              Applying clustering correlation before RFE search? If true, an hierarchal clustering
              is applied on the feature         space aimed to remove features that are correlated
              before the actual RFE search is performed. This approach can stabilize and
              accelerate the process in case of large feature spaces (e.g > 500 features).
  \default{False}

            \item \xmlNode{applyCrossCorrelation}: \xmlDesc{[True, Yes, 1, False, No, 0, t, y, 1, f, n, 0]}, 
              In case of subgroupping, should a cross correleation analysis should be performed
              cross sub-groups?         If it is activated, a cross correleation analysis is used to
              additionally filter the features selected for each         sub-groupping search.
  \default{False}

            \item \xmlNode{step}: \xmlDesc{float}, 
              If greater than or equal to 1, then step corresponds to the (integer) number
              of features to remove at each iteration. If within (0.0, 1.0), then step
              corresponds to the percentage (rounded down) of features to remove at         each
              iteration.
  \default{1}

            \item \xmlNode{subGroup}: \xmlDesc{comma-separated strings, integers, and floats}, 
              Subgroup of output variables on which to perform the search. Multiple nodes of this
              type can be inputted. The RFE search will be then performed in each ``subgroup''
              separately and then the the union of the different feature sets are used for the final
              ROM.
          \end{itemize}

        \item \xmlNode{VarianceThreshold}:
          The \xmlString{VarianceThreshold} is a feature selector that removes     all low-variance
          features. This feature selection algorithm looks only at the features and not     the
          desired outputs. The variance threshold can be set by the user.
          The \xmlNode{VarianceThreshold} node recognizes the following parameters:
            \begin{itemize}
              \item \xmlAttr{name}: \xmlDesc{string, required}, 
                User-defined name to designate this entity in the RAVEN input file.
              \item \xmlAttr{verbosity}: \xmlDesc{[silent, quiet, all, debug], optional}, 
                Desired verbosity of messages coming from this entity
          \end{itemize}

          The \xmlNode{VarianceThreshold} node recognizes the following subnodes:
          \begin{itemize}
            \item \xmlNode{parametersToInclude}: \xmlDesc{comma-separated strings}, 
              List of IDs of features/variables to include in the search.
  \default{None}

            \item \xmlNode{whichSpace}: \xmlDesc{[Feature, feature, Target, target]}, 
              Which space to search? Target or Feature (this is temporary till DataSet training is
              implemented)
  \default{feature}

            \item \xmlNode{threshold}: \xmlDesc{float}, 
              Features with a training-set variance lower than this threshold                   will
              be removed. The default is to keep all features with non-zero
              variance, i.e. remove the features that have the same value in all
              samples.
  \default{0.0}
          \end{itemize}
      \end{itemize}

    \item \xmlNode{featureSpaceTransformation}:
      Use dimensionality reduction technique to perform a trasformation of the training dataset
      into an uncorrelated one. The dimensionality of the problem will not be reduced but
      the data will be transformed in the transformed space. E.g if the number of features
      are 5, the method projects such features into a new uncorrelated space (still 5-dimensional).
      In case of time-dependent ROMs, all the samples are concatenated in a global 2D matrix
      (n\_samples*n\_timesteps,n\_features) before applying the transformation and then reconstructed
      back into the original shape (before fitting the model).

      The \xmlNode{featureSpaceTransformation} node recognizes the following subnodes:
      \begin{itemize}
        \item \xmlNode{transformationMethod}: \xmlDesc{[PCA, KernelLinearPCA, KernelPolyPCA, KernelRbfPCA, KernelSigmoidPCA, KernelCosinePCA, ICA]}, 
          Transformation method to use. Eight options (5 Kernel PCAs) are available:
          \begin{itemize}                     \item \textit{PCA}, Principal Component Analysis;
          \item \textit{KernelLinearPCA}, Kernel (Linear) Principal component analysis;
          \item \textit{KernelPolyPCA}, Kernel (Poly) Principal component analysis;
          \item \textit{KernelRbfPCA}, Kernel(Rbf) Principal component analysis;
          \item \textit{KernelSigmoidPCA}, Kernel (Sigmoid) Principal component analysis;
          \item \textit{KernelCosinePCA}, Kernel (Cosine) Principal component analysis;
          \item \textit{ICA}, Independent component analysis;                    \end{itemize}
  \default{PCA}

        \item \xmlNode{parametersToInclude}: \xmlDesc{comma-separated strings}, 
          List of IDs of features/variables to include in the transformation process.
  \default{None}

        \item \xmlNode{whichSpace}: \xmlDesc{[Feature, feature, Target, target]}, 
          Which space to search? Target or Feature?
  \default{Feature}
      \end{itemize}

    \item \xmlNode{CV}: \xmlDesc{string}, 
      The text portion of this node needs to contain the name of the \xmlNode{PostProcessor} with
      \xmlAttr{subType}         ``CrossValidation``.
      The \xmlNode{CV} node recognizes the following parameters:
        \begin{itemize}
          \item \xmlAttr{class}: \xmlDesc{string, optional}, 
            should be set to \xmlString{Model}
          \item \xmlAttr{type}: \xmlDesc{string, optional}, 
            should be set to \xmlString{PostProcessor}
      \end{itemize}

    \item \xmlNode{alias}: \xmlDesc{string}, 
      specifies alias for         any variable of interest in the input or output space. These
      aliases can be used anywhere in the RAVEN input to         refer to the variables. In the body
      of this node the user specifies the name of the variable that the model is going to use
      (during its execution).
      The \xmlNode{alias} node recognizes the following parameters:
        \begin{itemize}
          \item \xmlAttr{variable}: \xmlDesc{string, required}, 
            define the actual alias, usable throughout the RAVEN input
          \item \xmlAttr{type}: \xmlDesc{[input, output], required}, 
            either ``input'' or ``output''.
      \end{itemize}
  \end{itemize}

\hspace{24pt}
Example:
\begin{lstlisting}[style=XML]
<Simulation>
  ...
  <Models>
    ...
    <ROM name='aUserDefinedName' subType='NDspline'>
       <Features>var1,var2,var3</Features>
       <Target>result1,result2</Target>
     </ROM>
    ...
  </Models>
  ...
</Simulation>
\end{lstlisting}


\subsubsection{pickledROM}
  It is not uncommon for a reduced-order model (ROM) to be created and trained in one RAVEN run,
  then     serialized to file (\emph{pickled}), then loaded into another RAVEN run to be used as a
  model.  When this is     the case, a \xmlNode{ROM} with subtype \xmlString{pickledROM} is used to
  hold the place of the ROM that will     be loaded from file.  The notation for this ROM is much
  less than a typical ROM; it usually only requires a name and     its subtype.     \\     Note that
  when loading ROMs from file, RAVEN will not perform any checks on the expected inputs or outputs
  of     a ROM; it is expected that a user know at least the I/O of a ROM before trying to use it as
  a model.     However, RAVEN does require that pickled ROMs be trained before pickling in the first
  place.     \\     Initially, a pickledROM is not usable.  It cannot be trained or sampled;
  attempting to do so will raise an     error.  An \xmlNode{IOStep} is used to load the ROM from
  file, at which point the ROM will have all the same     characteristics as when it was pickled in
  a previous RAVEN run.

  The \xmlNode{pickledROM} node recognizes the following parameters:
    \begin{itemize}
      \item \xmlAttr{name}: \xmlDesc{string, required}, 
        User-defined name to designate this entity in the RAVEN input file.
      \item \xmlAttr{verbosity}: \xmlDesc{[silent, quiet, all, debug], optional}, 
        Desired verbosity of messages coming from this entity
      \item \xmlAttr{subType}: \xmlDesc{string, required}, 
        specify the type of ROM that will be used
  \end{itemize}

  The \xmlNode{pickledROM} node recognizes the following subnodes:
  \begin{itemize}
    \item \xmlNode{Features}: \xmlDesc{comma-separated strings}, 
      specifies the names of the features of this ROM.         \nb These parameters are going to be
      requested for the training of this object         (see Section~\ref{subsec:stepRomTrainer})

    \item \xmlNode{Target}: \xmlDesc{comma-separated strings}, 
      contains a comma separated list of the targets of this ROM. These parameters         are the
      Figures of Merit (FOMs) this ROM is supposed to predict.         \nb These parameters are
      going to be requested for the training of this         object (see Section
      \ref{subsec:stepRomTrainer}).

    \item \xmlNode{pivotParameter}: \xmlDesc{string}, 
      If a time-dependent ROM is requested, please specifies the pivot         variable (e.g. time,
      etc) used in the input HistorySet.
  \default{time}

    \item \xmlNode{featureSelection}:
      Apply feature selection algorithm

      The \xmlNode{featureSelection} node recognizes the following subnodes:
      \begin{itemize}
        \item \xmlNode{RFE}:
          The \xmlString{RFE} (Recursive Feature Elimination) is a feature selection algorithm.
          Feature selection refers to techniques that select a subset of the most relevant features
          for a model (ROM).         Fewer features can allow ROMs to run more efficiently (less
          space or time complexity) and be more effective.         Indeed, some ROMs (machine
          learning algorithms) can be misled by irrelevant input features, resulting in worse
          predictive performance.         RFE is a wrapper-type feature selection algorithm. This
          means that a different ROM is given and used in the core of the         method,         is
          wrapped by RFE, and used to help select features.         \\RFE works by searching for a
          subset of features by starting with all features in the training dataset and successfully
          removing         features until the desired number remains.         This is achieved by
          fitting the given ROM used in the core of the model, ranking features by importance,
          discarding the least important features, and re-fitting the model. This process is
          repeated until a specified number of         features remains.         When the full model
          is created, a measure of variable importance is computed that ranks the predictors from
          most         important to least.         At each stage of the search, the least important
          predictors are iteratively eliminated prior to rebuilding the model.         Features are
          scored either using the ROM model (if the model provides a mean to compute feature
          importances) or by         using a statistical method.         \\In RAVEN the
          \xmlString{RFE} class refers to an augmentation of the basic algorithm, since it allows,
          optionally,         to perform the search on multiple groups of targets (separately) and
          then combine the results of the search in a         single set. In addition, when the RFE
          search is concluded, the user can request to identify the set of features         that
          bring to a minimization of the score (i.e. maximimization of the accuracy).         In
          addition, using the ``applyClusteringFiltering'' option, the algorithm can, using an
          hierarchal clustering algorithm,         identify highly correlated features to speed up
          the subsequential search.
          The \xmlNode{RFE} node recognizes the following parameters:
            \begin{itemize}
              \item \xmlAttr{name}: \xmlDesc{string, required}, 
                User-defined name to designate this entity in the RAVEN input file.
              \item \xmlAttr{verbosity}: \xmlDesc{[silent, quiet, all, debug], optional}, 
                Desired verbosity of messages coming from this entity
          \end{itemize}

          The \xmlNode{RFE} node recognizes the following subnodes:
          \begin{itemize}
            \item \xmlNode{parametersToInclude}: \xmlDesc{comma-separated strings}, 
              List of IDs of features/variables to include in the search.
  \default{None}

            \item \xmlNode{whichSpace}: \xmlDesc{[Feature, feature, Target, target]}, 
              Which space to search? Target or Feature (this is temporary till DataSet training is
              implemented)
  \default{feature}

            \item \xmlNode{nFeaturesToSelect}: \xmlDesc{integer}, 
              Exact Number of features to select. If not inputted, ``nFeaturesToSelect'' will be set
              to $1/2$ of the features in the training dataset.
  \default{None}

            \item \xmlNode{maxNumberFeatures}: \xmlDesc{integer}, 
              Maximum Number of features to select, the algorithm will automatically determine the
              feature list to minimize a total score.
  \default{None}

            \item \xmlNode{onlyOutputScore}: \xmlDesc{[True, Yes, 1, False, No, 0, t, y, 1, f, n, 0]}, 
              If maxNumberFeatures is on, only output score should beconsidered? Or, in case of
              particular models (e.g. DMDC), state variable space score should be considered as
              well.
  \default{False}

            \item \xmlNode{applyClusteringFiltering}: \xmlDesc{[True, Yes, 1, False, No, 0, t, y, 1, f, n, 0]}, 
              Applying clustering correlation before RFE search? If true, an hierarchal clustering
              is applied on the feature         space aimed to remove features that are correlated
              before the actual RFE search is performed. This approach can stabilize and
              accelerate the process in case of large feature spaces (e.g > 500 features).
  \default{False}

            \item \xmlNode{applyCrossCorrelation}: \xmlDesc{[True, Yes, 1, False, No, 0, t, y, 1, f, n, 0]}, 
              In case of subgroupping, should a cross correleation analysis should be performed
              cross sub-groups?         If it is activated, a cross correleation analysis is used to
              additionally filter the features selected for each         sub-groupping search.
  \default{False}

            \item \xmlNode{step}: \xmlDesc{float}, 
              If greater than or equal to 1, then step corresponds to the (integer) number
              of features to remove at each iteration. If within (0.0, 1.0), then step
              corresponds to the percentage (rounded down) of features to remove at         each
              iteration.
  \default{1}

            \item \xmlNode{subGroup}: \xmlDesc{comma-separated strings, integers, and floats}, 
              Subgroup of output variables on which to perform the search. Multiple nodes of this
              type can be inputted. The RFE search will be then performed in each ``subgroup''
              separately and then the the union of the different feature sets are used for the final
              ROM.
          \end{itemize}

        \item \xmlNode{VarianceThreshold}:
          The \xmlString{VarianceThreshold} is a feature selector that removes     all low-variance
          features. This feature selection algorithm looks only at the features and not     the
          desired outputs. The variance threshold can be set by the user.
          The \xmlNode{VarianceThreshold} node recognizes the following parameters:
            \begin{itemize}
              \item \xmlAttr{name}: \xmlDesc{string, required}, 
                User-defined name to designate this entity in the RAVEN input file.
              \item \xmlAttr{verbosity}: \xmlDesc{[silent, quiet, all, debug], optional}, 
                Desired verbosity of messages coming from this entity
          \end{itemize}

          The \xmlNode{VarianceThreshold} node recognizes the following subnodes:
          \begin{itemize}
            \item \xmlNode{parametersToInclude}: \xmlDesc{comma-separated strings}, 
              List of IDs of features/variables to include in the search.
  \default{None}

            \item \xmlNode{whichSpace}: \xmlDesc{[Feature, feature, Target, target]}, 
              Which space to search? Target or Feature (this is temporary till DataSet training is
              implemented)
  \default{feature}

            \item \xmlNode{threshold}: \xmlDesc{float}, 
              Features with a training-set variance lower than this threshold                   will
              be removed. The default is to keep all features with non-zero
              variance, i.e. remove the features that have the same value in all
              samples.
  \default{0.0}
          \end{itemize}
      \end{itemize}

    \item \xmlNode{featureSpaceTransformation}:
      Use dimensionality reduction technique to perform a trasformation of the training dataset
      into an uncorrelated one. The dimensionality of the problem will not be reduced but
      the data will be transformed in the transformed space. E.g if the number of features
      are 5, the method projects such features into a new uncorrelated space (still 5-dimensional).
      In case of time-dependent ROMs, all the samples are concatenated in a global 2D matrix
      (n\_samples*n\_timesteps,n\_features) before applying the transformation and then reconstructed
      back into the original shape (before fitting the model).

      The \xmlNode{featureSpaceTransformation} node recognizes the following subnodes:
      \begin{itemize}
        \item \xmlNode{transformationMethod}: \xmlDesc{[PCA, KernelLinearPCA, KernelPolyPCA, KernelRbfPCA, KernelSigmoidPCA, KernelCosinePCA, ICA]}, 
          Transformation method to use. Eight options (5 Kernel PCAs) are available:
          \begin{itemize}                     \item \textit{PCA}, Principal Component Analysis;
          \item \textit{KernelLinearPCA}, Kernel (Linear) Principal component analysis;
          \item \textit{KernelPolyPCA}, Kernel (Poly) Principal component analysis;
          \item \textit{KernelRbfPCA}, Kernel(Rbf) Principal component analysis;
          \item \textit{KernelSigmoidPCA}, Kernel (Sigmoid) Principal component analysis;
          \item \textit{KernelCosinePCA}, Kernel (Cosine) Principal component analysis;
          \item \textit{ICA}, Independent component analysis;                    \end{itemize}
  \default{PCA}

        \item \xmlNode{parametersToInclude}: \xmlDesc{comma-separated strings}, 
          List of IDs of features/variables to include in the transformation process.
  \default{None}

        \item \xmlNode{whichSpace}: \xmlDesc{[Feature, feature, Target, target]}, 
          Which space to search? Target or Feature?
  \default{Feature}
      \end{itemize}

    \item \xmlNode{CV}: \xmlDesc{string}, 
      The text portion of this node needs to contain the name of the \xmlNode{PostProcessor} with
      \xmlAttr{subType}         ``CrossValidation``.
      The \xmlNode{CV} node recognizes the following parameters:
        \begin{itemize}
          \item \xmlAttr{class}: \xmlDesc{string, optional}, 
            should be set to \xmlString{Model}
          \item \xmlAttr{type}: \xmlDesc{string, optional}, 
            should be set to \xmlString{PostProcessor}
      \end{itemize}

    \item \xmlNode{alias}: \xmlDesc{string}, 
      specifies alias for         any variable of interest in the input or output space. These
      aliases can be used anywhere in the RAVEN input to         refer to the variables. In the body
      of this node the user specifies the name of the variable that the model is going to use
      (during its execution).
      The \xmlNode{alias} node recognizes the following parameters:
        \begin{itemize}
          \item \xmlAttr{variable}: \xmlDesc{string, required}, 
            define the actual alias, usable throughout the RAVEN input
          \item \xmlAttr{type}: \xmlDesc{[input, output], required}, 
            either ``input'' or ``output''.
      \end{itemize}

    \item \xmlNode{seed}: \xmlDesc{integer}, 
      provides seed for VARMA and ARMA sampling.
      Must be provided before training. If no seed is assigned,
      then a random number will be used.
  \default{None}

    \item \xmlNode{Multicycle}: \xmlDesc{string}, 
      indicates that each sample of the ARMA should yield
      multiple sequential samples. For example, if an ARMA model is trained to produce a year's
      worth of data,                                                    enabling
      \xmlNode{Multicycle} causes it to produce several successive years of data. Multicycle
      sampling                                                    is independent of ROM training,
      and only changes how samples of the ARMA are created.
      \nb The output of a multicycle ARMA must be stored in a \xmlNode{DataSet}, as the targets will
      depend                                                    on both the \xmlNode{pivotParameter}
      as well as the cycle, \xmlString{Cycle}. The cycle is a second
      \xmlNode{Index} that all targets should depend on, with variable name \xmlString{Cycle}.
  \default{None}

      The \xmlNode{Multicycle} node recognizes the following subnodes:
      \begin{itemize}
        \item \xmlNode{cycles}: \xmlDesc{integer}, 
          the number of cycles the ARMA should produce
          each time it yields a sample.

        \item \xmlNode{growth}: \xmlDesc{float}, 
          if provided then the histories produced by
          the ARMA will be increased by the growth factor for successive cycles. This node can be
          added                                                    multiple times with different
          settings for different targets.                                                    The
          text of this node is the growth factor in percentage. Some examples are in
          Table~\ref{tab:arma multicycle growth}, where \emph{Growth factor} is the value used in
          the RAVEN                                                    input and \emph{Scaling
          factor} is the value by which the history will be multiplied.
          \begin{table}[h!]                                                      \centering
          \begin{tabular}{r c l}                                                        Growth
          factor & Scaling factor & Description \\ \hline
          50 & 1.5 & growing by 50\% each cycle \\
          -50 & 0.5 & shrinking by 50\% each cycle \\
          150 & 2.5 & growing by 150\% each cycle \\
          \end{tabular}                                                      \caption{ARMA Growth
          Factor Examples}                                                      \label{tab:arma
          multicycle growth}                                                    \end{table}
  \default{None}
          The \xmlNode{growth} node recognizes the following parameters:
            \begin{itemize}
              \item \xmlAttr{targets}: \xmlDesc{comma-separated strings, required}, 
                lists the targets                     in this ARMA that this growth factor should
                apply to.
              \item \xmlAttr{start\_index}: \xmlDesc{integer, optional}, 
                -- no description yet --
              \item \xmlAttr{end\_index}: \xmlDesc{integer, optional}, 
                -- no description yet --
              \item \xmlAttr{mode}: \xmlDesc{[exponential, linear], required}, 
                either \xmlString{linear} or                     \xmlString{exponential}, determines
                the manner in which the growth factor is applied.                     If
                \xmlString{linear}, then the scaling factor is $(1+y\cdot g/100)$;
                if \xmlString{exponential}, then the scaling factor is $(1+g/100)^y$;
                where $y$ is the cycle after the first and $g$ is the provided scaling factor.
          \end{itemize}
      \end{itemize}

    \item \xmlNode{clusterEvalMode}: \xmlDesc{[clustered, truncated, full]}, 
      changes the structure of the samples for Clustered
      Segmented ROMs. These are identical to the options for \xmlNode{evalMode}
      node under \xmlNode{Segmented}
  \default{None}

    \item \xmlNode{maxCycles}: \xmlDesc{integer}, 
      maximum number of cycles to run (default no limit)
  \default{None}
  \end{itemize}

\hspace{24pt}
Example:
For this example the ROM has already been created and trained in another RAVEN run, then pickled to a file
called \texttt{rom\_pickle.pk}.  In the example, the file is identified in \xmlNode{Files}, the model is
defined in \xmlNode{Models}, and the model loaded in \xmlNode{Steps}.
\begin{lstlisting}[style=XML]
<Simulation>
  ...
  <Files>
    <Input name="rompk" type="">rom_pickle.pk</Input>
  </Files>
  ...
  <Models>
    ...
    <ROM name="myRom" subType="pickledROM"/>
    ...
  </Models>
  ...
  <Steps>
    ...
    <IOStep name="loadROM">
      <Input class="Files" type="">rompk</Input>
      <Output class="Models" type="ROM">myRom</Output>
    </IOStep>
    ...
  </Steps>
  ...
</Simulation>
\end{lstlisting}


\subsubsection{GaussPolynomialRom}
  The \xmlString{GaussPolynomialRom} is based on a                         characteristic Gaussian
  polynomial fitting scheme: generalized polynomial chaos                         expansion (gPC).
  \\                         In gPC, sets of polynomials orthogonal with respect to the distribution
  of uncertainty                         are used to represent the original model.  The method
  converges moments of the original                         model faster than Monte Carlo for small-
  dimension uncertainty spaces ($N<15$).                         In order to use this ROM, the
  \xmlNode{ROM} attribute \xmlAttr{subType} needs to                         be
  \xmlString{GaussPolynomialRom}.                         \\                         The
  GaussPolynomialRom is dependent on specific sampling; thus, this ROM cannot be trained unless a
  SparseGridCollocation or similar Sampler specifies this ROM in its input and is sampled in a
  MultiRun step.                         \begin{table}[htb]                           \centering
  \begin{tabular}{c | c c}                             Unc. Distribution & Default Quadrature &
  Default Polynomials \\ \hline                             Uniform & Legendre & Legendre \\
  Normal & Hermite & Hermite \\ \hline                             Gamma & Laguerre & Laguerre \\
  Beta & Jacobi & Jacobi \\ \hline                             Other & Legendre* & Legendre*
  \end{tabular}                           \caption{GaussPolynomialRom defaults}
  \label{tab:gpcCompatible}                         \end{table}                         \nb This ROM
  type must be trained from a collocation quadrature set.                         Thus, it can only
  be trained from the outcomes of a SparseGridCollocation sampler.                         Also,
  this ROM must be referenced in the SparseGridCollocation sampler in order to
  accurately produce the necessary sparse grid points to train this ROM.
  \zNormalizationNotPerformed{GaussPolynomialRom}                         \\
  When Printing this ROM via a Print OutStream (see \ref{sec:printing}), the available metrics are:
  \begin{itemize}                           \item \xmlString{mean}, the mean value of the ROM output
  within the input space it was trained,                           \item \xmlString{variance}, the
  variance of the ROM output within the input space it was trained,                           \item
  \xmlString{samples}, the number of distinct model runs required to construct the ROM,
  \item \xmlString{indices}, the Sobol sensitivity indices (in percent), Sobol total indices, and
  partial variances,                           \item \xmlString{polyCoeffs}, the polynomial
  expansion coefficients (PCE moments) of the ROM.  These are                             listed by
  each polynomial combination, with the polynomial order tags listed in the order of the variables
  shown in the XML print.                         \end{itemize}

  The \xmlNode{GaussPolynomialRom} node recognizes the following parameters:
    \begin{itemize}
      \item \xmlAttr{name}: \xmlDesc{string, required}, 
        User-defined name to designate this entity in the RAVEN input file.
      \item \xmlAttr{verbosity}: \xmlDesc{[silent, quiet, all, debug], optional}, 
        Desired verbosity of messages coming from this entity
      \item \xmlAttr{subType}: \xmlDesc{string, required}, 
        specify the type of ROM that will be used
  \end{itemize}

  The \xmlNode{GaussPolynomialRom} node recognizes the following subnodes:
  \begin{itemize}
    \item \xmlNode{Features}: \xmlDesc{comma-separated strings}, 
      specifies the names of the features of this ROM.         \nb These parameters are going to be
      requested for the training of this object         (see Section~\ref{subsec:stepRomTrainer})

    \item \xmlNode{Target}: \xmlDesc{comma-separated strings}, 
      contains a comma separated list of the targets of this ROM. These parameters         are the
      Figures of Merit (FOMs) this ROM is supposed to predict.         \nb These parameters are
      going to be requested for the training of this         object (see Section
      \ref{subsec:stepRomTrainer}).

    \item \xmlNode{pivotParameter}: \xmlDesc{string}, 
      If a time-dependent ROM is requested, please specifies the pivot         variable (e.g. time,
      etc) used in the input HistorySet.
  \default{time}

    \item \xmlNode{featureSelection}:
      Apply feature selection algorithm

      The \xmlNode{featureSelection} node recognizes the following subnodes:
      \begin{itemize}
        \item \xmlNode{RFE}:
          The \xmlString{RFE} (Recursive Feature Elimination) is a feature selection algorithm.
          Feature selection refers to techniques that select a subset of the most relevant features
          for a model (ROM).         Fewer features can allow ROMs to run more efficiently (less
          space or time complexity) and be more effective.         Indeed, some ROMs (machine
          learning algorithms) can be misled by irrelevant input features, resulting in worse
          predictive performance.         RFE is a wrapper-type feature selection algorithm. This
          means that a different ROM is given and used in the core of the         method,         is
          wrapped by RFE, and used to help select features.         \\RFE works by searching for a
          subset of features by starting with all features in the training dataset and successfully
          removing         features until the desired number remains.         This is achieved by
          fitting the given ROM used in the core of the model, ranking features by importance,
          discarding the least important features, and re-fitting the model. This process is
          repeated until a specified number of         features remains.         When the full model
          is created, a measure of variable importance is computed that ranks the predictors from
          most         important to least.         At each stage of the search, the least important
          predictors are iteratively eliminated prior to rebuilding the model.         Features are
          scored either using the ROM model (if the model provides a mean to compute feature
          importances) or by         using a statistical method.         \\In RAVEN the
          \xmlString{RFE} class refers to an augmentation of the basic algorithm, since it allows,
          optionally,         to perform the search on multiple groups of targets (separately) and
          then combine the results of the search in a         single set. In addition, when the RFE
          search is concluded, the user can request to identify the set of features         that
          bring to a minimization of the score (i.e. maximimization of the accuracy).         In
          addition, using the ``applyClusteringFiltering'' option, the algorithm can, using an
          hierarchal clustering algorithm,         identify highly correlated features to speed up
          the subsequential search.
          The \xmlNode{RFE} node recognizes the following parameters:
            \begin{itemize}
              \item \xmlAttr{name}: \xmlDesc{string, required}, 
                User-defined name to designate this entity in the RAVEN input file.
              \item \xmlAttr{verbosity}: \xmlDesc{[silent, quiet, all, debug], optional}, 
                Desired verbosity of messages coming from this entity
          \end{itemize}

          The \xmlNode{RFE} node recognizes the following subnodes:
          \begin{itemize}
            \item \xmlNode{parametersToInclude}: \xmlDesc{comma-separated strings}, 
              List of IDs of features/variables to include in the search.
  \default{None}

            \item \xmlNode{whichSpace}: \xmlDesc{[Feature, feature, Target, target]}, 
              Which space to search? Target or Feature (this is temporary till DataSet training is
              implemented)
  \default{feature}

            \item \xmlNode{nFeaturesToSelect}: \xmlDesc{integer}, 
              Exact Number of features to select. If not inputted, ``nFeaturesToSelect'' will be set
              to $1/2$ of the features in the training dataset.
  \default{None}

            \item \xmlNode{maxNumberFeatures}: \xmlDesc{integer}, 
              Maximum Number of features to select, the algorithm will automatically determine the
              feature list to minimize a total score.
  \default{None}

            \item \xmlNode{onlyOutputScore}: \xmlDesc{[True, Yes, 1, False, No, 0, t, y, 1, f, n, 0]}, 
              If maxNumberFeatures is on, only output score should beconsidered? Or, in case of
              particular models (e.g. DMDC), state variable space score should be considered as
              well.
  \default{False}

            \item \xmlNode{applyClusteringFiltering}: \xmlDesc{[True, Yes, 1, False, No, 0, t, y, 1, f, n, 0]}, 
              Applying clustering correlation before RFE search? If true, an hierarchal clustering
              is applied on the feature         space aimed to remove features that are correlated
              before the actual RFE search is performed. This approach can stabilize and
              accelerate the process in case of large feature spaces (e.g > 500 features).
  \default{False}

            \item \xmlNode{applyCrossCorrelation}: \xmlDesc{[True, Yes, 1, False, No, 0, t, y, 1, f, n, 0]}, 
              In case of subgroupping, should a cross correleation analysis should be performed
              cross sub-groups?         If it is activated, a cross correleation analysis is used to
              additionally filter the features selected for each         sub-groupping search.
  \default{False}

            \item \xmlNode{step}: \xmlDesc{float}, 
              If greater than or equal to 1, then step corresponds to the (integer) number
              of features to remove at each iteration. If within (0.0, 1.0), then step
              corresponds to the percentage (rounded down) of features to remove at         each
              iteration.
  \default{1}

            \item \xmlNode{subGroup}: \xmlDesc{comma-separated strings, integers, and floats}, 
              Subgroup of output variables on which to perform the search. Multiple nodes of this
              type can be inputted. The RFE search will be then performed in each ``subgroup''
              separately and then the the union of the different feature sets are used for the final
              ROM.
          \end{itemize}

        \item \xmlNode{VarianceThreshold}:
          The \xmlString{VarianceThreshold} is a feature selector that removes     all low-variance
          features. This feature selection algorithm looks only at the features and not     the
          desired outputs. The variance threshold can be set by the user.
          The \xmlNode{VarianceThreshold} node recognizes the following parameters:
            \begin{itemize}
              \item \xmlAttr{name}: \xmlDesc{string, required}, 
                User-defined name to designate this entity in the RAVEN input file.
              \item \xmlAttr{verbosity}: \xmlDesc{[silent, quiet, all, debug], optional}, 
                Desired verbosity of messages coming from this entity
          \end{itemize}

          The \xmlNode{VarianceThreshold} node recognizes the following subnodes:
          \begin{itemize}
            \item \xmlNode{parametersToInclude}: \xmlDesc{comma-separated strings}, 
              List of IDs of features/variables to include in the search.
  \default{None}

            \item \xmlNode{whichSpace}: \xmlDesc{[Feature, feature, Target, target]}, 
              Which space to search? Target or Feature (this is temporary till DataSet training is
              implemented)
  \default{feature}

            \item \xmlNode{threshold}: \xmlDesc{float}, 
              Features with a training-set variance lower than this threshold                   will
              be removed. The default is to keep all features with non-zero
              variance, i.e. remove the features that have the same value in all
              samples.
  \default{0.0}
          \end{itemize}
      \end{itemize}

    \item \xmlNode{featureSpaceTransformation}:
      Use dimensionality reduction technique to perform a trasformation of the training dataset
      into an uncorrelated one. The dimensionality of the problem will not be reduced but
      the data will be transformed in the transformed space. E.g if the number of features
      are 5, the method projects such features into a new uncorrelated space (still 5-dimensional).
      In case of time-dependent ROMs, all the samples are concatenated in a global 2D matrix
      (n\_samples*n\_timesteps,n\_features) before applying the transformation and then reconstructed
      back into the original shape (before fitting the model).

      The \xmlNode{featureSpaceTransformation} node recognizes the following subnodes:
      \begin{itemize}
        \item \xmlNode{transformationMethod}: \xmlDesc{[PCA, KernelLinearPCA, KernelPolyPCA, KernelRbfPCA, KernelSigmoidPCA, KernelCosinePCA, ICA]}, 
          Transformation method to use. Eight options (5 Kernel PCAs) are available:
          \begin{itemize}                     \item \textit{PCA}, Principal Component Analysis;
          \item \textit{KernelLinearPCA}, Kernel (Linear) Principal component analysis;
          \item \textit{KernelPolyPCA}, Kernel (Poly) Principal component analysis;
          \item \textit{KernelRbfPCA}, Kernel(Rbf) Principal component analysis;
          \item \textit{KernelSigmoidPCA}, Kernel (Sigmoid) Principal component analysis;
          \item \textit{KernelCosinePCA}, Kernel (Cosine) Principal component analysis;
          \item \textit{ICA}, Independent component analysis;                    \end{itemize}
  \default{PCA}

        \item \xmlNode{parametersToInclude}: \xmlDesc{comma-separated strings}, 
          List of IDs of features/variables to include in the transformation process.
  \default{None}

        \item \xmlNode{whichSpace}: \xmlDesc{[Feature, feature, Target, target]}, 
          Which space to search? Target or Feature?
  \default{Feature}
      \end{itemize}

    \item \xmlNode{alias}: \xmlDesc{string}, 
      specifies alias for         any variable of interest in the input or output space. These
      aliases can be used anywhere in the RAVEN input to         refer to the variables. In the body
      of this node the user specifies the name of the variable that the model is going to use
      (during its execution).
      The \xmlNode{alias} node recognizes the following parameters:
        \begin{itemize}
          \item \xmlAttr{variable}: \xmlDesc{string, required}, 
            define the actual alias, usable throughout the RAVEN input
          \item \xmlAttr{type}: \xmlDesc{[input, output], required}, 
            either ``input'' or ``output''.
      \end{itemize}

    \item \xmlNode{IndexSet}: \xmlDesc{[TensorProduct, TotalDegree, HyperbolicCross, Custom]}, 
      specifies the rules by which to construct multidimensional polynomials.  The options are
      \xmlString{TensorProduct}, \xmlString{TotalDegree},\\
      \xmlString{HyperbolicCross}, and \xmlString{Custom}.
      Total degree is efficient for                                                  uncertain
      inputs with a large degree of regularity, while hyperbolic cross is more efficient
      for low-regularity input spaces.                                                  If
      \xmlString{Custom} is chosen, the \xmlNode{IndexPoints} is required.

    \item \xmlNode{PolynomialOrder}: \xmlDesc{integer}, 
      indicates the maximum polynomial order in any one dimension to use in the
      polynomial chaos expansion. \nb If non-equal importance weights are supplied in the optional
      \xmlNode{Interpolation} node, the actual polynomial order in dimensions with high
      importance might exceed this value; however, this value is still used to limit the
      relative overall order.

    \item \xmlNode{SparseGrid}: \xmlDesc{[smolyak, tensor]}, 
      allows specification of the multidimensional
      quadrature construction strategy.  Options are \xmlString{smolyak} and \xmlString{tensor}.
  \default{smolyak}

    \item \xmlNode{IndexPoints}: \xmlDesc{comma-separated list of comma separated integer tuples}, 
      used to specify the index set points in a \xmlString{Custom} index set.  The tuples are
      entered as comma-separated values between parenthesis, with each tuple separated by a comma.
      Any amount of whitespace is acceptable.  For example,
      \xmlNode{IndexPoints}\verb'(0,1),(0,2),(1,1),(4,0)'\xmlNode{/IndexPoints}
      \nb{Using custom index sets                                                  does not
      guarantee accurate convergence.}
  \default{None}

    \item \xmlNode{Interpolation}: \xmlDesc{string}, 
      offers the option to specify quadrature, polynomials, and importance weights for the given
      variable name.  The ROM accepts any number of \xmlNode{Interpolation} nodes up to the
      dimensionality of the input space.
  \default{None}
      The \xmlNode{Interpolation} node recognizes the following parameters:
        \begin{itemize}
          \item \xmlAttr{quad}: \xmlDesc{string, optional}, 
            specifies the quadrature type to use for collocation in this dimension.  The default
            options                   depend on the uncertainty distribution of the input dimension,
            as shown in Table                   \ref{tab:gpcCompatible}. Additionally, Clenshaw
            Curtis quadrature can be used for any                   distribution that doesn't
            include an infinite bound.                   \default{see Table
            \ref{tab:gpcCompatible}.}                   \nb For an uncertain distribution aside from
            the four listed on Table                   \ref{tab:gpcCompatible}, this ROM
            makes use of the uniform-like range of the distribution's CDF to apply quadrature that
            is                   suited uniform uncertainty (Legendre).  It converges more slowly
            than the four listed, but are                   viable choices.  Choosing polynomial
            type Legendre for any non-uniform distribution will                   enable this
            formulation automatically.
          \item \xmlAttr{poly}: \xmlDesc{string, optional}, 
            specifies the interpolating polynomial family to use for the polynomial expansion in
            this                   dimension.  The default options depend on the quadrature type
            chosen, as shown in Table                   \ref{tab:gpcCompatible}.  Currently, no
            polynomials are available outside the                   default. \default{see Table
            \ref{tab:gpcCompatible}.}
          \item \xmlAttr{weight}: \xmlDesc{float, optional}, 
            delineates the importance weighting of this dimension.  A larger importance weight will
            result in increased resolution for this dimension at the cost of resolution in lower-
            weighted                   dimensions.  The algorithm normalizes weights at run-time. \default{1}
      \end{itemize}
  \end{itemize}

\hspace{24pt}
Example:
\begin{lstlisting}[style=XML,morekeywords={name,subType}]
<Simulation>
  ...
  <Samplers>
    ...
    <SparseGridCollocation name="mySG" parallel="0">
      <variable name="x1">
        <distribution>myDist1</distribution>
      </variable>
      <variable name="x2">
        <distribution>myDist2</distribution>
      </variable>
      <ROM class = 'Models' type = 'ROM' >myROM</ROM>
    </SparseGridCollocation>
    ...
  </Samplers>
  ...
  <Models>
    ...
    <ROM name='myRom' subType='GaussPolynomialRom'>
      <Target>ans</Target>
      <Features>x1,x2</Features>
      <IndexSet>TotalDegree</IndexSet>
      <PolynomialOrder>4</PolynomialOrder>
      <Interpolation quad='Legendre' poly='Legendre' weight='1'>x1</Interpolation>
      <Interpolation quad='ClenshawCurtis' poly='Jacobi' weight='2'>x2</Interpolation>
    </ROM>
    ...
  </Models>
  ...
</Simulation>
\end{lstlisting}


\subsubsection{HDMRRom}
  The \xmlString{HDMRRom} is based on a Sobol decomposition scheme.                         In Sobol
  decomposition, also known as high-density model reduction (HDMR, specifically Cut-HDMR),
  a model is approximated as as the sum of increasing-complexity interactions.  At its lowest level
  (order 1), it treats the function as a sum of the reference case plus a functional of each input
  dimesion separately.  At order 2, it adds functionals to consider the pairing of each dimension
  with each other dimension.  The benefit to this approach is considering several functions of small
  input cardinality instead of a single function with large input cardinality.  This allows reduced
  order models like generalized polynomial chaos (see \ref{subsubsec:GaussPolynomialRom}) to
  approximate the functionals accurately with few computations runs.                         In
  order to use this ROM, the \xmlNode{ROM} attribute \xmlAttr{subType} needs to
  be \xmlString{HDMRRom}.                         \\                         The HDMRRom is
  dependent on specific sampling; thus, this ROM cannot be trained unless a
  Sobol or similar Sampler specifies this ROM in its input and is sampled in a MultiRun step.
  \\                         \nb This ROM type must be trained from a Sobol decomposition training
  set.                         Thus, it can only be trained from the outcomes of a Sobol sampler.
  Also, this ROM must be referenced in the Sobol sampler in order to
  accurately produce the necessary sparse grid points to train this ROM.
  Experience has shown order 2 Sobol decompositions to include the great majority of
  uncertainty in most models.                         \zNormalizationNotPerformed{HDMRRom}
  \\                         When Printing this ROM via an OutStream (see \ref{sec:printing}), the
  available metrics are:                         \begin{itemize}                           \item
  \xmlString{mean}, the mean value of the ROM output within the input space it was trained,
  \item \xmlString{variance}, the ANOVA-calculated variance of the ROM output within the input space
  it                             was trained.                           \item \xmlString{samples},
  the number of distinct model runs required to construct the ROM,                           \item
  \xmlString{indices}, the Sobol sensitivity indices (in percent), Sobol total indices, and partial
  variances.                         \end{itemize}

  The \xmlNode{HDMRRom} node recognizes the following parameters:
    \begin{itemize}
      \item \xmlAttr{name}: \xmlDesc{string, required}, 
        User-defined name to designate this entity in the RAVEN input file.
      \item \xmlAttr{verbosity}: \xmlDesc{[silent, quiet, all, debug], optional}, 
        Desired verbosity of messages coming from this entity
      \item \xmlAttr{subType}: \xmlDesc{string, required}, 
        specify the type of ROM that will be used
  \end{itemize}

  The \xmlNode{HDMRRom} node recognizes the following subnodes:
  \begin{itemize}
    \item \xmlNode{Features}: \xmlDesc{comma-separated strings}, 
      specifies the names of the features of this ROM.         \nb These parameters are going to be
      requested for the training of this object         (see Section~\ref{subsec:stepRomTrainer})

    \item \xmlNode{Target}: \xmlDesc{comma-separated strings}, 
      contains a comma separated list of the targets of this ROM. These parameters         are the
      Figures of Merit (FOMs) this ROM is supposed to predict.         \nb These parameters are
      going to be requested for the training of this         object (see Section
      \ref{subsec:stepRomTrainer}).

    \item \xmlNode{pivotParameter}: \xmlDesc{string}, 
      If a time-dependent ROM is requested, please specifies the pivot         variable (e.g. time,
      etc) used in the input HistorySet.
  \default{time}

    \item \xmlNode{featureSelection}:
      Apply feature selection algorithm

      The \xmlNode{featureSelection} node recognizes the following subnodes:
      \begin{itemize}
        \item \xmlNode{RFE}:
          The \xmlString{RFE} (Recursive Feature Elimination) is a feature selection algorithm.
          Feature selection refers to techniques that select a subset of the most relevant features
          for a model (ROM).         Fewer features can allow ROMs to run more efficiently (less
          space or time complexity) and be more effective.         Indeed, some ROMs (machine
          learning algorithms) can be misled by irrelevant input features, resulting in worse
          predictive performance.         RFE is a wrapper-type feature selection algorithm. This
          means that a different ROM is given and used in the core of the         method,         is
          wrapped by RFE, and used to help select features.         \\RFE works by searching for a
          subset of features by starting with all features in the training dataset and successfully
          removing         features until the desired number remains.         This is achieved by
          fitting the given ROM used in the core of the model, ranking features by importance,
          discarding the least important features, and re-fitting the model. This process is
          repeated until a specified number of         features remains.         When the full model
          is created, a measure of variable importance is computed that ranks the predictors from
          most         important to least.         At each stage of the search, the least important
          predictors are iteratively eliminated prior to rebuilding the model.         Features are
          scored either using the ROM model (if the model provides a mean to compute feature
          importances) or by         using a statistical method.         \\In RAVEN the
          \xmlString{RFE} class refers to an augmentation of the basic algorithm, since it allows,
          optionally,         to perform the search on multiple groups of targets (separately) and
          then combine the results of the search in a         single set. In addition, when the RFE
          search is concluded, the user can request to identify the set of features         that
          bring to a minimization of the score (i.e. maximimization of the accuracy).         In
          addition, using the ``applyClusteringFiltering'' option, the algorithm can, using an
          hierarchal clustering algorithm,         identify highly correlated features to speed up
          the subsequential search.
          The \xmlNode{RFE} node recognizes the following parameters:
            \begin{itemize}
              \item \xmlAttr{name}: \xmlDesc{string, required}, 
                User-defined name to designate this entity in the RAVEN input file.
              \item \xmlAttr{verbosity}: \xmlDesc{[silent, quiet, all, debug], optional}, 
                Desired verbosity of messages coming from this entity
          \end{itemize}

          The \xmlNode{RFE} node recognizes the following subnodes:
          \begin{itemize}
            \item \xmlNode{parametersToInclude}: \xmlDesc{comma-separated strings}, 
              List of IDs of features/variables to include in the search.
  \default{None}

            \item \xmlNode{whichSpace}: \xmlDesc{[Feature, feature, Target, target]}, 
              Which space to search? Target or Feature (this is temporary till DataSet training is
              implemented)
  \default{feature}

            \item \xmlNode{nFeaturesToSelect}: \xmlDesc{integer}, 
              Exact Number of features to select. If not inputted, ``nFeaturesToSelect'' will be set
              to $1/2$ of the features in the training dataset.
  \default{None}

            \item \xmlNode{maxNumberFeatures}: \xmlDesc{integer}, 
              Maximum Number of features to select, the algorithm will automatically determine the
              feature list to minimize a total score.
  \default{None}

            \item \xmlNode{onlyOutputScore}: \xmlDesc{[True, Yes, 1, False, No, 0, t, y, 1, f, n, 0]}, 
              If maxNumberFeatures is on, only output score should beconsidered? Or, in case of
              particular models (e.g. DMDC), state variable space score should be considered as
              well.
  \default{False}

            \item \xmlNode{applyClusteringFiltering}: \xmlDesc{[True, Yes, 1, False, No, 0, t, y, 1, f, n, 0]}, 
              Applying clustering correlation before RFE search? If true, an hierarchal clustering
              is applied on the feature         space aimed to remove features that are correlated
              before the actual RFE search is performed. This approach can stabilize and
              accelerate the process in case of large feature spaces (e.g > 500 features).
  \default{False}

            \item \xmlNode{applyCrossCorrelation}: \xmlDesc{[True, Yes, 1, False, No, 0, t, y, 1, f, n, 0]}, 
              In case of subgroupping, should a cross correleation analysis should be performed
              cross sub-groups?         If it is activated, a cross correleation analysis is used to
              additionally filter the features selected for each         sub-groupping search.
  \default{False}

            \item \xmlNode{step}: \xmlDesc{float}, 
              If greater than or equal to 1, then step corresponds to the (integer) number
              of features to remove at each iteration. If within (0.0, 1.0), then step
              corresponds to the percentage (rounded down) of features to remove at         each
              iteration.
  \default{1}

            \item \xmlNode{subGroup}: \xmlDesc{comma-separated strings, integers, and floats}, 
              Subgroup of output variables on which to perform the search. Multiple nodes of this
              type can be inputted. The RFE search will be then performed in each ``subgroup''
              separately and then the the union of the different feature sets are used for the final
              ROM.
          \end{itemize}

        \item \xmlNode{VarianceThreshold}:
          The \xmlString{VarianceThreshold} is a feature selector that removes     all low-variance
          features. This feature selection algorithm looks only at the features and not     the
          desired outputs. The variance threshold can be set by the user.
          The \xmlNode{VarianceThreshold} node recognizes the following parameters:
            \begin{itemize}
              \item \xmlAttr{name}: \xmlDesc{string, required}, 
                User-defined name to designate this entity in the RAVEN input file.
              \item \xmlAttr{verbosity}: \xmlDesc{[silent, quiet, all, debug], optional}, 
                Desired verbosity of messages coming from this entity
          \end{itemize}

          The \xmlNode{VarianceThreshold} node recognizes the following subnodes:
          \begin{itemize}
            \item \xmlNode{parametersToInclude}: \xmlDesc{comma-separated strings}, 
              List of IDs of features/variables to include in the search.
  \default{None}

            \item \xmlNode{whichSpace}: \xmlDesc{[Feature, feature, Target, target]}, 
              Which space to search? Target or Feature (this is temporary till DataSet training is
              implemented)
  \default{feature}

            \item \xmlNode{threshold}: \xmlDesc{float}, 
              Features with a training-set variance lower than this threshold                   will
              be removed. The default is to keep all features with non-zero
              variance, i.e. remove the features that have the same value in all
              samples.
  \default{0.0}
          \end{itemize}
      \end{itemize}

    \item \xmlNode{featureSpaceTransformation}:
      Use dimensionality reduction technique to perform a trasformation of the training dataset
      into an uncorrelated one. The dimensionality of the problem will not be reduced but
      the data will be transformed in the transformed space. E.g if the number of features
      are 5, the method projects such features into a new uncorrelated space (still 5-dimensional).
      In case of time-dependent ROMs, all the samples are concatenated in a global 2D matrix
      (n\_samples*n\_timesteps,n\_features) before applying the transformation and then reconstructed
      back into the original shape (before fitting the model).

      The \xmlNode{featureSpaceTransformation} node recognizes the following subnodes:
      \begin{itemize}
        \item \xmlNode{transformationMethod}: \xmlDesc{[PCA, KernelLinearPCA, KernelPolyPCA, KernelRbfPCA, KernelSigmoidPCA, KernelCosinePCA, ICA]}, 
          Transformation method to use. Eight options (5 Kernel PCAs) are available:
          \begin{itemize}                     \item \textit{PCA}, Principal Component Analysis;
          \item \textit{KernelLinearPCA}, Kernel (Linear) Principal component analysis;
          \item \textit{KernelPolyPCA}, Kernel (Poly) Principal component analysis;
          \item \textit{KernelRbfPCA}, Kernel(Rbf) Principal component analysis;
          \item \textit{KernelSigmoidPCA}, Kernel (Sigmoid) Principal component analysis;
          \item \textit{KernelCosinePCA}, Kernel (Cosine) Principal component analysis;
          \item \textit{ICA}, Independent component analysis;                    \end{itemize}
  \default{PCA}

        \item \xmlNode{parametersToInclude}: \xmlDesc{comma-separated strings}, 
          List of IDs of features/variables to include in the transformation process.
  \default{None}

        \item \xmlNode{whichSpace}: \xmlDesc{[Feature, feature, Target, target]}, 
          Which space to search? Target or Feature?
  \default{Feature}
      \end{itemize}

    \item \xmlNode{alias}: \xmlDesc{string}, 
      specifies alias for         any variable of interest in the input or output space. These
      aliases can be used anywhere in the RAVEN input to         refer to the variables. In the body
      of this node the user specifies the name of the variable that the model is going to use
      (during its execution).
      The \xmlNode{alias} node recognizes the following parameters:
        \begin{itemize}
          \item \xmlAttr{variable}: \xmlDesc{string, required}, 
            define the actual alias, usable throughout the RAVEN input
          \item \xmlAttr{type}: \xmlDesc{[input, output], required}, 
            either ``input'' or ``output''.
      \end{itemize}

    \item \xmlNode{IndexSet}: \xmlDesc{[TensorProduct, TotalDegree, HyperbolicCross, Custom]}, 
      specifies the rules by which to construct multidimensional polynomials.  The options are
      \xmlString{TensorProduct}, \xmlString{TotalDegree},\\
      \xmlString{HyperbolicCross}, and \xmlString{Custom}.
      Total degree is efficient for                                                  uncertain
      inputs with a large degree of regularity, while hyperbolic cross is more efficient
      for low-regularity input spaces.                                                  If
      \xmlString{Custom} is chosen, the \xmlNode{IndexPoints} is required.

    \item \xmlNode{PolynomialOrder}: \xmlDesc{integer}, 
      indicates the maximum polynomial order in any one dimension to use in the
      polynomial chaos expansion. \nb If non-equal importance weights are supplied in the optional
      \xmlNode{Interpolation} node, the actual polynomial order in dimensions with high
      importance might exceed this value; however, this value is still used to limit the
      relative overall order.

    \item \xmlNode{SparseGrid}: \xmlDesc{[smolyak, tensor]}, 
      allows specification of the multidimensional
      quadrature construction strategy.  Options are \xmlString{smolyak} and \xmlString{tensor}.
  \default{smolyak}

    \item \xmlNode{IndexPoints}: \xmlDesc{comma-separated list of comma separated integer tuples}, 
      used to specify the index set points in a \xmlString{Custom} index set.  The tuples are
      entered as comma-separated values between parenthesis, with each tuple separated by a comma.
      Any amount of whitespace is acceptable.  For example,
      \xmlNode{IndexPoints}\verb'(0,1),(0,2),(1,1),(4,0)'\xmlNode{/IndexPoints}
      \nb{Using custom index sets                                                  does not
      guarantee accurate convergence.}
  \default{None}

    \item \xmlNode{Interpolation}: \xmlDesc{string}, 
      offers the option to specify quadrature, polynomials, and importance weights for the given
      variable name.  The ROM accepts any number of \xmlNode{Interpolation} nodes up to the
      dimensionality of the input space.
  \default{None}
      The \xmlNode{Interpolation} node recognizes the following parameters:
        \begin{itemize}
          \item \xmlAttr{quad}: \xmlDesc{string, optional}, 
            specifies the quadrature type to use for collocation in this dimension.  The default
            options                   depend on the uncertainty distribution of the input dimension,
            as shown in Table                   \ref{tab:gpcCompatible}. Additionally, Clenshaw
            Curtis quadrature can be used for any                   distribution that doesn't
            include an infinite bound.                   \default{see Table
            \ref{tab:gpcCompatible}.}                   \nb For an uncertain distribution aside from
            the four listed on Table                   \ref{tab:gpcCompatible}, this ROM
            makes use of the uniform-like range of the distribution's CDF to apply quadrature that
            is                   suited uniform uncertainty (Legendre).  It converges more slowly
            than the four listed, but are                   viable choices.  Choosing polynomial
            type Legendre for any non-uniform distribution will                   enable this
            formulation automatically.
          \item \xmlAttr{poly}: \xmlDesc{string, optional}, 
            specifies the interpolating polynomial family to use for the polynomial expansion in
            this                   dimension.  The default options depend on the quadrature type
            chosen, as shown in Table                   \ref{tab:gpcCompatible}.  Currently, no
            polynomials are available outside the                   default. \default{see Table
            \ref{tab:gpcCompatible}.}
          \item \xmlAttr{weight}: \xmlDesc{float, optional}, 
            delineates the importance weighting of this dimension.  A larger importance weight will
            result in increased resolution for this dimension at the cost of resolution in lower-
            weighted                   dimensions.  The algorithm normalizes weights at run-time. \default{1}
      \end{itemize}

    \item \xmlNode{SobolOrder}: \xmlDesc{integer}, 
      indicates the maximum cardinality of the input space used in the subset functionals.  For
      example, order 1                                                  includes only functionals of
      each independent dimension separately, while order 2 considers pair-wise interactions.
  \end{itemize}

\hspace{24pt}
Example:
\begin{lstlisting}[style=XML,morekeywords={name,subType}]
  <Samplers>
    ...
    <Sobol name="mySobol" parallel="0">
      <variable name="x1">
        <distribution>myDist1</distribution>
      </variable>
      <variable name="x2">
        <distribution>myDist2</distribution>
      </variable>
      <ROM class = 'Models' type = 'ROM' >myHDMR</ROM>
    </Sobol>
    ...
  </Samplers>
  ...
  <Models>
    ...
    <ROM name='myHDMR' subType='HDMRRom'>
      <Target>ans</Target>
      <Features>x1,x2</Features>
      <SobolOrder>2</SobolOrder>
      <IndexSet>TotalDegree</IndexSet>
      <PolynomialOrder>4</PolynomialOrder>
      <Interpolation quad='Legendre' poly='Legendre' weight='1'>x1</Interpolation>
      <Interpolation quad='ClenshawCurtis' poly='Jacobi' weight='2'>x2</Interpolation>
    </ROM>
    ...
  </Models>
\end{lstlisting}


\subsubsection{MSR}
  The \xmlNode{MSR} contains a class of ROMs that perform a topological
  decomposition of the data into approximately monotonic regions and fits weighted
  linear patches to the identified monotonic regions of the input space. Query
  points have estimated probabilities that they belong to each cluster. These
  probabilities can eitehr be used to give a smooth, weighted prediction based on
  the associated linear models, or a hard categorization  to a particular local
  linear model which is then used for prediction. Currently, the probability
  prediction can be done using kernel density estimation (KDE) or through a
  one-versus-one support vector machine (SVM).                             \\
  \zNormalizationNotPerformed{MSR}                             \\                             In
  order to use this ROM, the \xmlNode{ROM} attribute \xmlAttr{subType} needs to
  be \xmlString{MSR}

  The \xmlNode{MSR} node recognizes the following parameters:
    \begin{itemize}
      \item \xmlAttr{name}: \xmlDesc{string, required}, 
        User-defined name to designate this entity in the RAVEN input file.
      \item \xmlAttr{verbosity}: \xmlDesc{[silent, quiet, all, debug], optional}, 
        Desired verbosity of messages coming from this entity
      \item \xmlAttr{subType}: \xmlDesc{string, required}, 
        specify the type of ROM that will be used
  \end{itemize}

  The \xmlNode{MSR} node recognizes the following subnodes:
  \begin{itemize}
    \item \xmlNode{Features}: \xmlDesc{comma-separated strings}, 
      specifies the names of the features of this ROM.         \nb These parameters are going to be
      requested for the training of this object         (see Section~\ref{subsec:stepRomTrainer})

    \item \xmlNode{Target}: \xmlDesc{comma-separated strings}, 
      contains a comma separated list of the targets of this ROM. These parameters         are the
      Figures of Merit (FOMs) this ROM is supposed to predict.         \nb These parameters are
      going to be requested for the training of this         object (see Section
      \ref{subsec:stepRomTrainer}).

    \item \xmlNode{pivotParameter}: \xmlDesc{string}, 
      If a time-dependent ROM is requested, please specifies the pivot         variable (e.g. time,
      etc) used in the input HistorySet.
  \default{time}

    \item \xmlNode{featureSelection}:
      Apply feature selection algorithm

      The \xmlNode{featureSelection} node recognizes the following subnodes:
      \begin{itemize}
        \item \xmlNode{RFE}:
          The \xmlString{RFE} (Recursive Feature Elimination) is a feature selection algorithm.
          Feature selection refers to techniques that select a subset of the most relevant features
          for a model (ROM).         Fewer features can allow ROMs to run more efficiently (less
          space or time complexity) and be more effective.         Indeed, some ROMs (machine
          learning algorithms) can be misled by irrelevant input features, resulting in worse
          predictive performance.         RFE is a wrapper-type feature selection algorithm. This
          means that a different ROM is given and used in the core of the         method,         is
          wrapped by RFE, and used to help select features.         \\RFE works by searching for a
          subset of features by starting with all features in the training dataset and successfully
          removing         features until the desired number remains.         This is achieved by
          fitting the given ROM used in the core of the model, ranking features by importance,
          discarding the least important features, and re-fitting the model. This process is
          repeated until a specified number of         features remains.         When the full model
          is created, a measure of variable importance is computed that ranks the predictors from
          most         important to least.         At each stage of the search, the least important
          predictors are iteratively eliminated prior to rebuilding the model.         Features are
          scored either using the ROM model (if the model provides a mean to compute feature
          importances) or by         using a statistical method.         \\In RAVEN the
          \xmlString{RFE} class refers to an augmentation of the basic algorithm, since it allows,
          optionally,         to perform the search on multiple groups of targets (separately) and
          then combine the results of the search in a         single set. In addition, when the RFE
          search is concluded, the user can request to identify the set of features         that
          bring to a minimization of the score (i.e. maximimization of the accuracy).         In
          addition, using the ``applyClusteringFiltering'' option, the algorithm can, using an
          hierarchal clustering algorithm,         identify highly correlated features to speed up
          the subsequential search.
          The \xmlNode{RFE} node recognizes the following parameters:
            \begin{itemize}
              \item \xmlAttr{name}: \xmlDesc{string, required}, 
                User-defined name to designate this entity in the RAVEN input file.
              \item \xmlAttr{verbosity}: \xmlDesc{[silent, quiet, all, debug], optional}, 
                Desired verbosity of messages coming from this entity
          \end{itemize}

          The \xmlNode{RFE} node recognizes the following subnodes:
          \begin{itemize}
            \item \xmlNode{parametersToInclude}: \xmlDesc{comma-separated strings}, 
              List of IDs of features/variables to include in the search.
  \default{None}

            \item \xmlNode{whichSpace}: \xmlDesc{[Feature, feature, Target, target]}, 
              Which space to search? Target or Feature (this is temporary till DataSet training is
              implemented)
  \default{feature}

            \item \xmlNode{nFeaturesToSelect}: \xmlDesc{integer}, 
              Exact Number of features to select. If not inputted, ``nFeaturesToSelect'' will be set
              to $1/2$ of the features in the training dataset.
  \default{None}

            \item \xmlNode{maxNumberFeatures}: \xmlDesc{integer}, 
              Maximum Number of features to select, the algorithm will automatically determine the
              feature list to minimize a total score.
  \default{None}

            \item \xmlNode{onlyOutputScore}: \xmlDesc{[True, Yes, 1, False, No, 0, t, y, 1, f, n, 0]}, 
              If maxNumberFeatures is on, only output score should beconsidered? Or, in case of
              particular models (e.g. DMDC), state variable space score should be considered as
              well.
  \default{False}

            \item \xmlNode{applyClusteringFiltering}: \xmlDesc{[True, Yes, 1, False, No, 0, t, y, 1, f, n, 0]}, 
              Applying clustering correlation before RFE search? If true, an hierarchal clustering
              is applied on the feature         space aimed to remove features that are correlated
              before the actual RFE search is performed. This approach can stabilize and
              accelerate the process in case of large feature spaces (e.g > 500 features).
  \default{False}

            \item \xmlNode{applyCrossCorrelation}: \xmlDesc{[True, Yes, 1, False, No, 0, t, y, 1, f, n, 0]}, 
              In case of subgroupping, should a cross correleation analysis should be performed
              cross sub-groups?         If it is activated, a cross correleation analysis is used to
              additionally filter the features selected for each         sub-groupping search.
  \default{False}

            \item \xmlNode{step}: \xmlDesc{float}, 
              If greater than or equal to 1, then step corresponds to the (integer) number
              of features to remove at each iteration. If within (0.0, 1.0), then step
              corresponds to the percentage (rounded down) of features to remove at         each
              iteration.
  \default{1}

            \item \xmlNode{subGroup}: \xmlDesc{comma-separated strings, integers, and floats}, 
              Subgroup of output variables on which to perform the search. Multiple nodes of this
              type can be inputted. The RFE search will be then performed in each ``subgroup''
              separately and then the the union of the different feature sets are used for the final
              ROM.
          \end{itemize}

        \item \xmlNode{VarianceThreshold}:
          The \xmlString{VarianceThreshold} is a feature selector that removes     all low-variance
          features. This feature selection algorithm looks only at the features and not     the
          desired outputs. The variance threshold can be set by the user.
          The \xmlNode{VarianceThreshold} node recognizes the following parameters:
            \begin{itemize}
              \item \xmlAttr{name}: \xmlDesc{string, required}, 
                User-defined name to designate this entity in the RAVEN input file.
              \item \xmlAttr{verbosity}: \xmlDesc{[silent, quiet, all, debug], optional}, 
                Desired verbosity of messages coming from this entity
          \end{itemize}

          The \xmlNode{VarianceThreshold} node recognizes the following subnodes:
          \begin{itemize}
            \item \xmlNode{parametersToInclude}: \xmlDesc{comma-separated strings}, 
              List of IDs of features/variables to include in the search.
  \default{None}

            \item \xmlNode{whichSpace}: \xmlDesc{[Feature, feature, Target, target]}, 
              Which space to search? Target or Feature (this is temporary till DataSet training is
              implemented)
  \default{feature}

            \item \xmlNode{threshold}: \xmlDesc{float}, 
              Features with a training-set variance lower than this threshold                   will
              be removed. The default is to keep all features with non-zero
              variance, i.e. remove the features that have the same value in all
              samples.
  \default{0.0}
          \end{itemize}
      \end{itemize}

    \item \xmlNode{featureSpaceTransformation}:
      Use dimensionality reduction technique to perform a trasformation of the training dataset
      into an uncorrelated one. The dimensionality of the problem will not be reduced but
      the data will be transformed in the transformed space. E.g if the number of features
      are 5, the method projects such features into a new uncorrelated space (still 5-dimensional).
      In case of time-dependent ROMs, all the samples are concatenated in a global 2D matrix
      (n\_samples*n\_timesteps,n\_features) before applying the transformation and then reconstructed
      back into the original shape (before fitting the model).

      The \xmlNode{featureSpaceTransformation} node recognizes the following subnodes:
      \begin{itemize}
        \item \xmlNode{transformationMethod}: \xmlDesc{[PCA, KernelLinearPCA, KernelPolyPCA, KernelRbfPCA, KernelSigmoidPCA, KernelCosinePCA, ICA]}, 
          Transformation method to use. Eight options (5 Kernel PCAs) are available:
          \begin{itemize}                     \item \textit{PCA}, Principal Component Analysis;
          \item \textit{KernelLinearPCA}, Kernel (Linear) Principal component analysis;
          \item \textit{KernelPolyPCA}, Kernel (Poly) Principal component analysis;
          \item \textit{KernelRbfPCA}, Kernel(Rbf) Principal component analysis;
          \item \textit{KernelSigmoidPCA}, Kernel (Sigmoid) Principal component analysis;
          \item \textit{KernelCosinePCA}, Kernel (Cosine) Principal component analysis;
          \item \textit{ICA}, Independent component analysis;                    \end{itemize}
  \default{PCA}

        \item \xmlNode{parametersToInclude}: \xmlDesc{comma-separated strings}, 
          List of IDs of features/variables to include in the transformation process.
  \default{None}

        \item \xmlNode{whichSpace}: \xmlDesc{[Feature, feature, Target, target]}, 
          Which space to search? Target or Feature?
  \default{Feature}
      \end{itemize}

    \item \xmlNode{CV}: \xmlDesc{string}, 
      The text portion of this node needs to contain the name of the \xmlNode{PostProcessor} with
      \xmlAttr{subType}         ``CrossValidation``.
      The \xmlNode{CV} node recognizes the following parameters:
        \begin{itemize}
          \item \xmlAttr{class}: \xmlDesc{string, optional}, 
            should be set to \xmlString{Model}
          \item \xmlAttr{type}: \xmlDesc{string, optional}, 
            should be set to \xmlString{PostProcessor}
      \end{itemize}

    \item \xmlNode{alias}: \xmlDesc{string}, 
      specifies alias for         any variable of interest in the input or output space. These
      aliases can be used anywhere in the RAVEN input to         refer to the variables. In the body
      of this node the user specifies the name of the variable that the model is going to use
      (during its execution).
      The \xmlNode{alias} node recognizes the following parameters:
        \begin{itemize}
          \item \xmlAttr{variable}: \xmlDesc{string, required}, 
            define the actual alias, usable throughout the RAVEN input
          \item \xmlAttr{type}: \xmlDesc{[input, output], required}, 
            either ``input'' or ``output''.
      \end{itemize}

    \item \xmlNode{persistence}: \xmlDesc{string}, 
      specifies how                                                  to define the hierarchical
      simplification by assigning a value to each local
      minimum and maximum according to the one of the strategy options below:
      \begin{itemize}                                                    \item \texttt{difference} -
      The function value difference between the
      extremum and its closest-valued neighboring saddle.
      \item \texttt{probability} - The probability integral computed as the
      sum of the probability of each point in a cluster divided by the count of
      the cluster.                                                    \item \texttt{count} - The
      count of points that flow to or from the
      extremum.                                                  \end{itemize}
  \default{difference}

    \item \xmlNode{gradient}: \xmlDesc{string}, 
      specifies the                                                  method used for estimating the
      gradient, available options are:
      \begin{itemize}                                                    \item \texttt{steepest}
      \end{itemize}
  \default{steepest}

    \item \xmlNode{simplification}: \xmlDesc{float}, 
      specifies the                                                  amount of noise reduction to
      apply before returning labels.
  \default{0}

    \item \xmlNode{graph}: \xmlDesc{string}, 
      specifies the type                                                  of neighborhood graph used
      in the algorithm, available options are:
      \begin{itemize}                                                    \item \texttt{beta
      skeleton}                                                    \item \texttt{relaxed beta
      skeleton}                                                    \item \texttt{approximate knn}
      \end{itemize}
  \default{beta skeleton}

    \item \xmlNode{beta}: \xmlDesc{float}, 
      in range: $(0, 2])$. It is                                                  only used when the
      \xmlNode{graph} is set to \texttt{beta skeleton} or
      \texttt{relaxed beta skeleton}.
  \default{1.0}

    \item \xmlNode{knn}: \xmlDesc{integer}, 
      is the number of                                                  neighbors when using the
      \texttt{approximate knn} for the \xmlNode{graph}
      sub-node and used to speed up the computation of other graphs by using the
      approximate knn graph as a starting point for pruning. -1 means use a fully
      connected graph.
  \default{-1}

    \item \xmlNode{weighted}: \xmlDesc{[True, Yes, 1, False, No, 0, t, y, 1, f, n, 0]}, 
      a flag that specifies                                                  whether the regression
      models should be probability weighted.
  \default{False}

    \item \xmlNode{partitionPredictor}: \xmlDesc{string}, 
      a flag that                                                  specifies how the predictions for
      query point categorization  should be
      performed. Available options are:
      \begin{itemize}                                                    \item \texttt{kde}
      \item \texttt{svm}                                                  \end{itemize}
  \default{kde}

    \item \xmlNode{smooth}: \xmlDesc{[True, Yes, 1, False, No, 0, t, y, 1, f, n, 0]}, 
      if this node is present, the ROM will blend the
      estimates of all of the local linear models weighted by the probability the
      query point is categorized as belonging to that partition of the input space.
  \default{False}

    \item \xmlNode{kernel}: \xmlDesc{string}, 
      this option is only                                                  used when the
      \xmlNode{partitionPredictor} is set to \texttt{kde} and
      specifies the type of kernel to use in the kernel density estimation.
      Available options are:                                                  \begin{itemize}
      \item \texttt{uniform}                                                    \item
      \texttt{triangular}                                                    \item \texttt{gaussian}
      \item \texttt{epanechnikov}                                                    \item
      \texttt{biweight} or \texttt{quartic}                                                    \item
      \texttt{triweight}                                                    \item \texttt{tricube}
      \item \texttt{cosine}                                                    \item
      \texttt{logistic}                                                    \item \texttt{silverman}
      \item \texttt{exponential}                                                  \end{itemize}
  \default{gaussian}

    \item \xmlNode{bandwidth}: \xmlDesc{float or string}, 
      this                                                  option is only used when the
      \xmlNode{partitionPredictor} is set to
      \texttt{kde} and specifies the scale of the fall-off. A higher bandwidth
      implies a smooother blending. If set to \texttt{variable}, then the bandwidth
      will be set to the distance of the $k$-nearest neighbor of the query point
      where $k$ is set by the \xmlNode{knn} parameter.
  \default{1.0}
  \end{itemize}

\hspace{24pt}
Example:
\begin{lstlisting}[style=XML,morekeywords={name,subType}]
<Simulation>
  ...
  <Models>
    ...
    </ROM>
    <ROM name='aUserDefinedName' subType='MSR'>
       <Features>var1,var2,var3</Features>
       <Target>result1,result2</Target>
       <!-- <weighted>true</weighted> -->
       <simplification>0.0</simplification>
       <persistence>difference</persistence>
       <gradient>steepest</gradient>
       <graph>beta skeleton</graph>
       <beta>1</beta>
       <knn>8</knn>
       <partitionPredictor>kde</partitionPredictor>
       <kernel>gaussian</kernel>
       <smooth/>
       <bandwidth>0.2</bandwidth>
     </ROM>
    ...
  </Models>
  ...
</Simulation>
\end{lstlisting}


\subsubsection{NDinvDistWeight}
  The \xmlNode{NDinvDistWeight} is based on an                             $N$-dimensional inverse
  distance weighting formulation.                             Inverse distance weighting (IDW) is a
  type of deterministic method for                             multivariate interpolation with a
  known scattered set of points.                             The assigned values to unknown points
  are calculated via a weighted average of                             the values available at the
  known points.                             \\
  \zNormalizationPerformed{NDinvDistWeight}                             \\
  In order to use this Reduced Order Model, the \xmlNode{ROM} attribute
  \xmlAttr{subType} needs to be \xmlString{NDinvDistWeight}.

  The \xmlNode{NDinvDistWeight} node recognizes the following parameters:
    \begin{itemize}
      \item \xmlAttr{name}: \xmlDesc{string, required}, 
        User-defined name to designate this entity in the RAVEN input file.
      \item \xmlAttr{verbosity}: \xmlDesc{[silent, quiet, all, debug], optional}, 
        Desired verbosity of messages coming from this entity
      \item \xmlAttr{subType}: \xmlDesc{string, required}, 
        specify the type of ROM that will be used
  \end{itemize}

  The \xmlNode{NDinvDistWeight} node recognizes the following subnodes:
  \begin{itemize}
    \item \xmlNode{Features}: \xmlDesc{comma-separated strings}, 
      specifies the names of the features of this ROM.         \nb These parameters are going to be
      requested for the training of this object         (see Section~\ref{subsec:stepRomTrainer})

    \item \xmlNode{Target}: \xmlDesc{comma-separated strings}, 
      contains a comma separated list of the targets of this ROM. These parameters         are the
      Figures of Merit (FOMs) this ROM is supposed to predict.         \nb These parameters are
      going to be requested for the training of this         object (see Section
      \ref{subsec:stepRomTrainer}).

    \item \xmlNode{pivotParameter}: \xmlDesc{string}, 
      If a time-dependent ROM is requested, please specifies the pivot         variable (e.g. time,
      etc) used in the input HistorySet.
  \default{time}

    \item \xmlNode{featureSelection}:
      Apply feature selection algorithm

      The \xmlNode{featureSelection} node recognizes the following subnodes:
      \begin{itemize}
        \item \xmlNode{RFE}:
          The \xmlString{RFE} (Recursive Feature Elimination) is a feature selection algorithm.
          Feature selection refers to techniques that select a subset of the most relevant features
          for a model (ROM).         Fewer features can allow ROMs to run more efficiently (less
          space or time complexity) and be more effective.         Indeed, some ROMs (machine
          learning algorithms) can be misled by irrelevant input features, resulting in worse
          predictive performance.         RFE is a wrapper-type feature selection algorithm. This
          means that a different ROM is given and used in the core of the         method,         is
          wrapped by RFE, and used to help select features.         \\RFE works by searching for a
          subset of features by starting with all features in the training dataset and successfully
          removing         features until the desired number remains.         This is achieved by
          fitting the given ROM used in the core of the model, ranking features by importance,
          discarding the least important features, and re-fitting the model. This process is
          repeated until a specified number of         features remains.         When the full model
          is created, a measure of variable importance is computed that ranks the predictors from
          most         important to least.         At each stage of the search, the least important
          predictors are iteratively eliminated prior to rebuilding the model.         Features are
          scored either using the ROM model (if the model provides a mean to compute feature
          importances) or by         using a statistical method.         \\In RAVEN the
          \xmlString{RFE} class refers to an augmentation of the basic algorithm, since it allows,
          optionally,         to perform the search on multiple groups of targets (separately) and
          then combine the results of the search in a         single set. In addition, when the RFE
          search is concluded, the user can request to identify the set of features         that
          bring to a minimization of the score (i.e. maximimization of the accuracy).         In
          addition, using the ``applyClusteringFiltering'' option, the algorithm can, using an
          hierarchal clustering algorithm,         identify highly correlated features to speed up
          the subsequential search.
          The \xmlNode{RFE} node recognizes the following parameters:
            \begin{itemize}
              \item \xmlAttr{name}: \xmlDesc{string, required}, 
                User-defined name to designate this entity in the RAVEN input file.
              \item \xmlAttr{verbosity}: \xmlDesc{[silent, quiet, all, debug], optional}, 
                Desired verbosity of messages coming from this entity
          \end{itemize}

          The \xmlNode{RFE} node recognizes the following subnodes:
          \begin{itemize}
            \item \xmlNode{parametersToInclude}: \xmlDesc{comma-separated strings}, 
              List of IDs of features/variables to include in the search.
  \default{None}

            \item \xmlNode{whichSpace}: \xmlDesc{[Feature, feature, Target, target]}, 
              Which space to search? Target or Feature (this is temporary till DataSet training is
              implemented)
  \default{feature}

            \item \xmlNode{nFeaturesToSelect}: \xmlDesc{integer}, 
              Exact Number of features to select. If not inputted, ``nFeaturesToSelect'' will be set
              to $1/2$ of the features in the training dataset.
  \default{None}

            \item \xmlNode{maxNumberFeatures}: \xmlDesc{integer}, 
              Maximum Number of features to select, the algorithm will automatically determine the
              feature list to minimize a total score.
  \default{None}

            \item \xmlNode{onlyOutputScore}: \xmlDesc{[True, Yes, 1, False, No, 0, t, y, 1, f, n, 0]}, 
              If maxNumberFeatures is on, only output score should beconsidered? Or, in case of
              particular models (e.g. DMDC), state variable space score should be considered as
              well.
  \default{False}

            \item \xmlNode{applyClusteringFiltering}: \xmlDesc{[True, Yes, 1, False, No, 0, t, y, 1, f, n, 0]}, 
              Applying clustering correlation before RFE search? If true, an hierarchal clustering
              is applied on the feature         space aimed to remove features that are correlated
              before the actual RFE search is performed. This approach can stabilize and
              accelerate the process in case of large feature spaces (e.g > 500 features).
  \default{False}

            \item \xmlNode{applyCrossCorrelation}: \xmlDesc{[True, Yes, 1, False, No, 0, t, y, 1, f, n, 0]}, 
              In case of subgroupping, should a cross correleation analysis should be performed
              cross sub-groups?         If it is activated, a cross correleation analysis is used to
              additionally filter the features selected for each         sub-groupping search.
  \default{False}

            \item \xmlNode{step}: \xmlDesc{float}, 
              If greater than or equal to 1, then step corresponds to the (integer) number
              of features to remove at each iteration. If within (0.0, 1.0), then step
              corresponds to the percentage (rounded down) of features to remove at         each
              iteration.
  \default{1}

            \item \xmlNode{subGroup}: \xmlDesc{comma-separated strings, integers, and floats}, 
              Subgroup of output variables on which to perform the search. Multiple nodes of this
              type can be inputted. The RFE search will be then performed in each ``subgroup''
              separately and then the the union of the different feature sets are used for the final
              ROM.
          \end{itemize}

        \item \xmlNode{VarianceThreshold}:
          The \xmlString{VarianceThreshold} is a feature selector that removes     all low-variance
          features. This feature selection algorithm looks only at the features and not     the
          desired outputs. The variance threshold can be set by the user.
          The \xmlNode{VarianceThreshold} node recognizes the following parameters:
            \begin{itemize}
              \item \xmlAttr{name}: \xmlDesc{string, required}, 
                User-defined name to designate this entity in the RAVEN input file.
              \item \xmlAttr{verbosity}: \xmlDesc{[silent, quiet, all, debug], optional}, 
                Desired verbosity of messages coming from this entity
          \end{itemize}

          The \xmlNode{VarianceThreshold} node recognizes the following subnodes:
          \begin{itemize}
            \item \xmlNode{parametersToInclude}: \xmlDesc{comma-separated strings}, 
              List of IDs of features/variables to include in the search.
  \default{None}

            \item \xmlNode{whichSpace}: \xmlDesc{[Feature, feature, Target, target]}, 
              Which space to search? Target or Feature (this is temporary till DataSet training is
              implemented)
  \default{feature}

            \item \xmlNode{threshold}: \xmlDesc{float}, 
              Features with a training-set variance lower than this threshold                   will
              be removed. The default is to keep all features with non-zero
              variance, i.e. remove the features that have the same value in all
              samples.
  \default{0.0}
          \end{itemize}
      \end{itemize}

    \item \xmlNode{featureSpaceTransformation}:
      Use dimensionality reduction technique to perform a trasformation of the training dataset
      into an uncorrelated one. The dimensionality of the problem will not be reduced but
      the data will be transformed in the transformed space. E.g if the number of features
      are 5, the method projects such features into a new uncorrelated space (still 5-dimensional).
      In case of time-dependent ROMs, all the samples are concatenated in a global 2D matrix
      (n\_samples*n\_timesteps,n\_features) before applying the transformation and then reconstructed
      back into the original shape (before fitting the model).

      The \xmlNode{featureSpaceTransformation} node recognizes the following subnodes:
      \begin{itemize}
        \item \xmlNode{transformationMethod}: \xmlDesc{[PCA, KernelLinearPCA, KernelPolyPCA, KernelRbfPCA, KernelSigmoidPCA, KernelCosinePCA, ICA]}, 
          Transformation method to use. Eight options (5 Kernel PCAs) are available:
          \begin{itemize}                     \item \textit{PCA}, Principal Component Analysis;
          \item \textit{KernelLinearPCA}, Kernel (Linear) Principal component analysis;
          \item \textit{KernelPolyPCA}, Kernel (Poly) Principal component analysis;
          \item \textit{KernelRbfPCA}, Kernel(Rbf) Principal component analysis;
          \item \textit{KernelSigmoidPCA}, Kernel (Sigmoid) Principal component analysis;
          \item \textit{KernelCosinePCA}, Kernel (Cosine) Principal component analysis;
          \item \textit{ICA}, Independent component analysis;                    \end{itemize}
  \default{PCA}

        \item \xmlNode{parametersToInclude}: \xmlDesc{comma-separated strings}, 
          List of IDs of features/variables to include in the transformation process.
  \default{None}

        \item \xmlNode{whichSpace}: \xmlDesc{[Feature, feature, Target, target]}, 
          Which space to search? Target or Feature?
  \default{Feature}
      \end{itemize}

    \item \xmlNode{CV}: \xmlDesc{string}, 
      The text portion of this node needs to contain the name of the \xmlNode{PostProcessor} with
      \xmlAttr{subType}         ``CrossValidation``.
      The \xmlNode{CV} node recognizes the following parameters:
        \begin{itemize}
          \item \xmlAttr{class}: \xmlDesc{string, optional}, 
            should be set to \xmlString{Model}
          \item \xmlAttr{type}: \xmlDesc{string, optional}, 
            should be set to \xmlString{PostProcessor}
      \end{itemize}

    \item \xmlNode{alias}: \xmlDesc{string}, 
      specifies alias for         any variable of interest in the input or output space. These
      aliases can be used anywhere in the RAVEN input to         refer to the variables. In the body
      of this node the user specifies the name of the variable that the model is going to use
      (during its execution).
      The \xmlNode{alias} node recognizes the following parameters:
        \begin{itemize}
          \item \xmlAttr{variable}: \xmlDesc{string, required}, 
            define the actual alias, usable throughout the RAVEN input
          \item \xmlAttr{type}: \xmlDesc{[input, output], required}, 
            either ``input'' or ``output''.
      \end{itemize}

    \item \xmlNode{p}: \xmlDesc{integer}, 
      must be greater than zero and represents the ``power parameter''.
      For the choice of value for \xmlNode{p},it is necessary to consider the degree
      of smoothing desired in the interpolation/extrapolation, the density and
      distribution of samples being interpolated, and the maximum distance over
      which an individual sample is allowed to influence the surrounding ones (lower
      $p$ means greater importance for points far away).
  \end{itemize}

\hspace{24pt}
Example:
\begin{lstlisting}[style=XML,morekeywords={name,subType}]
<Simulation>
  ...
  <Models>
    ...
    <ROM name='aUserDefinedName' subType='NDinvDistWeight'>
      <Features>var1,var2,var3</Features>
      <Target>result1,result2</Target>
      <p>3</p>
     </ROM>
    ...
  </Models>
  ...
</Simulation>
\end{lstlisting}


\subsubsection{SyntheticHistory}
  A ROM for characterizing and generating synthetic histories. This ROM makes use of         a
  variety of TimeSeriesAnalysis (TSA) algorithms to characterize and generate new         signals
  based on training signal sets. It is a more general implementation of the ARMA ROM. The available
  algorithms are discussed in more detail below. The SyntheticHistory ROM uses the TSA algorithms to
  characterize then reproduce time series in sequence; for example, if using Fourier then ARMA, the
  SyntheticHistory ROM will characterize the Fourier properties using the Fourier TSA algorithm on a
  training signal, then send the residual to the ARMA TSA algorithm for characterization. Generating
  new signals works in reverse, first generating a signal using the ARMA TSA algorithm then
  superimposing the Fourier TSA algorithm.         //         In order to use this Reduced Order
  Model, the \xmlNode{ROM} attribute         \xmlAttr{subType} needs to be
  \xmlString{SyntheticHistory}.

  The \xmlNode{SyntheticHistory} node recognizes the following parameters:
    \begin{itemize}
      \item \xmlAttr{name}: \xmlDesc{string, required}, 
        User-defined name to designate this entity in the RAVEN input file.
      \item \xmlAttr{verbosity}: \xmlDesc{[silent, quiet, all, debug], optional}, 
        Desired verbosity of messages coming from this entity
      \item \xmlAttr{subType}: \xmlDesc{string, required}, 
        specify the type of ROM that will be used
  \end{itemize}

  The \xmlNode{SyntheticHistory} node recognizes the following subnodes:
  \begin{itemize}
    \item \xmlNode{Features}: \xmlDesc{comma-separated strings}, 
      specifies the names of the features of this ROM.         \nb These parameters are going to be
      requested for the training of this object         (see Section~\ref{subsec:stepRomTrainer})

    \item \xmlNode{Target}: \xmlDesc{comma-separated strings}, 
      contains a comma separated list of the targets of this ROM. These parameters         are the
      Figures of Merit (FOMs) this ROM is supposed to predict.         \nb These parameters are
      going to be requested for the training of this         object (see Section
      \ref{subsec:stepRomTrainer}).

    \item \xmlNode{pivotParameter}: \xmlDesc{string}, 
      If a time-dependent ROM is requested, please specifies the pivot         variable (e.g. time,
      etc) used in the input HistorySet.
  \default{time}

    \item \xmlNode{featureSelection}:
      Apply feature selection algorithm

      The \xmlNode{featureSelection} node recognizes the following subnodes:
      \begin{itemize}
        \item \xmlNode{RFE}:
          The \xmlString{RFE} (Recursive Feature Elimination) is a feature selection algorithm.
          Feature selection refers to techniques that select a subset of the most relevant features
          for a model (ROM).         Fewer features can allow ROMs to run more efficiently (less
          space or time complexity) and be more effective.         Indeed, some ROMs (machine
          learning algorithms) can be misled by irrelevant input features, resulting in worse
          predictive performance.         RFE is a wrapper-type feature selection algorithm. This
          means that a different ROM is given and used in the core of the         method,         is
          wrapped by RFE, and used to help select features.         \\RFE works by searching for a
          subset of features by starting with all features in the training dataset and successfully
          removing         features until the desired number remains.         This is achieved by
          fitting the given ROM used in the core of the model, ranking features by importance,
          discarding the least important features, and re-fitting the model. This process is
          repeated until a specified number of         features remains.         When the full model
          is created, a measure of variable importance is computed that ranks the predictors from
          most         important to least.         At each stage of the search, the least important
          predictors are iteratively eliminated prior to rebuilding the model.         Features are
          scored either using the ROM model (if the model provides a mean to compute feature
          importances) or by         using a statistical method.         \\In RAVEN the
          \xmlString{RFE} class refers to an augmentation of the basic algorithm, since it allows,
          optionally,         to perform the search on multiple groups of targets (separately) and
          then combine the results of the search in a         single set. In addition, when the RFE
          search is concluded, the user can request to identify the set of features         that
          bring to a minimization of the score (i.e. maximimization of the accuracy).         In
          addition, using the ``applyClusteringFiltering'' option, the algorithm can, using an
          hierarchal clustering algorithm,         identify highly correlated features to speed up
          the subsequential search.
          The \xmlNode{RFE} node recognizes the following parameters:
            \begin{itemize}
              \item \xmlAttr{name}: \xmlDesc{string, required}, 
                User-defined name to designate this entity in the RAVEN input file.
              \item \xmlAttr{verbosity}: \xmlDesc{[silent, quiet, all, debug], optional}, 
                Desired verbosity of messages coming from this entity
          \end{itemize}

          The \xmlNode{RFE} node recognizes the following subnodes:
          \begin{itemize}
            \item \xmlNode{parametersToInclude}: \xmlDesc{comma-separated strings}, 
              List of IDs of features/variables to include in the search.
  \default{None}

            \item \xmlNode{whichSpace}: \xmlDesc{[Feature, feature, Target, target]}, 
              Which space to search? Target or Feature (this is temporary till DataSet training is
              implemented)
  \default{feature}

            \item \xmlNode{nFeaturesToSelect}: \xmlDesc{integer}, 
              Exact Number of features to select. If not inputted, ``nFeaturesToSelect'' will be set
              to $1/2$ of the features in the training dataset.
  \default{None}

            \item \xmlNode{maxNumberFeatures}: \xmlDesc{integer}, 
              Maximum Number of features to select, the algorithm will automatically determine the
              feature list to minimize a total score.
  \default{None}

            \item \xmlNode{onlyOutputScore}: \xmlDesc{[True, Yes, 1, False, No, 0, t, y, 1, f, n, 0]}, 
              If maxNumberFeatures is on, only output score should beconsidered? Or, in case of
              particular models (e.g. DMDC), state variable space score should be considered as
              well.
  \default{False}

            \item \xmlNode{applyClusteringFiltering}: \xmlDesc{[True, Yes, 1, False, No, 0, t, y, 1, f, n, 0]}, 
              Applying clustering correlation before RFE search? If true, an hierarchal clustering
              is applied on the feature         space aimed to remove features that are correlated
              before the actual RFE search is performed. This approach can stabilize and
              accelerate the process in case of large feature spaces (e.g > 500 features).
  \default{False}

            \item \xmlNode{applyCrossCorrelation}: \xmlDesc{[True, Yes, 1, False, No, 0, t, y, 1, f, n, 0]}, 
              In case of subgroupping, should a cross correleation analysis should be performed
              cross sub-groups?         If it is activated, a cross correleation analysis is used to
              additionally filter the features selected for each         sub-groupping search.
  \default{False}

            \item \xmlNode{step}: \xmlDesc{float}, 
              If greater than or equal to 1, then step corresponds to the (integer) number
              of features to remove at each iteration. If within (0.0, 1.0), then step
              corresponds to the percentage (rounded down) of features to remove at         each
              iteration.
  \default{1}

            \item \xmlNode{subGroup}: \xmlDesc{comma-separated strings, integers, and floats}, 
              Subgroup of output variables on which to perform the search. Multiple nodes of this
              type can be inputted. The RFE search will be then performed in each ``subgroup''
              separately and then the the union of the different feature sets are used for the final
              ROM.
          \end{itemize}

        \item \xmlNode{VarianceThreshold}:
          The \xmlString{VarianceThreshold} is a feature selector that removes     all low-variance
          features. This feature selection algorithm looks only at the features and not     the
          desired outputs. The variance threshold can be set by the user.
          The \xmlNode{VarianceThreshold} node recognizes the following parameters:
            \begin{itemize}
              \item \xmlAttr{name}: \xmlDesc{string, required}, 
                User-defined name to designate this entity in the RAVEN input file.
              \item \xmlAttr{verbosity}: \xmlDesc{[silent, quiet, all, debug], optional}, 
                Desired verbosity of messages coming from this entity
          \end{itemize}

          The \xmlNode{VarianceThreshold} node recognizes the following subnodes:
          \begin{itemize}
            \item \xmlNode{parametersToInclude}: \xmlDesc{comma-separated strings}, 
              List of IDs of features/variables to include in the search.
  \default{None}

            \item \xmlNode{whichSpace}: \xmlDesc{[Feature, feature, Target, target]}, 
              Which space to search? Target or Feature (this is temporary till DataSet training is
              implemented)
  \default{feature}

            \item \xmlNode{threshold}: \xmlDesc{float}, 
              Features with a training-set variance lower than this threshold                   will
              be removed. The default is to keep all features with non-zero
              variance, i.e. remove the features that have the same value in all
              samples.
  \default{0.0}
          \end{itemize}
      \end{itemize}

    \item \xmlNode{featureSpaceTransformation}:
      Use dimensionality reduction technique to perform a trasformation of the training dataset
      into an uncorrelated one. The dimensionality of the problem will not be reduced but
      the data will be transformed in the transformed space. E.g if the number of features
      are 5, the method projects such features into a new uncorrelated space (still 5-dimensional).
      In case of time-dependent ROMs, all the samples are concatenated in a global 2D matrix
      (n\_samples*n\_timesteps,n\_features) before applying the transformation and then reconstructed
      back into the original shape (before fitting the model).

      The \xmlNode{featureSpaceTransformation} node recognizes the following subnodes:
      \begin{itemize}
        \item \xmlNode{transformationMethod}: \xmlDesc{[PCA, KernelLinearPCA, KernelPolyPCA, KernelRbfPCA, KernelSigmoidPCA, KernelCosinePCA, ICA]}, 
          Transformation method to use. Eight options (5 Kernel PCAs) are available:
          \begin{itemize}                     \item \textit{PCA}, Principal Component Analysis;
          \item \textit{KernelLinearPCA}, Kernel (Linear) Principal component analysis;
          \item \textit{KernelPolyPCA}, Kernel (Poly) Principal component analysis;
          \item \textit{KernelRbfPCA}, Kernel(Rbf) Principal component analysis;
          \item \textit{KernelSigmoidPCA}, Kernel (Sigmoid) Principal component analysis;
          \item \textit{KernelCosinePCA}, Kernel (Cosine) Principal component analysis;
          \item \textit{ICA}, Independent component analysis;                    \end{itemize}
  \default{PCA}

        \item \xmlNode{parametersToInclude}: \xmlDesc{comma-separated strings}, 
          List of IDs of features/variables to include in the transformation process.
  \default{None}

        \item \xmlNode{whichSpace}: \xmlDesc{[Feature, feature, Target, target]}, 
          Which space to search? Target or Feature?
  \default{Feature}
      \end{itemize}

    \item \xmlNode{CV}: \xmlDesc{string}, 
      The text portion of this node needs to contain the name of the \xmlNode{PostProcessor} with
      \xmlAttr{subType}         ``CrossValidation``.
      The \xmlNode{CV} node recognizes the following parameters:
        \begin{itemize}
          \item \xmlAttr{class}: \xmlDesc{string, optional}, 
            should be set to \xmlString{Model}
          \item \xmlAttr{type}: \xmlDesc{string, optional}, 
            should be set to \xmlString{PostProcessor}
      \end{itemize}

    \item \xmlNode{alias}: \xmlDesc{string}, 
      specifies alias for         any variable of interest in the input or output space. These
      aliases can be used anywhere in the RAVEN input to         refer to the variables. In the body
      of this node the user specifies the name of the variable that the model is going to use
      (during its execution).
      The \xmlNode{alias} node recognizes the following parameters:
        \begin{itemize}
          \item \xmlAttr{variable}: \xmlDesc{string, required}, 
            define the actual alias, usable throughout the RAVEN input
          \item \xmlAttr{type}: \xmlDesc{[input, output], required}, 
            either ``input'' or ``output''.
      \end{itemize}

    \item \xmlNode{pivotParameter}: \xmlDesc{string}, 
      If a time-dependent ROM is requested, please specifies the pivot         variable (e.g. time,
      etc) used in the input HistorySet.
  \default{time}

    \item \xmlNode{fourier}:
      TimeSeriesAnalysis algorithm for determining the strength and phase of
      specified Fourier periods within training signals. The Fourier signals take
      the form $C\sin(\frac{2\pi}{k}+\phi)$, where $C$ is the calculated strength
      or amplitude, $k$ is the user-specified period(s) to search for, and $\phi$
      is the calculated phase shift. The resulting characterization and synthetic
      history generation is deterministic given a single training signal.
      The \xmlNode{fourier} node recognizes the following parameters:
        \begin{itemize}
          \item \xmlAttr{target}: \xmlDesc{comma-separated strings, required}, 
            indicates the variables for which this algorithm will be used for characterization.
          \item \xmlAttr{seed}: \xmlDesc{integer, optional}, 
            sets a seed for the underlying random number generator, if present.
      \end{itemize}

      The \xmlNode{fourier} node recognizes the following subnodes:
      \begin{itemize}
        \item \xmlNode{periods}: \xmlDesc{comma-separated floats}, 
          Specifies the periods (inverse of frequencies) that should be searched
          for within the training signal.
      \end{itemize}

    \item \xmlNode{arma}:
      characterizes the signal using Auto-Regressive and Moving Average         coefficients to
      stochastically fit the training signal.         The ARMA representation has the following
      form:         \begin{equation*}           A\_t = \sum\_{i=1}^P \phi\_i A\_{t-i} + \epsilon\_t +
      \sum\_{j=1}^Q \theta\_j \epsilon\_{t-j},         \end{equation*}         where $t$ indicates a
      discrete time step, $\phi$ are the signal lag (or auto-regressive)         coefficients, $P$
      is the number of signal lag terms to consider, $\epsilon$ is a random noise         term,
      $\theta$ are the noise lag (or moving average) coefficients, and $Q$ is the number of
      noise lag terms to consider. The ARMA algorithms are developed in RAVEN using the
      \texttt{statsmodels} Python library.
      The \xmlNode{arma} node recognizes the following parameters:
        \begin{itemize}
          \item \xmlAttr{target}: \xmlDesc{comma-separated strings, required}, 
            indicates the variables for which this algorithm will be used for characterization.
          \item \xmlAttr{seed}: \xmlDesc{integer, optional}, 
            sets a seed for the underlying random number generator, if present.
          \item \xmlAttr{reduce\_memory}: \xmlDesc{[True, Yes, 1, False, No, 0, t, y, 1, f, n, 0], optional}, 
            activates a lower memory usage ARMA training. This does tend to result
            in a slightly slower training time, at the benefit of lower memory usage. For
            example, in one 1000-length history test, low memory reduced memory usage by 2.3
            MiB, but increased training time by 0.4 seconds. No change in results has been
            observed switching between modes. Note that the ARMA must be
            retrained to change this property; it cannot be applied to serialized ARMAs. \default{False}
          \item \xmlAttr{gaussianize}: \xmlDesc{[True, Yes, 1, False, No, 0, t, y, 1, f, n, 0], optional}, 
            activates a transformation of the signal to a normal distribution before
            training. This is done by fitting a CDF to the data and then transforming the
            data to a normal distribution using the CDF. The CDF is saved and used during
            sampling to back-transform the data to the original distribution. This is
            recommended for non-normal data, but is not required. Note that the ARMA must be
            retrained to change this property; it cannot be applied to serialized ARMAs.
            Note: New models wishing to apply this transformation should use a
            \xmlNode{gaussianize} node preceding the \xmlNode{arma} node instead of this
            option. \default{False}
      \end{itemize}

      The \xmlNode{arma} node recognizes the following subnodes:
      \begin{itemize}
        \item \xmlNode{SignalLag}: \xmlDesc{integer}, 
          the number of terms in the AutoRegressive term to retain in the
          regression; typically represented as $P$ in literature.

        \item \xmlNode{NoiseLag}: \xmlDesc{integer}, 
          the number of terms in the Moving Average term to retain in the
          regression; typically represented as $Q$ in literature.
      \end{itemize}

    \item \xmlNode{varma}:
      characterizes the vector-valued signal using Auto-Regressive and Moving         Average
      coefficients to stochastically fit the training signal.         The VARMA representation has
      the following form:         \begin{equation*}           A\_t = \sum\_{i=1}^P \phi\_i A\_{t-i} +
      \epsilon\_t + \sum\_{j=1}^Q \theta\_j \epsilon\_{t-j},         \end{equation*}         where $t$
      indicates a discrete time step, $\phi$ are the signal lag (or auto-regressive)
      coefficients, $P$ is the number of signal lag terms to consider, $\epsilon$ is a random noise
      term, $\theta$ are the noise lag (or moving average) coefficients, and $Q$ is the number of
      noise lag terms to consider. For signal $A\_t$ which is a $k \times 1$ vector, each $\phi\_i$
      and $\theta\_j$ are $k \times k$ matrices, and $\epsilon\_t$ is characterized by the         $k
      \times k$ covariance matrix $\Sigma$. The VARMA algorithms are developed in RAVEN using the
      \texttt{statsmodels} Python library.
      The \xmlNode{varma} node recognizes the following parameters:
        \begin{itemize}
          \item \xmlAttr{target}: \xmlDesc{comma-separated strings, required}, 
            indicates the variables for which this algorithm will be used for characterization.
          \item \xmlAttr{seed}: \xmlDesc{integer, optional}, 
            sets a seed for the underlying random number generator, if present.
      \end{itemize}

      The \xmlNode{varma} node recognizes the following subnodes:
      \begin{itemize}
        \item \xmlNode{P}: \xmlDesc{integer}, 
          the number of terms in the AutoRegressive term to retain in the
          regression; typically represented as $P$ in literature.

        \item \xmlNode{Q}: \xmlDesc{integer}, 
          the number of terms in the Moving Average term to retain in the
          regression; typically represented as $Q$ in literature.
      \end{itemize}

    \item \xmlNode{MarkovAR}:
      characterizes the signal using autoregressive (AR) coefficients conditioned         on the
      state of a hidden Markov model (HMM) to stochastically fit the training signal.         The
      Markov-switching autoregressive model (MSAR) has the following form:         \begin{equation*}
      Y\_t = \mu\_{S\_t} \sum\_{i=1}^p \phi\_{i,{S\_t}} \left(Y\_{t-i} - \mu\_{S\_{t-i}}\right) +
      \varepsilon\_{t,{S\_t}},         \end{equation*}         where $t$ indicates a discrete time
      step, $\phi$ are the signal lag (or auto-regressive)         coefficients, $p$ is the number
      of signal lag terms to consider, $\varepsilon$ is a random noise         term with mean 0 and
      variance $\sigma^2\_{S\_t}$, and $S\_t$ is the HMM state at time $t$.         The HMM state is
      determined by the transition probabilities between states, which are conditioned         on
      the previous state. The transition probabilities are stored in a transition matrix $P$,
      where entry $p\_{ij}$ is the probability of transitioning from state $i$ to state $j$
      conditional         on being in state $i$. For a MSAR model with HMM state dimensionality $r$,
      the transition matrix         $P$ is of size $r \times r$. Each of the mean, autoregressive,
      and noise variance terms may be         switching or non-switching parameters.
      The \xmlNode{MarkovAR} node recognizes the following parameters:
        \begin{itemize}
          \item \xmlAttr{target}: \xmlDesc{comma-separated strings, required}, 
            indicates the variables for which this algorithm will be used for characterization.
          \item \xmlAttr{seed}: \xmlDesc{integer, optional}, 
            sets a seed for the underlying random number generator, if present.
          \item \xmlAttr{switching\_ar}: \xmlDesc{[True, Yes, 1, False, No, 0, t, y, 1, f, n, 0], optional}, 
            indicates whether the autoregressive coefficients are switching parameters. \default{True}
          \item \xmlAttr{switching\_variance}: \xmlDesc{[True, Yes, 1, False, No, 0, t, y, 1, f, n, 0], optional}, 
            indicates whether the noise variance is a switching parameter. \default{True}
          \item \xmlAttr{switching\_trend}: \xmlDesc{[True, Yes, 1, False, No, 0, t, y, 1, f, n, 0], optional}, 
            indicates whether the mean is a switching parameter. \default{True}
      \end{itemize}

      The \xmlNode{MarkovAR} node recognizes the following subnodes:
      \begin{itemize}
        \item \xmlNode{P}: \xmlDesc{integer}, 
          the number of terms in the AutoRegressive term to retain in the
          regression; typically represented as $P$ in literature.

        \item \xmlNode{MarkovStates}: \xmlDesc{integer}, 
          the number of states in the hidden Markov model.
      \end{itemize}

    \item \xmlNode{STL}:
       Decomposes the signal into trend, seasonal, and residual components using the STL method of
      Cleveland et al. (1990).
      The \xmlNode{STL} node recognizes the following parameters:
        \begin{itemize}
          \item \xmlAttr{target}: \xmlDesc{comma-separated strings, required}, 
            indicates the variables for which this algorithm will be used for characterization.
          \item \xmlAttr{seed}: \xmlDesc{integer, optional}, 
            sets a seed for the underlying random number generator, if present.
      \end{itemize}

      The \xmlNode{STL} node recognizes the following subnodes:
      \begin{itemize}
        \item \xmlNode{seasonal}: \xmlDesc{integer}, 
          the length of the seasonal smoother.

        \item \xmlNode{period}: \xmlDesc{integer}, 
          periodicity of the sequence.

        \item \xmlNode{trend}: \xmlDesc{integer}, 
          the length of the trend smoother. Must be an odd integer.
      \end{itemize}

    \item \xmlNode{wavelet}:
      Discrete Wavelet TimeSeriesAnalysis algorithm. Performs a discrete wavelet transform
      on time-dependent data. Note: This TSA module requires pywavelets to be installed within your
      python environment.
      The \xmlNode{wavelet} node recognizes the following parameters:
        \begin{itemize}
          \item \xmlAttr{target}: \xmlDesc{comma-separated strings, required}, 
            indicates the variables for which this algorithm will be used for characterization.
          \item \xmlAttr{seed}: \xmlDesc{integer, optional}, 
            sets a seed for the underlying random number generator, if present.
      \end{itemize}

      The \xmlNode{wavelet} node recognizes the following subnodes:
      \begin{itemize}
        \item \xmlNode{family}: \xmlDesc{string}, 
          The type of wavelet to use for the transformation.     There are several possible families
          to choose from, and most families contain     more than one variation. For more
          information regarding the wavelet families,     refer to the Pywavelets documentation
          located at:     https://pywavelets.readthedocs.io/en/latest/ref/wavelets.html (wavelet-
          families)     \\     Possible values are:     \begin{itemize}       \item \textbf{haar
          family}: haar       \item \textbf{db family}: db1, db2, db3, db4, db5, db6, db7, db8, db9,
          db10, db11,         db12, db13, db14, db15, db16, db17, db18, db19, db20, db21, db22,
          db23,         db24, db25, db26, db27, db28, db29, db30, db31, db32, db33, db34, db35,
          db36, db37, db38       \item \textbf{sym family}: sym2, sym3, sym4, sym5, sym6, sym7,
          sym8, sym9, sym10,         sym11, sym12, sym13, sym14, sym15, sym16, sym17, sym18, sym19,
          sym20       \item \textbf{coif family}: coif1, coif2, coif3, coif4, coif5, coif6, coif7,
          coif8,         coif9, coif10, coif11, coif12, coif13, coif14, coif15, coif16, coif17
          \item \textbf{bior family}: bior1.1, bior1.3, bior1.5, bior2.2, bior2.4, bior2.6,
          bior2.8, bior3.1, bior3.3, bior3.5, bior3.7, bior3.9, bior4.4, bior5.5,         bior6.8
          \item \textbf{rbio family}: rbio1.1, rbio1.3, rbio1.5, rbio2.2, rbio2.4, rbio2.6,
          rbio2.8, rbio3.1, rbio3.3, rbio3.5, rbio3.7, rbio3.9, rbio4.4, rbio5.5,         rbio6.8
          \item \textbf{dmey family}: dmey       \item \textbf{gaus family}: gaus1, gaus2, gaus3,
          gaus4, gaus5, gaus6, gaus7, gaus8       \item \textbf{mexh family}: mexh       \item
          \textbf{morl family}: morl       \item \textbf{cgau family}: cgau1, cgau2, cgau3, cgau4,
          cgau5, cgau6, cgau7, cgau8       \item \textbf{shan family}: shan       \item \textbf{fbsp
          family}: fbsp       \item \textbf{cmor family}: cmor     \end{itemize}
      \end{itemize}

    \item \xmlNode{PolynomialRegression}:
      fits time-series data using a polynomial function of degree one or greater.
      The \xmlNode{PolynomialRegression} node recognizes the following parameters:
        \begin{itemize}
          \item \xmlAttr{target}: \xmlDesc{comma-separated strings, required}, 
            indicates the variables for which this algorithm will be used for characterization.
          \item \xmlAttr{seed}: \xmlDesc{integer, optional}, 
            sets a seed for the underlying random number generator, if present.
      \end{itemize}

      The \xmlNode{PolynomialRegression} node recognizes the following subnodes:
      \begin{itemize}
        \item \xmlNode{degree}: \xmlDesc{integer}, 
          Specifies the degree polynomial to fit the data with.
      \end{itemize}

    \item \xmlNode{rwd}:
      TimeSeriesAnalysis algorithm for sliding window snapshots to generate features
      The \xmlNode{rwd} node recognizes the following parameters:
        \begin{itemize}
          \item \xmlAttr{target}: \xmlDesc{comma-separated strings, required}, 
            indicates the variables for which this algorithm will be used for characterization.
          \item \xmlAttr{seed}: \xmlDesc{integer, optional}, 
            sets a seed for the underlying random number generator, if present.
      \end{itemize}

      The \xmlNode{rwd} node recognizes the following subnodes:
      \begin{itemize}
        \item \xmlNode{signatureWindowLength}: \xmlDesc{integer}, 
          the size of signature window, which represents as a snapshot for a certain time step;
          typically represented as $w$ in literature, or $w\_sig$ in the code.

        \item \xmlNode{featureIndex}: \xmlDesc{integer}, 
          Index used for feature selection, which requires pre-analysis for now, will be addresses
          via other non human work required method

        \item \xmlNode{sampleType}: \xmlDesc{integer}, 
          Indicating the type of sampling.

        \item \xmlNode{seed}: \xmlDesc{integer}, 
          Indicating random seed.
      \end{itemize}

    \item \xmlNode{maxabsscaler}:
      scales the data to the interval $[-1, 1]$. This is done by dividing by     the largest
      absolute value of the data.
      The \xmlNode{maxabsscaler} node recognizes the following parameters:
        \begin{itemize}
          \item \xmlAttr{target}: \xmlDesc{comma-separated strings, required}, 
            indicates the variables for which this algorithm will be used for characterization.
          \item \xmlAttr{seed}: \xmlDesc{integer, optional}, 
            sets a seed for the underlying random number generator, if present.
      \end{itemize}

    \item \xmlNode{minmaxscaler}:
      scales the data to the interval $[0, 1]$. This is done by subtracting the
      minimum value from each point and dividing by the range.
      The \xmlNode{minmaxscaler} node recognizes the following parameters:
        \begin{itemize}
          \item \xmlAttr{target}: \xmlDesc{comma-separated strings, required}, 
            indicates the variables for which this algorithm will be used for characterization.
          \item \xmlAttr{seed}: \xmlDesc{integer, optional}, 
            sets a seed for the underlying random number generator, if present.
      \end{itemize}

    \item \xmlNode{robustscaler}:
      centers and scales the data by subtracting the median and dividing by     the interquartile
      range.
      The \xmlNode{robustscaler} node recognizes the following parameters:
        \begin{itemize}
          \item \xmlAttr{target}: \xmlDesc{comma-separated strings, required}, 
            indicates the variables for which this algorithm will be used for characterization.
          \item \xmlAttr{seed}: \xmlDesc{integer, optional}, 
            sets a seed for the underlying random number generator, if present.
      \end{itemize}

    \item \xmlNode{standardscaler}:
      centers and scales the data by subtracting the mean and dividing by     the standard
      deviation.
      The \xmlNode{standardscaler} node recognizes the following parameters:
        \begin{itemize}
          \item \xmlAttr{target}: \xmlDesc{comma-separated strings, required}, 
            indicates the variables for which this algorithm will be used for characterization.
          \item \xmlAttr{seed}: \xmlDesc{integer, optional}, 
            sets a seed for the underlying random number generator, if present.
      \end{itemize}

    \item \xmlNode{preserveCDF}:
      forces generated data provided to the inverse transformation function to
      have the same CDF as the original data through quantile mapping. If this
      transformer is used as part of a SyntheticHistory ROM, it should likely
      be used as the first transformer in the chain.
      The \xmlNode{preserveCDF} node recognizes the following parameters:
        \begin{itemize}
          \item \xmlAttr{target}: \xmlDesc{comma-separated strings, required}, 
            indicates the variables for which this algorithm will be used for characterization.
          \item \xmlAttr{seed}: \xmlDesc{integer, optional}, 
            sets a seed for the underlying random number generator, if present.
      \end{itemize}

    \item \xmlNode{differencing}:
      applies Nth order differencing to the data.
      The \xmlNode{differencing} node recognizes the following parameters:
        \begin{itemize}
          \item \xmlAttr{target}: \xmlDesc{comma-separated strings, required}, 
            indicates the variables for which this algorithm will be used for characterization.
          \item \xmlAttr{seed}: \xmlDesc{integer, optional}, 
            sets a seed for the underlying random number generator, if present.
      \end{itemize}

      The \xmlNode{differencing} node recognizes the following subnodes:
      \begin{itemize}
        \item \xmlNode{order}: \xmlDesc{integer}, 
          differencing order.
      \end{itemize}

    \item \xmlNode{zerofilter}:
      masks values that are near zero. The masked values are replaced with NaN     values. Caution
      should be used when using this algorithm because not all algorithms can handle     NaN values!
      A warning will be issued if NaN values are detected in the input of an algorithm that     does
      not support them.
      The \xmlNode{zerofilter} node recognizes the following parameters:
        \begin{itemize}
          \item \xmlAttr{target}: \xmlDesc{comma-separated strings, required}, 
            indicates the variables for which this algorithm will be used for characterization.
          \item \xmlAttr{seed}: \xmlDesc{integer, optional}, 
            sets a seed for the underlying random number generator, if present.
      \end{itemize}

    \item \xmlNode{logtransformer}:
      applies the natural logarithm to the data and inverts by applying the
      exponential function.
      The \xmlNode{logtransformer} node recognizes the following parameters:
        \begin{itemize}
          \item \xmlAttr{target}: \xmlDesc{comma-separated strings, required}, 
            indicates the variables for which this algorithm will be used for characterization.
          \item \xmlAttr{seed}: \xmlDesc{integer, optional}, 
            sets a seed for the underlying random number generator, if present.
      \end{itemize}

    \item \xmlNode{arcsinhtransformer}:
      applies the inverse hyperbolic sine to the data and inverts by applying
      the hyperbolic sine.
      The \xmlNode{arcsinhtransformer} node recognizes the following parameters:
        \begin{itemize}
          \item \xmlAttr{target}: \xmlDesc{comma-separated strings, required}, 
            indicates the variables for which this algorithm will be used for characterization.
          \item \xmlAttr{seed}: \xmlDesc{integer, optional}, 
            sets a seed for the underlying random number generator, if present.
      \end{itemize}

    \item \xmlNode{tanhtransformer}:
      applies the hyperbolic tangent to the data and inverts by applying the
      inverse hyperbolic tangent.
      The \xmlNode{tanhtransformer} node recognizes the following parameters:
        \begin{itemize}
          \item \xmlAttr{target}: \xmlDesc{comma-separated strings, required}, 
            indicates the variables for which this algorithm will be used for characterization.
          \item \xmlAttr{seed}: \xmlDesc{integer, optional}, 
            sets a seed for the underlying random number generator, if present.
      \end{itemize}

    \item \xmlNode{sigmoidtransformer}:
      applies the sigmoid (expit) function to the data and inverts by applying
      the logit function.
      The \xmlNode{sigmoidtransformer} node recognizes the following parameters:
        \begin{itemize}
          \item \xmlAttr{target}: \xmlDesc{comma-separated strings, required}, 
            indicates the variables for which this algorithm will be used for characterization.
          \item \xmlAttr{seed}: \xmlDesc{integer, optional}, 
            sets a seed for the underlying random number generator, if present.
      \end{itemize}

    \item \xmlNode{outtruncation}:
      limits the data to either positive or negative values by "reflecting" the
      out-of-range values back into the desired range.
      The \xmlNode{outtruncation} node recognizes the following parameters:
        \begin{itemize}
          \item \xmlAttr{target}: \xmlDesc{comma-separated strings, required}, 
            indicates the variables for which this algorithm will be used for characterization.
          \item \xmlAttr{seed}: \xmlDesc{integer, optional}, 
            sets a seed for the underlying random number generator, if present.
          \item \xmlAttr{domain}: \xmlDesc{[positive, negative], required}, 
            -- no description yet --
      \end{itemize}

    \item \xmlNode{gaussianize}:
      transforms the data into a normal distribution using quantile mapping.
      The \xmlNode{gaussianize} node recognizes the following parameters:
        \begin{itemize}
          \item \xmlAttr{target}: \xmlDesc{comma-separated strings, required}, 
            indicates the variables for which this algorithm will be used for characterization.
          \item \xmlAttr{seed}: \xmlDesc{integer, optional}, 
            sets a seed for the underlying random number generator, if present.
          \item \xmlAttr{nQuantiles}: \xmlDesc{integer, optional}, 
            number of quantiles to use in the transformation. If \xmlAttr{nQuantiles}
            is greater than the number of data, then the number of data is used instead. \default{1000}
      \end{itemize}

    \item \xmlNode{quantiletransformer}:
      transforms the data to fit a given distribution by mapping the data to     a uniform
      distribution and then to the desired distribution.
      The \xmlNode{quantiletransformer} node recognizes the following parameters:
        \begin{itemize}
          \item \xmlAttr{target}: \xmlDesc{comma-separated strings, required}, 
            indicates the variables for which this algorithm will be used for characterization.
          \item \xmlAttr{seed}: \xmlDesc{integer, optional}, 
            sets a seed for the underlying random number generator, if present.
          \item \xmlAttr{nQuantiles}: \xmlDesc{integer, optional}, 
            number of quantiles to use in the transformation. If \xmlAttr{nQuantiles}
            is greater than the number of data, then the number of data is used instead. \default{1000}
          \item \xmlAttr{outputDistribution}: \xmlDesc{[normal, uniform], optional}, 
            distribution to transform to. \default{normal}
      \end{itemize}
  \end{itemize}

\hspace{24pt}
Example:
\begin{lstlisting}[style=XML,morekeywords={name,subType,pivotLength,shift,target,threshold,period,width}]
<Simulation>
  ...
  <Models>
    ...
    <ROM name="synth" subType="SyntheticHistory">
      <Target>signal1, signal2, hour</Target>
      <Features>scaling</Features>
      <pivotParameter>hour</pivotParameter>
      <fourier target="signal1, signal2">
        <periods>12, 24</periods>
      </fourier>
      <arma target="signal1, signal2" seed='42'>
        <SignalLag>2</SignalLag>
        <NoiseLag>3</NoiseLag>
      </arma>
    </ROM>
    ...
  </Models>
  ...
</Simulation>
\end{lstlisting}


\subsubsection{ARMA}
  The \xmlString{ARMA} ROM is based on an autoregressive moving average time series model with
  Fourier signal processing, sometimes referred to as a FARMA.                         ARMA is a
  type of time dependent model that characterizes the autocorrelation between time series data. The
  mathematic description of ARMA is given as                         \begin{equation*}
  x\_t = \sum\_{i=1}^p\phi\_ix\_{t-i}+\alpha\_t+\sum\_{j=1}^q\theta\_j\alpha\_{t-j},
  \end{equation*}                         where $x$ is a vector of dimension $n$, and $\phi\_i$ and
  $\theta\_j$ are both $n$ by $n$ matrices. When $q=0$, the above is
  autoregressive (AR); when $p=0$, the above is moving average (MA).                         When
  training an ARMA, the input needs to be a synchronized HistorySet. For unsynchronized data, use
  PostProcessor methods to                         synchronize the data before training an ARMA.
  The ARMA model implemented allows an option to use Fourier series to detrend the time series
  before fitting to ARMA model to                         train. The Fourier trend will be stored in
  the trained ARMA model for data generation. The following equation
  describes the detrending                         process.
  \begin{equation*}                         \begin{aligned}                         x\_t &= y\_t -
  \sum\_m\left\{a\_m\sin(2\pi f\_mt)+b\_m\cos(2\pi f\_mt)\right\}  \\                         &=y\_t -
  \sum\_m\ c\_m\sin(2\pi f\_mt+\phi\_m)                         \end{aligned}
  \end{equation*}                         where $1/f\_m$ is defined by the user parameter
  \xmlNode{Fourier}. \nb $a\_m$ and $b\_m$ will be calculated then transformed to
  $c\_m$ and $\phi$. The $c\_m$ will be stored as \xmlString{amplitude}, and $\phi$ will be stored as
  \xmlString{phase}.                         By default, each target in the training will be
  considered independent and have an unique ARMA for each                         target.
  Correlated targets can be specified through the \xmlNode{correlate} node, at which point
  the correlated targets will be trained together using a vector ARMA (or VARMA). Due to limitations
  in                         the VARMA, in order to seed samples the VARMA must be trained with the
  node \xmlNode{seed}, which acts                         independently from the global random seed
  used by other RAVEN entities.                         Both the ARMA and VARMA make use of the
  \texttt{statsmodels} python package.                         In order to use this Reduced Order
  Model, the \xmlNode{ROM} attribute                         \xmlAttr{subType} needs to be
  \xmlString{ARMA}.

  The \xmlNode{ARMA} node recognizes the following parameters:
    \begin{itemize}
      \item \xmlAttr{name}: \xmlDesc{string, required}, 
        User-defined name to designate this entity in the RAVEN input file.
      \item \xmlAttr{verbosity}: \xmlDesc{[silent, quiet, all, debug], optional}, 
        Desired verbosity of messages coming from this entity
      \item \xmlAttr{subType}: \xmlDesc{string, required}, 
        specify the type of ROM that will be used
  \end{itemize}

  The \xmlNode{ARMA} node recognizes the following subnodes:
  \begin{itemize}
    \item \xmlNode{Features}: \xmlDesc{comma-separated strings}, 
      specifies the names of the features of this ROM.         \nb These parameters are going to be
      requested for the training of this object         (see Section~\ref{subsec:stepRomTrainer})

    \item \xmlNode{Target}: \xmlDesc{comma-separated strings}, 
      contains a comma separated list of the targets of this ROM. These parameters         are the
      Figures of Merit (FOMs) this ROM is supposed to predict.         \nb These parameters are
      going to be requested for the training of this         object (see Section
      \ref{subsec:stepRomTrainer}).

    \item \xmlNode{pivotParameter}: \xmlDesc{string}, 
      If a time-dependent ROM is requested, please specifies the pivot         variable (e.g. time,
      etc) used in the input HistorySet.
  \default{time}

    \item \xmlNode{featureSelection}:
      Apply feature selection algorithm

      The \xmlNode{featureSelection} node recognizes the following subnodes:
      \begin{itemize}
        \item \xmlNode{RFE}:
          The \xmlString{RFE} (Recursive Feature Elimination) is a feature selection algorithm.
          Feature selection refers to techniques that select a subset of the most relevant features
          for a model (ROM).         Fewer features can allow ROMs to run more efficiently (less
          space or time complexity) and be more effective.         Indeed, some ROMs (machine
          learning algorithms) can be misled by irrelevant input features, resulting in worse
          predictive performance.         RFE is a wrapper-type feature selection algorithm. This
          means that a different ROM is given and used in the core of the         method,         is
          wrapped by RFE, and used to help select features.         \\RFE works by searching for a
          subset of features by starting with all features in the training dataset and successfully
          removing         features until the desired number remains.         This is achieved by
          fitting the given ROM used in the core of the model, ranking features by importance,
          discarding the least important features, and re-fitting the model. This process is
          repeated until a specified number of         features remains.         When the full model
          is created, a measure of variable importance is computed that ranks the predictors from
          most         important to least.         At each stage of the search, the least important
          predictors are iteratively eliminated prior to rebuilding the model.         Features are
          scored either using the ROM model (if the model provides a mean to compute feature
          importances) or by         using a statistical method.         \\In RAVEN the
          \xmlString{RFE} class refers to an augmentation of the basic algorithm, since it allows,
          optionally,         to perform the search on multiple groups of targets (separately) and
          then combine the results of the search in a         single set. In addition, when the RFE
          search is concluded, the user can request to identify the set of features         that
          bring to a minimization of the score (i.e. maximimization of the accuracy).         In
          addition, using the ``applyClusteringFiltering'' option, the algorithm can, using an
          hierarchal clustering algorithm,         identify highly correlated features to speed up
          the subsequential search.
          The \xmlNode{RFE} node recognizes the following parameters:
            \begin{itemize}
              \item \xmlAttr{name}: \xmlDesc{string, required}, 
                User-defined name to designate this entity in the RAVEN input file.
              \item \xmlAttr{verbosity}: \xmlDesc{[silent, quiet, all, debug], optional}, 
                Desired verbosity of messages coming from this entity
          \end{itemize}

          The \xmlNode{RFE} node recognizes the following subnodes:
          \begin{itemize}
            \item \xmlNode{parametersToInclude}: \xmlDesc{comma-separated strings}, 
              List of IDs of features/variables to include in the search.
  \default{None}

            \item \xmlNode{whichSpace}: \xmlDesc{[Feature, feature, Target, target]}, 
              Which space to search? Target or Feature (this is temporary till DataSet training is
              implemented)
  \default{feature}

            \item \xmlNode{nFeaturesToSelect}: \xmlDesc{integer}, 
              Exact Number of features to select. If not inputted, ``nFeaturesToSelect'' will be set
              to $1/2$ of the features in the training dataset.
  \default{None}

            \item \xmlNode{maxNumberFeatures}: \xmlDesc{integer}, 
              Maximum Number of features to select, the algorithm will automatically determine the
              feature list to minimize a total score.
  \default{None}

            \item \xmlNode{onlyOutputScore}: \xmlDesc{[True, Yes, 1, False, No, 0, t, y, 1, f, n, 0]}, 
              If maxNumberFeatures is on, only output score should beconsidered? Or, in case of
              particular models (e.g. DMDC), state variable space score should be considered as
              well.
  \default{False}

            \item \xmlNode{applyClusteringFiltering}: \xmlDesc{[True, Yes, 1, False, No, 0, t, y, 1, f, n, 0]}, 
              Applying clustering correlation before RFE search? If true, an hierarchal clustering
              is applied on the feature         space aimed to remove features that are correlated
              before the actual RFE search is performed. This approach can stabilize and
              accelerate the process in case of large feature spaces (e.g > 500 features).
  \default{False}

            \item \xmlNode{applyCrossCorrelation}: \xmlDesc{[True, Yes, 1, False, No, 0, t, y, 1, f, n, 0]}, 
              In case of subgroupping, should a cross correleation analysis should be performed
              cross sub-groups?         If it is activated, a cross correleation analysis is used to
              additionally filter the features selected for each         sub-groupping search.
  \default{False}

            \item \xmlNode{step}: \xmlDesc{float}, 
              If greater than or equal to 1, then step corresponds to the (integer) number
              of features to remove at each iteration. If within (0.0, 1.0), then step
              corresponds to the percentage (rounded down) of features to remove at         each
              iteration.
  \default{1}

            \item \xmlNode{subGroup}: \xmlDesc{comma-separated strings, integers, and floats}, 
              Subgroup of output variables on which to perform the search. Multiple nodes of this
              type can be inputted. The RFE search will be then performed in each ``subgroup''
              separately and then the the union of the different feature sets are used for the final
              ROM.
          \end{itemize}

        \item \xmlNode{VarianceThreshold}:
          The \xmlString{VarianceThreshold} is a feature selector that removes     all low-variance
          features. This feature selection algorithm looks only at the features and not     the
          desired outputs. The variance threshold can be set by the user.
          The \xmlNode{VarianceThreshold} node recognizes the following parameters:
            \begin{itemize}
              \item \xmlAttr{name}: \xmlDesc{string, required}, 
                User-defined name to designate this entity in the RAVEN input file.
              \item \xmlAttr{verbosity}: \xmlDesc{[silent, quiet, all, debug], optional}, 
                Desired verbosity of messages coming from this entity
          \end{itemize}

          The \xmlNode{VarianceThreshold} node recognizes the following subnodes:
          \begin{itemize}
            \item \xmlNode{parametersToInclude}: \xmlDesc{comma-separated strings}, 
              List of IDs of features/variables to include in the search.
  \default{None}

            \item \xmlNode{whichSpace}: \xmlDesc{[Feature, feature, Target, target]}, 
              Which space to search? Target or Feature (this is temporary till DataSet training is
              implemented)
  \default{feature}

            \item \xmlNode{threshold}: \xmlDesc{float}, 
              Features with a training-set variance lower than this threshold                   will
              be removed. The default is to keep all features with non-zero
              variance, i.e. remove the features that have the same value in all
              samples.
  \default{0.0}
          \end{itemize}
      \end{itemize}

    \item \xmlNode{featureSpaceTransformation}:
      Use dimensionality reduction technique to perform a trasformation of the training dataset
      into an uncorrelated one. The dimensionality of the problem will not be reduced but
      the data will be transformed in the transformed space. E.g if the number of features
      are 5, the method projects such features into a new uncorrelated space (still 5-dimensional).
      In case of time-dependent ROMs, all the samples are concatenated in a global 2D matrix
      (n\_samples*n\_timesteps,n\_features) before applying the transformation and then reconstructed
      back into the original shape (before fitting the model).

      The \xmlNode{featureSpaceTransformation} node recognizes the following subnodes:
      \begin{itemize}
        \item \xmlNode{transformationMethod}: \xmlDesc{[PCA, KernelLinearPCA, KernelPolyPCA, KernelRbfPCA, KernelSigmoidPCA, KernelCosinePCA, ICA]}, 
          Transformation method to use. Eight options (5 Kernel PCAs) are available:
          \begin{itemize}                     \item \textit{PCA}, Principal Component Analysis;
          \item \textit{KernelLinearPCA}, Kernel (Linear) Principal component analysis;
          \item \textit{KernelPolyPCA}, Kernel (Poly) Principal component analysis;
          \item \textit{KernelRbfPCA}, Kernel(Rbf) Principal component analysis;
          \item \textit{KernelSigmoidPCA}, Kernel (Sigmoid) Principal component analysis;
          \item \textit{KernelCosinePCA}, Kernel (Cosine) Principal component analysis;
          \item \textit{ICA}, Independent component analysis;                    \end{itemize}
  \default{PCA}

        \item \xmlNode{parametersToInclude}: \xmlDesc{comma-separated strings}, 
          List of IDs of features/variables to include in the transformation process.
  \default{None}

        \item \xmlNode{whichSpace}: \xmlDesc{[Feature, feature, Target, target]}, 
          Which space to search? Target or Feature?
  \default{Feature}
      \end{itemize}

    \item \xmlNode{CV}: \xmlDesc{string}, 
      The text portion of this node needs to contain the name of the \xmlNode{PostProcessor} with
      \xmlAttr{subType}         ``CrossValidation``.
      The \xmlNode{CV} node recognizes the following parameters:
        \begin{itemize}
          \item \xmlAttr{class}: \xmlDesc{string, optional}, 
            should be set to \xmlString{Model}
          \item \xmlAttr{type}: \xmlDesc{string, optional}, 
            should be set to \xmlString{PostProcessor}
      \end{itemize}

    \item \xmlNode{alias}: \xmlDesc{string}, 
      specifies alias for         any variable of interest in the input or output space. These
      aliases can be used anywhere in the RAVEN input to         refer to the variables. In the body
      of this node the user specifies the name of the variable that the model is going to use
      (during its execution).
      The \xmlNode{alias} node recognizes the following parameters:
        \begin{itemize}
          \item \xmlAttr{variable}: \xmlDesc{string, required}, 
            define the actual alias, usable throughout the RAVEN input
          \item \xmlAttr{type}: \xmlDesc{[input, output], required}, 
            either ``input'' or ``output''.
      \end{itemize}

    \item \xmlNode{pivotParameter}: \xmlDesc{string}, 
      defines the pivot variable (e.g., time) that is non-decreasing in
      the input HistorySet.
  \default{time}

    \item \xmlNode{correlate}: \xmlDesc{comma-separated strings}, 
      indicates the listed variables                                                    should be
      considered as influencing each other, and trained together instead of independently.  This
      node                                                    can only be listed once, so all
      variables that are desired for correlation should be included.  \nb The
      correlated VARMA takes notably longer to train than the independent ARMAs for the same number
      of targets.
  \default{None}

    \item \xmlNode{seed}: \xmlDesc{integer}, 
      provides seed for VARMA and ARMA sampling.
      Must be provided before training. If no seed is assigned,
      then a random number will be used.
  \default{None}

    \item \xmlNode{reseedCopies}: \xmlDesc{[True, Yes, 1, False, No, 0, t, y, 1, f, n, 0]}, 
      if \xmlString{True} then whenever the ARMA is loaded from file, a
      random reseeding will be performed to ensure different histories. \nb If
      reproducible histories are desired for an ARMA loaded from file,
      \xmlNode{reseedCopies} should be set to \xmlString{False}, and in the
      \xmlNode{RunInfo} block \xmlNode{batchSize} needs to be 1
      and \xmlNode{internalParallel} should be
      \xmlString{False} for RAVEN runs sampling the trained ARMA model.
      If \xmlNode{InternalParallel} is \xmlString{True} and the ARMA has
      \xmlNode{reseedCopies} as \xmlString{False}, an identical ARMA history
      will always be provided regardless of how many samples are taken.
      If \xmlNode{InternalParallel} is \xmlString{False} and \xmlNode{batchSize}
      is more than 1, it is not possible to guarantee the order of RNG usage by
      the separate processes, so it is not possible to guarantee reproducible
      histories are generated.
  \default{True}

    \item \xmlNode{P}: \xmlDesc{integer}, 
      defines the value of $p$.
  \default{3}

    \item \xmlNode{Q}: \xmlDesc{integer}, 
      defines the value of $q$.
  \default{3}

    \item \xmlNode{Fourier}: \xmlDesc{comma-separated integers}, 
      must be positive integers. This defines the
      based period that will be used for Fourier detrending, i.e., this
      field defines $1/f\_m$ in the above equation.
      When this filed is not specified, the ARMA considers no Fourier detrend.
  \default{None}

    \item \xmlNode{Peaks}: \xmlDesc{string}, 
      designed to estimate the peaks in signals that repeat with some frequency,
      often in periodic data.
  \default{None}
      The \xmlNode{Peaks} node recognizes the following parameters:
        \begin{itemize}
          \item \xmlAttr{target}: \xmlDesc{string, required}, 
            defines the name of one target (besides the                         pivot parameter)
            expected to have periodic peaks.
          \item \xmlAttr{threshold}: \xmlDesc{float, required}, 
            user-defined minimum required                         height of peaks (absolute value).
          \item \xmlAttr{period}: \xmlDesc{float, required}, 
            user-defined expected period for target variable.
      \end{itemize}

      The \xmlNode{Peaks} node recognizes the following subnodes:
      \begin{itemize}
        \item \xmlNode{nbin}: \xmlDesc{integer}, 
          -- no description yet --
  \default{5}

        \item \xmlNode{window}: \xmlDesc{comma-separated floats}, 
          lists the window of time within each period in which a peak should be discovered.
          The text of this node is the upper and lower boundary of this
          window \emph{relative to} the start of the period, separated by a comma.
          User can define the lower bound to be a negative
          number if the window passes through one side of one period. For example, if the period is
          24                                                  hours, the window can be -2,2 which is
          equivalent to 22, 2.
          The \xmlNode{window} node recognizes the following parameters:
            \begin{itemize}
              \item \xmlAttr{width}: \xmlDesc{float, required}, 
                The user defined  width of peaks in that window. The width is in the unit of the
                signal as well.
          \end{itemize}
      \end{itemize}

    \item \xmlNode{preserveInputCDF}: \xmlDesc{[True, Yes, 1, False, No, 0, t, y, 1, f, n, 0]}, 
      enables a final transform on sampled                                                    data
      coercing it to have the same distribution as the original data. If \xmlString{True}, then
      every                                                    sample generated by this ARMA after
      training will have a distribution of values that conforms within
      numerical accuracy to the original data. This is especially useful when variance is desired
      not to stretch                                                    the most extreme events
      (high or low signal values), but instead the sequence of events throughout this
      history. For example, this transform can preserve the load duration curve for a load signal.
  \default{False}

    \item \xmlNode{SpecificFourier}: \xmlDesc{string}, 
      provides a means to specify different Fourier
      decomposition for different target variables.  Values given in the subnodes of this node will
      supercede                                                    the defaults set by the
      \xmlNode{Fourier} and \xmlNode{FourierOrder} nodes.
  \default{None}
      The \xmlNode{SpecificFourier} node recognizes the following parameters:
        \begin{itemize}
          \item \xmlAttr{variables}: \xmlDesc{comma-separated strings, required}, 
            lists the variables to whom                     the \xmlNode{SpecificFourier} parameters
            will apply.
      \end{itemize}

      The \xmlNode{SpecificFourier} node recognizes the following subnodes:
      \begin{itemize}
        \item \xmlNode{periods}: \xmlDesc{comma-separated integers}, 
          lists the (fundamental)                                                    periodic
          wavelength of the Fourier decomposition for these variables,
          as in the \xmlNode{Fourier} general node.
      \end{itemize}

    \item \xmlNode{Multicycle}: \xmlDesc{string}, 
      indicates that each sample of the ARMA should yield
      multiple sequential samples. For example, if an ARMA model is trained to produce a year's
      worth of data,                                                    enabling
      \xmlNode{Multicycle} causes it to produce several successive years of data. Multicycle
      sampling                                                    is independent of ROM training,
      and only changes how samples of the ARMA are created.
      \nb The output of a multicycle ARMA must be stored in a \xmlNode{DataSet}, as the targets will
      depend                                                    on both the \xmlNode{pivotParameter}
      as well as the cycle, \xmlString{Cycle}. The cycle is a second
      \xmlNode{Index} that all targets should depend on, with variable name \xmlString{Cycle}.
  \default{None}

      The \xmlNode{Multicycle} node recognizes the following subnodes:
      \begin{itemize}
        \item \xmlNode{cycles}: \xmlDesc{integer}, 
          the number of cycles the ARMA should produce
          each time it yields a sample.

        \item \xmlNode{growth}: \xmlDesc{float}, 
          if provided then the histories produced by
          the ARMA will be increased by the growth factor for successive cycles. This node can be
          added                                                    multiple times with different
          settings for different targets.                                                    The
          text of this node is the growth factor in percentage. Some examples are in
          Table~\ref{tab:arma multicycle growth}, where \emph{Growth factor} is the value used in
          the RAVEN                                                    input and \emph{Scaling
          factor} is the value by which the history will be multiplied.
          \begin{table}[h!]                                                      \centering
          \begin{tabular}{r c l}                                                        Growth
          factor & Scaling factor & Description \\ \hline
          50 & 1.5 & growing by 50\% each cycle \\
          -50 & 0.5 & shrinking by 50\% each cycle \\
          150 & 2.5 & growing by 150\% each cycle \\
          \end{tabular}                                                      \caption{ARMA Growth
          Factor Examples}                                                      \label{tab:arma
          multicycle growth}                                                    \end{table}
  \default{None}
          The \xmlNode{growth} node recognizes the following parameters:
            \begin{itemize}
              \item \xmlAttr{targets}: \xmlDesc{comma-separated strings, required}, 
                lists the targets                     in this ARMA that this growth factor should
                apply to.
              \item \xmlAttr{start\_index}: \xmlDesc{integer, optional}, 
                -- no description yet --
              \item \xmlAttr{end\_index}: \xmlDesc{integer, optional}, 
                -- no description yet --
              \item \xmlAttr{mode}: \xmlDesc{[exponential, linear], required}, 
                either \xmlString{linear} or                     \xmlString{exponential}, determines
                the manner in which the growth factor is applied.                     If
                \xmlString{linear}, then the scaling factor is $(1+y\cdot g/100)$;
                if \xmlString{exponential}, then the scaling factor is $(1+g/100)^y$;
                where $y$ is the cycle after the first and $g$ is the provided scaling factor.
          \end{itemize}
      \end{itemize}

    \item \xmlNode{nyquistScalar}: \xmlDesc{integer}, 
      -- no description yet --
  \default{1}

    \item \xmlNode{ZeroFilter}: \xmlDesc{string}, 
      turns on \emph{zero filtering}                                                  for the listed
      targets. Zero filtering is a very specific algorithm, and should not be used without
      understanding its application.  When zero filtering is enabled, the ARMA will remove all the
      values from                                                  the training data equal to zero
      for the target, then train on the remaining data (including Fourier detrending
      if applicable). If the target is set as correlated to another target, the second target will
      be treated as                                                  two distinct series: one
      containing times in which the original target is zero, and one in the remaining
      times. The results from separated ARMAs are recombined after sampling. This can be a
      methodology for                                                  treating histories with long
      zero-value segments punctuated periodically by peaks.
  \default{None}
      The \xmlNode{ZeroFilter} node recognizes the following parameters:
        \begin{itemize}
          \item \xmlAttr{tol}: \xmlDesc{float, optional}, 
            -- no description yet --
      \end{itemize}

    \item \xmlNode{outTruncation}: \xmlDesc{comma-separated strings}, 
      defines whether and how output                                                    time series
      are limited in domain. This node has one attribute, \xmlAttr{domain}, whose value can be
      \xmlString{positive} or \xmlString{negative}. The value of this node contains the list of
      targets to whom                                                    this domain limitation
      should be applied. In the event a negative value is discovered in a target whose
      domain is strictly positive, the absolute value of the original negative value will be used
      instead, and                                                    similarly for the negative
      domain.
  \default{None}
      The \xmlNode{outTruncation} node recognizes the following parameters:
        \begin{itemize}
          \item \xmlAttr{domain}: \xmlDesc{[positive, negative], required}, 
            -- no description yet --
      \end{itemize}
  \end{itemize}

In addition, \xmlNode{Segment} can be used to divided the ROM. In order to enable the segmentation, the
user need to specify following information for \xmlNode{Segment}:
\begin{itemize}
  \item \xmlNode{Segment}, \xmlDesc{node, optional}, provides an alternative way to build the ROM. When
    this mode is enabled, the subspace of the ROM (e.g. ``time'') will be divided into segments as
    requested, then a distinct ROM will be trained on each of the segments. This is especially helpful if
    during the subspace the ROM representation of the signal changes significantly. For example, if the signal
    is different during summer and winter, then a signal can be divided and a distinct ROM trained on the
    segments. By default, no segmentation occurs.

    To futher enable clustering of the segments, the \xmlNode{Segment} has the following attributes:
    \begin{itemize}
      \item \xmlAttr{grouping}, \xmlDesc{string, optional field} enables the use of ROM subspace clustering in
        addition to segmenting if set to \xmlString{cluster}. If set to \xmlString{segment}, then performs
        segmentation without clustering. If clustering, then an additional node needs to be included in the
        \xmlNode{Segment} node, as described below.
        \default{segment}
    \end{itemize}

    This node takes the following subnodes:
    \begin{itemize}
      \item \xmlNode{subspace}, \xmlDesc{string, required field} designates the subspace to divide. This
        should be the pivot parameter (often ``time'') for the ROM. This node also requires an attribute
        to determine how the subspace is divided, as well as other attributes, described below:
        \begin{itemize}
          \item \xmlAttr{pivotLength}, \xmlDesc{float, optional field}, provides the value in the subspace
            that each segment should attempt to represent, independently of how the data is stored. For
            example, if the subspace has hourly resolution, is measured in seconds, and the desired
            segmentation is daily, the \xmlAttr{pivotLength} would be 86400.
            Either this option or \xmlAttr{divisions} must be provided.
          \item \xmlAttr{divisions}, \xmlDesc{integer, optional field}, as an alternative to
            \xmlAttr{pivotLength}, this attribute can be used to specify how many data points to include in
            each subdivision, rather than use the pivot values. The algorithm will attempt to split the data
            points as equally as possible.
            Either this option or \xmlAttr{pivotLength} must be provided.
          \item \xmlAttr{shift}, \xmlDesc{string, optional field}, governs the way in which the subspace is
            treated in each segment. By default, the subspace retains its actual values for each segment; for
            example, if each segment is 4 hours long, the first segment starts at time 0, the second at 4
            hours, the third at 8 hours, and so forth. Options to change this behavior are \xmlString{zero}
            and \xmlString{first}. In the case of \xmlString{zero}, each segment restarts the pivot with the
            subspace value as 0, shifting all other values similarly. In the example above, the first segment
            would start at 0, the second at 0, and the third at 0, with each ending at 4 hours. Note that the
            pivot values are restored when the ROM is evaluated. Using \xmlString{first}, each segment
            subspace restarts at the value of the first segment. This is useful in the event subspace 0 is not
            a desirable value.
        \end{itemize}
      \item \xmlNode{Classifier}, \xmlDesc{string, optional field} associates a \xmlNode{PostProcessor}
        defined in the \xmlNode{Models} block to this segmentation. If clustering is enabled (see
        \xmlAttr{grouping} above), then this associated Classifier will be used to cluster the segmented ROM
        subspaces. The attributes \xmlAttr{class}=\xmlString{Models} and
        \xmlAttr{type}=\xmlString{PostProcessor} must be set, and the text of this node is the \xmlAttr{name}
        of the requested Classifier. Note this Classifier must be a valid Classifier; not all PostProcessors
        are suitable. For example, see the DataMining PostProcessor subtype Clustering.
      \item \xmlNode{clusterFeatures}, \xmlDesc{string, optional field}, if clustering then delineates
        the fundamental ROM features that should be considered while clustering. The available features are
        ROM-dependent, and an exception is raised if an unrecognized request is given. See individual ROMs
        for options. \default All ROM-specific options.
      \item \xmlNode{evalMode}, \xmlDesc{string, optional field}, one of \xmlString{truncated},
        \xmlString{full}, or \xmlString{clustered}, determines how the evaluations are
        represented, as follows:
        \begin{itemize}
          \item \xmlString{full}, reproduce the full signal using representative cluster segments,
          \item \xmlString{truncated}, reproduce a history containing exactly segment from each
            cluster placed back-to-back, with the \xmlNode{pivotParameter} spanning the clustered
            dimension. Note this will almost surely not be the same length as the original signal;
            information about indexing can be found in the ROM's XML metadata.
          \item \xmlString{clustered}, reproduce a N-dimensional object with the variable
            \texttt{\_ROM\_cluster} as one of the indexes for the ROM's sampled variables. Note that
            in order to use the option, the receiving \xmlNode{DataObject} should be of type
            \xmlNode{DataSet} with one of the indices being \texttt{\_ROM\_cluster}.
        \end{itemize}
     \item \xmlNode{evaluationClusterChoice}, \xmlDesc{string, optional field}, one of \xmlString{first} or
        \xmlString{random}, determines, if \xmlAttr{grouping}$=cluster$, which
        strategy needs to be followed for the evaluation stage. If ``first'', the
        first ROM (representative segmented ROM),in each cluster, is considered to
         be representative of the full space in the cluster (i.e. the evaluation is always performed
         interrogating the first ROM in each cluster); If ``random'', a random ROM, in each cluster,
         is choosen when an evaluation is requested.
   \nb if ``first'' is used, there is \emph{substantial} memory savings when compared to using
   ``random''.
         %If ``centroid'', a ROM ``trained" on the centroids
         %information of each cluster is used for the evaluation (\nb ``centroid'' option is not
         %available yet).
         \default{first}
    \end{itemize}
\end{itemize}

\hspace{24pt}
General ARMA Example:
\begin{lstlisting}[style=XML, morekeywords={name,subType,pivotLength,shift,target,threshold,period,width}]
<Simulation>
  ...
  <Models>
    ...
    <ROM name='aUserDefinedName' subType='ARMA'>
      <pivotParameter>Time</pivotParameter>
      <Features>scaling</Features>
      <Target>Speed1,Speed2</Target>
      <P>5</P>
      <Q>4</Q>
      <Segment>
        <subspace pivotLength="1296000" shift="first">Time</subspace>
      </Segment>
      <preserveInputCDF>True</preserveInputCDF>
      <Fourier>604800,86400</Fourier>
      <FourierOrder>2, 4</FourierOrder>
      <Peaks target='Speed1' threshold='0.1' period='86400'>
        <window width='14400' >-7200,10800</window>
        <window width='18000' >64800,75600</window>
      </Peaks>
     </ROM>
    ...
  </Models>
  ...
</Simulation>
\end{lstlisting}


\subsubsection{PolyExponential}
  The \xmlNode{PolyExponential} contains a single ROM type, aimed to construct a     time-dependent
  (or any other monotonic variable) surrogate model based on polynomial sum of exponential term.
  This surrogate have the form:     \begin{equation}       SM(X,z) = \sum\_{i=1}^{N} P\_{i}(X) \times
  \exp ( - Q\_{i}(X) \times z )     \end{equation}     where:     \begin{itemize}       \item
  $\mathbf{z}$ is the independent  monotonic variable (e.g. time)       \item $\mathbf{X}$  is the
  vector of the other independent (parametric) variables  (Features)       \item $\mathbf{P\_{i}}(X)$
  is a polynomial of rank M function of the parametric space X       \item  $\mathbf{Q\_{i}}(X)$ is a
  polynomial of rank M function of the parametric space X       \item  $\mathbf{N}$ is the number of
  requested exponential terms.     \end{itemize}     It is crucial to notice that this model is
  quite suitable for FOMs whose drivers are characterized by an exponential-like behavior.     In
  addition, it is important to notice that the exponential terms' coefficients are computed running
  a genetic-algorithm optimization     problem, which is quite slow in case of increasing number of
  ``numberExpTerms''.     In order to use this Reduced Order Model, the \xmlNode{ROM} attribute
  \xmlAttr{subType} needs to be set equal to \xmlString{PolyExponential}.     \\     Once the ROM is
  trained (\textbf{Step} \xmlNode{RomTrainer}), its coefficients can be exported into an XML file
  via an \xmlNode{OutStream} of type \xmlAttr{Print}. The following variable/parameters can be
  exported (i.e. \xmlNode{what} node     in \xmlNode{OutStream} of type \xmlAttr{Print}):
  \begin{itemize}       \item \xmlNode{expTerms}, see XML input specifications above, inquired pre-
  pending the keyword ``output|'' (e.g. output| expTerms)       \item \xmlNode{coeffRegressor}, see
  XML input specifications above       \item \xmlNode{polyOrder}, see XML input specifications above
  \item \xmlNode{features}, see XML input specifications above       \item \xmlNode{timeScale}, XML
  node containing the array of the training time steps values       \item \xmlNode{coefficients},
  XML node containing the exponential terms' coefficients for each realization     \end{itemize}

  The \xmlNode{PolyExponential} node recognizes the following parameters:
    \begin{itemize}
      \item \xmlAttr{name}: \xmlDesc{string, required}, 
        User-defined name to designate this entity in the RAVEN input file.
      \item \xmlAttr{verbosity}: \xmlDesc{[silent, quiet, all, debug], optional}, 
        Desired verbosity of messages coming from this entity
      \item \xmlAttr{subType}: \xmlDesc{string, required}, 
        specify the type of ROM that will be used
  \end{itemize}

  The \xmlNode{PolyExponential} node recognizes the following subnodes:
  \begin{itemize}
    \item \xmlNode{Features}: \xmlDesc{comma-separated strings}, 
      specifies the names of the features of this ROM.         \nb These parameters are going to be
      requested for the training of this object         (see Section~\ref{subsec:stepRomTrainer})

    \item \xmlNode{Target}: \xmlDesc{comma-separated strings}, 
      contains a comma separated list of the targets of this ROM. These parameters         are the
      Figures of Merit (FOMs) this ROM is supposed to predict.         \nb These parameters are
      going to be requested for the training of this         object (see Section
      \ref{subsec:stepRomTrainer}).

    \item \xmlNode{pivotParameter}: \xmlDesc{string}, 
      If a time-dependent ROM is requested, please specifies the pivot         variable (e.g. time,
      etc) used in the input HistorySet.
  \default{time}

    \item \xmlNode{featureSelection}:
      Apply feature selection algorithm

      The \xmlNode{featureSelection} node recognizes the following subnodes:
      \begin{itemize}
        \item \xmlNode{RFE}:
          The \xmlString{RFE} (Recursive Feature Elimination) is a feature selection algorithm.
          Feature selection refers to techniques that select a subset of the most relevant features
          for a model (ROM).         Fewer features can allow ROMs to run more efficiently (less
          space or time complexity) and be more effective.         Indeed, some ROMs (machine
          learning algorithms) can be misled by irrelevant input features, resulting in worse
          predictive performance.         RFE is a wrapper-type feature selection algorithm. This
          means that a different ROM is given and used in the core of the         method,         is
          wrapped by RFE, and used to help select features.         \\RFE works by searching for a
          subset of features by starting with all features in the training dataset and successfully
          removing         features until the desired number remains.         This is achieved by
          fitting the given ROM used in the core of the model, ranking features by importance,
          discarding the least important features, and re-fitting the model. This process is
          repeated until a specified number of         features remains.         When the full model
          is created, a measure of variable importance is computed that ranks the predictors from
          most         important to least.         At each stage of the search, the least important
          predictors are iteratively eliminated prior to rebuilding the model.         Features are
          scored either using the ROM model (if the model provides a mean to compute feature
          importances) or by         using a statistical method.         \\In RAVEN the
          \xmlString{RFE} class refers to an augmentation of the basic algorithm, since it allows,
          optionally,         to perform the search on multiple groups of targets (separately) and
          then combine the results of the search in a         single set. In addition, when the RFE
          search is concluded, the user can request to identify the set of features         that
          bring to a minimization of the score (i.e. maximimization of the accuracy).         In
          addition, using the ``applyClusteringFiltering'' option, the algorithm can, using an
          hierarchal clustering algorithm,         identify highly correlated features to speed up
          the subsequential search.
          The \xmlNode{RFE} node recognizes the following parameters:
            \begin{itemize}
              \item \xmlAttr{name}: \xmlDesc{string, required}, 
                User-defined name to designate this entity in the RAVEN input file.
              \item \xmlAttr{verbosity}: \xmlDesc{[silent, quiet, all, debug], optional}, 
                Desired verbosity of messages coming from this entity
          \end{itemize}

          The \xmlNode{RFE} node recognizes the following subnodes:
          \begin{itemize}
            \item \xmlNode{parametersToInclude}: \xmlDesc{comma-separated strings}, 
              List of IDs of features/variables to include in the search.
  \default{None}

            \item \xmlNode{whichSpace}: \xmlDesc{[Feature, feature, Target, target]}, 
              Which space to search? Target or Feature (this is temporary till DataSet training is
              implemented)
  \default{feature}

            \item \xmlNode{nFeaturesToSelect}: \xmlDesc{integer}, 
              Exact Number of features to select. If not inputted, ``nFeaturesToSelect'' will be set
              to $1/2$ of the features in the training dataset.
  \default{None}

            \item \xmlNode{maxNumberFeatures}: \xmlDesc{integer}, 
              Maximum Number of features to select, the algorithm will automatically determine the
              feature list to minimize a total score.
  \default{None}

            \item \xmlNode{onlyOutputScore}: \xmlDesc{[True, Yes, 1, False, No, 0, t, y, 1, f, n, 0]}, 
              If maxNumberFeatures is on, only output score should beconsidered? Or, in case of
              particular models (e.g. DMDC), state variable space score should be considered as
              well.
  \default{False}

            \item \xmlNode{applyClusteringFiltering}: \xmlDesc{[True, Yes, 1, False, No, 0, t, y, 1, f, n, 0]}, 
              Applying clustering correlation before RFE search? If true, an hierarchal clustering
              is applied on the feature         space aimed to remove features that are correlated
              before the actual RFE search is performed. This approach can stabilize and
              accelerate the process in case of large feature spaces (e.g > 500 features).
  \default{False}

            \item \xmlNode{applyCrossCorrelation}: \xmlDesc{[True, Yes, 1, False, No, 0, t, y, 1, f, n, 0]}, 
              In case of subgroupping, should a cross correleation analysis should be performed
              cross sub-groups?         If it is activated, a cross correleation analysis is used to
              additionally filter the features selected for each         sub-groupping search.
  \default{False}

            \item \xmlNode{step}: \xmlDesc{float}, 
              If greater than or equal to 1, then step corresponds to the (integer) number
              of features to remove at each iteration. If within (0.0, 1.0), then step
              corresponds to the percentage (rounded down) of features to remove at         each
              iteration.
  \default{1}

            \item \xmlNode{subGroup}: \xmlDesc{comma-separated strings, integers, and floats}, 
              Subgroup of output variables on which to perform the search. Multiple nodes of this
              type can be inputted. The RFE search will be then performed in each ``subgroup''
              separately and then the the union of the different feature sets are used for the final
              ROM.
          \end{itemize}

        \item \xmlNode{VarianceThreshold}:
          The \xmlString{VarianceThreshold} is a feature selector that removes     all low-variance
          features. This feature selection algorithm looks only at the features and not     the
          desired outputs. The variance threshold can be set by the user.
          The \xmlNode{VarianceThreshold} node recognizes the following parameters:
            \begin{itemize}
              \item \xmlAttr{name}: \xmlDesc{string, required}, 
                User-defined name to designate this entity in the RAVEN input file.
              \item \xmlAttr{verbosity}: \xmlDesc{[silent, quiet, all, debug], optional}, 
                Desired verbosity of messages coming from this entity
          \end{itemize}

          The \xmlNode{VarianceThreshold} node recognizes the following subnodes:
          \begin{itemize}
            \item \xmlNode{parametersToInclude}: \xmlDesc{comma-separated strings}, 
              List of IDs of features/variables to include in the search.
  \default{None}

            \item \xmlNode{whichSpace}: \xmlDesc{[Feature, feature, Target, target]}, 
              Which space to search? Target or Feature (this is temporary till DataSet training is
              implemented)
  \default{feature}

            \item \xmlNode{threshold}: \xmlDesc{float}, 
              Features with a training-set variance lower than this threshold                   will
              be removed. The default is to keep all features with non-zero
              variance, i.e. remove the features that have the same value in all
              samples.
  \default{0.0}
          \end{itemize}
      \end{itemize}

    \item \xmlNode{featureSpaceTransformation}:
      Use dimensionality reduction technique to perform a trasformation of the training dataset
      into an uncorrelated one. The dimensionality of the problem will not be reduced but
      the data will be transformed in the transformed space. E.g if the number of features
      are 5, the method projects such features into a new uncorrelated space (still 5-dimensional).
      In case of time-dependent ROMs, all the samples are concatenated in a global 2D matrix
      (n\_samples*n\_timesteps,n\_features) before applying the transformation and then reconstructed
      back into the original shape (before fitting the model).

      The \xmlNode{featureSpaceTransformation} node recognizes the following subnodes:
      \begin{itemize}
        \item \xmlNode{transformationMethod}: \xmlDesc{[PCA, KernelLinearPCA, KernelPolyPCA, KernelRbfPCA, KernelSigmoidPCA, KernelCosinePCA, ICA]}, 
          Transformation method to use. Eight options (5 Kernel PCAs) are available:
          \begin{itemize}                     \item \textit{PCA}, Principal Component Analysis;
          \item \textit{KernelLinearPCA}, Kernel (Linear) Principal component analysis;
          \item \textit{KernelPolyPCA}, Kernel (Poly) Principal component analysis;
          \item \textit{KernelRbfPCA}, Kernel(Rbf) Principal component analysis;
          \item \textit{KernelSigmoidPCA}, Kernel (Sigmoid) Principal component analysis;
          \item \textit{KernelCosinePCA}, Kernel (Cosine) Principal component analysis;
          \item \textit{ICA}, Independent component analysis;                    \end{itemize}
  \default{PCA}

        \item \xmlNode{parametersToInclude}: \xmlDesc{comma-separated strings}, 
          List of IDs of features/variables to include in the transformation process.
  \default{None}

        \item \xmlNode{whichSpace}: \xmlDesc{[Feature, feature, Target, target]}, 
          Which space to search? Target or Feature?
  \default{Feature}
      \end{itemize}

    \item \xmlNode{CV}: \xmlDesc{string}, 
      The text portion of this node needs to contain the name of the \xmlNode{PostProcessor} with
      \xmlAttr{subType}         ``CrossValidation``.
      The \xmlNode{CV} node recognizes the following parameters:
        \begin{itemize}
          \item \xmlAttr{class}: \xmlDesc{string, optional}, 
            should be set to \xmlString{Model}
          \item \xmlAttr{type}: \xmlDesc{string, optional}, 
            should be set to \xmlString{PostProcessor}
      \end{itemize}

    \item \xmlNode{alias}: \xmlDesc{string}, 
      specifies alias for         any variable of interest in the input or output space. These
      aliases can be used anywhere in the RAVEN input to         refer to the variables. In the body
      of this node the user specifies the name of the variable that the model is going to use
      (during its execution).
      The \xmlNode{alias} node recognizes the following parameters:
        \begin{itemize}
          \item \xmlAttr{variable}: \xmlDesc{string, required}, 
            define the actual alias, usable throughout the RAVEN input
          \item \xmlAttr{type}: \xmlDesc{[input, output], required}, 
            either ``input'' or ``output''.
      \end{itemize}

    \item \xmlNode{pivotParameter}: \xmlDesc{string}, 
      defines the pivot variable (e.g., time) that represents the
      independent monotonic variable
  \default{time}

    \item \xmlNode{numberExpTerms}: \xmlDesc{integer}, 
      the number of exponential terms to be used ($N$ above)
  \default{3}

    \item \xmlNode{coeffRegressor}: \xmlDesc{[poly, spline, nearest]}, 
      defines which regressor to use for interpolating the
      exponential coefficient. Available are ``spline'',``poly'' and ``nearest''.
  \default{spline}

    \item \xmlNode{polyOrder}: \xmlDesc{integer}, 
      the polynomial order to be used for interpolating the exponential
      coefficients. Only valid in case of  \xmlNode{coeffRegressor} set to ``poly''.
  \default{3}

    \item \xmlNode{tol}: \xmlDesc{float}, 
      relative tolerance of the optimization problem (differential evolution optimizer)
  \default{0.001}

    \item \xmlNode{max\_iter}: \xmlDesc{integer}, 
      maximum number of iterations (generations) for the
      optimization problem  (differential evolution optimizer)
  \default{5000}
  \end{itemize}

\hspace{24pt}
Example:
\begin{lstlisting}[style=XML,morekeywords={name,subType}]
<Simulation>
  ...
  <Models>
    ...
   <ROM name='PolyExp' subType='PolyExponential'>
     <Target>time,decay_heat, xe135_dens</Target>
     <Features>enrichment,bu</Features>
     <pivotParameter>time</pivotParameter>
     <numberExpTerms>5</numberExpTerms>
     <max_iter>1000000</max_iter>
     <tol>0.000001</tol>
  </ROM>
    ...
  </Models>
  ...
</Simulation>
\end{lstlisting}

Example to export the coefficients of trained PolyExponential ROM:
\begin{lstlisting}[style=XML,morekeywords={name,subType}]
<Simulation>
  ...
  <OutStreams>
    ...
    <Print name = 'dumpAllCoefficients'>
      <type>xml</type>
      <source>PolyExp</source>
      <!--
        here the <what> node is omitted. All the available params/coefficients
        are going to be printed out
      -->
    </Print>
    <Print name = 'dumpSomeCoefficients'>
      <type>xml</type>
      <source>PolyExp</source>
      <what>coefficients,timeScale</what>
    </Print>
    ...
  </OutStreams>
  ...
</Simulation>
\end{lstlisting}


\subsubsection{DMD}
  The \xmlString{DMD} ROM aimed to construct a time-dependent (or any other monotonic
  variable) surrogate model based on Dynamic Mode Decomposition         This surrogate is aimed to
  perform a ``dimensionality reduction regression'', where, given time         series (or any
  monotonic-dependent variable) of data, a set of modes each of which is associated         with a
  fixed oscillation frequency and decay/growth rate is computed         in order to represent the
  data-set.         In order to use this Reduced Order Model, the \xmlNode{ROM} attribute
  \xmlAttr{subType} needs to be set equal to \xmlString{DMD}.         \\         Once the ROM  is
  trained (\textbf{Step} \xmlNode{RomTrainer}), its parameters/coefficients can be exported into an
  XML file         via an \xmlNode{OutStream} of type \xmlAttr{Print}. The following
  variable/parameters can be exported (i.e. \xmlNode{what} node         in \xmlNode{OutStream} of
  type \xmlAttr{Print}):         \begin{itemize}           \item \xmlNode{rankSVD}, see XML input
  specifications below           \item \xmlNode{energyRankSVD}, see XML input specifications below
  \item \xmlNode{rankTLSQ}, see XML input specifications below           \item \xmlNode{exactModes},
  see XML input specifications below           \item \xmlNode{optimized}, see XML input
  specifications below           \item \xmlNode{features}, see XML input specifications below
  \item \xmlNode{timeScale}, XML node containing the array of the training time steps values
  \item \xmlNode{dmdTimeScale}, XML node containing the array of time scale in the DMD space (can be
  used as mapping           between the  \xmlNode{timeScale} and \xmlNode{dmdTimeScale})
  \item \xmlNode{eigs}, XML node containing the eigenvalues (imaginary and real part)
  \item \xmlNode{amplitudes}, XML node containing the amplitudes (imaginary and real part)
  \item \xmlNode{modes}, XML node containing the dynamic modes (imaginary and real part)
  \end{itemize}

  The \xmlNode{DMD} node recognizes the following parameters:
    \begin{itemize}
      \item \xmlAttr{name}: \xmlDesc{string, required}, 
        User-defined name to designate this entity in the RAVEN input file.
      \item \xmlAttr{verbosity}: \xmlDesc{[silent, quiet, all, debug], optional}, 
        Desired verbosity of messages coming from this entity
      \item \xmlAttr{subType}: \xmlDesc{string, required}, 
        specify the type of ROM that will be used
  \end{itemize}

  The \xmlNode{DMD} node recognizes the following subnodes:
  \begin{itemize}
    \item \xmlNode{Features}: \xmlDesc{comma-separated strings}, 
      specifies the names of the features of this ROM.         \nb These parameters are going to be
      requested for the training of this object         (see Section~\ref{subsec:stepRomTrainer})

    \item \xmlNode{Target}: \xmlDesc{comma-separated strings}, 
      contains a comma separated list of the targets of this ROM. These parameters         are the
      Figures of Merit (FOMs) this ROM is supposed to predict.         \nb These parameters are
      going to be requested for the training of this         object (see Section
      \ref{subsec:stepRomTrainer}).

    \item \xmlNode{pivotParameter}: \xmlDesc{string}, 
      If a time-dependent ROM is requested, please specifies the pivot         variable (e.g. time,
      etc) used in the input HistorySet.
  \default{time}

    \item \xmlNode{featureSelection}:
      Apply feature selection algorithm

      The \xmlNode{featureSelection} node recognizes the following subnodes:
      \begin{itemize}
        \item \xmlNode{RFE}:
          The \xmlString{RFE} (Recursive Feature Elimination) is a feature selection algorithm.
          Feature selection refers to techniques that select a subset of the most relevant features
          for a model (ROM).         Fewer features can allow ROMs to run more efficiently (less
          space or time complexity) and be more effective.         Indeed, some ROMs (machine
          learning algorithms) can be misled by irrelevant input features, resulting in worse
          predictive performance.         RFE is a wrapper-type feature selection algorithm. This
          means that a different ROM is given and used in the core of the         method,         is
          wrapped by RFE, and used to help select features.         \\RFE works by searching for a
          subset of features by starting with all features in the training dataset and successfully
          removing         features until the desired number remains.         This is achieved by
          fitting the given ROM used in the core of the model, ranking features by importance,
          discarding the least important features, and re-fitting the model. This process is
          repeated until a specified number of         features remains.         When the full model
          is created, a measure of variable importance is computed that ranks the predictors from
          most         important to least.         At each stage of the search, the least important
          predictors are iteratively eliminated prior to rebuilding the model.         Features are
          scored either using the ROM model (if the model provides a mean to compute feature
          importances) or by         using a statistical method.         \\In RAVEN the
          \xmlString{RFE} class refers to an augmentation of the basic algorithm, since it allows,
          optionally,         to perform the search on multiple groups of targets (separately) and
          then combine the results of the search in a         single set. In addition, when the RFE
          search is concluded, the user can request to identify the set of features         that
          bring to a minimization of the score (i.e. maximimization of the accuracy).         In
          addition, using the ``applyClusteringFiltering'' option, the algorithm can, using an
          hierarchal clustering algorithm,         identify highly correlated features to speed up
          the subsequential search.
          The \xmlNode{RFE} node recognizes the following parameters:
            \begin{itemize}
              \item \xmlAttr{name}: \xmlDesc{string, required}, 
                User-defined name to designate this entity in the RAVEN input file.
              \item \xmlAttr{verbosity}: \xmlDesc{[silent, quiet, all, debug], optional}, 
                Desired verbosity of messages coming from this entity
          \end{itemize}

          The \xmlNode{RFE} node recognizes the following subnodes:
          \begin{itemize}
            \item \xmlNode{parametersToInclude}: \xmlDesc{comma-separated strings}, 
              List of IDs of features/variables to include in the search.
  \default{None}

            \item \xmlNode{whichSpace}: \xmlDesc{[Feature, feature, Target, target]}, 
              Which space to search? Target or Feature (this is temporary till DataSet training is
              implemented)
  \default{feature}

            \item \xmlNode{nFeaturesToSelect}: \xmlDesc{integer}, 
              Exact Number of features to select. If not inputted, ``nFeaturesToSelect'' will be set
              to $1/2$ of the features in the training dataset.
  \default{None}

            \item \xmlNode{maxNumberFeatures}: \xmlDesc{integer}, 
              Maximum Number of features to select, the algorithm will automatically determine the
              feature list to minimize a total score.
  \default{None}

            \item \xmlNode{onlyOutputScore}: \xmlDesc{[True, Yes, 1, False, No, 0, t, y, 1, f, n, 0]}, 
              If maxNumberFeatures is on, only output score should beconsidered? Or, in case of
              particular models (e.g. DMDC), state variable space score should be considered as
              well.
  \default{False}

            \item \xmlNode{applyClusteringFiltering}: \xmlDesc{[True, Yes, 1, False, No, 0, t, y, 1, f, n, 0]}, 
              Applying clustering correlation before RFE search? If true, an hierarchal clustering
              is applied on the feature         space aimed to remove features that are correlated
              before the actual RFE search is performed. This approach can stabilize and
              accelerate the process in case of large feature spaces (e.g > 500 features).
  \default{False}

            \item \xmlNode{applyCrossCorrelation}: \xmlDesc{[True, Yes, 1, False, No, 0, t, y, 1, f, n, 0]}, 
              In case of subgroupping, should a cross correleation analysis should be performed
              cross sub-groups?         If it is activated, a cross correleation analysis is used to
              additionally filter the features selected for each         sub-groupping search.
  \default{False}

            \item \xmlNode{step}: \xmlDesc{float}, 
              If greater than or equal to 1, then step corresponds to the (integer) number
              of features to remove at each iteration. If within (0.0, 1.0), then step
              corresponds to the percentage (rounded down) of features to remove at         each
              iteration.
  \default{1}

            \item \xmlNode{subGroup}: \xmlDesc{comma-separated strings, integers, and floats}, 
              Subgroup of output variables on which to perform the search. Multiple nodes of this
              type can be inputted. The RFE search will be then performed in each ``subgroup''
              separately and then the the union of the different feature sets are used for the final
              ROM.
          \end{itemize}

        \item \xmlNode{VarianceThreshold}:
          The \xmlString{VarianceThreshold} is a feature selector that removes     all low-variance
          features. This feature selection algorithm looks only at the features and not     the
          desired outputs. The variance threshold can be set by the user.
          The \xmlNode{VarianceThreshold} node recognizes the following parameters:
            \begin{itemize}
              \item \xmlAttr{name}: \xmlDesc{string, required}, 
                User-defined name to designate this entity in the RAVEN input file.
              \item \xmlAttr{verbosity}: \xmlDesc{[silent, quiet, all, debug], optional}, 
                Desired verbosity of messages coming from this entity
          \end{itemize}

          The \xmlNode{VarianceThreshold} node recognizes the following subnodes:
          \begin{itemize}
            \item \xmlNode{parametersToInclude}: \xmlDesc{comma-separated strings}, 
              List of IDs of features/variables to include in the search.
  \default{None}

            \item \xmlNode{whichSpace}: \xmlDesc{[Feature, feature, Target, target]}, 
              Which space to search? Target or Feature (this is temporary till DataSet training is
              implemented)
  \default{feature}

            \item \xmlNode{threshold}: \xmlDesc{float}, 
              Features with a training-set variance lower than this threshold                   will
              be removed. The default is to keep all features with non-zero
              variance, i.e. remove the features that have the same value in all
              samples.
  \default{0.0}
          \end{itemize}
      \end{itemize}

    \item \xmlNode{featureSpaceTransformation}:
      Use dimensionality reduction technique to perform a trasformation of the training dataset
      into an uncorrelated one. The dimensionality of the problem will not be reduced but
      the data will be transformed in the transformed space. E.g if the number of features
      are 5, the method projects such features into a new uncorrelated space (still 5-dimensional).
      In case of time-dependent ROMs, all the samples are concatenated in a global 2D matrix
      (n\_samples*n\_timesteps,n\_features) before applying the transformation and then reconstructed
      back into the original shape (before fitting the model).

      The \xmlNode{featureSpaceTransformation} node recognizes the following subnodes:
      \begin{itemize}
        \item \xmlNode{transformationMethod}: \xmlDesc{[PCA, KernelLinearPCA, KernelPolyPCA, KernelRbfPCA, KernelSigmoidPCA, KernelCosinePCA, ICA]}, 
          Transformation method to use. Eight options (5 Kernel PCAs) are available:
          \begin{itemize}                     \item \textit{PCA}, Principal Component Analysis;
          \item \textit{KernelLinearPCA}, Kernel (Linear) Principal component analysis;
          \item \textit{KernelPolyPCA}, Kernel (Poly) Principal component analysis;
          \item \textit{KernelRbfPCA}, Kernel(Rbf) Principal component analysis;
          \item \textit{KernelSigmoidPCA}, Kernel (Sigmoid) Principal component analysis;
          \item \textit{KernelCosinePCA}, Kernel (Cosine) Principal component analysis;
          \item \textit{ICA}, Independent component analysis;                    \end{itemize}
  \default{PCA}

        \item \xmlNode{parametersToInclude}: \xmlDesc{comma-separated strings}, 
          List of IDs of features/variables to include in the transformation process.
  \default{None}

        \item \xmlNode{whichSpace}: \xmlDesc{[Feature, feature, Target, target]}, 
          Which space to search? Target or Feature?
  \default{Feature}
      \end{itemize}

    \item \xmlNode{CV}: \xmlDesc{string}, 
      The text portion of this node needs to contain the name of the \xmlNode{PostProcessor} with
      \xmlAttr{subType}         ``CrossValidation``.
      The \xmlNode{CV} node recognizes the following parameters:
        \begin{itemize}
          \item \xmlAttr{class}: \xmlDesc{string, optional}, 
            should be set to \xmlString{Model}
          \item \xmlAttr{type}: \xmlDesc{string, optional}, 
            should be set to \xmlString{PostProcessor}
      \end{itemize}

    \item \xmlNode{alias}: \xmlDesc{string}, 
      specifies alias for         any variable of interest in the input or output space. These
      aliases can be used anywhere in the RAVEN input to         refer to the variables. In the body
      of this node the user specifies the name of the variable that the model is going to use
      (during its execution).
      The \xmlNode{alias} node recognizes the following parameters:
        \begin{itemize}
          \item \xmlAttr{variable}: \xmlDesc{string, required}, 
            define the actual alias, usable throughout the RAVEN input
          \item \xmlAttr{type}: \xmlDesc{[input, output], required}, 
            either ``input'' or ``output''.
      \end{itemize}

    \item \xmlNode{dmdType}: \xmlDesc{[dmd, hodmd]}, 
      the type of Dynamic Mode Decomposition to apply.Available are:
      \begin{itemize}                                                     \item \textit{dmd}, for
      classical DMD                                                     \item \textit{hodmd}, for
      high order DMD.                                                   \end{itemize}
  \default{dmd}

    \item \xmlNode{pivotParameter}: \xmlDesc{string}, 
      defines the pivot variable (e.g., time) that represents the
      independent monotonic variable
  \default{time}

    \item \xmlNode{rankSVD}: \xmlDesc{integer}, 
      defines the truncation rank to be used for the SVD.
      Available options are:                                                  \begin{itemize}
      \item \textit{-1}, no truncation is performed
      \item \textit{0}, optimal rank is internally computed
      \item \textit{>1}, this rank is going to be used for the truncation
      \end{itemize}
  \default{None}

    \item \xmlNode{energyRankSVD}: \xmlDesc{float}, 
      energy level ($0.0 < float < 1.0$) used to compute the rank such
      as computed rank is the number of the biggest singular values needed to reach the energy
      identified by                                                    \xmlNode{energyRankSVD}. This
      node has always priority over  \xmlNode{rankSVD}
  \default{None}

    \item \xmlNode{rankTLSQ}: \xmlDesc{integer}, 
      $int > 0$ that defines the truncation rank to be used for the total
      least square problem. If not inputted, no truncation is applied
  \default{None}

    \item \xmlNode{exactModes}: \xmlDesc{[True, Yes, 1, False, No, 0, t, y, 1, f, n, 0]}, 
      True if the exact modes need to be computed (eigenvalues and
      eigenvectors),   otherwise the projected ones (using the left-singular matrix after SVD).
  \default{True}

    \item \xmlNode{optimized}: \xmlDesc{float}, 
      True if the amplitudes need to be computed minimizing the error
      between the modes and all the time-steps or False, if only the 1st timestep only needs to be
      considered
  \default{False}
  \end{itemize}

\hspace{24pt}
Example:
\textbf{Example:}
\begin{lstlisting}[style=XML,morekeywords={name,subType}]
<Simulation>
  ...
  <Models>
    ...
   <ROM name='DMD' subType='DMD'>
      <Target>time,totals_watts, xe135_dens</Target>
      <Features>enrichment,bu</Features>
      <dmdType>dmd</dmdType>
      <pivotParameter>time</pivotParameter>
      <rankSVD>0</rankSVD>
      <rankTLSQ>5</rankTLSQ>
      <exactModes>False</exactModes>
      <optimized>True</optimized>
    </ROM
    ...
  </Models>
  ...
</Simulation>
\end{lstlisting}

Example to export the coefficients of trained DMD ROM:
\begin{lstlisting}[style=XML,morekeywords={name,subType}]
<Simulation>
  ...
  <OutStreams>
    ...
    <Print name = 'dumpAllCoefficients'>
      <type>xml</type>
      <source>DMD</source>
      <!--
        here the <what> node is omitted. All the available params/coefficients
        are going to be printed out
      -->
    </Print>
    <Print name = 'dumpSomeCoefficients'>
      <type>xml</type>
      <source>DMD</source>
      <what>eigs,amplitudes,modes</what>
    </Print>
    ...
  </OutStreams>
  ...
</Simulation>
\end{lstlisting}


\subsubsection{DMDC}
  The \xmlString{DMDC} contains a single ROM type similar to DMD, aimed to         construct a time-
  dependent surrogate model based on Dynamic         Mode Decomposition with Control (ref.
  \cite{proctor2016dynamic}).         In addition to perform a ``dimensionality reduction
  regression'' like DMD, this surrogate will         calculate the state-space representation
  matrices A, B and  C in a discrete time domain:         \begin{itemize}           \item
  $x[k+1]=A*x[k]+B*u[k]$           \item $y[k+1]=C*x[k+1]$         \end{itemize}          In order
  to use this Reduced Order Model, the \xmlNode{ROM} attribute         \xmlAttr{subType} needs to be
  set equal to \xmlString{DMDC}.         \\         Once the ROM  is trained (\textbf{Step}
  \xmlNode{RomTrainer}), its         parameters/coefficients can be exported into an XML file
  via an \xmlNode{OutStream} of type \xmlAttr{Print}. The following variable/parameters can be
  exported (i.e.         \xmlNode{what} node         in \xmlNode{OutStream} of type
  \xmlAttr{Print}):         \begin{itemize}           \item \xmlNode{rankSVD}, see XML input
  specifications below           \item \xmlNode{actuators}, XML node containing the list of actuator
  variables (u),                 see XML input specifications below           \item
  \xmlNode{stateVariables}, XML node containing the list of system state variables (x),
  see XML input specifications below           \item \xmlNode{initStateVariables}, XML node
  containing the list of system state variables                 (x\_init) that are used for
  initializing the model in ``evaluation'' mode,                 see XML input specifications below
  \item \xmlNode{outputs}, XML node containing the list of system output variables (y)
  \item \xmlNode{dmdTimeScale}, XML node containing the the array of time scale in the DMD space,
  which is time axis in traning data (Time)           \item \xmlNode{UNorm}, XML node containing the
  norminal values of actuators,                 which are the initial actuator values in the
  training data           \item \xmlNode{XNorm}, XML node containing the norminal values of state
  variables,                 which are the initial state values in the training data           \item
  \xmlNode{XLast}, XML node containing the last value of state variables,                 which are
  the final state values in the training data (before nominal value subtraction)           \item
  \xmlNode{YNorm}, XML node containing the norminal values of output variables,
  which are the initial output values in the training data           \item \xmlNode{Atilde},  XML
  node containing the A matrix in discrete time domain                 (imaginary part, matrix
  shape, and real part)           \item \xmlNode{Btilde}, XML node containing the B matrix in
  discrete time domain                 (imaginary part, matrix shape, and real part)           \item
  \xmlNode{Ctilde}, XML node containing the C matrix in discrete time domain
  (imaginary part, matrix shape, and real part)         \end{itemize}

  The \xmlNode{DMDC} node recognizes the following parameters:
    \begin{itemize}
      \item \xmlAttr{name}: \xmlDesc{string, required}, 
        User-defined name to designate this entity in the RAVEN input file.
      \item \xmlAttr{verbosity}: \xmlDesc{[silent, quiet, all, debug], optional}, 
        Desired verbosity of messages coming from this entity
      \item \xmlAttr{subType}: \xmlDesc{string, required}, 
        specify the type of ROM that will be used
  \end{itemize}

  The \xmlNode{DMDC} node recognizes the following subnodes:
  \begin{itemize}
    \item \xmlNode{Features}: \xmlDesc{comma-separated strings}, 
      specifies the names of the features of this ROM.         \nb These parameters are going to be
      requested for the training of this object         (see Section~\ref{subsec:stepRomTrainer})

    \item \xmlNode{Target}: \xmlDesc{comma-separated strings}, 
      contains a comma separated list of the targets of this ROM. These parameters         are the
      Figures of Merit (FOMs) this ROM is supposed to predict.         \nb These parameters are
      going to be requested for the training of this         object (see Section
      \ref{subsec:stepRomTrainer}).

    \item \xmlNode{pivotParameter}: \xmlDesc{string}, 
      If a time-dependent ROM is requested, please specifies the pivot         variable (e.g. time,
      etc) used in the input HistorySet.
  \default{time}

    \item \xmlNode{featureSelection}:
      Apply feature selection algorithm

      The \xmlNode{featureSelection} node recognizes the following subnodes:
      \begin{itemize}
        \item \xmlNode{RFE}:
          The \xmlString{RFE} (Recursive Feature Elimination) is a feature selection algorithm.
          Feature selection refers to techniques that select a subset of the most relevant features
          for a model (ROM).         Fewer features can allow ROMs to run more efficiently (less
          space or time complexity) and be more effective.         Indeed, some ROMs (machine
          learning algorithms) can be misled by irrelevant input features, resulting in worse
          predictive performance.         RFE is a wrapper-type feature selection algorithm. This
          means that a different ROM is given and used in the core of the         method,         is
          wrapped by RFE, and used to help select features.         \\RFE works by searching for a
          subset of features by starting with all features in the training dataset and successfully
          removing         features until the desired number remains.         This is achieved by
          fitting the given ROM used in the core of the model, ranking features by importance,
          discarding the least important features, and re-fitting the model. This process is
          repeated until a specified number of         features remains.         When the full model
          is created, a measure of variable importance is computed that ranks the predictors from
          most         important to least.         At each stage of the search, the least important
          predictors are iteratively eliminated prior to rebuilding the model.         Features are
          scored either using the ROM model (if the model provides a mean to compute feature
          importances) or by         using a statistical method.         \\In RAVEN the
          \xmlString{RFE} class refers to an augmentation of the basic algorithm, since it allows,
          optionally,         to perform the search on multiple groups of targets (separately) and
          then combine the results of the search in a         single set. In addition, when the RFE
          search is concluded, the user can request to identify the set of features         that
          bring to a minimization of the score (i.e. maximimization of the accuracy).         In
          addition, using the ``applyClusteringFiltering'' option, the algorithm can, using an
          hierarchal clustering algorithm,         identify highly correlated features to speed up
          the subsequential search.
          The \xmlNode{RFE} node recognizes the following parameters:
            \begin{itemize}
              \item \xmlAttr{name}: \xmlDesc{string, required}, 
                User-defined name to designate this entity in the RAVEN input file.
              \item \xmlAttr{verbosity}: \xmlDesc{[silent, quiet, all, debug], optional}, 
                Desired verbosity of messages coming from this entity
          \end{itemize}

          The \xmlNode{RFE} node recognizes the following subnodes:
          \begin{itemize}
            \item \xmlNode{parametersToInclude}: \xmlDesc{comma-separated strings}, 
              List of IDs of features/variables to include in the search.
  \default{None}

            \item \xmlNode{whichSpace}: \xmlDesc{[Feature, feature, Target, target]}, 
              Which space to search? Target or Feature (this is temporary till DataSet training is
              implemented)
  \default{feature}

            \item \xmlNode{nFeaturesToSelect}: \xmlDesc{integer}, 
              Exact Number of features to select. If not inputted, ``nFeaturesToSelect'' will be set
              to $1/2$ of the features in the training dataset.
  \default{None}

            \item \xmlNode{maxNumberFeatures}: \xmlDesc{integer}, 
              Maximum Number of features to select, the algorithm will automatically determine the
              feature list to minimize a total score.
  \default{None}

            \item \xmlNode{onlyOutputScore}: \xmlDesc{[True, Yes, 1, False, No, 0, t, y, 1, f, n, 0]}, 
              If maxNumberFeatures is on, only output score should beconsidered? Or, in case of
              particular models (e.g. DMDC), state variable space score should be considered as
              well.
  \default{False}

            \item \xmlNode{applyClusteringFiltering}: \xmlDesc{[True, Yes, 1, False, No, 0, t, y, 1, f, n, 0]}, 
              Applying clustering correlation before RFE search? If true, an hierarchal clustering
              is applied on the feature         space aimed to remove features that are correlated
              before the actual RFE search is performed. This approach can stabilize and
              accelerate the process in case of large feature spaces (e.g > 500 features).
  \default{False}

            \item \xmlNode{applyCrossCorrelation}: \xmlDesc{[True, Yes, 1, False, No, 0, t, y, 1, f, n, 0]}, 
              In case of subgroupping, should a cross correleation analysis should be performed
              cross sub-groups?         If it is activated, a cross correleation analysis is used to
              additionally filter the features selected for each         sub-groupping search.
  \default{False}

            \item \xmlNode{step}: \xmlDesc{float}, 
              If greater than or equal to 1, then step corresponds to the (integer) number
              of features to remove at each iteration. If within (0.0, 1.0), then step
              corresponds to the percentage (rounded down) of features to remove at         each
              iteration.
  \default{1}

            \item \xmlNode{subGroup}: \xmlDesc{comma-separated strings, integers, and floats}, 
              Subgroup of output variables on which to perform the search. Multiple nodes of this
              type can be inputted. The RFE search will be then performed in each ``subgroup''
              separately and then the the union of the different feature sets are used for the final
              ROM.
          \end{itemize}

        \item \xmlNode{VarianceThreshold}:
          The \xmlString{VarianceThreshold} is a feature selector that removes     all low-variance
          features. This feature selection algorithm looks only at the features and not     the
          desired outputs. The variance threshold can be set by the user.
          The \xmlNode{VarianceThreshold} node recognizes the following parameters:
            \begin{itemize}
              \item \xmlAttr{name}: \xmlDesc{string, required}, 
                User-defined name to designate this entity in the RAVEN input file.
              \item \xmlAttr{verbosity}: \xmlDesc{[silent, quiet, all, debug], optional}, 
                Desired verbosity of messages coming from this entity
          \end{itemize}

          The \xmlNode{VarianceThreshold} node recognizes the following subnodes:
          \begin{itemize}
            \item \xmlNode{parametersToInclude}: \xmlDesc{comma-separated strings}, 
              List of IDs of features/variables to include in the search.
  \default{None}

            \item \xmlNode{whichSpace}: \xmlDesc{[Feature, feature, Target, target]}, 
              Which space to search? Target or Feature (this is temporary till DataSet training is
              implemented)
  \default{feature}

            \item \xmlNode{threshold}: \xmlDesc{float}, 
              Features with a training-set variance lower than this threshold                   will
              be removed. The default is to keep all features with non-zero
              variance, i.e. remove the features that have the same value in all
              samples.
  \default{0.0}
          \end{itemize}
      \end{itemize}

    \item \xmlNode{featureSpaceTransformation}:
      Use dimensionality reduction technique to perform a trasformation of the training dataset
      into an uncorrelated one. The dimensionality of the problem will not be reduced but
      the data will be transformed in the transformed space. E.g if the number of features
      are 5, the method projects such features into a new uncorrelated space (still 5-dimensional).
      In case of time-dependent ROMs, all the samples are concatenated in a global 2D matrix
      (n\_samples*n\_timesteps,n\_features) before applying the transformation and then reconstructed
      back into the original shape (before fitting the model).

      The \xmlNode{featureSpaceTransformation} node recognizes the following subnodes:
      \begin{itemize}
        \item \xmlNode{transformationMethod}: \xmlDesc{[PCA, KernelLinearPCA, KernelPolyPCA, KernelRbfPCA, KernelSigmoidPCA, KernelCosinePCA, ICA]}, 
          Transformation method to use. Eight options (5 Kernel PCAs) are available:
          \begin{itemize}                     \item \textit{PCA}, Principal Component Analysis;
          \item \textit{KernelLinearPCA}, Kernel (Linear) Principal component analysis;
          \item \textit{KernelPolyPCA}, Kernel (Poly) Principal component analysis;
          \item \textit{KernelRbfPCA}, Kernel(Rbf) Principal component analysis;
          \item \textit{KernelSigmoidPCA}, Kernel (Sigmoid) Principal component analysis;
          \item \textit{KernelCosinePCA}, Kernel (Cosine) Principal component analysis;
          \item \textit{ICA}, Independent component analysis;                    \end{itemize}
  \default{PCA}

        \item \xmlNode{parametersToInclude}: \xmlDesc{comma-separated strings}, 
          List of IDs of features/variables to include in the transformation process.
  \default{None}

        \item \xmlNode{whichSpace}: \xmlDesc{[Feature, feature, Target, target]}, 
          Which space to search? Target or Feature?
  \default{Feature}
      \end{itemize}

    \item \xmlNode{CV}: \xmlDesc{string}, 
      The text portion of this node needs to contain the name of the \xmlNode{PostProcessor} with
      \xmlAttr{subType}         ``CrossValidation``.
      The \xmlNode{CV} node recognizes the following parameters:
        \begin{itemize}
          \item \xmlAttr{class}: \xmlDesc{string, optional}, 
            should be set to \xmlString{Model}
          \item \xmlAttr{type}: \xmlDesc{string, optional}, 
            should be set to \xmlString{PostProcessor}
      \end{itemize}

    \item \xmlNode{alias}: \xmlDesc{string}, 
      specifies alias for         any variable of interest in the input or output space. These
      aliases can be used anywhere in the RAVEN input to         refer to the variables. In the body
      of this node the user specifies the name of the variable that the model is going to use
      (during its execution).
      The \xmlNode{alias} node recognizes the following parameters:
        \begin{itemize}
          \item \xmlAttr{variable}: \xmlDesc{string, required}, 
            define the actual alias, usable throughout the RAVEN input
          \item \xmlAttr{type}: \xmlDesc{[input, output], required}, 
            either ``input'' or ``output''.
      \end{itemize}

    \item \xmlNode{pivotParameter}: \xmlDesc{string}, 
      defines the pivot variable (e.g., time) that represents the
      independent monotonic variable
  \default{time}

    \item \xmlNode{rankSVD}: \xmlDesc{integer}, 
      defines the truncation rank to be used for the SVD.
      Available options are:                                                  \begin{itemize}
      \item \textit{-1}, no truncation is performed
      \item \textit{0}, optimal rank is internally computed
      \item \textit{>1}, this rank is going to be used for the truncation
      \end{itemize}
  \default{None}

    \item \xmlNode{energyRankSVD}: \xmlDesc{float}, 
      energy level ($0.0 < float < 1.0$) used to compute the rank such
      as computed rank is the number of the biggest singular values needed to reach the energy
      identified by                                                    \xmlNode{energyRankSVD}. This
      node has always priority over  \xmlNode{rankSVD}
  \default{None}

    \item \xmlNode{rankTLSQ}: \xmlDesc{integer}, 
      $int > 0$ that defines the truncation rank to be used for the total
      least square problem. If not inputted, no truncation is applied
  \default{None}

    \item \xmlNode{exactModes}: \xmlDesc{[True, Yes, 1, False, No, 0, t, y, 1, f, n, 0]}, 
      True if the exact modes need to be computed (eigenvalues and
      eigenvectors),   otherwise the projected ones (using the left-singular matrix after SVD).
  \default{True}

    \item \xmlNode{optimized}: \xmlDesc{float}, 
      True if the amplitudes need to be computed minimizing the error
      between the modes and all the time-steps or False, if only the 1st timestep only needs to be
      considered
  \default{False}

    \item \xmlNode{actuators}: \xmlDesc{comma-separated strings}, 
      defines the actuators (i.e. system input parameters)
      of this model. Each actuator variable (u1, u2, etc.) needs to
      be listed here.

    \item \xmlNode{stateVariables}: \xmlDesc{comma-separated strings}, 
      defines the state variables (i.e. system variable vectors)
      of this model. Each state variable (x1, x2, etc.) needs to be listed
      here. The variables indicated in \xmlNode{stateVariables} must be
      listed in the \xmlNode{Target} node too.

    \item \xmlNode{initStateVariables}: \xmlDesc{comma-separated strings}, 
      defines the state variables' ids  that should be used as
      initialization variable                                                   in the evaluation
      stage (for the evaluation of the model).
      These variables are used for the first time step to initiate
      the rolling time-step prediction of the state variables, ``exited''
      by the \xmlNode{actuators} signal. The variables listed in
      \xmlNode{initStateVariables} must be listed in the  \xmlNode{Features}
      node too.                                                   \nb The
      \xmlNode{initStateVariables} MUST be named appending ``\_init'' to
      the stateVariables listed in \xmlNode{stateVariables} XML node
  \default{[]}

    \item \xmlNode{subtractNormUXY}: \xmlDesc{[True, Yes, 1, False, No, 0, t, y, 1, f, n, 0]}, 
      True if the initial values need to be subtracted from the
      actuators (u), state (x) and outputs (y) if any. False if the subtraction
      is not needed.
  \default{False}

    \item \xmlNode{singleValuesTruncationTol}: \xmlDesc{float}, 
      Truncation threshold to apply to singular values vector
  \default{1e-09}
  \end{itemize}

\hspace{24pt}
Example of DMDc ROM definition, with 1 actuator variable (u1), 3 state variables (x1, x2, x3), 2 output variables (y1, y2), and 2 scheduling parameters (mod, flow):
\begin{lstlisting}[style=XML,morekeywords={name,subType}]
<Simulation>
   ...
   <Models>
     ...
    <ROM name="DMDrom" subType="DMDC">
      <!-- Target contains Time, StateVariable Names (x) and OutputVariable Names (y) in training data -->
      <Target>Time,x1,x2,x3,y1,y2</Target>
      <!-- Actuator Variable Names (u) -->
      <actuators>u1</actuators>
      <!-- StateVariables Names (x) -->
      <stateVariables>x1,x2,x3</stateVariables>
      <!-- Pivot variable (e.g. Time) -->
      <pivotParameter>Time</pivotParameter>
      <!-- rankSVD: -1 = No truncation; 0 = optimized truncation; pos. int = truncation level -->
      <rankSVD>1</rankSVD>
      <!-- SubtractNormUXY: True = will subtract the initial values from U,X,Y -->
      <subtractNormUXY>True</subtractNormUXY>

      <!-- Features are the variable names for predictions: Actuator "u", scheduling parameters, and initial states -->
      <Features>u1,mod,flow,x1_init,x2_init,x3_init</Features>
      <!-- Initialization Variables-->
      <initStateVariables>
        x1_init,x2_init,x3_init
      </initStateVariables>
    </ROM>
     ...
   </Models>
   ...
 </Simulation>

\end{lstlisting}

Example to export the coefficients of trained DMDC ROM:
\begin{lstlisting}[style=XML,morekeywords={name,subType}]
<Simulation>
  ...
  <OutStreams>
    ...
    <Print name = 'dumpAllCoefficients'>
      <type>xml</type>
      <source>DMDc</source>
      <!--
        here the <what> node is omitted. All the available params/coefficients
        are going to be printed out
      -->
    </Print>
    <Print name = 'dumpSomeCoefficients'>
      <type>xml</type>
      <source>DMDc</source>
      <what>rankSVD,UNorm,XNorm,XLast,Atilde,Btilde</what>
    </Print>
    ...
  </OutStreams>
  ...
</Simulation>
\end{lstlisting}




\subsubsection{LinearDiscriminantAnalysisClassifier}
  The \xmlNode{LinearDiscriminantAnalysisClassifier} is a classifier with a linear decision
  boundary,     generated by fitting class conditional densities to the data and using Bayes' rule.
  The model fits a Gaussian density to each class, assuming that all classes share the same
  covariance matrix.     The fitted model can also be used to reduce the dimensionality of the input
  by projecting it to the most discriminative     directions, using the transform method.
  \zNormalizationNotPerformed{LinearDiscriminantAnalysisClassifier}

  The \xmlNode{LinearDiscriminantAnalysisClassifier} node recognizes the following parameters:
    \begin{itemize}
      \item \xmlAttr{name}: \xmlDesc{string, required}, 
        User-defined name to designate this entity in the RAVEN input file.
      \item \xmlAttr{verbosity}: \xmlDesc{[silent, quiet, all, debug], optional}, 
        Desired verbosity of messages coming from this entity
      \item \xmlAttr{subType}: \xmlDesc{string, required}, 
        specify the type of ROM that will be used
  \end{itemize}

  The \xmlNode{LinearDiscriminantAnalysisClassifier} node recognizes the following subnodes:
  \begin{itemize}
    \item \xmlNode{Features}: \xmlDesc{comma-separated strings}, 
      specifies the names of the features of this ROM.         \nb These parameters are going to be
      requested for the training of this object         (see Section~\ref{subsec:stepRomTrainer})

    \item \xmlNode{Target}: \xmlDesc{comma-separated strings}, 
      contains a comma separated list of the targets of this ROM. These parameters         are the
      Figures of Merit (FOMs) this ROM is supposed to predict.         \nb These parameters are
      going to be requested for the training of this         object (see Section
      \ref{subsec:stepRomTrainer}).

    \item \xmlNode{pivotParameter}: \xmlDesc{string}, 
      If a time-dependent ROM is requested, please specifies the pivot         variable (e.g. time,
      etc) used in the input HistorySet.
  \default{time}

    \item \xmlNode{CV}: \xmlDesc{string}, 
      The text portion of this node needs to contain the name of the \xmlNode{PostProcessor} with
      \xmlAttr{subType}         ``CrossValidation``.
      The \xmlNode{CV} node recognizes the following parameters:
        \begin{itemize}
          \item \xmlAttr{class}: \xmlDesc{string, optional}, 
            should be set to \xmlString{Model}
          \item \xmlAttr{type}: \xmlDesc{string, optional}, 
            should be set to \xmlString{PostProcessor}
      \end{itemize}

    \item \xmlNode{alias}: \xmlDesc{string}, 
      specifies alias for         any variable of interest in the input or output space. These
      aliases can be used anywhere in the RAVEN input to         refer to the variables. In the body
      of this node the user specifies the name of the variable that the model is going to use
      (during its execution).
      The \xmlNode{alias} node recognizes the following parameters:
        \begin{itemize}
          \item \xmlAttr{variable}: \xmlDesc{string, required}, 
            define the actual alias, usable throughout the RAVEN input
          \item \xmlAttr{type}: \xmlDesc{[input, output], required}, 
            either ``input'' or ``output''.
      \end{itemize}

    \item \xmlNode{solver}: \xmlDesc{string}, 
      Solver to use, possible values:
      \begin{itemize}                                                    \item svd: Singular value
      decomposition (default). Does not compute the covariance matrix,
      therefore this solver is recommended for data with a large number of features.
      \item lsqr: Least squares solution. Can be combined with shrinkage or custom covariance
      estimator.                                                    \item eigen: Eigenvalue
      decomposition. Can be combined with shrinkage or custom covariance estimator.
      \end{itemize}
  \default{svd}

    \item \xmlNode{Shrinkage}: \xmlDesc{float or string}, 
      Shrinkage parameter, possible values: 1) None: no shrinkage (default),
      2) `auto': automatic shrinkage using the Ledoit-Wolf lemma,
      3) float between 0 an d1: fixed shrinkage parameter.
      This should be left to None if covariance\_estimator is used. Note that shrinkage works
      only with `lsqr' and `eigen' solvers.
  \default{None}

    \item \xmlNode{priors}: \xmlDesc{comma-separated floats}, 
      The class prior probabilities. By default, the class proportions are inferred from the
      training data.
  \default{None}

    \item \xmlNode{n\_components}: \xmlDesc{integer}, 
      Number of components (<= min(n\_classes - 1, n\_features)) for dimensionality reduction.
      If None, will be set to min(n\_classes - 1, n\_features). This parameter only affects the
      transform                                                  method.
  \default{None}

    \item \xmlNode{store\_covariance}: \xmlDesc{[True, Yes, 1, False, No, 0, t, y, 1, f, n, 0]}, 
      If True, explicitely compute the weighted within-class covariance matrix when solver
      is `svd'. The matrix is always computed and stored for the other solvers.
  \default{False}

    \item \xmlNode{tol}: \xmlDesc{float}, 
      Absolute threshold for a singular value of X to be considered significant, used to estimate
      the rank of X.                                                  Dimensions whose singular
      values are non-significant are discarded. Only used if solver is `svd'.
  \default{0.0001}

    \item \xmlNode{covariance\_estimator}: \xmlDesc{integer}, 
      covariance estimator (not supported)
  \default{None}
  \end{itemize}


\subsubsection{QuadraticDiscriminantAnalysisClassifier}
  The \xmlNode{QuadraticDiscriminantAnalysisClassifier} is a classifier with a quadratic decision
  boundary,     generated by fitting class conditional densities to the data and using Bayes' rule.
  The model fits a Gaussian density to each class
  \zNormalizationNotPerformed{QuadraticDiscriminantAnalysisClassifier}

  The \xmlNode{QuadraticDiscriminantAnalysisClassifier} node recognizes the following parameters:
    \begin{itemize}
      \item \xmlAttr{name}: \xmlDesc{string, required}, 
        User-defined name to designate this entity in the RAVEN input file.
      \item \xmlAttr{verbosity}: \xmlDesc{[silent, quiet, all, debug], optional}, 
        Desired verbosity of messages coming from this entity
      \item \xmlAttr{subType}: \xmlDesc{string, required}, 
        specify the type of ROM that will be used
  \end{itemize}

  The \xmlNode{QuadraticDiscriminantAnalysisClassifier} node recognizes the following subnodes:
  \begin{itemize}
    \item \xmlNode{Features}: \xmlDesc{comma-separated strings}, 
      specifies the names of the features of this ROM.         \nb These parameters are going to be
      requested for the training of this object         (see Section~\ref{subsec:stepRomTrainer})

    \item \xmlNode{Target}: \xmlDesc{comma-separated strings}, 
      contains a comma separated list of the targets of this ROM. These parameters         are the
      Figures of Merit (FOMs) this ROM is supposed to predict.         \nb These parameters are
      going to be requested for the training of this         object (see Section
      \ref{subsec:stepRomTrainer}).

    \item \xmlNode{pivotParameter}: \xmlDesc{string}, 
      If a time-dependent ROM is requested, please specifies the pivot         variable (e.g. time,
      etc) used in the input HistorySet.
  \default{time}

    \item \xmlNode{CV}: \xmlDesc{string}, 
      The text portion of this node needs to contain the name of the \xmlNode{PostProcessor} with
      \xmlAttr{subType}         ``CrossValidation``.
      The \xmlNode{CV} node recognizes the following parameters:
        \begin{itemize}
          \item \xmlAttr{class}: \xmlDesc{string, optional}, 
            should be set to \xmlString{Model}
          \item \xmlAttr{type}: \xmlDesc{string, optional}, 
            should be set to \xmlString{PostProcessor}
      \end{itemize}

    \item \xmlNode{alias}: \xmlDesc{string}, 
      specifies alias for         any variable of interest in the input or output space. These
      aliases can be used anywhere in the RAVEN input to         refer to the variables. In the body
      of this node the user specifies the name of the variable that the model is going to use
      (during its execution).
      The \xmlNode{alias} node recognizes the following parameters:
        \begin{itemize}
          \item \xmlAttr{variable}: \xmlDesc{string, required}, 
            define the actual alias, usable throughout the RAVEN input
          \item \xmlAttr{type}: \xmlDesc{[input, output], required}, 
            either ``input'' or ``output''.
      \end{itemize}

    \item \xmlNode{priors}: \xmlDesc{comma-separated floats}, 
      The class prior probabilities. By default, the class
      proportions are inferred from the training data.
  \default{None}

    \item \xmlNode{reg\_param}: \xmlDesc{float}, 
      Regularizes the per-class covariance estimates by transforming
      S2 as S2 = (1 - reg\_param) * S2 + reg\_param * np.eye(n\_features),
      where S2 corresponds to the                                                  scaling\_
      attribute of a given class.
  \default{0.0}

    \item \xmlNode{store\_covariance}: \xmlDesc{[True, Yes, 1, False, No, 0, t, y, 1, f, n, 0]}, 
      If True, the class covariance matrices are explicitely
      computed and stored in the self.covariance\_ attribute.
  \default{False}

    \item \xmlNode{tol}: \xmlDesc{float}, 
      Absolute threshold for a singular value to be considered
      significant, used to estimate the rank of Xk where Xk is the centered
      matrix of samples in class k. This parameter does not affect
      the predictions. It only controls a warning that is raised when
      features are considered to be colinear.
  \default{0.0001}
  \end{itemize}


\subsubsection{ARDRegression}
  The \xmlNode{ARDRegression} is Bayesian ARD regression.                             Fit the
  weights of a regression model, using an ARD prior. The weights of the
  regression model are assumed to be in Gaussian distributions. Also estimate the
  parameters lambda (precisions of the distributions of the weights) and
  alpha (precision of the distribution of the noise).                             The estimation is
  done by an iterative procedures (Evidence Maximization).
  \zNormalizationNotPerformed{ARDRegression}

  The \xmlNode{ARDRegression} node recognizes the following parameters:
    \begin{itemize}
      \item \xmlAttr{name}: \xmlDesc{string, required}, 
        User-defined name to designate this entity in the RAVEN input file.
      \item \xmlAttr{verbosity}: \xmlDesc{[silent, quiet, all, debug], optional}, 
        Desired verbosity of messages coming from this entity
      \item \xmlAttr{subType}: \xmlDesc{string, required}, 
        specify the type of ROM that will be used
  \end{itemize}

  The \xmlNode{ARDRegression} node recognizes the following subnodes:
  \begin{itemize}
    \item \xmlNode{Features}: \xmlDesc{comma-separated strings}, 
      specifies the names of the features of this ROM.         \nb These parameters are going to be
      requested for the training of this object         (see Section~\ref{subsec:stepRomTrainer})

    \item \xmlNode{Target}: \xmlDesc{comma-separated strings}, 
      contains a comma separated list of the targets of this ROM. These parameters         are the
      Figures of Merit (FOMs) this ROM is supposed to predict.         \nb These parameters are
      going to be requested for the training of this         object (see Section
      \ref{subsec:stepRomTrainer}).

    \item \xmlNode{pivotParameter}: \xmlDesc{string}, 
      If a time-dependent ROM is requested, please specifies the pivot         variable (e.g. time,
      etc) used in the input HistorySet.
  \default{time}

    \item \xmlNode{CV}: \xmlDesc{string}, 
      The text portion of this node needs to contain the name of the \xmlNode{PostProcessor} with
      \xmlAttr{subType}         ``CrossValidation``.
      The \xmlNode{CV} node recognizes the following parameters:
        \begin{itemize}
          \item \xmlAttr{class}: \xmlDesc{string, optional}, 
            should be set to \xmlString{Model}
          \item \xmlAttr{type}: \xmlDesc{string, optional}, 
            should be set to \xmlString{PostProcessor}
      \end{itemize}

    \item \xmlNode{alias}: \xmlDesc{string}, 
      specifies alias for         any variable of interest in the input or output space. These
      aliases can be used anywhere in the RAVEN input to         refer to the variables. In the body
      of this node the user specifies the name of the variable that the model is going to use
      (during its execution).
      The \xmlNode{alias} node recognizes the following parameters:
        \begin{itemize}
          \item \xmlAttr{variable}: \xmlDesc{string, required}, 
            define the actual alias, usable throughout the RAVEN input
          \item \xmlAttr{type}: \xmlDesc{[input, output], required}, 
            either ``input'' or ``output''.
      \end{itemize}

    \item \xmlNode{n\_iter}: \xmlDesc{integer}, 
      Maximum number of iterations.
  \default{300}

    \item \xmlNode{tol}: \xmlDesc{float}, 
      Tolerance for stopping criterion
  \default{0.001}

    \item \xmlNode{alpha\_1}: \xmlDesc{float}, 
      Hyper-parameter : shape parameter for the Gamma
      distribution prior over the alpha parameter.
  \default{1e-06}

    \item \xmlNode{alpha\_2}: \xmlDesc{float}, 
      Hyper-parameter : inverse scale parameter (rate parameter)
      for the Gamma distribution prior over the alpha parameter.
  \default{1e-06}

    \item \xmlNode{lambda\_1}: \xmlDesc{float}, 
      Hyper-parameter : shape parameter for the Gamma distribution
      prior over the lambda parameter.
  \default{1e-06}

    \item \xmlNode{lambda\_2}: \xmlDesc{float}, 
      Hyper-parameter : inverse scale parameter (rate parameter) for
      the Gamma distribution prior over the lambda parameter.
  \default{1e-06}

    \item \xmlNode{compute\_score}: \xmlDesc{[True, Yes, 1, False, No, 0, t, y, 1, f, n, 0]}, 
      If True, compute the objective function at each step of the
      model.
  \default{False}

    \item \xmlNode{threshold\_lambda}: \xmlDesc{float}, 
      threshold for removing (pruning) weights with
      shigh precision from the computation..
  \default{10000}

    \item \xmlNode{fit\_intercept}: \xmlDesc{[True, Yes, 1, False, No, 0, t, y, 1, f, n, 0]}, 
      Whether to calculate the intercept for this model. Specifies if a constant (a.k.a. bias or
      intercept)                                                   should be added to the decision
      function.
  \default{True}

    \item \xmlNode{normalize}: \xmlDesc{[True, Yes, 1, False, No, 0, t, y, 1, f, n, 0]}, 
      This parameter is ignored when fit\_intercept is set to False. If True,
      the regressors X will be normalized before regression by subtracting the mean and
      dividing by the l2-norm.
  \default{True}

    \item \xmlNode{verbose}: \xmlDesc{[True, Yes, 1, False, No, 0, t, y, 1, f, n, 0]}, 
      Verbose mode when fitting the model.
  \default{False}
  \end{itemize}


\subsubsection{BayesianRidge}
  The \xmlNode{BayesianRidge} is Bayesian Ridge regression.                         It estimates a
  probabilistic model of the regression problem as                         described above. The
  prior for the coefficient is given by a                         spherical Gaussian:
  $p(w|\lambda) = \mathcal{N}(w|0,\lambda^{-1}\mathbf{I}\_{p})$                         The
  parameters $w$, $\alpha$ and $\lambda$ are estimated jointly during                         the
  fit of the model, the regularization parameters $\alpha$ and $\lambda$
  being estimated by maximizing the log marginal likelihood.
  \zNormalizationNotPerformed{BayesianRidge}

  The \xmlNode{BayesianRidge} node recognizes the following parameters:
    \begin{itemize}
      \item \xmlAttr{name}: \xmlDesc{string, required}, 
        User-defined name to designate this entity in the RAVEN input file.
      \item \xmlAttr{verbosity}: \xmlDesc{[silent, quiet, all, debug], optional}, 
        Desired verbosity of messages coming from this entity
      \item \xmlAttr{subType}: \xmlDesc{string, required}, 
        specify the type of ROM that will be used
  \end{itemize}

  The \xmlNode{BayesianRidge} node recognizes the following subnodes:
  \begin{itemize}
    \item \xmlNode{Features}: \xmlDesc{comma-separated strings}, 
      specifies the names of the features of this ROM.         \nb These parameters are going to be
      requested for the training of this object         (see Section~\ref{subsec:stepRomTrainer})

    \item \xmlNode{Target}: \xmlDesc{comma-separated strings}, 
      contains a comma separated list of the targets of this ROM. These parameters         are the
      Figures of Merit (FOMs) this ROM is supposed to predict.         \nb These parameters are
      going to be requested for the training of this         object (see Section
      \ref{subsec:stepRomTrainer}).

    \item \xmlNode{pivotParameter}: \xmlDesc{string}, 
      If a time-dependent ROM is requested, please specifies the pivot         variable (e.g. time,
      etc) used in the input HistorySet.
  \default{time}

    \item \xmlNode{CV}: \xmlDesc{string}, 
      The text portion of this node needs to contain the name of the \xmlNode{PostProcessor} with
      \xmlAttr{subType}         ``CrossValidation``.
      The \xmlNode{CV} node recognizes the following parameters:
        \begin{itemize}
          \item \xmlAttr{class}: \xmlDesc{string, optional}, 
            should be set to \xmlString{Model}
          \item \xmlAttr{type}: \xmlDesc{string, optional}, 
            should be set to \xmlString{PostProcessor}
      \end{itemize}

    \item \xmlNode{alias}: \xmlDesc{string}, 
      specifies alias for         any variable of interest in the input or output space. These
      aliases can be used anywhere in the RAVEN input to         refer to the variables. In the body
      of this node the user specifies the name of the variable that the model is going to use
      (during its execution).
      The \xmlNode{alias} node recognizes the following parameters:
        \begin{itemize}
          \item \xmlAttr{variable}: \xmlDesc{string, required}, 
            define the actual alias, usable throughout the RAVEN input
          \item \xmlAttr{type}: \xmlDesc{[input, output], required}, 
            either ``input'' or ``output''.
      \end{itemize}

    \item \xmlNode{n\_iter}: \xmlDesc{integer}, 
      Maximum number of iterations.
  \default{300}

    \item \xmlNode{tol}: \xmlDesc{float}, 
      Tolerance for stopping criterion
  \default{0.001}

    \item \xmlNode{alpha\_1}: \xmlDesc{float}, 
      Hyper-parameter : shape parameter for the Gamma
      distribution prior over the alpha parameter.
  \default{1e-06}

    \item \xmlNode{alpha\_2}: \xmlDesc{float}, 
      Hyper-parameter : inverse scale parameter (rate parameter)
      for the Gamma distribution prior over the alpha parameter.
  \default{1e-06}

    \item \xmlNode{lambda\_1}: \xmlDesc{float}, 
      Hyper-parameter : shape parameter for the Gamma distribution
      prior over the lambda parameter.
  \default{1e-06}

    \item \xmlNode{lambda\_2}: \xmlDesc{float}, 
      Hyper-parameter : inverse scale parameter (rate parameter) for
      the Gamma distribution prior over the lambda parameter.
  \default{1e-06}

    \item \xmlNode{alpha\_init}: \xmlDesc{float}, 
      Initial value for alpha (precision of the noise).
      If not set, alpha\_init is $1/Var(y)$.
  \default{None}

    \item \xmlNode{lambda\_init}: \xmlDesc{float}, 
      Initial value for lambda (precision of the weights).
  \default{1.0}

    \item \xmlNode{compute\_score}: \xmlDesc{[True, Yes, 1, False, No, 0, t, y, 1, f, n, 0]}, 
      If True, compute the objective function at each step of the
      model.
  \default{False}

    \item \xmlNode{fit\_intercept}: \xmlDesc{[True, Yes, 1, False, No, 0, t, y, 1, f, n, 0]}, 
      Whether to calculate the intercept for this model. Specifies if a constant (a.k.a. bias or
      intercept)                                                   should be added to the decision
      function.
  \default{True}

    \item \xmlNode{normalize}: \xmlDesc{[True, Yes, 1, False, No, 0, t, y, 1, f, n, 0]}, 
      This parameter is ignored when fit\_intercept is set to False. If True,
      the regressors X will be normalized before regression by subtracting the mean and
      dividing by the l2-norm.
  \default{False}

    \item \xmlNode{verbose}: \xmlDesc{[True, Yes, 1, False, No, 0, t, y, 1, f, n, 0]}, 
      Verbose mode when fitting the model.
  \default{False}
  \end{itemize}


\subsubsection{ElasticNet}
  The \xmlNode{ElasticNet} employs                         Linear regression with combined L1 and L2
  priors as regularizer.                         It minimizes the objective function:
  \begin{equation}                         1/(2*n\_{samples}) *||y - Xw||^2\_2+alpha*l1\_ratio*||w||\_1
  + 0.5 *alpha*(1 - l1\_ratio)*||w||^2\_2                         \end{equation}
  \zNormalizationNotPerformed{ElasticNet}

  The \xmlNode{ElasticNet} node recognizes the following parameters:
    \begin{itemize}
      \item \xmlAttr{name}: \xmlDesc{string, required}, 
        User-defined name to designate this entity in the RAVEN input file.
      \item \xmlAttr{verbosity}: \xmlDesc{[silent, quiet, all, debug], optional}, 
        Desired verbosity of messages coming from this entity
      \item \xmlAttr{subType}: \xmlDesc{string, required}, 
        specify the type of ROM that will be used
  \end{itemize}

  The \xmlNode{ElasticNet} node recognizes the following subnodes:
  \begin{itemize}
    \item \xmlNode{Features}: \xmlDesc{comma-separated strings}, 
      specifies the names of the features of this ROM.         \nb These parameters are going to be
      requested for the training of this object         (see Section~\ref{subsec:stepRomTrainer})

    \item \xmlNode{Target}: \xmlDesc{comma-separated strings}, 
      contains a comma separated list of the targets of this ROM. These parameters         are the
      Figures of Merit (FOMs) this ROM is supposed to predict.         \nb These parameters are
      going to be requested for the training of this         object (see Section
      \ref{subsec:stepRomTrainer}).

    \item \xmlNode{pivotParameter}: \xmlDesc{string}, 
      If a time-dependent ROM is requested, please specifies the pivot         variable (e.g. time,
      etc) used in the input HistorySet.
  \default{time}

    \item \xmlNode{CV}: \xmlDesc{string}, 
      The text portion of this node needs to contain the name of the \xmlNode{PostProcessor} with
      \xmlAttr{subType}         ``CrossValidation``.
      The \xmlNode{CV} node recognizes the following parameters:
        \begin{itemize}
          \item \xmlAttr{class}: \xmlDesc{string, optional}, 
            should be set to \xmlString{Model}
          \item \xmlAttr{type}: \xmlDesc{string, optional}, 
            should be set to \xmlString{PostProcessor}
      \end{itemize}

    \item \xmlNode{alias}: \xmlDesc{string}, 
      specifies alias for         any variable of interest in the input or output space. These
      aliases can be used anywhere in the RAVEN input to         refer to the variables. In the body
      of this node the user specifies the name of the variable that the model is going to use
      (during its execution).
      The \xmlNode{alias} node recognizes the following parameters:
        \begin{itemize}
          \item \xmlAttr{variable}: \xmlDesc{string, required}, 
            define the actual alias, usable throughout the RAVEN input
          \item \xmlAttr{type}: \xmlDesc{[input, output], required}, 
            either ``input'' or ``output''.
      \end{itemize}

    \item \xmlNode{tol}: \xmlDesc{float}, 
      Tolerance for stopping criterion
  \default{0.0001}

    \item \xmlNode{alpha}: \xmlDesc{float}, 
      specifies a constant                                                  that multiplies the
      penalty terms.                                                  $alpha = 0$ is equivalent to
      an ordinary least square, solved by the
      \textbf{LinearRegression} object.
  \default{1.0}

    \item \xmlNode{l1\_ratio}: \xmlDesc{float}, 
      specifies the                                                  ElasticNet mixing parameter,
      with $0 <= l1\_ratio <= 1$.                                                  For $l1\_ratio =
      0$ the penalty is an L2 penalty.                                                  For
      $l1\_ratio = 1$ it is an L1 penalty.                                                  For $0 <
      l1\_ratio < 1$, the penalty is a combination of L1 and L2.
  \default{0.5}

    \item \xmlNode{fit\_intercept}: \xmlDesc{[True, Yes, 1, False, No, 0, t, y, 1, f, n, 0]}, 
      Whether the intercept should be estimated or not. If False,
      the data is assumed to be already centered.
  \default{True}

    \item \xmlNode{precompute}: \xmlDesc{[True, Yes, 1, False, No, 0, t, y, 1, f, n, 0]}, 
      Whether to use a precomputed Gram matrix to speed up calculations.
      For sparse input this option is always True to preserve sparsity.
  \default{False}

    \item \xmlNode{max\_iter}: \xmlDesc{integer}, 
      The maximum number of iterations.
  \default{1000}

    \item \xmlNode{positive}: \xmlDesc{[True, Yes, 1, False, No, 0, t, y, 1, f, n, 0]}, 
      When set to True, forces the coefficients to be positive.
  \default{True}

    \item \xmlNode{selection}: \xmlDesc{[cyclic, random]}, 
      If set to ``random'', a random coefficient is updated every iteration
      rather than looping over features sequentially by default. This (setting to `random'')
      often leads to significantly faster convergence especially when tol is higher than $1e-4$
  \default{cyclic}

    \item \xmlNode{normalize}: \xmlDesc{[True, Yes, 1, False, No, 0, t, y, 1, f, n, 0]}, 
      This parameter is ignored when fit\_intercept is set to False. If True,
      the regressors X will be normalized before regression by subtracting the mean and
      dividing by the l2-norm.
  \default{False}

    \item \xmlNode{warm\_start}: \xmlDesc{[True, Yes, 1, False, No, 0, t, y, 1, f, n, 0]}, 
      When set to True, reuse the solution of the previous call
      to fit as initialization, otherwise, just erase the previous solution.
  \default{False}
  \end{itemize}


\subsubsection{ElasticNetCV}
  The \xmlNode{ElasticNetCV} employs                         Linear regression with combined L1 and
  L2 priors as regularizer.                         This model is similar to the
  \xmlNode{ElasticNet}                         with the addition of an iterative fitting along a
  regularization path (via cross-validation).
  \zNormalizationNotPerformed{ElasticNetCV}

  The \xmlNode{ElasticNetCV} node recognizes the following parameters:
    \begin{itemize}
      \item \xmlAttr{name}: \xmlDesc{string, required}, 
        User-defined name to designate this entity in the RAVEN input file.
      \item \xmlAttr{verbosity}: \xmlDesc{[silent, quiet, all, debug], optional}, 
        Desired verbosity of messages coming from this entity
      \item \xmlAttr{subType}: \xmlDesc{string, required}, 
        specify the type of ROM that will be used
  \end{itemize}

  The \xmlNode{ElasticNetCV} node recognizes the following subnodes:
  \begin{itemize}
    \item \xmlNode{Features}: \xmlDesc{comma-separated strings}, 
      specifies the names of the features of this ROM.         \nb These parameters are going to be
      requested for the training of this object         (see Section~\ref{subsec:stepRomTrainer})

    \item \xmlNode{Target}: \xmlDesc{comma-separated strings}, 
      contains a comma separated list of the targets of this ROM. These parameters         are the
      Figures of Merit (FOMs) this ROM is supposed to predict.         \nb These parameters are
      going to be requested for the training of this         object (see Section
      \ref{subsec:stepRomTrainer}).

    \item \xmlNode{pivotParameter}: \xmlDesc{string}, 
      If a time-dependent ROM is requested, please specifies the pivot         variable (e.g. time,
      etc) used in the input HistorySet.
  \default{time}

    \item \xmlNode{CV}: \xmlDesc{string}, 
      The text portion of this node needs to contain the name of the \xmlNode{PostProcessor} with
      \xmlAttr{subType}         ``CrossValidation``.
      The \xmlNode{CV} node recognizes the following parameters:
        \begin{itemize}
          \item \xmlAttr{class}: \xmlDesc{string, optional}, 
            should be set to \xmlString{Model}
          \item \xmlAttr{type}: \xmlDesc{string, optional}, 
            should be set to \xmlString{PostProcessor}
      \end{itemize}

    \item \xmlNode{alias}: \xmlDesc{string}, 
      specifies alias for         any variable of interest in the input or output space. These
      aliases can be used anywhere in the RAVEN input to         refer to the variables. In the body
      of this node the user specifies the name of the variable that the model is going to use
      (during its execution).
      The \xmlNode{alias} node recognizes the following parameters:
        \begin{itemize}
          \item \xmlAttr{variable}: \xmlDesc{string, required}, 
            define the actual alias, usable throughout the RAVEN input
          \item \xmlAttr{type}: \xmlDesc{[input, output], required}, 
            either ``input'' or ``output''.
      \end{itemize}

    \item \xmlNode{tol}: \xmlDesc{float}, 
      Tolerance for stopping criterion
  \default{0.0001}

    \item \xmlNode{eps}: \xmlDesc{float}, 
      Length of the path. $eps=1e-3$ means that
      $alpha\_min / alpha\_max = 1e-3$.
  \default{0.001}

    \item \xmlNode{l1\_ratio}: \xmlDesc{float}, 
      specifies the                                                  float between 0 and 1 passed to
      ElasticNet (scaling between l1 and l2 penalties).
      For $l1\_ratio = 0$ the penalty is an L2 penalty. For $l1\_ratio = 1$ it is
      an L1 penalty. For $0 < l1\_ratio < 1$, the penalty is a combination of L1
      and L2 This parameter can be a list, in which case the different values
      are tested by cross-validation and the one giving the best prediction score
      is used. Note that a good choice of list of values for l1\_ratio is often to
      put more values close to 1 (i.e. Lasso) and less close to 0 (i.e. Ridge),
      as in $[.1, .5, .7, .9, .95, .99, 1]$.
  \default{0.5}

    \item \xmlNode{fit\_intercept}: \xmlDesc{[True, Yes, 1, False, No, 0, t, y, 1, f, n, 0]}, 
      Whether the intercept should be estimated or not. If False,
      the data is assumed to be already centered.
  \default{True}

    \item \xmlNode{precompute}: \xmlDesc{string}, 
      Whether to use a precomputed Gram matrix to speed up calculations.
      For sparse input this option is always True to preserve sparsity.
  \default{auto}

    \item \xmlNode{max\_iter}: \xmlDesc{integer}, 
      The maximum number of iterations.
  \default{1000}

    \item \xmlNode{cv}: \xmlDesc{integer}, 
      Determines the cross-validation splitting strategy.
      It specifies the number of folds.
  \default{None}

    \item \xmlNode{positive}: \xmlDesc{[True, Yes, 1, False, No, 0, t, y, 1, f, n, 0]}, 
      When set to True, forces the coefficients to be positive.
  \default{True}

    \item \xmlNode{selection}: \xmlDesc{[cyclic, random]}, 
      If set to ``random'', a random coefficient is updated every iteration
      rather than looping over features sequentially by default. This (setting to `random'')
      often leads to significantly faster convergence especially when tol is higher than $1e-4$
  \default{cyclic}

    \item \xmlNode{normalize}: \xmlDesc{[True, Yes, 1, False, No, 0, t, y, 1, f, n, 0]}, 
      This parameter is ignored when fit\_intercept is set to False. If True,
      the regressors X will be normalized before regression by subtracting the mean and
      dividing by the l2-norm.
  \default{False}

    \item \xmlNode{n\_alphas}: \xmlDesc{integer}, 
      Number of alphas along the regularization path,
      used for each l1\_ratio.
  \default{100}
  \end{itemize}


\subsubsection{Lars}
  The \xmlNode{Lars} (\textit{Least Angle Regression model})                         is a regression
  algorithm for high-dimensional data.                         The LARS algorithm provides a means
  of producing an estimate of which variables                         to include, as well as their
  coefficients, when a response variable is                         determined by a linear
  combination of a subset of potential covariates.
  \zNormalizationNotPerformed{Lars}

  The \xmlNode{Lars} node recognizes the following parameters:
    \begin{itemize}
      \item \xmlAttr{name}: \xmlDesc{string, required}, 
        User-defined name to designate this entity in the RAVEN input file.
      \item \xmlAttr{verbosity}: \xmlDesc{[silent, quiet, all, debug], optional}, 
        Desired verbosity of messages coming from this entity
      \item \xmlAttr{subType}: \xmlDesc{string, required}, 
        specify the type of ROM that will be used
  \end{itemize}

  The \xmlNode{Lars} node recognizes the following subnodes:
  \begin{itemize}
    \item \xmlNode{Features}: \xmlDesc{comma-separated strings}, 
      specifies the names of the features of this ROM.         \nb These parameters are going to be
      requested for the training of this object         (see Section~\ref{subsec:stepRomTrainer})

    \item \xmlNode{Target}: \xmlDesc{comma-separated strings}, 
      contains a comma separated list of the targets of this ROM. These parameters         are the
      Figures of Merit (FOMs) this ROM is supposed to predict.         \nb These parameters are
      going to be requested for the training of this         object (see Section
      \ref{subsec:stepRomTrainer}).

    \item \xmlNode{pivotParameter}: \xmlDesc{string}, 
      If a time-dependent ROM is requested, please specifies the pivot         variable (e.g. time,
      etc) used in the input HistorySet.
  \default{time}

    \item \xmlNode{CV}: \xmlDesc{string}, 
      The text portion of this node needs to contain the name of the \xmlNode{PostProcessor} with
      \xmlAttr{subType}         ``CrossValidation``.
      The \xmlNode{CV} node recognizes the following parameters:
        \begin{itemize}
          \item \xmlAttr{class}: \xmlDesc{string, optional}, 
            should be set to \xmlString{Model}
          \item \xmlAttr{type}: \xmlDesc{string, optional}, 
            should be set to \xmlString{PostProcessor}
      \end{itemize}

    \item \xmlNode{alias}: \xmlDesc{string}, 
      specifies alias for         any variable of interest in the input or output space. These
      aliases can be used anywhere in the RAVEN input to         refer to the variables. In the body
      of this node the user specifies the name of the variable that the model is going to use
      (during its execution).
      The \xmlNode{alias} node recognizes the following parameters:
        \begin{itemize}
          \item \xmlAttr{variable}: \xmlDesc{string, required}, 
            define the actual alias, usable throughout the RAVEN input
          \item \xmlAttr{type}: \xmlDesc{[input, output], required}, 
            either ``input'' or ``output''.
      \end{itemize}

    \item \xmlNode{eps}: \xmlDesc{float}, 
      The machine-precision regularization in the computation of the Cholesky
      diagonal factors. Increase this for very ill-conditioned systems. Unlike the tol
      parameter in some iterative optimization-based algorithms, this parameter does not
      control the tolerance of the optimization.
  \default{2.220446049250313e-16}

    \item \xmlNode{fit\_intercept}: \xmlDesc{[True, Yes, 1, False, No, 0, t, y, 1, f, n, 0]}, 
      Whether the intercept should be estimated or not. If False,
      the data is assumed to be already centered.
  \default{True}

    \item \xmlNode{precompute}: \xmlDesc{string}, 
      Whether to use a precomputed Gram matrix to speed up calculations.
      For sparse input this option is always True to preserve sparsity.
  \default{auto}

    \item \xmlNode{normalize}: \xmlDesc{[True, Yes, 1, False, No, 0, t, y, 1, f, n, 0]}, 
      This parameter is ignored when fit\_intercept is set to False. If True,
      the regressors X will be normalized before regression by subtracting the mean and
      dividing by the l2-norm.
  \default{True}

    \item \xmlNode{n\_nonzero\_coefs}: \xmlDesc{integer}, 
      Target number of non-zero coefficients.
  \default{500}

    \item \xmlNode{verbose}: \xmlDesc{[True, Yes, 1, False, No, 0, t, y, 1, f, n, 0]}, 
      Sets the verbosity amount.
  \default{False}

    \item \xmlNode{fit\_path}: \xmlDesc{[True, Yes, 1, False, No, 0, t, y, 1, f, n, 0]}, 
      If True the full path is stored in the coef\_path\_ attribute.
      If you compute the solution for a large problem or many targets,
      setting fit\_path to False will lead to a speedup, especially with a
      small alpha.
  \default{True}
  \end{itemize}


\subsubsection{LarsCV}
  The \xmlNode{LarsCV} is Cross-validated \textit{Least Angle Regression model} model
  is a regression algorithm for high-dimensional data.                         The LARS algorithm
  provides a means of producing an estimate of which variables                         to include,
  as well as their coefficients, when a response variable is                         determined by a
  linear combination of a subset of potential covariates.                         This method is an
  augmentation of the Lars method with the addition of cross-validation
  embedded tecniques.                         \zNormalizationNotPerformed{LarsCV}

  The \xmlNode{LarsCV} node recognizes the following parameters:
    \begin{itemize}
      \item \xmlAttr{name}: \xmlDesc{string, required}, 
        User-defined name to designate this entity in the RAVEN input file.
      \item \xmlAttr{verbosity}: \xmlDesc{[silent, quiet, all, debug], optional}, 
        Desired verbosity of messages coming from this entity
      \item \xmlAttr{subType}: \xmlDesc{string, required}, 
        specify the type of ROM that will be used
  \end{itemize}

  The \xmlNode{LarsCV} node recognizes the following subnodes:
  \begin{itemize}
    \item \xmlNode{Features}: \xmlDesc{comma-separated strings}, 
      specifies the names of the features of this ROM.         \nb These parameters are going to be
      requested for the training of this object         (see Section~\ref{subsec:stepRomTrainer})

    \item \xmlNode{Target}: \xmlDesc{comma-separated strings}, 
      contains a comma separated list of the targets of this ROM. These parameters         are the
      Figures of Merit (FOMs) this ROM is supposed to predict.         \nb These parameters are
      going to be requested for the training of this         object (see Section
      \ref{subsec:stepRomTrainer}).

    \item \xmlNode{pivotParameter}: \xmlDesc{string}, 
      If a time-dependent ROM is requested, please specifies the pivot         variable (e.g. time,
      etc) used in the input HistorySet.
  \default{time}

    \item \xmlNode{CV}: \xmlDesc{string}, 
      The text portion of this node needs to contain the name of the \xmlNode{PostProcessor} with
      \xmlAttr{subType}         ``CrossValidation``.
      The \xmlNode{CV} node recognizes the following parameters:
        \begin{itemize}
          \item \xmlAttr{class}: \xmlDesc{string, optional}, 
            should be set to \xmlString{Model}
          \item \xmlAttr{type}: \xmlDesc{string, optional}, 
            should be set to \xmlString{PostProcessor}
      \end{itemize}

    \item \xmlNode{alias}: \xmlDesc{string}, 
      specifies alias for         any variable of interest in the input or output space. These
      aliases can be used anywhere in the RAVEN input to         refer to the variables. In the body
      of this node the user specifies the name of the variable that the model is going to use
      (during its execution).
      The \xmlNode{alias} node recognizes the following parameters:
        \begin{itemize}
          \item \xmlAttr{variable}: \xmlDesc{string, required}, 
            define the actual alias, usable throughout the RAVEN input
          \item \xmlAttr{type}: \xmlDesc{[input, output], required}, 
            either ``input'' or ``output''.
      \end{itemize}

    \item \xmlNode{eps}: \xmlDesc{float}, 
      The machine-precision regularization in the computation of the Cholesky
      diagonal factors. Increase this for very ill-conditioned systems. Unlike the tol
      parameter in some iterative optimization-based algorithms, this parameter does not
      control the tolerance of the optimization.
  \default{2.220446049250313e-16}

    \item \xmlNode{fit\_intercept}: \xmlDesc{[True, Yes, 1, False, No, 0, t, y, 1, f, n, 0]}, 
      Whether the intercept should be estimated or not. If False,
      the data is assumed to be already centered.
  \default{True}

    \item \xmlNode{precompute}: \xmlDesc{string}, 
      Whether to use a precomputed Gram matrix to speed up calculations.
      For sparse input this option is always True to preserve sparsity.
  \default{auto}

    \item \xmlNode{normalize}: \xmlDesc{[True, Yes, 1, False, No, 0, t, y, 1, f, n, 0]}, 
      This parameter is ignored when fit\_intercept is set to False. If True,
      the regressors X will be normalized before regression by subtracting the mean and
      dividing by the l2-norm.
  \default{True}

    \item \xmlNode{max\_n\_alphas}: \xmlDesc{integer}, 
      The maximum number of points on the path used to compute the
      residuals in the cross-validation.
  \default{1000}

    \item \xmlNode{cv}: \xmlDesc{integer}, 
      Determines the cross-validation splitting strategy.
      It specifies the number of folds..
  \default{None}

    \item \xmlNode{verbose}: \xmlDesc{[True, Yes, 1, False, No, 0, t, y, 1, f, n, 0]}, 
      Sets the verbosity amount.
  \default{False}

    \item \xmlNode{max\_iter}: \xmlDesc{integer}, 
      Maximum number of iterations to perform.
  \default{500}
  \end{itemize}


\subsubsection{Lasso}
  The \xmlNode{Lasso} (\textit{Linear Model trained with L1 prior as regularizer})
  is an algorithm for regression problem                         It minimizes the usual sum of
  squared errors, with a bound on the sum of the                         absolute values of the
  coefficients:                         \begin{equation}                          (1 / (2 *
  n\_samples)) * ||y - Xw||^2\_2 + alpha * ||w||\_1                         \end{equation}
  \zNormalizationNotPerformed{Lasso}

  The \xmlNode{Lasso} node recognizes the following parameters:
    \begin{itemize}
      \item \xmlAttr{name}: \xmlDesc{string, required}, 
        User-defined name to designate this entity in the RAVEN input file.
      \item \xmlAttr{verbosity}: \xmlDesc{[silent, quiet, all, debug], optional}, 
        Desired verbosity of messages coming from this entity
      \item \xmlAttr{subType}: \xmlDesc{string, required}, 
        specify the type of ROM that will be used
  \end{itemize}

  The \xmlNode{Lasso} node recognizes the following subnodes:
  \begin{itemize}
    \item \xmlNode{Features}: \xmlDesc{comma-separated strings}, 
      specifies the names of the features of this ROM.         \nb These parameters are going to be
      requested for the training of this object         (see Section~\ref{subsec:stepRomTrainer})

    \item \xmlNode{Target}: \xmlDesc{comma-separated strings}, 
      contains a comma separated list of the targets of this ROM. These parameters         are the
      Figures of Merit (FOMs) this ROM is supposed to predict.         \nb These parameters are
      going to be requested for the training of this         object (see Section
      \ref{subsec:stepRomTrainer}).

    \item \xmlNode{pivotParameter}: \xmlDesc{string}, 
      If a time-dependent ROM is requested, please specifies the pivot         variable (e.g. time,
      etc) used in the input HistorySet.
  \default{time}

    \item \xmlNode{CV}: \xmlDesc{string}, 
      The text portion of this node needs to contain the name of the \xmlNode{PostProcessor} with
      \xmlAttr{subType}         ``CrossValidation``.
      The \xmlNode{CV} node recognizes the following parameters:
        \begin{itemize}
          \item \xmlAttr{class}: \xmlDesc{string, optional}, 
            should be set to \xmlString{Model}
          \item \xmlAttr{type}: \xmlDesc{string, optional}, 
            should be set to \xmlString{PostProcessor}
      \end{itemize}

    \item \xmlNode{alias}: \xmlDesc{string}, 
      specifies alias for         any variable of interest in the input or output space. These
      aliases can be used anywhere in the RAVEN input to         refer to the variables. In the body
      of this node the user specifies the name of the variable that the model is going to use
      (during its execution).
      The \xmlNode{alias} node recognizes the following parameters:
        \begin{itemize}
          \item \xmlAttr{variable}: \xmlDesc{string, required}, 
            define the actual alias, usable throughout the RAVEN input
          \item \xmlAttr{type}: \xmlDesc{[input, output], required}, 
            either ``input'' or ``output''.
      \end{itemize}

    \item \xmlNode{alpha}: \xmlDesc{float}, 
      Constant that multiplies the L1 term. Defaults to 1.0.
      $alpha = 0$ is equivalent to an ordinary least square, solved by
      the LinearRegression object. For numerical reasons, using $alpha = 0$
      with the Lasso object is not advised.
  \default{1.0}

    \item \xmlNode{tol}: \xmlDesc{float}, 
      The tolerance for the optimization: if the updates are smaller
      than tol, the optimization code checks the dual gap for optimality and
      continues until it is smaller than tol..
  \default{0.0001}

    \item \xmlNode{fit\_intercept}: \xmlDesc{[True, Yes, 1, False, No, 0, t, y, 1, f, n, 0]}, 
      Whether the intercept should be estimated or not. If False,
      the data is assumed to be already centered.
  \default{True}

    \item \xmlNode{precompute}: \xmlDesc{[True, Yes, 1, False, No, 0, t, y, 1, f, n, 0]}, 
      Whether to use a precomputed Gram matrix to speed up calculations.
      For sparse input this option is always True to preserve sparsity.
  \default{False}

    \item \xmlNode{normalize}: \xmlDesc{[True, Yes, 1, False, No, 0, t, y, 1, f, n, 0]}, 
      This parameter is ignored when fit\_intercept is set to False. If True,
      the regressors X will be normalized before regression by subtracting the mean and
      dividing by the l2-norm.
  \default{False}

    \item \xmlNode{max\_iter}: \xmlDesc{integer}, 
      The maximum number of iterations.
  \default{1000}

    \item \xmlNode{positive}: \xmlDesc{[True, Yes, 1, False, No, 0, t, y, 1, f, n, 0]}, 
      When set to True, forces the coefficients to be positive.
  \default{False}

    \item \xmlNode{selection}: \xmlDesc{[cyclic, random]}, 
      If set to ``random'', a random coefficient is updated every iteration
      rather than looping over features sequentially by default. This (setting to `random'')
      often leads to significantly faster convergence especially when tol is higher than $1e-4$
  \default{cyclic}

    \item \xmlNode{warm\_start}: \xmlDesc{[True, Yes, 1, False, No, 0, t, y, 1, f, n, 0]}, 
      When set to True, reuse the solution of the previous call
      to fit as initialization, otherwise, just erase the previous solution.
  \default{False}
  \end{itemize}


\subsubsection{LassoCV}
  The \xmlNode{LassoCV} (\textit{Lasso linear model with iterative fitting along a regularization
  path})                         is an algorithm for regression problem. The best model is selected
  by cross-validation.                         It minimizes the usual sum of squared errors, with a
  bound on the sum of the                         absolute values of the coefficients:
  \begin{equation}                          (1 / (2 * n\_samples)) * ||y - Xw||^2\_2 + alpha *
  ||w||\_1                         \end{equation}
  \zNormalizationNotPerformed{LassoCV}

  The \xmlNode{LassoCV} node recognizes the following parameters:
    \begin{itemize}
      \item \xmlAttr{name}: \xmlDesc{string, required}, 
        User-defined name to designate this entity in the RAVEN input file.
      \item \xmlAttr{verbosity}: \xmlDesc{[silent, quiet, all, debug], optional}, 
        Desired verbosity of messages coming from this entity
      \item \xmlAttr{subType}: \xmlDesc{string, required}, 
        specify the type of ROM that will be used
  \end{itemize}

  The \xmlNode{LassoCV} node recognizes the following subnodes:
  \begin{itemize}
    \item \xmlNode{Features}: \xmlDesc{comma-separated strings}, 
      specifies the names of the features of this ROM.         \nb These parameters are going to be
      requested for the training of this object         (see Section~\ref{subsec:stepRomTrainer})

    \item \xmlNode{Target}: \xmlDesc{comma-separated strings}, 
      contains a comma separated list of the targets of this ROM. These parameters         are the
      Figures of Merit (FOMs) this ROM is supposed to predict.         \nb These parameters are
      going to be requested for the training of this         object (see Section
      \ref{subsec:stepRomTrainer}).

    \item \xmlNode{pivotParameter}: \xmlDesc{string}, 
      If a time-dependent ROM is requested, please specifies the pivot         variable (e.g. time,
      etc) used in the input HistorySet.
  \default{time}

    \item \xmlNode{CV}: \xmlDesc{string}, 
      The text portion of this node needs to contain the name of the \xmlNode{PostProcessor} with
      \xmlAttr{subType}         ``CrossValidation``.
      The \xmlNode{CV} node recognizes the following parameters:
        \begin{itemize}
          \item \xmlAttr{class}: \xmlDesc{string, optional}, 
            should be set to \xmlString{Model}
          \item \xmlAttr{type}: \xmlDesc{string, optional}, 
            should be set to \xmlString{PostProcessor}
      \end{itemize}

    \item \xmlNode{alias}: \xmlDesc{string}, 
      specifies alias for         any variable of interest in the input or output space. These
      aliases can be used anywhere in the RAVEN input to         refer to the variables. In the body
      of this node the user specifies the name of the variable that the model is going to use
      (during its execution).
      The \xmlNode{alias} node recognizes the following parameters:
        \begin{itemize}
          \item \xmlAttr{variable}: \xmlDesc{string, required}, 
            define the actual alias, usable throughout the RAVEN input
          \item \xmlAttr{type}: \xmlDesc{[input, output], required}, 
            either ``input'' or ``output''.
      \end{itemize}

    \item \xmlNode{tol}: \xmlDesc{float}, 
      Tolerance for stopping criterion
  \default{0.0001}

    \item \xmlNode{eps}: \xmlDesc{float}, 
      Length of the path. $eps=1e-3$ means that
      $alpha\_min / alpha\_max = 1e-3$.
  \default{0.001}

    \item \xmlNode{n\_alphas}: \xmlDesc{integer}, 
      The maximum number of iterations.
  \default{100}

    \item \xmlNode{fit\_intercept}: \xmlDesc{[True, Yes, 1, False, No, 0, t, y, 1, f, n, 0]}, 
      Whether the intercept should be estimated or not. If False,
      the data is assumed to be already centered.
  \default{True}

    \item \xmlNode{normalize}: \xmlDesc{[True, Yes, 1, False, No, 0, t, y, 1, f, n, 0]}, 
      This parameter is ignored when fit\_intercept is set to False. If True,
      the regressors X will be normalized before regression by subtracting the mean and
      dividing by the l2-norm.
  \default{False}

    \item \xmlNode{precompute}: \xmlDesc{string}, 
      Whether to use a precomputed Gram matrix to speed up calculations.
      For sparse input this option is always True to preserve sparsity.
  \default{auto}

    \item \xmlNode{max\_iter}: \xmlDesc{integer}, 
      The maximum number of iterations.
  \default{1000}

    \item \xmlNode{positive}: \xmlDesc{[True, Yes, 1, False, No, 0, t, y, 1, f, n, 0]}, 
      When set to True, forces the coefficients to be positive.
  \default{False}

    \item \xmlNode{selection}: \xmlDesc{[cyclic, random]}, 
      If set to ``random'', a random coefficient is updated every iteration
      rather than looping over features sequentially by default. This (setting to `random'')
      often leads to significantly faster convergence especially when tol is higher than $1e-4$
  \default{cyclic}

    \item \xmlNode{cv}: \xmlDesc{integer}, 
      Determines the cross-validation splitting strategy.
      It specifies the number of folds..
  \default{None}

    \item \xmlNode{alphas}: \xmlDesc{comma-separated floats}, 
      List of alphas where to compute the models. If None alphas
      are set automatically.
  \default{None}

    \item \xmlNode{verbose}: \xmlDesc{[True, Yes, 1, False, No, 0, t, y, 1, f, n, 0]}, 
      Amount of verbosity.
  \default{False}
  \end{itemize}


\subsubsection{LassoLars}
  The \xmlNode{LassoLars} (\textit{Lasso model fit with Least Angle Regression})
  It is a Linear Model trained with an L1 prior as regularizer.                         The
  optimization objective for Lasso is:                         \begin{equation}
  (1 / (2 * n\_samples)) * ||y - Xw||^2\_2 + alpha * ||w||\_1                         \end{equation}
  \zNormalizationNotPerformed{LassoLars}

  The \xmlNode{LassoLars} node recognizes the following parameters:
    \begin{itemize}
      \item \xmlAttr{name}: \xmlDesc{string, required}, 
        User-defined name to designate this entity in the RAVEN input file.
      \item \xmlAttr{verbosity}: \xmlDesc{[silent, quiet, all, debug], optional}, 
        Desired verbosity of messages coming from this entity
      \item \xmlAttr{subType}: \xmlDesc{string, required}, 
        specify the type of ROM that will be used
  \end{itemize}

  The \xmlNode{LassoLars} node recognizes the following subnodes:
  \begin{itemize}
    \item \xmlNode{Features}: \xmlDesc{comma-separated strings}, 
      specifies the names of the features of this ROM.         \nb These parameters are going to be
      requested for the training of this object         (see Section~\ref{subsec:stepRomTrainer})

    \item \xmlNode{Target}: \xmlDesc{comma-separated strings}, 
      contains a comma separated list of the targets of this ROM. These parameters         are the
      Figures of Merit (FOMs) this ROM is supposed to predict.         \nb These parameters are
      going to be requested for the training of this         object (see Section
      \ref{subsec:stepRomTrainer}).

    \item \xmlNode{pivotParameter}: \xmlDesc{string}, 
      If a time-dependent ROM is requested, please specifies the pivot         variable (e.g. time,
      etc) used in the input HistorySet.
  \default{time}

    \item \xmlNode{CV}: \xmlDesc{string}, 
      The text portion of this node needs to contain the name of the \xmlNode{PostProcessor} with
      \xmlAttr{subType}         ``CrossValidation``.
      The \xmlNode{CV} node recognizes the following parameters:
        \begin{itemize}
          \item \xmlAttr{class}: \xmlDesc{string, optional}, 
            should be set to \xmlString{Model}
          \item \xmlAttr{type}: \xmlDesc{string, optional}, 
            should be set to \xmlString{PostProcessor}
      \end{itemize}

    \item \xmlNode{alias}: \xmlDesc{string}, 
      specifies alias for         any variable of interest in the input or output space. These
      aliases can be used anywhere in the RAVEN input to         refer to the variables. In the body
      of this node the user specifies the name of the variable that the model is going to use
      (during its execution).
      The \xmlNode{alias} node recognizes the following parameters:
        \begin{itemize}
          \item \xmlAttr{variable}: \xmlDesc{string, required}, 
            define the actual alias, usable throughout the RAVEN input
          \item \xmlAttr{type}: \xmlDesc{[input, output], required}, 
            either ``input'' or ``output''.
      \end{itemize}

    \item \xmlNode{alpha}: \xmlDesc{float}, 
      Constant that multiplies the L1 term. Defaults to 1.0.
      $alpha = 0$ is equivalent to an ordinary least square, solved by
      the LinearRegression object. For numerical reasons, using $alpha = 0$
      with the Lasso object is not advised.
  \default{1.0}

    \item \xmlNode{fit\_intercept}: \xmlDesc{[True, Yes, 1, False, No, 0, t, y, 1, f, n, 0]}, 
      Whether the intercept should be estimated or not. If False,
      the data is assumed to be already centered.
  \default{True}

    \item \xmlNode{normalize}: \xmlDesc{[True, Yes, 1, False, No, 0, t, y, 1, f, n, 0]}, 
      This parameter is ignored when fit\_intercept is set to False. If True,
      the regressors X will be normalized before regression by subtracting the mean and
      dividing by the l2-norm.
  \default{False}

    \item \xmlNode{precompute}: \xmlDesc{string}, 
      Whether to use a precomputed Gram matrix to speed up calculations.
      For sparse input this option is always True to preserve sparsity.
  \default{auto}

    \item \xmlNode{max\_iter}: \xmlDesc{integer}, 
      The maximum number of iterations.
  \default{500}

    \item \xmlNode{eps}: \xmlDesc{float}, 
      The machine-precision regularization in the computation of the Cholesky
      diagonal factors. Increase this for very ill-conditioned systems. Unlike the tol
      parameter in some iterative optimization-based algorithms, this parameter does not
      control the tolerance of the optimization.
  \default{2.220446049250313e-16}

    \item \xmlNode{positive}: \xmlDesc{[True, Yes, 1, False, No, 0, t, y, 1, f, n, 0]}, 
      When set to True, forces the coefficients to be positive.
  \default{False}

    \item \xmlNode{verbose}: \xmlDesc{[True, Yes, 1, False, No, 0, t, y, 1, f, n, 0]}, 
      Amount of verbosity.
  \default{False}
  \end{itemize}


\subsubsection{LassoLarsCV}
  The \xmlNode{LassoLarsCV} (\textit{Cross-validated Lasso model fit with Least Angle Regression})
  This model is an augomentation of the LassoLars model with the addition of
  cross validation tecniques.                         The optimization objective for Lasso is:
  \begin{equation}                          (1 / (2 * n\_samples)) * ||y - Xw||^2\_2 + alpha *
  ||w||\_1                         \end{equation}
  \zNormalizationNotPerformed{LassoLarsCV}

  The \xmlNode{LassoLarsCV} node recognizes the following parameters:
    \begin{itemize}
      \item \xmlAttr{name}: \xmlDesc{string, required}, 
        User-defined name to designate this entity in the RAVEN input file.
      \item \xmlAttr{verbosity}: \xmlDesc{[silent, quiet, all, debug], optional}, 
        Desired verbosity of messages coming from this entity
      \item \xmlAttr{subType}: \xmlDesc{string, required}, 
        specify the type of ROM that will be used
  \end{itemize}

  The \xmlNode{LassoLarsCV} node recognizes the following subnodes:
  \begin{itemize}
    \item \xmlNode{Features}: \xmlDesc{comma-separated strings}, 
      specifies the names of the features of this ROM.         \nb These parameters are going to be
      requested for the training of this object         (see Section~\ref{subsec:stepRomTrainer})

    \item \xmlNode{Target}: \xmlDesc{comma-separated strings}, 
      contains a comma separated list of the targets of this ROM. These parameters         are the
      Figures of Merit (FOMs) this ROM is supposed to predict.         \nb These parameters are
      going to be requested for the training of this         object (see Section
      \ref{subsec:stepRomTrainer}).

    \item \xmlNode{pivotParameter}: \xmlDesc{string}, 
      If a time-dependent ROM is requested, please specifies the pivot         variable (e.g. time,
      etc) used in the input HistorySet.
  \default{time}

    \item \xmlNode{CV}: \xmlDesc{string}, 
      The text portion of this node needs to contain the name of the \xmlNode{PostProcessor} with
      \xmlAttr{subType}         ``CrossValidation``.
      The \xmlNode{CV} node recognizes the following parameters:
        \begin{itemize}
          \item \xmlAttr{class}: \xmlDesc{string, optional}, 
            should be set to \xmlString{Model}
          \item \xmlAttr{type}: \xmlDesc{string, optional}, 
            should be set to \xmlString{PostProcessor}
      \end{itemize}

    \item \xmlNode{alias}: \xmlDesc{string}, 
      specifies alias for         any variable of interest in the input or output space. These
      aliases can be used anywhere in the RAVEN input to         refer to the variables. In the body
      of this node the user specifies the name of the variable that the model is going to use
      (during its execution).
      The \xmlNode{alias} node recognizes the following parameters:
        \begin{itemize}
          \item \xmlAttr{variable}: \xmlDesc{string, required}, 
            define the actual alias, usable throughout the RAVEN input
          \item \xmlAttr{type}: \xmlDesc{[input, output], required}, 
            either ``input'' or ``output''.
      \end{itemize}

    \item \xmlNode{fit\_intercept}: \xmlDesc{[True, Yes, 1, False, No, 0, t, y, 1, f, n, 0]}, 
      Whether the intercept should be estimated or not. If False,
      the data is assumed to be already centered.
  \default{True}

    \item \xmlNode{max\_iter}: \xmlDesc{integer}, 
      The maximum number of iterations.
  \default{500}

    \item \xmlNode{normalize}: \xmlDesc{[True, Yes, 1, False, No, 0, t, y, 1, f, n, 0]}, 
      This parameter is ignored when fit\_intercept is set to False. If True,
      the regressors X will be normalized before regression by subtracting the mean and
      dividing by the l2-norm.
  \default{True}

    \item \xmlNode{precompute}: \xmlDesc{string}, 
      Whether to use a precomputed Gram matrix to speed up calculations.
      For sparse input this option is always True to preserve sparsity.
  \default{auto}

    \item \xmlNode{max\_n\_alphas}: \xmlDesc{integer}, 
      The maximum number of points on the path used to compute the residuals in
      the cross-validation
  \default{1000}

    \item \xmlNode{eps}: \xmlDesc{float}, 
      The machine-precision regularization in the computation of the Cholesky
      diagonal factors. Increase this for very ill-conditioned systems. Unlike the tol
      parameter in some iterative optimization-based algorithms, this parameter does not
      control the tolerance of the optimization.
  \default{2.220446049250313e-16}

    \item \xmlNode{positive}: \xmlDesc{[True, Yes, 1, False, No, 0, t, y, 1, f, n, 0]}, 
      When set to True, forces the coefficients to be positive.
  \default{False}

    \item \xmlNode{cv}: \xmlDesc{integer}, 
      Determines the cross-validation splitting strategy.
      It specifies the number of folds..
  \default{None}

    \item \xmlNode{verbose}: \xmlDesc{[True, Yes, 1, False, No, 0, t, y, 1, f, n, 0]}, 
      Amount of verbosity.
  \default{False}
  \end{itemize}


\subsubsection{LassoLarsIC}
  The \xmlNode{LassoLarsIC} (\textit{Lasso model fit with Lars using BIC or AIC for model
  selection})                         is a Lasso model fit with Lars using BIC or AIC for model
  selection.                         The optimization objective for Lasso is:
  $(1 / (2 * n\_samples)) * ||y - Xw||^2\_2 + alpha * ||w||\_1$                         AIC is the
  Akaike information criterion and BIC is the Bayes Information criterion. Such criteria
  are useful to select the value of the regularization parameter by making a trade-off between the
  goodness of fit and the complexity of the model. A good model should explain well the data
  while being simple.                         \zNormalizationNotPerformed{LassoLarsIC}

  The \xmlNode{LassoLarsIC} node recognizes the following parameters:
    \begin{itemize}
      \item \xmlAttr{name}: \xmlDesc{string, required}, 
        User-defined name to designate this entity in the RAVEN input file.
      \item \xmlAttr{verbosity}: \xmlDesc{[silent, quiet, all, debug], optional}, 
        Desired verbosity of messages coming from this entity
      \item \xmlAttr{subType}: \xmlDesc{string, required}, 
        specify the type of ROM that will be used
  \end{itemize}

  The \xmlNode{LassoLarsIC} node recognizes the following subnodes:
  \begin{itemize}
    \item \xmlNode{Features}: \xmlDesc{comma-separated strings}, 
      specifies the names of the features of this ROM.         \nb These parameters are going to be
      requested for the training of this object         (see Section~\ref{subsec:stepRomTrainer})

    \item \xmlNode{Target}: \xmlDesc{comma-separated strings}, 
      contains a comma separated list of the targets of this ROM. These parameters         are the
      Figures of Merit (FOMs) this ROM is supposed to predict.         \nb These parameters are
      going to be requested for the training of this         object (see Section
      \ref{subsec:stepRomTrainer}).

    \item \xmlNode{pivotParameter}: \xmlDesc{string}, 
      If a time-dependent ROM is requested, please specifies the pivot         variable (e.g. time,
      etc) used in the input HistorySet.
  \default{time}

    \item \xmlNode{CV}: \xmlDesc{string}, 
      The text portion of this node needs to contain the name of the \xmlNode{PostProcessor} with
      \xmlAttr{subType}         ``CrossValidation``.
      The \xmlNode{CV} node recognizes the following parameters:
        \begin{itemize}
          \item \xmlAttr{class}: \xmlDesc{string, optional}, 
            should be set to \xmlString{Model}
          \item \xmlAttr{type}: \xmlDesc{string, optional}, 
            should be set to \xmlString{PostProcessor}
      \end{itemize}

    \item \xmlNode{alias}: \xmlDesc{string}, 
      specifies alias for         any variable of interest in the input or output space. These
      aliases can be used anywhere in the RAVEN input to         refer to the variables. In the body
      of this node the user specifies the name of the variable that the model is going to use
      (during its execution).
      The \xmlNode{alias} node recognizes the following parameters:
        \begin{itemize}
          \item \xmlAttr{variable}: \xmlDesc{string, required}, 
            define the actual alias, usable throughout the RAVEN input
          \item \xmlAttr{type}: \xmlDesc{[input, output], required}, 
            either ``input'' or ``output''.
      \end{itemize}

    \item \xmlNode{criterion}: \xmlDesc{[bic, aic]}, 
      The type of criterion to use.
  \default{aic}

    \item \xmlNode{fit\_intercept}: \xmlDesc{[True, Yes, 1, False, No, 0, t, y, 1, f, n, 0]}, 
      Whether the intercept should be estimated or not. If False,
      the data is assumed to be already centered.
  \default{True}

    \item \xmlNode{normalize}: \xmlDesc{[True, Yes, 1, False, No, 0, t, y, 1, f, n, 0]}, 
      This parameter is ignored when fit\_intercept is set to False. If True,
      the regressors X will be normalized before regression by subtracting the mean and
      dividing by the l2-norm.
  \default{True}

    \item \xmlNode{max\_iter}: \xmlDesc{integer}, 
      The maximum number of iterations.
  \default{500}

    \item \xmlNode{precompute}: \xmlDesc{string}, 
      Whether to use a precomputed Gram matrix to speed up calculations.
      For sparse input this option is always True to preserve sparsity.
  \default{auto}

    \item \xmlNode{eps}: \xmlDesc{float}, 
      The machine-precision regularization in the computation of the Cholesky
      diagonal factors. Increase this for very ill-conditioned systems. Unlike the tol
      parameter in some iterative optimization-based algorithms, this parameter does not
      control the tolerance of the optimization.
  \default{2.220446049250313e-16}

    \item \xmlNode{positive}: \xmlDesc{[True, Yes, 1, False, No, 0, t, y, 1, f, n, 0]}, 
      When set to True, forces the coefficients to be positive.
  \default{False}

    \item \xmlNode{verbose}: \xmlDesc{[True, Yes, 1, False, No, 0, t, y, 1, f, n, 0]}, 
      Amount of verbosity.
  \default{False}
  \end{itemize}


\subsubsection{LinearRegression}
  The \xmlNode{LinearRegression}                         is an Ordinary least squares Linear
  Regression.                         LinearRegression fits a linear model with coefficients $w =
  (w1, …, wp)$ to                         minimize the residual sum of squares between the observed
  targets in the                         dataset, and the targets predicted by the linear
  approximation.                         \zNormalizationNotPerformed{LinearRegression}

  The \xmlNode{LinearRegression} node recognizes the following parameters:
    \begin{itemize}
      \item \xmlAttr{name}: \xmlDesc{string, required}, 
        User-defined name to designate this entity in the RAVEN input file.
      \item \xmlAttr{verbosity}: \xmlDesc{[silent, quiet, all, debug], optional}, 
        Desired verbosity of messages coming from this entity
      \item \xmlAttr{subType}: \xmlDesc{string, required}, 
        specify the type of ROM that will be used
  \end{itemize}

  The \xmlNode{LinearRegression} node recognizes the following subnodes:
  \begin{itemize}
    \item \xmlNode{Features}: \xmlDesc{comma-separated strings}, 
      specifies the names of the features of this ROM.         \nb These parameters are going to be
      requested for the training of this object         (see Section~\ref{subsec:stepRomTrainer})

    \item \xmlNode{Target}: \xmlDesc{comma-separated strings}, 
      contains a comma separated list of the targets of this ROM. These parameters         are the
      Figures of Merit (FOMs) this ROM is supposed to predict.         \nb These parameters are
      going to be requested for the training of this         object (see Section
      \ref{subsec:stepRomTrainer}).

    \item \xmlNode{pivotParameter}: \xmlDesc{string}, 
      If a time-dependent ROM is requested, please specifies the pivot         variable (e.g. time,
      etc) used in the input HistorySet.
  \default{time}

    \item \xmlNode{CV}: \xmlDesc{string}, 
      The text portion of this node needs to contain the name of the \xmlNode{PostProcessor} with
      \xmlAttr{subType}         ``CrossValidation``.
      The \xmlNode{CV} node recognizes the following parameters:
        \begin{itemize}
          \item \xmlAttr{class}: \xmlDesc{string, optional}, 
            should be set to \xmlString{Model}
          \item \xmlAttr{type}: \xmlDesc{string, optional}, 
            should be set to \xmlString{PostProcessor}
      \end{itemize}

    \item \xmlNode{alias}: \xmlDesc{string}, 
      specifies alias for         any variable of interest in the input or output space. These
      aliases can be used anywhere in the RAVEN input to         refer to the variables. In the body
      of this node the user specifies the name of the variable that the model is going to use
      (during its execution).
      The \xmlNode{alias} node recognizes the following parameters:
        \begin{itemize}
          \item \xmlAttr{variable}: \xmlDesc{string, required}, 
            define the actual alias, usable throughout the RAVEN input
          \item \xmlAttr{type}: \xmlDesc{[input, output], required}, 
            either ``input'' or ``output''.
      \end{itemize}

    \item \xmlNode{fit\_intercept}: \xmlDesc{[True, Yes, 1, False, No, 0, t, y, 1, f, n, 0]}, 
      Whether the intercept should be estimated or not. If False,
      the data is assumed to be already centered.
  \default{True}

    \item \xmlNode{normalize}: \xmlDesc{[True, Yes, 1, False, No, 0, t, y, 1, f, n, 0]}, 
      This parameter is ignored when fit\_intercept is set to False. If True,
      the regressors X will be normalized before regression by subtracting the mean and
      dividing by the l2-norm.
  \default{False}
  \end{itemize}


\subsubsection{LogisticRegression}
  The \xmlNode{LogisticRegression}  is                             a logit, MaxEnt classifier.
  In the multiclass case, the training algorithm uses the one-vs-rest (OvR) scheme
  if the ``multi\_class'' option is set to ``ovr'', and uses the cross-entropy loss if the
  ``multi\_class'' option is set to ``multinomial''. (Currently the ``multinomial'' option
  is supported only by the ``lbfgs'', ``sag'', ``saga'' and ``newton-cg'' solvers.)
  This class implements regularized logistic regression using the ``liblinear'' library, ``newton-
  cg'',                             ``sag'', ``saga'' and ``lbfgs'' solvers. Regularization is
  applied by default. It can handle both dense and sparse input.                             The
  ``newton-cg'', ``sag'', and ``lbfgs'' solvers support only L2 regularization with primal
  formulation,                             or no regularization. The ``liblinear'' solver supports
  both L1 and L2 regularization, with a dual formulation                             only for the L2
  penalty. The Elastic-Net regularization is only supported by the ``saga'' solver.
  \zNormalizationPerformed{LogisticRegression}

  The \xmlNode{LogisticRegression} node recognizes the following parameters:
    \begin{itemize}
      \item \xmlAttr{name}: \xmlDesc{string, required}, 
        User-defined name to designate this entity in the RAVEN input file.
      \item \xmlAttr{verbosity}: \xmlDesc{[silent, quiet, all, debug], optional}, 
        Desired verbosity of messages coming from this entity
      \item \xmlAttr{subType}: \xmlDesc{string, required}, 
        specify the type of ROM that will be used
  \end{itemize}

  The \xmlNode{LogisticRegression} node recognizes the following subnodes:
  \begin{itemize}
    \item \xmlNode{Features}: \xmlDesc{comma-separated strings}, 
      specifies the names of the features of this ROM.         \nb These parameters are going to be
      requested for the training of this object         (see Section~\ref{subsec:stepRomTrainer})

    \item \xmlNode{Target}: \xmlDesc{comma-separated strings}, 
      contains a comma separated list of the targets of this ROM. These parameters         are the
      Figures of Merit (FOMs) this ROM is supposed to predict.         \nb These parameters are
      going to be requested for the training of this         object (see Section
      \ref{subsec:stepRomTrainer}).

    \item \xmlNode{pivotParameter}: \xmlDesc{string}, 
      If a time-dependent ROM is requested, please specifies the pivot         variable (e.g. time,
      etc) used in the input HistorySet.
  \default{time}

    \item \xmlNode{CV}: \xmlDesc{string}, 
      The text portion of this node needs to contain the name of the \xmlNode{PostProcessor} with
      \xmlAttr{subType}         ``CrossValidation``.
      The \xmlNode{CV} node recognizes the following parameters:
        \begin{itemize}
          \item \xmlAttr{class}: \xmlDesc{string, optional}, 
            should be set to \xmlString{Model}
          \item \xmlAttr{type}: \xmlDesc{string, optional}, 
            should be set to \xmlString{PostProcessor}
      \end{itemize}

    \item \xmlNode{alias}: \xmlDesc{string}, 
      specifies alias for         any variable of interest in the input or output space. These
      aliases can be used anywhere in the RAVEN input to         refer to the variables. In the body
      of this node the user specifies the name of the variable that the model is going to use
      (during its execution).
      The \xmlNode{alias} node recognizes the following parameters:
        \begin{itemize}
          \item \xmlAttr{variable}: \xmlDesc{string, required}, 
            define the actual alias, usable throughout the RAVEN input
          \item \xmlAttr{type}: \xmlDesc{[input, output], required}, 
            either ``input'' or ``output''.
      \end{itemize}

    \item \xmlNode{penalty}: \xmlDesc{[l1, l2, elasticnet, none]}, 
      Used to specify the norm used in the penalization. The newton-cg, sag and lbfgs solvers
      support only l2 penalties. elasticnet is only supported by the saga solver. If none (
      not supported by the liblinear solver), no regularization is applied.
  \default{l2}

    \item \xmlNode{dual}: \xmlDesc{[True, Yes, 1, False, No, 0, t, y, 1, f, n, 0]}, 
      Select the algorithm to either solve the dual or primal optimization problem.
      Prefer dual=False when $n\_samples > n\_features$.
  \default{True}

    \item \xmlNode{C}: \xmlDesc{float}, 
      Regularization parameter. The strength of the regularization is inversely
      proportional to C.Must be strictly positive.
  \default{1.0}

    \item \xmlNode{tol}: \xmlDesc{float}, 
      Tolerance for stopping criterion
  \default{0.0001}

    \item \xmlNode{fit\_intercept}: \xmlDesc{[True, Yes, 1, False, No, 0, t, y, 1, f, n, 0]}, 
      Whether to calculate the intercept for this model. Specifies if a constant (a.k.a. bias or
      intercept) should be added to the decision function.
  \default{True}

    \item \xmlNode{intercept\_scaling}: \xmlDesc{float}, 
      When fit\_intercept is True, instance vector x becomes $[x, intercept\_scaling]$,
      i.e. a “synthetic” feature with constant value equals to intercept\_scaling is appended
      to the instance vector. The intercept becomes $intercept\_scaling * synthetic\_feature\_weight$
      \nb the synthetic feature weight is subject to $l1/l2$ regularization as all other features.
      To lessen the effect of regularization on synthetic feature weight (and therefore on the
      intercept)                                                  $intercept\_scaling$ has to be
      increased.
  \default{1.0}

    \item \xmlNode{solver}: \xmlDesc{[newton-cg, lbfgs, liblinear, sag, saga]}, 
      Algorithm to use in the optimization problem.
      \begin{itemize}                                                    \item For small datasets,
      ``liblinear'' is a good choice, whereas ``sag'' and ``saga'' are faster for large ones.
      \item For multiclass problems, only ``newton-cg'', ``sag'', ``saga'' and ``lbfgs'' handle
      multinomial loss; `                                                    `liblinear'' is limited
      to one-versus-rest schemes.                                                    \item ``newton-
      cg'', ``lbfgs'', ``sag'' and ``saga'' handle L2 or no penalty
      \item ``liblinear'' and ``saga'' also handle L1 penalty
      \item ``saga'' also supports ``elasticnet'' penalty
      \item ``liblinear'' does not support setting penalty=``none''
      \end{itemize}
  \default{lbfgs}

    \item \xmlNode{max\_iter}: \xmlDesc{integer}, 
      Hard limit on iterations within solver.``-1'' for no limit
  \default{100}

    \item \xmlNode{multi\_class}: \xmlDesc{[auto, ovr, multinomial]}, 
      If the option chosen is ``ovr'', then a binary problem is fit for each label. For
      ``multinomial''                                                  the loss minimised is the
      multinomial loss fit across the entire probability distribution, even when the
      data is binary. ``multinomial' is unavailable when solver=``liblinear''. ``auto'' selects
      ``ovr'' if the data is                                                  binary, or if
      solver=``liblinear'', and otherwise selects ``multinomial''.
  \default{auto}

    \item \xmlNode{l1\_ratio}: \xmlDesc{float}, 
      The Elastic-Net mixing parameter, with $0 <= l1\_ratio <= 1$. Only used if
      penalty=``elasticnet''.                                                  Setting $l1\_ratio=0$
      is equivalent to using penalty=``l2'', while setting $l1\_ratio=1$ is equivalent to using
      $penalty=``l1''$. For $0 < l1\_ratio <1$, the penalty is a combination of L1 and L2.
  \default{0.5}

    \item \xmlNode{class\_weight}: \xmlDesc{[balanced]}, 
      If not given, all classes are supposed to have weight one.
      The “balanced” mode uses the values of y to automatically adjust weights
      inversely proportional to class frequencies in the input data
  \default{None}

    \item \xmlNode{random\_state}: \xmlDesc{integer}, 
      Used when solver == ‘sag’, ‘saga’ or ‘liblinear’ to shuffle the data.
  \default{None}
  \end{itemize}


\subsubsection{MultiTaskElasticNet}
  The \xmlNode{MultiTaskElasticNet} employs                         Linear regression with combined
  L1 and L2 priors as regularizer.                         The optimization objective for
  MultiTaskElasticNet is:                         $(1 / (2 * n\_samples)) * ||Y - XW||^{Fro}\_2
  + alpha * l1\_ratio * ||W||\_{21}                         + 0.5 * alpha * (1 - l1\_ratio) *
  ||W||\_{Fro}^2$                         \\Where:                         $||W||\_{21} = \sum\_i
  \sqrt{\sum\_j w\_{ij}^2}$                         i.e. the sum of norm of each row.
  \zNormalizationNotPerformed{MultiTaskElasticNet}

  The \xmlNode{MultiTaskElasticNet} node recognizes the following parameters:
    \begin{itemize}
      \item \xmlAttr{name}: \xmlDesc{string, required}, 
        User-defined name to designate this entity in the RAVEN input file.
      \item \xmlAttr{verbosity}: \xmlDesc{[silent, quiet, all, debug], optional}, 
        Desired verbosity of messages coming from this entity
      \item \xmlAttr{subType}: \xmlDesc{string, required}, 
        specify the type of ROM that will be used
  \end{itemize}

  The \xmlNode{MultiTaskElasticNet} node recognizes the following subnodes:
  \begin{itemize}
    \item \xmlNode{Features}: \xmlDesc{comma-separated strings}, 
      specifies the names of the features of this ROM.         \nb These parameters are going to be
      requested for the training of this object         (see Section~\ref{subsec:stepRomTrainer})

    \item \xmlNode{Target}: \xmlDesc{comma-separated strings}, 
      contains a comma separated list of the targets of this ROM. These parameters         are the
      Figures of Merit (FOMs) this ROM is supposed to predict.         \nb These parameters are
      going to be requested for the training of this         object (see Section
      \ref{subsec:stepRomTrainer}).

    \item \xmlNode{pivotParameter}: \xmlDesc{string}, 
      If a time-dependent ROM is requested, please specifies the pivot         variable (e.g. time,
      etc) used in the input HistorySet.
  \default{time}

    \item \xmlNode{CV}: \xmlDesc{string}, 
      The text portion of this node needs to contain the name of the \xmlNode{PostProcessor} with
      \xmlAttr{subType}         ``CrossValidation``.
      The \xmlNode{CV} node recognizes the following parameters:
        \begin{itemize}
          \item \xmlAttr{class}: \xmlDesc{string, optional}, 
            should be set to \xmlString{Model}
          \item \xmlAttr{type}: \xmlDesc{string, optional}, 
            should be set to \xmlString{PostProcessor}
      \end{itemize}

    \item \xmlNode{alias}: \xmlDesc{string}, 
      specifies alias for         any variable of interest in the input or output space. These
      aliases can be used anywhere in the RAVEN input to         refer to the variables. In the body
      of this node the user specifies the name of the variable that the model is going to use
      (during its execution).
      The \xmlNode{alias} node recognizes the following parameters:
        \begin{itemize}
          \item \xmlAttr{variable}: \xmlDesc{string, required}, 
            define the actual alias, usable throughout the RAVEN input
          \item \xmlAttr{type}: \xmlDesc{[input, output], required}, 
            either ``input'' or ``output''.
      \end{itemize}

    \item \xmlNode{tol}: \xmlDesc{float}, 
      Tolerance for stopping criterion
  \default{0.0001}

    \item \xmlNode{alpha}: \xmlDesc{float}, 
      specifies a constant                                                  that multiplies the
      penalty terms.                                                  $alpha = 0$ is equivalent to
      an ordinary least square, solved by the
      \textbf{LinearRegression} object.
  \default{1.0}

    \item \xmlNode{l1\_ratio}: \xmlDesc{float}, 
      specifies the                                                  ElasticNet mixing parameter,
      with $0 <= l1\_ratio <= 1$.                                                  For $l1\_ratio =
      0$ the penalty is an L2 penalty.                                                  For
      $l1\_ratio = 1$ it is an L1 penalty.                                                  For $0 <
      l1\_ratio < 1$, the penalty is a combination of L1 and L2.
  \default{0.5}

    \item \xmlNode{fit\_intercept}: \xmlDesc{[True, Yes, 1, False, No, 0, t, y, 1, f, n, 0]}, 
      Whether the intercept should be estimated or not. If False,
      the data is assumed to be already centered.
  \default{True}

    \item \xmlNode{max\_iter}: \xmlDesc{integer}, 
      The maximum number of iterations.
  \default{1000}

    \item \xmlNode{selection}: \xmlDesc{[cyclic, random]}, 
      If set to ``random'', a random coefficient is updated every iteration
      rather than looping over features sequentially by default. This (setting to `random'')
      often leads to significantly faster convergence especially when tol is higher than $1e-4$
  \default{cyclic}

    \item \xmlNode{normalize}: \xmlDesc{[True, Yes, 1, False, No, 0, t, y, 1, f, n, 0]}, 
      This parameter is ignored when fit\_intercept is set to False. If True,
      the regressors X will be normalized before regression by subtracting the mean and
      dividing by the l2-norm.
  \default{False}

    \item \xmlNode{warm\_start}: \xmlDesc{[True, Yes, 1, False, No, 0, t, y, 1, f, n, 0]}, 
      When set to True, reuse the solution of the previous call
      to fit as initialization, otherwise, just erase the previous solution.
  \default{False}
  \end{itemize}


\subsubsection{MultiTaskElasticNetCV}
  The \xmlNode{MultiTaskElasticNetCV} employs                         linear regression with
  combined L1 and L2 priors as regularizer.                         The optimization objective for
  MultiTaskElasticNet is:                         $(1 / (2 * n\_samples)) * ||Y - XW||^{Fro}\_2
  + alpha * l1\_ratio * ||W||\_{21}                         + 0.5 * alpha * (1 - l1\_ratio) *
  ||W||\_{Fro}^2$                         \\Where:                         $||W||\_{21} = \sum\_i
  \sqrt{\sum\_j w\_{ij}^2}$                         In this model, the cross-validation is embedded
  for the automatic selection                         of the best hyper-parameters.
  \zNormalizationNotPerformed{MultiTaskElasticNetCV}

  The \xmlNode{MultiTaskElasticNetCV} node recognizes the following parameters:
    \begin{itemize}
      \item \xmlAttr{name}: \xmlDesc{string, required}, 
        User-defined name to designate this entity in the RAVEN input file.
      \item \xmlAttr{verbosity}: \xmlDesc{[silent, quiet, all, debug], optional}, 
        Desired verbosity of messages coming from this entity
      \item \xmlAttr{subType}: \xmlDesc{string, required}, 
        specify the type of ROM that will be used
  \end{itemize}

  The \xmlNode{MultiTaskElasticNetCV} node recognizes the following subnodes:
  \begin{itemize}
    \item \xmlNode{Features}: \xmlDesc{comma-separated strings}, 
      specifies the names of the features of this ROM.         \nb These parameters are going to be
      requested for the training of this object         (see Section~\ref{subsec:stepRomTrainer})

    \item \xmlNode{Target}: \xmlDesc{comma-separated strings}, 
      contains a comma separated list of the targets of this ROM. These parameters         are the
      Figures of Merit (FOMs) this ROM is supposed to predict.         \nb These parameters are
      going to be requested for the training of this         object (see Section
      \ref{subsec:stepRomTrainer}).

    \item \xmlNode{pivotParameter}: \xmlDesc{string}, 
      If a time-dependent ROM is requested, please specifies the pivot         variable (e.g. time,
      etc) used in the input HistorySet.
  \default{time}

    \item \xmlNode{CV}: \xmlDesc{string}, 
      The text portion of this node needs to contain the name of the \xmlNode{PostProcessor} with
      \xmlAttr{subType}         ``CrossValidation``.
      The \xmlNode{CV} node recognizes the following parameters:
        \begin{itemize}
          \item \xmlAttr{class}: \xmlDesc{string, optional}, 
            should be set to \xmlString{Model}
          \item \xmlAttr{type}: \xmlDesc{string, optional}, 
            should be set to \xmlString{PostProcessor}
      \end{itemize}

    \item \xmlNode{alias}: \xmlDesc{string}, 
      specifies alias for         any variable of interest in the input or output space. These
      aliases can be used anywhere in the RAVEN input to         refer to the variables. In the body
      of this node the user specifies the name of the variable that the model is going to use
      (during its execution).
      The \xmlNode{alias} node recognizes the following parameters:
        \begin{itemize}
          \item \xmlAttr{variable}: \xmlDesc{string, required}, 
            define the actual alias, usable throughout the RAVEN input
          \item \xmlAttr{type}: \xmlDesc{[input, output], required}, 
            either ``input'' or ``output''.
      \end{itemize}

    \item \xmlNode{eps}: \xmlDesc{float}, 
      Length of the path. $eps=1e-3$ means that $alpha\_min / alpha\_max = 1e-3$.
  \default{0.001}

    \item \xmlNode{tol}: \xmlDesc{float}, 
      Tolerance for stopping criterion
  \default{0.0001}

    \item \xmlNode{n\_alpha}: \xmlDesc{integer}, 
      Number of alphas along the regularization path.
  \default{100}

    \item \xmlNode{l1\_ratio}: \xmlDesc{float}, 
      specifies the                                                  ElasticNet mixing parameter,
      with $0 <= l1\_ratio <= 1$.                                                  For $l1\_ratio =
      0$ the penalty is an L2 penalty.                                                  For
      $l1\_ratio = 1$ it is an L1 penalty.                                                  For $0 <
      l1\_ratio < 1$, the penalty is a combination of L1 and L2.
  \default{0.5}

    \item \xmlNode{fit\_intercept}: \xmlDesc{[True, Yes, 1, False, No, 0, t, y, 1, f, n, 0]}, 
      Whether the intercept should be estimated or not. If False,
      the data is assumed to be already centered.
  \default{True}

    \item \xmlNode{max\_iter}: \xmlDesc{integer}, 
      The maximum number of iterations.
  \default{1000}

    \item \xmlNode{selection}: \xmlDesc{[cyclic, random]}, 
      If set to ``random'', a random coefficient is updated every iteration
      rather than looping over features sequentially by default. This (setting to `random'')
      often leads to significantly faster convergence especially when tol is higher than $1e-4$
  \default{cyclic}

    \item \xmlNode{normalize}: \xmlDesc{[True, Yes, 1, False, No, 0, t, y, 1, f, n, 0]}, 
      This parameter is ignored when fit\_intercept is set to False. If True,
      the regressors X will be normalized before regression by subtracting the mean and
      dividing by the l2-norm.
  \default{False}

    \item \xmlNode{cv}: \xmlDesc{integer}, 
      Determines the cross-validation splitting strategy.
      It specifies the number of folds..
  \default{5}
  \end{itemize}


\subsubsection{MultiTaskLasso}
  The \xmlNode{MultiTaskLasso} (\textit{Multi-task Lasso model trained                         with
  L1/L2 mixed-norm as regularizer}) is an algorithm for regression problem
  where the optimization objective for Lasso is:                         $(1 / (2 * n\_samples)) *
  ||Y - XW||^2\_{Fro} + alpha * ||W||\_{21}$                         \\Where:
  $||W||\_{21} = \sum\_i \sqrt{\sum\_j w\_{ij}^2}$                         i.e. the sum of norm of each
  row.                         \zNormalizationNotPerformed{MultiTaskLasso}

  The \xmlNode{MultiTaskLasso} node recognizes the following parameters:
    \begin{itemize}
      \item \xmlAttr{name}: \xmlDesc{string, required}, 
        User-defined name to designate this entity in the RAVEN input file.
      \item \xmlAttr{verbosity}: \xmlDesc{[silent, quiet, all, debug], optional}, 
        Desired verbosity of messages coming from this entity
      \item \xmlAttr{subType}: \xmlDesc{string, required}, 
        specify the type of ROM that will be used
  \end{itemize}

  The \xmlNode{MultiTaskLasso} node recognizes the following subnodes:
  \begin{itemize}
    \item \xmlNode{Features}: \xmlDesc{comma-separated strings}, 
      specifies the names of the features of this ROM.         \nb These parameters are going to be
      requested for the training of this object         (see Section~\ref{subsec:stepRomTrainer})

    \item \xmlNode{Target}: \xmlDesc{comma-separated strings}, 
      contains a comma separated list of the targets of this ROM. These parameters         are the
      Figures of Merit (FOMs) this ROM is supposed to predict.         \nb These parameters are
      going to be requested for the training of this         object (see Section
      \ref{subsec:stepRomTrainer}).

    \item \xmlNode{pivotParameter}: \xmlDesc{string}, 
      If a time-dependent ROM is requested, please specifies the pivot         variable (e.g. time,
      etc) used in the input HistorySet.
  \default{time}

    \item \xmlNode{CV}: \xmlDesc{string}, 
      The text portion of this node needs to contain the name of the \xmlNode{PostProcessor} with
      \xmlAttr{subType}         ``CrossValidation``.
      The \xmlNode{CV} node recognizes the following parameters:
        \begin{itemize}
          \item \xmlAttr{class}: \xmlDesc{string, optional}, 
            should be set to \xmlString{Model}
          \item \xmlAttr{type}: \xmlDesc{string, optional}, 
            should be set to \xmlString{PostProcessor}
      \end{itemize}

    \item \xmlNode{alias}: \xmlDesc{string}, 
      specifies alias for         any variable of interest in the input or output space. These
      aliases can be used anywhere in the RAVEN input to         refer to the variables. In the body
      of this node the user specifies the name of the variable that the model is going to use
      (during its execution).
      The \xmlNode{alias} node recognizes the following parameters:
        \begin{itemize}
          \item \xmlAttr{variable}: \xmlDesc{string, required}, 
            define the actual alias, usable throughout the RAVEN input
          \item \xmlAttr{type}: \xmlDesc{[input, output], required}, 
            either ``input'' or ``output''.
      \end{itemize}

    \item \xmlNode{alpha}: \xmlDesc{float}, 
      Constant that multiplies the L1 term. Defaults to 1.0.
      $alpha = 0$ is equivalent to an ordinary least square, solved by
      the LinearRegression object. For numerical reasons, using $alpha = 0$
      with the Lasso object is not advised.
  \default{1.0}

    \item \xmlNode{tol}: \xmlDesc{float}, 
      The tolerance for the optimization: if the updates are smaller
      than tol, the optimization code checks the dual gap for optimality and
      continues until it is smaller than tol..
  \default{0.0001}

    \item \xmlNode{fit\_intercept}: \xmlDesc{[True, Yes, 1, False, No, 0, t, y, 1, f, n, 0]}, 
      Whether the intercept should be estimated or not. If False,
      the data is assumed to be already centered.
  \default{True}

    \item \xmlNode{normalize}: \xmlDesc{[True, Yes, 1, False, No, 0, t, y, 1, f, n, 0]}, 
      This parameter is ignored when fit\_intercept is set to False. If True,
      the regressors X will be normalized before regression by subtracting the mean and
      dividing by the l2-norm.
  \default{False}

    \item \xmlNode{max\_iter}: \xmlDesc{integer}, 
      The maximum number of iterations.
  \default{1000}

    \item \xmlNode{selection}: \xmlDesc{[cyclic, random]}, 
      If set to ``random'', a random coefficient is updated every iteration
      rather than looping over features sequentially by default. This setting
      often leads to significantly faster convergence especially when tol is higher than $1e-4$
  \default{cyclic}

    \item \xmlNode{warm\_start}: \xmlDesc{[True, Yes, 1, False, No, 0, t, y, 1, f, n, 0]}, 
      When set to True, reuse the solution of the previous call
      to fit as initialization, otherwise, just erase the previous solution.
  \default{False}
  \end{itemize}


\subsubsection{MultiTaskLassoCV}
  The \xmlNode{MultiTaskLassoCV} (\textit{Multi-task Lasso model trained
  with L1/L2 mixed-norm as regularizer}) is an algorithm for regression problem
  where the optimization objective for Lasso is:                         $(1 / (2 * n\_samples)) *
  ||Y - XW||^2\_{Fro} + alpha * ||W||\_{21}$                         \\Where:
  $||W||\_{21} = \sum\_i \sqrt{\sum\_j w\_{ij}^2}$                         i.e. the sum of norm of each
  row.                         In this model, the cross-validation is embedded for the automatic
  selection                         of the best hyper-parameters.
  \zNormalizationNotPerformed{MultiTaskLassoCV}

  The \xmlNode{MultiTaskLassoCV} node recognizes the following parameters:
    \begin{itemize}
      \item \xmlAttr{name}: \xmlDesc{string, required}, 
        User-defined name to designate this entity in the RAVEN input file.
      \item \xmlAttr{verbosity}: \xmlDesc{[silent, quiet, all, debug], optional}, 
        Desired verbosity of messages coming from this entity
      \item \xmlAttr{subType}: \xmlDesc{string, required}, 
        specify the type of ROM that will be used
  \end{itemize}

  The \xmlNode{MultiTaskLassoCV} node recognizes the following subnodes:
  \begin{itemize}
    \item \xmlNode{Features}: \xmlDesc{comma-separated strings}, 
      specifies the names of the features of this ROM.         \nb These parameters are going to be
      requested for the training of this object         (see Section~\ref{subsec:stepRomTrainer})

    \item \xmlNode{Target}: \xmlDesc{comma-separated strings}, 
      contains a comma separated list of the targets of this ROM. These parameters         are the
      Figures of Merit (FOMs) this ROM is supposed to predict.         \nb These parameters are
      going to be requested for the training of this         object (see Section
      \ref{subsec:stepRomTrainer}).

    \item \xmlNode{pivotParameter}: \xmlDesc{string}, 
      If a time-dependent ROM is requested, please specifies the pivot         variable (e.g. time,
      etc) used in the input HistorySet.
  \default{time}

    \item \xmlNode{CV}: \xmlDesc{string}, 
      The text portion of this node needs to contain the name of the \xmlNode{PostProcessor} with
      \xmlAttr{subType}         ``CrossValidation``.
      The \xmlNode{CV} node recognizes the following parameters:
        \begin{itemize}
          \item \xmlAttr{class}: \xmlDesc{string, optional}, 
            should be set to \xmlString{Model}
          \item \xmlAttr{type}: \xmlDesc{string, optional}, 
            should be set to \xmlString{PostProcessor}
      \end{itemize}

    \item \xmlNode{alias}: \xmlDesc{string}, 
      specifies alias for         any variable of interest in the input or output space. These
      aliases can be used anywhere in the RAVEN input to         refer to the variables. In the body
      of this node the user specifies the name of the variable that the model is going to use
      (during its execution).
      The \xmlNode{alias} node recognizes the following parameters:
        \begin{itemize}
          \item \xmlAttr{variable}: \xmlDesc{string, required}, 
            define the actual alias, usable throughout the RAVEN input
          \item \xmlAttr{type}: \xmlDesc{[input, output], required}, 
            either ``input'' or ``output''.
      \end{itemize}

    \item \xmlNode{eps}: \xmlDesc{float}, 
      Length of the path. $eps=1e-3$ means that $alpha\_min / alpha\_max = 1e-3$.
  \default{0.001}

    \item \xmlNode{n\_alpha}: \xmlDesc{integer}, 
      Number of alphas along the regularization path.
  \default{100}

    \item \xmlNode{fit\_intercept}: \xmlDesc{[True, Yes, 1, False, No, 0, t, y, 1, f, n, 0]}, 
      Whether the intercept should be estimated or not. If False,
      the data is assumed to be already centered.
  \default{True}

    \item \xmlNode{normalize}: \xmlDesc{[True, Yes, 1, False, No, 0, t, y, 1, f, n, 0]}, 
      This parameter is ignored when fit\_intercept is set to False. If True,
      the regressors X will be normalized before regression by subtracting the mean and
      dividing by the l2-norm.
  \default{False}

    \item \xmlNode{max\_iter}: \xmlDesc{integer}, 
      The maximum number of iterations.
  \default{1000}

    \item \xmlNode{tol}: \xmlDesc{float}, 
      Tolerance for stopping criterion
  \default{0.0001}

    \item \xmlNode{selection}: \xmlDesc{[cyclic, random]}, 
      If set to ``random'', a random coefficient is updated every iteration
      rather than looping over features sequentially by default. This (setting to `random'')
      often leads to significantly faster convergence especially when tol is higher than $1e-4$
  \default{cyclic}

    \item \xmlNode{cv}: \xmlDesc{integer}, 
      Determines the cross-validation splitting strategy.
      It specifies the number of folds..
  \default{5}
  \end{itemize}


\subsubsection{OrthogonalMatchingPursuit}
  The \xmlNode{OrthogonalMatchingPursuit}                         implements the OMP algorithm for
  approximating the fit of a                         linear model with constraints imposed on the
  number of non-zero                         coefficients (ie. the $\ell\_0$ pseudo-norm). OMP is
  based on a greedy                         algorithm that includes at each step the atom most
  highly correlated                         with the current residual. It is similar to the simpler
  matching                         pursuit (MP) method, but better in that at each iteration, the
  residual                         is recomputed using an orthogonal projection on the space of the
  previously chosen dictionary elements.
  \zNormalizationNotPerformed{OrthogonalMatchingPursuit}

  The \xmlNode{OrthogonalMatchingPursuit} node recognizes the following parameters:
    \begin{itemize}
      \item \xmlAttr{name}: \xmlDesc{string, required}, 
        User-defined name to designate this entity in the RAVEN input file.
      \item \xmlAttr{verbosity}: \xmlDesc{[silent, quiet, all, debug], optional}, 
        Desired verbosity of messages coming from this entity
      \item \xmlAttr{subType}: \xmlDesc{string, required}, 
        specify the type of ROM that will be used
  \end{itemize}

  The \xmlNode{OrthogonalMatchingPursuit} node recognizes the following subnodes:
  \begin{itemize}
    \item \xmlNode{Features}: \xmlDesc{comma-separated strings}, 
      specifies the names of the features of this ROM.         \nb These parameters are going to be
      requested for the training of this object         (see Section~\ref{subsec:stepRomTrainer})

    \item \xmlNode{Target}: \xmlDesc{comma-separated strings}, 
      contains a comma separated list of the targets of this ROM. These parameters         are the
      Figures of Merit (FOMs) this ROM is supposed to predict.         \nb These parameters are
      going to be requested for the training of this         object (see Section
      \ref{subsec:stepRomTrainer}).

    \item \xmlNode{pivotParameter}: \xmlDesc{string}, 
      If a time-dependent ROM is requested, please specifies the pivot         variable (e.g. time,
      etc) used in the input HistorySet.
  \default{time}

    \item \xmlNode{CV}: \xmlDesc{string}, 
      The text portion of this node needs to contain the name of the \xmlNode{PostProcessor} with
      \xmlAttr{subType}         ``CrossValidation``.
      The \xmlNode{CV} node recognizes the following parameters:
        \begin{itemize}
          \item \xmlAttr{class}: \xmlDesc{string, optional}, 
            should be set to \xmlString{Model}
          \item \xmlAttr{type}: \xmlDesc{string, optional}, 
            should be set to \xmlString{PostProcessor}
      \end{itemize}

    \item \xmlNode{alias}: \xmlDesc{string}, 
      specifies alias for         any variable of interest in the input or output space. These
      aliases can be used anywhere in the RAVEN input to         refer to the variables. In the body
      of this node the user specifies the name of the variable that the model is going to use
      (during its execution).
      The \xmlNode{alias} node recognizes the following parameters:
        \begin{itemize}
          \item \xmlAttr{variable}: \xmlDesc{string, required}, 
            define the actual alias, usable throughout the RAVEN input
          \item \xmlAttr{type}: \xmlDesc{[input, output], required}, 
            either ``input'' or ``output''.
      \end{itemize}

    \item \xmlNode{n\_nonzero\_coefs}: \xmlDesc{integer}, 
      Desired number of non-zero entries in the solution. If None (by default)
      this value is set to ten-percent of n\_features.
  \default{None}

    \item \xmlNode{tol}: \xmlDesc{float}, 
      Maximum norm of the residual.
  \default{None}

    \item \xmlNode{fit\_intercept}: \xmlDesc{[True, Yes, 1, False, No, 0, t, y, 1, f, n, 0]}, 
      Whether the intercept should be estimated or not. If False,
      the data is assumed to be already centered.
  \default{True}

    \item \xmlNode{normalize}: \xmlDesc{[True, Yes, 1, False, No, 0, t, y, 1, f, n, 0]}, 
      This parameter is ignored when fit\_intercept is set to False. If True,
      the regressors X will be normalized before regression by subtracting the mean and
      dividing by the l2-norm.
  \default{True}

    \item \xmlNode{precompute}: \xmlDesc{string}, 
      Whether to use a precomputed Gram and Xy matrix to speed up calculations.
      Improves performance when n\_targets or n\_samples is very large.
  \default{auto}
  \end{itemize}


\subsubsection{OrthogonalMatchingPursuitCV}
  The \xmlNode{OrthogonalMatchingPursuitCV}                         implements the OMP algorithm for
  approximating the fit of a                         linear model with constraints imposed on the
  number of non-zero                         coefficients (ie. the $\ell\_0$ pseudo-norm). OMP is
  based on a greedy                         algorithm that includes at each step the atom most
  highly correlated                         with the current residual. It is similar to the simpler
  matching                         pursuit (MP) method, but better in that at each iteration, the
  residual                         is recomputed using an orthogonal projection on the space of the
  previously chosen dictionary elements.                         In this model, the cross-validation
  is embedded for the automatic selection                         of the best hyper-parameters.
  \zNormalizationNotPerformed{OrthogonalMatchingPursuitCV}

  The \xmlNode{OrthogonalMatchingPursuitCV} node recognizes the following parameters:
    \begin{itemize}
      \item \xmlAttr{name}: \xmlDesc{string, required}, 
        User-defined name to designate this entity in the RAVEN input file.
      \item \xmlAttr{verbosity}: \xmlDesc{[silent, quiet, all, debug], optional}, 
        Desired verbosity of messages coming from this entity
      \item \xmlAttr{subType}: \xmlDesc{string, required}, 
        specify the type of ROM that will be used
  \end{itemize}

  The \xmlNode{OrthogonalMatchingPursuitCV} node recognizes the following subnodes:
  \begin{itemize}
    \item \xmlNode{Features}: \xmlDesc{comma-separated strings}, 
      specifies the names of the features of this ROM.         \nb These parameters are going to be
      requested for the training of this object         (see Section~\ref{subsec:stepRomTrainer})

    \item \xmlNode{Target}: \xmlDesc{comma-separated strings}, 
      contains a comma separated list of the targets of this ROM. These parameters         are the
      Figures of Merit (FOMs) this ROM is supposed to predict.         \nb These parameters are
      going to be requested for the training of this         object (see Section
      \ref{subsec:stepRomTrainer}).

    \item \xmlNode{pivotParameter}: \xmlDesc{string}, 
      If a time-dependent ROM is requested, please specifies the pivot         variable (e.g. time,
      etc) used in the input HistorySet.
  \default{time}

    \item \xmlNode{CV}: \xmlDesc{string}, 
      The text portion of this node needs to contain the name of the \xmlNode{PostProcessor} with
      \xmlAttr{subType}         ``CrossValidation``.
      The \xmlNode{CV} node recognizes the following parameters:
        \begin{itemize}
          \item \xmlAttr{class}: \xmlDesc{string, optional}, 
            should be set to \xmlString{Model}
          \item \xmlAttr{type}: \xmlDesc{string, optional}, 
            should be set to \xmlString{PostProcessor}
      \end{itemize}

    \item \xmlNode{alias}: \xmlDesc{string}, 
      specifies alias for         any variable of interest in the input or output space. These
      aliases can be used anywhere in the RAVEN input to         refer to the variables. In the body
      of this node the user specifies the name of the variable that the model is going to use
      (during its execution).
      The \xmlNode{alias} node recognizes the following parameters:
        \begin{itemize}
          \item \xmlAttr{variable}: \xmlDesc{string, required}, 
            define the actual alias, usable throughout the RAVEN input
          \item \xmlAttr{type}: \xmlDesc{[input, output], required}, 
            either ``input'' or ``output''.
      \end{itemize}

    \item \xmlNode{fit\_intercept}: \xmlDesc{[True, Yes, 1, False, No, 0, t, y, 1, f, n, 0]}, 
      Whether the intercept should be estimated or not. If False,
      the data is assumed to be already centered.
  \default{True}

    \item \xmlNode{normalize}: \xmlDesc{[True, Yes, 1, False, No, 0, t, y, 1, f, n, 0]}, 
      This parameter is ignored when fit\_intercept is set to False. If True,
      the regressors X will be normalized before regression by subtracting the mean and
      dividing by the l2-norm.
  \default{True}

    \item \xmlNode{max\_iter}: \xmlDesc{integer}, 
      Maximum numbers of iterations to perform, therefore maximum
      features to include. Ten-percent of n\_features but at least 5 if available.
  \default{None}

    \item \xmlNode{cv}: \xmlDesc{integer}, 
      Determines the cross-validation splitting strategy.
      It specifies the number of folds..
  \default{None}

    \item \xmlNode{verbose}: \xmlDesc{[True, Yes, 1, False, No, 0, t, y, 1, f, n, 0]}, 
      Amount of verbosity.
  \default{False}
  \end{itemize}


\subsubsection{PassiveAggressiveClassifier}
  The \xmlNode{PassiveAggressiveClassifier}                         is a principled approach to
  linear                         classification that advocates minimal weight updates i.e., the
  least required                         to correctly classify the current training instance.
  \\The passive-aggressive algorithms are a family of algorithms for                         large-
  scale learning. They are similar to the Perceptron in that they                         do not
  require a learning rate. However, contrary to the Perceptron,                         they include
  a regularization parameter C.
  \zNormalizationPerformed{PassiveAggressiveClassifier}

  The \xmlNode{PassiveAggressiveClassifier} node recognizes the following parameters:
    \begin{itemize}
      \item \xmlAttr{name}: \xmlDesc{string, required}, 
        User-defined name to designate this entity in the RAVEN input file.
      \item \xmlAttr{verbosity}: \xmlDesc{[silent, quiet, all, debug], optional}, 
        Desired verbosity of messages coming from this entity
      \item \xmlAttr{subType}: \xmlDesc{string, required}, 
        specify the type of ROM that will be used
  \end{itemize}

  The \xmlNode{PassiveAggressiveClassifier} node recognizes the following subnodes:
  \begin{itemize}
    \item \xmlNode{Features}: \xmlDesc{comma-separated strings}, 
      specifies the names of the features of this ROM.         \nb These parameters are going to be
      requested for the training of this object         (see Section~\ref{subsec:stepRomTrainer})

    \item \xmlNode{Target}: \xmlDesc{comma-separated strings}, 
      contains a comma separated list of the targets of this ROM. These parameters         are the
      Figures of Merit (FOMs) this ROM is supposed to predict.         \nb These parameters are
      going to be requested for the training of this         object (see Section
      \ref{subsec:stepRomTrainer}).

    \item \xmlNode{pivotParameter}: \xmlDesc{string}, 
      If a time-dependent ROM is requested, please specifies the pivot         variable (e.g. time,
      etc) used in the input HistorySet.
  \default{time}

    \item \xmlNode{CV}: \xmlDesc{string}, 
      The text portion of this node needs to contain the name of the \xmlNode{PostProcessor} with
      \xmlAttr{subType}         ``CrossValidation``.
      The \xmlNode{CV} node recognizes the following parameters:
        \begin{itemize}
          \item \xmlAttr{class}: \xmlDesc{string, optional}, 
            should be set to \xmlString{Model}
          \item \xmlAttr{type}: \xmlDesc{string, optional}, 
            should be set to \xmlString{PostProcessor}
      \end{itemize}

    \item \xmlNode{alias}: \xmlDesc{string}, 
      specifies alias for         any variable of interest in the input or output space. These
      aliases can be used anywhere in the RAVEN input to         refer to the variables. In the body
      of this node the user specifies the name of the variable that the model is going to use
      (during its execution).
      The \xmlNode{alias} node recognizes the following parameters:
        \begin{itemize}
          \item \xmlAttr{variable}: \xmlDesc{string, required}, 
            define the actual alias, usable throughout the RAVEN input
          \item \xmlAttr{type}: \xmlDesc{[input, output], required}, 
            either ``input'' or ``output''.
      \end{itemize}

    \item \xmlNode{C}: \xmlDesc{float}, 
      Maximum step size (regularization).
  \default{1.0}

    \item \xmlNode{fit\_intercept}: \xmlDesc{[True, Yes, 1, False, No, 0, t, y, 1, f, n, 0]}, 
      Whether the intercept should be estimated or not. If False,
      the data is assumed to be already centered.
  \default{True}

    \item \xmlNode{max\_iter}: \xmlDesc{integer}, 
      The maximum number of passes over the training data (aka epochs).
  \default{1000}

    \item \xmlNode{tol}: \xmlDesc{float}, 
      The stopping criterion.
  \default{0.001}

    \item \xmlNode{early\_stopping}: \xmlDesc{[True, Yes, 1, False, No, 0, t, y, 1, f, n, 0]}, 
      hether to use early stopping to terminate training when validation score is not
      improving. If set to True, it will automatically set aside a stratified fraction of training
      data as validation and terminate training when validation score is not improving by at least
      tol for n\_iter\_no\_change consecutive epochs.
  \default{False}

    \item \xmlNode{validation\_fraction}: \xmlDesc{float}, 
      The proportion of training data to set aside as validation set for early stopping.
      Must be between 0 and 1. Only used if early\_stopping is True.
  \default{0.1}

    \item \xmlNode{n\_iter\_no\_change}: \xmlDesc{integer}, 
      Number of iterations with no improvement to wait before early stopping.
  \default{5}

    \item \xmlNode{shuffle}: \xmlDesc{[True, Yes, 1, False, No, 0, t, y, 1, f, n, 0]}, 
      Whether or not the training data should be shuffled after each epoch.
  \default{True}

    \item \xmlNode{loss}: \xmlDesc{[hinge,  squared\_hinge]}, 
      The loss function to be used: hinge: equivalent to PA-I.
      squared\_hinge: equivalent to PA-II.
  \default{hinge}

    \item \xmlNode{random\_state}: \xmlDesc{integer}, 
      Used to shuffle the training data, when shuffle is set to
      True. Pass an int for reproducible output across multiple function calls.
  \default{None}

    \item \xmlNode{verbose}: \xmlDesc{integer}, 
      The verbosity level
  \default{0}

    \item \xmlNode{warm\_start}: \xmlDesc{[True, Yes, 1, False, No, 0, t, y, 1, f, n, 0]}, 
      When set to True, reuse the solution of the previous call
      to fit as initialization, otherwise, just erase the previous solution.
  \default{False}
  \end{itemize}


\subsubsection{PassiveAggressiveRegressor}
  The \xmlNode{PassiveAggressiveRegressor}                         is a a regression algorithm
  similar to the Perceptron algorithm                         but with a regularization parameter C.
  \\The passive-aggressive algorithms are a family of algorithms for                         large-
  scale learning.                         \zNormalizationPerformed{PassiveAggressiveRegressor}

  The \xmlNode{PassiveAggressiveRegressor} node recognizes the following parameters:
    \begin{itemize}
      \item \xmlAttr{name}: \xmlDesc{string, required}, 
        User-defined name to designate this entity in the RAVEN input file.
      \item \xmlAttr{verbosity}: \xmlDesc{[silent, quiet, all, debug], optional}, 
        Desired verbosity of messages coming from this entity
      \item \xmlAttr{subType}: \xmlDesc{string, required}, 
        specify the type of ROM that will be used
  \end{itemize}

  The \xmlNode{PassiveAggressiveRegressor} node recognizes the following subnodes:
  \begin{itemize}
    \item \xmlNode{Features}: \xmlDesc{comma-separated strings}, 
      specifies the names of the features of this ROM.         \nb These parameters are going to be
      requested for the training of this object         (see Section~\ref{subsec:stepRomTrainer})

    \item \xmlNode{Target}: \xmlDesc{comma-separated strings}, 
      contains a comma separated list of the targets of this ROM. These parameters         are the
      Figures of Merit (FOMs) this ROM is supposed to predict.         \nb These parameters are
      going to be requested for the training of this         object (see Section
      \ref{subsec:stepRomTrainer}).

    \item \xmlNode{pivotParameter}: \xmlDesc{string}, 
      If a time-dependent ROM is requested, please specifies the pivot         variable (e.g. time,
      etc) used in the input HistorySet.
  \default{time}

    \item \xmlNode{CV}: \xmlDesc{string}, 
      The text portion of this node needs to contain the name of the \xmlNode{PostProcessor} with
      \xmlAttr{subType}         ``CrossValidation``.
      The \xmlNode{CV} node recognizes the following parameters:
        \begin{itemize}
          \item \xmlAttr{class}: \xmlDesc{string, optional}, 
            should be set to \xmlString{Model}
          \item \xmlAttr{type}: \xmlDesc{string, optional}, 
            should be set to \xmlString{PostProcessor}
      \end{itemize}

    \item \xmlNode{alias}: \xmlDesc{string}, 
      specifies alias for         any variable of interest in the input or output space. These
      aliases can be used anywhere in the RAVEN input to         refer to the variables. In the body
      of this node the user specifies the name of the variable that the model is going to use
      (during its execution).
      The \xmlNode{alias} node recognizes the following parameters:
        \begin{itemize}
          \item \xmlAttr{variable}: \xmlDesc{string, required}, 
            define the actual alias, usable throughout the RAVEN input
          \item \xmlAttr{type}: \xmlDesc{[input, output], required}, 
            either ``input'' or ``output''.
      \end{itemize}

    \item \xmlNode{C}: \xmlDesc{float}, 
      Maximum step size (regularization).
  \default{1.0}

    \item \xmlNode{fit\_intercept}: \xmlDesc{[True, Yes, 1, False, No, 0, t, y, 1, f, n, 0]}, 
      Whether the intercept should be estimated or not. If False,
      the data is assumed to be already centered.
  \default{True}

    \item \xmlNode{max\_iter}: \xmlDesc{integer}, 
      The maximum number of passes over the training data (aka epochs).
  \default{1000}

    \item \xmlNode{tol}: \xmlDesc{float}, 
      The stopping criterion.
  \default{0.001}

    \item \xmlNode{epsilon}: \xmlDesc{float}, 
      If the difference between the current prediction and the
      correct label is below this threshold, the model is not updated.
  \default{0.1}

    \item \xmlNode{early\_stopping}: \xmlDesc{[True, Yes, 1, False, No, 0, t, y, 1, f, n, 0]}, 
      hether to use early stopping to terminate training when validation score is not
      improving. If set to True, it will automatically set aside a stratified fraction of training
      data as validation and terminate training when validation score is not improving by at least
      tol for n\_iter\_no\_change consecutive epochs.
  \default{False}

    \item \xmlNode{validation\_fraction}: \xmlDesc{float}, 
      The proportion of training data to set aside as validation set for early stopping.
      Must be between 0 and 1. Only used if early\_stopping is True.
  \default{0.1}

    \item \xmlNode{n\_iter\_no\_change}: \xmlDesc{integer}, 
      Number of iterations with no improvement to wait before early stopping.
  \default{5}

    \item \xmlNode{shuffle}: \xmlDesc{[True, Yes, 1, False, No, 0, t, y, 1, f, n, 0]}, 
      Whether or not the training data should be shuffled after each epoch.
  \default{True}

    \item \xmlNode{loss}: \xmlDesc{[epsilon\_insensitive,  squared\_epsilon\_insensitive]}, 
      The loss function to be used: epsilon\_insensitive: equivalent to PA-I.
      squared\_epsilon\_insensitive: equivalent to PA-II.
  \default{epsilon\_insensitive}

    \item \xmlNode{random\_state}: \xmlDesc{integer}, 
      Used to shuffle the training data, when shuffle is set to
      True. Pass an int for reproducible output across multiple function calls.
  \default{None}

    \item \xmlNode{verbose}: \xmlDesc{integer}, 
      The verbosity level
  \default{0}

    \item \xmlNode{warm\_start}: \xmlDesc{[True, Yes, 1, False, No, 0, t, y, 1, f, n, 0]}, 
      When set to True, reuse the solution of the previous call
      to fit as initialization, otherwise, just erase the previous solution.
  \default{False}

    \item \xmlNode{average}: \xmlDesc{[True, Yes, 1, False, No, 0, t, y, 1, f, n, 0]}, 
      When set to True, computes the averaged SGD weights and
      stores the result in the coef\_ attribute.
  \default{False}
  \end{itemize}


\subsubsection{Perceptron}
  The \xmlNode{Perceptron} classifier is based on an                         algorithm for
  supervised classification of                         an input into one of several possible non-
  binary outputs.                         It is a type of linear classifier, i.e. a classification
  algorithm that makes                         its predictions based on a linear predictor function
  combining a set of weights                         with the feature vector.
  The algorithm allows for online learning, in that it processes elements in the
  training set one at a time.                         \zNormalizationPerformed{Perceptron}

  The \xmlNode{Perceptron} node recognizes the following parameters:
    \begin{itemize}
      \item \xmlAttr{name}: \xmlDesc{string, required}, 
        User-defined name to designate this entity in the RAVEN input file.
      \item \xmlAttr{verbosity}: \xmlDesc{[silent, quiet, all, debug], optional}, 
        Desired verbosity of messages coming from this entity
      \item \xmlAttr{subType}: \xmlDesc{string, required}, 
        specify the type of ROM that will be used
  \end{itemize}

  The \xmlNode{Perceptron} node recognizes the following subnodes:
  \begin{itemize}
    \item \xmlNode{Features}: \xmlDesc{comma-separated strings}, 
      specifies the names of the features of this ROM.         \nb These parameters are going to be
      requested for the training of this object         (see Section~\ref{subsec:stepRomTrainer})

    \item \xmlNode{Target}: \xmlDesc{comma-separated strings}, 
      contains a comma separated list of the targets of this ROM. These parameters         are the
      Figures of Merit (FOMs) this ROM is supposed to predict.         \nb These parameters are
      going to be requested for the training of this         object (see Section
      \ref{subsec:stepRomTrainer}).

    \item \xmlNode{pivotParameter}: \xmlDesc{string}, 
      If a time-dependent ROM is requested, please specifies the pivot         variable (e.g. time,
      etc) used in the input HistorySet.
  \default{time}

    \item \xmlNode{CV}: \xmlDesc{string}, 
      The text portion of this node needs to contain the name of the \xmlNode{PostProcessor} with
      \xmlAttr{subType}         ``CrossValidation``.
      The \xmlNode{CV} node recognizes the following parameters:
        \begin{itemize}
          \item \xmlAttr{class}: \xmlDesc{string, optional}, 
            should be set to \xmlString{Model}
          \item \xmlAttr{type}: \xmlDesc{string, optional}, 
            should be set to \xmlString{PostProcessor}
      \end{itemize}

    \item \xmlNode{alias}: \xmlDesc{string}, 
      specifies alias for         any variable of interest in the input or output space. These
      aliases can be used anywhere in the RAVEN input to         refer to the variables. In the body
      of this node the user specifies the name of the variable that the model is going to use
      (during its execution).
      The \xmlNode{alias} node recognizes the following parameters:
        \begin{itemize}
          \item \xmlAttr{variable}: \xmlDesc{string, required}, 
            define the actual alias, usable throughout the RAVEN input
          \item \xmlAttr{type}: \xmlDesc{[input, output], required}, 
            either ``input'' or ``output''.
      \end{itemize}

    \item \xmlNode{penalty}: \xmlDesc{[l2,  l1, elasticnet]}, 
      The penalty (aka regularization term) to be used.
  \default{None}

    \item \xmlNode{alpha}: \xmlDesc{float}, 
      Constant that multiplies the regularization term if regularization is used.
  \default{0.0001}

    \item \xmlNode{fit\_intercept}: \xmlDesc{[True, Yes, 1, False, No, 0, t, y, 1, f, n, 0]}, 
      Whether the intercept should be estimated or not. If False,
      the data is assumed to be already centered.
  \default{True}

    \item \xmlNode{max\_iter}: \xmlDesc{integer}, 
      The maximum number of passes over the training data (aka epochs).
  \default{1000}

    \item \xmlNode{tol}: \xmlDesc{float}, 
      The stopping criterion.
  \default{0.001}

    \item \xmlNode{n\_iter\_no\_change}: \xmlDesc{integer}, 
      Number of iterations with no improvement to wait before early stopping.
  \default{5}

    \item \xmlNode{shuffle}: \xmlDesc{[True, Yes, 1, False, No, 0, t, y, 1, f, n, 0]}, 
      Whether or not the training data should be shuffled after each epoch.
  \default{True}

    \item \xmlNode{eta0}: \xmlDesc{float}, 
      The stopping criterion.
  \default{1}

    \item \xmlNode{early\_stopping}: \xmlDesc{[True, Yes, 1, False, No, 0, t, y, 1, f, n, 0]}, 
      hether to use early stopping to terminate training when validation score is not
      improving. If set to True, it will automatically set aside a stratified fraction of training
      data as validation and terminate training when validation score is not improving by at least
      tol for n\_iter\_no\_change consecutive epochs.
  \default{False}

    \item \xmlNode{validation\_fraction}: \xmlDesc{float}, 
      The proportion of training data to set aside as validation set for early stopping.
      Must be between 0 and 1. Only used if early\_stopping is True.
  \default{0.1}

    \item \xmlNode{class\_weight}: \xmlDesc{[balanced]}, 
      If not given, all classes are supposed to have weight one.
      The “balanced” mode uses the values of y to automatically adjust weights
      inversely proportional to class frequencies in the input data
  \default{None}

    \item \xmlNode{random\_state}: \xmlDesc{integer}, 
      Used to shuffle the training data, when shuffle is set to
      True. Pass an int for reproducible output across multiple function calls.
  \default{None}

    \item \xmlNode{verbose}: \xmlDesc{integer}, 
      The verbosity level
  \default{0}

    \item \xmlNode{warm\_start}: \xmlDesc{[True, Yes, 1, False, No, 0, t, y, 1, f, n, 0]}, 
      When set to True, reuse the solution of the previous call
      to fit as initialization, otherwise, just erase the previous solution.
  \default{False}
  \end{itemize}


\subsubsection{Ridge}
  The \xmlNode{Ridge} regressor also known as                              \textit{linear least
  squares with l2 regularization} solves a regression                              model where the
  loss function is the linear least squares function and the
  regularization is given by the l2-norm.                              Also known as Ridge
  Regression or Tikhonov regularization.
  \zNormalizationNotPerformed{Ridge}

  The \xmlNode{Ridge} node recognizes the following parameters:
    \begin{itemize}
      \item \xmlAttr{name}: \xmlDesc{string, required}, 
        User-defined name to designate this entity in the RAVEN input file.
      \item \xmlAttr{verbosity}: \xmlDesc{[silent, quiet, all, debug], optional}, 
        Desired verbosity of messages coming from this entity
      \item \xmlAttr{subType}: \xmlDesc{string, required}, 
        specify the type of ROM that will be used
  \end{itemize}

  The \xmlNode{Ridge} node recognizes the following subnodes:
  \begin{itemize}
    \item \xmlNode{Features}: \xmlDesc{comma-separated strings}, 
      specifies the names of the features of this ROM.         \nb These parameters are going to be
      requested for the training of this object         (see Section~\ref{subsec:stepRomTrainer})

    \item \xmlNode{Target}: \xmlDesc{comma-separated strings}, 
      contains a comma separated list of the targets of this ROM. These parameters         are the
      Figures of Merit (FOMs) this ROM is supposed to predict.         \nb These parameters are
      going to be requested for the training of this         object (see Section
      \ref{subsec:stepRomTrainer}).

    \item \xmlNode{pivotParameter}: \xmlDesc{string}, 
      If a time-dependent ROM is requested, please specifies the pivot         variable (e.g. time,
      etc) used in the input HistorySet.
  \default{time}

    \item \xmlNode{CV}: \xmlDesc{string}, 
      The text portion of this node needs to contain the name of the \xmlNode{PostProcessor} with
      \xmlAttr{subType}         ``CrossValidation``.
      The \xmlNode{CV} node recognizes the following parameters:
        \begin{itemize}
          \item \xmlAttr{class}: \xmlDesc{string, optional}, 
            should be set to \xmlString{Model}
          \item \xmlAttr{type}: \xmlDesc{string, optional}, 
            should be set to \xmlString{PostProcessor}
      \end{itemize}

    \item \xmlNode{alias}: \xmlDesc{string}, 
      specifies alias for         any variable of interest in the input or output space. These
      aliases can be used anywhere in the RAVEN input to         refer to the variables. In the body
      of this node the user specifies the name of the variable that the model is going to use
      (during its execution).
      The \xmlNode{alias} node recognizes the following parameters:
        \begin{itemize}
          \item \xmlAttr{variable}: \xmlDesc{string, required}, 
            define the actual alias, usable throughout the RAVEN input
          \item \xmlAttr{type}: \xmlDesc{[input, output], required}, 
            either ``input'' or ``output''.
      \end{itemize}

    \item \xmlNode{alpha}: \xmlDesc{float}, 
      Regularization strength; must be a positive float. Regularization
      improves the conditioning of the problem and reduces the variance of the estimates.
      Larger values specify stronger regularization. Alpha corresponds to $1 / (2C)$ in other
      linear models such as LogisticRegression or LinearSVC.
  \default{1.0}

    \item \xmlNode{fit\_intercept}: \xmlDesc{[True, Yes, 1, False, No, 0, t, y, 1, f, n, 0]}, 
      Whether the intercept should be estimated or not. If False,
      the data is assumed to be already centered.
  \default{True}

    \item \xmlNode{normalize}: \xmlDesc{[True, Yes, 1, False, No, 0, t, y, 1, f, n, 0]}, 
      This parameter is ignored when fit\_intercept is set to False. If True, the
      regressors X will be normalized before regression by subtracting the mean and dividing
      by the l2-norm.
  \default{False}

    \item \xmlNode{max\_iter}: \xmlDesc{integer}, 
      Maximum number of iterations for conjugate gradient solver.
  \default{None}

    \item \xmlNode{tol}: \xmlDesc{float}, 
      Precision of the solution
  \default{0.001}

    \item \xmlNode{solver}: \xmlDesc{[auto, svd, cholesky, lsqr, sparse\_cg, sag, saga]}, 
      Solver to use in the computational routines:
      \begin{itemize}                                                    \item auto, chooses the
      solver automatically based on the type of data.
      \item svd, uses a Singular Value Decomposition of X to compute the Ridge coefficients. More
      stable for singular                                                                matrices
      than ``cholesky''.                                                    \item cholesky, uses the
      standard scipy.linalg.solve function to obtain a closed-form solution.
      \item sparse\_cg, uses the conjugate gradient solver as found in scipy.sparse.linalg.cg. As an
      iterative algorithm,                                                               this solver
      is more appropriate than ‘cholesky’ for large-scale data (possibility to set tol and
      max\_iter).                                                    \item lsqr, uses the dedicated
      regularized least-squares routine scipy.sparse.linalg.lsqr. It is the fastest and uses
      an iterative procedure.                                                    \item sag, uses a
      Stochastic Average Gradient descent, and ``saga'' uses its improved, unbiased version named
      SAGA.                                                               Both methods also use an
      iterative procedure, and are often faster than other solvers when both
      n\_samples and n\_features are large. Note that ``sag'' and ``saga'' fast convergence is only
      guaranteed on                                                               features with
      approximately the same scale. You can preprocess the data with a scaler from
      sklearn.preprocessing.                                                  \end{itemize}
  \default{auto}
  \end{itemize}


\subsubsection{RidgeCV}
  The \xmlNode{RidgeCV} regressor also known as                              \textit{linear least
  squares with l2 regularization} solves a regression                              model where the
  loss function is the linear least squares function and the
  regularization is given by the l2-norm.                              In addition, a cross-
  validation method is applied to optimize the hyper-parameter.
  \zNormalizationNotPerformed{RidgeCV}

  The \xmlNode{RidgeCV} node recognizes the following parameters:
    \begin{itemize}
      \item \xmlAttr{name}: \xmlDesc{string, required}, 
        User-defined name to designate this entity in the RAVEN input file.
      \item \xmlAttr{verbosity}: \xmlDesc{[silent, quiet, all, debug], optional}, 
        Desired verbosity of messages coming from this entity
      \item \xmlAttr{subType}: \xmlDesc{string, required}, 
        specify the type of ROM that will be used
  \end{itemize}

  The \xmlNode{RidgeCV} node recognizes the following subnodes:
  \begin{itemize}
    \item \xmlNode{Features}: \xmlDesc{comma-separated strings}, 
      specifies the names of the features of this ROM.         \nb These parameters are going to be
      requested for the training of this object         (see Section~\ref{subsec:stepRomTrainer})

    \item \xmlNode{Target}: \xmlDesc{comma-separated strings}, 
      contains a comma separated list of the targets of this ROM. These parameters         are the
      Figures of Merit (FOMs) this ROM is supposed to predict.         \nb These parameters are
      going to be requested for the training of this         object (see Section
      \ref{subsec:stepRomTrainer}).

    \item \xmlNode{pivotParameter}: \xmlDesc{string}, 
      If a time-dependent ROM is requested, please specifies the pivot         variable (e.g. time,
      etc) used in the input HistorySet.
  \default{time}

    \item \xmlNode{CV}: \xmlDesc{string}, 
      The text portion of this node needs to contain the name of the \xmlNode{PostProcessor} with
      \xmlAttr{subType}         ``CrossValidation``.
      The \xmlNode{CV} node recognizes the following parameters:
        \begin{itemize}
          \item \xmlAttr{class}: \xmlDesc{string, optional}, 
            should be set to \xmlString{Model}
          \item \xmlAttr{type}: \xmlDesc{string, optional}, 
            should be set to \xmlString{PostProcessor}
      \end{itemize}

    \item \xmlNode{alias}: \xmlDesc{string}, 
      specifies alias for         any variable of interest in the input or output space. These
      aliases can be used anywhere in the RAVEN input to         refer to the variables. In the body
      of this node the user specifies the name of the variable that the model is going to use
      (during its execution).
      The \xmlNode{alias} node recognizes the following parameters:
        \begin{itemize}
          \item \xmlAttr{variable}: \xmlDesc{string, required}, 
            define the actual alias, usable throughout the RAVEN input
          \item \xmlAttr{type}: \xmlDesc{[input, output], required}, 
            either ``input'' or ``output''.
      \end{itemize}

    \item \xmlNode{fit\_intercept}: \xmlDesc{[True, Yes, 1, False, No, 0, t, y, 1, f, n, 0]}, 
      Whether the intercept should be estimated or not. If False,
      the data is assumed to be already centered.
  \default{True}

    \item \xmlNode{normalize}: \xmlDesc{[True, Yes, 1, False, No, 0, t, y, 1, f, n, 0]}, 
      This parameter is ignored when fit\_intercept is set to False. If True, the
      regressors X will be normalized before regression by subtracting the mean and dividing
      by the l2-norm.
  \default{False}

    \item \xmlNode{gcv\_mode}: \xmlDesc{[auto, svd, eigen]}, 
      Flag indicating which strategy to use when performing Leave-One-Out Cross-Validation.
      Options are:                                                  \begin{itemize}
      \item \textit{auto}, use ``svd'' if $n\_samples > n\_features$, otherwise use ``eigen''
      \item \textit{svd}, force use of singular value decomposition of X when X is
      dense, eigenvalue decomposition of $X^T.X$ when X is sparse
      \item \textit{eigen}, force computation via eigendecomposition of $X.X^T$
      \end{itemize}
  \default{auto}

    \item \xmlNode{alpha\_per\_target}: \xmlDesc{[True, Yes, 1, False, No, 0, t, y, 1, f, n, 0]}, 
      Flag indicating whether to optimize the alpha value for each target separately
      (for multi-output settings: multiple prediction targets). When set to True, after fitting,
      the alpha\_ attribute will contain a value for each target. When set to False, a single alpha
      is used for all targets. New in version 0.24. (not used)
  \default{False}

    \item \xmlNode{cv}: \xmlDesc{integer}, 
      Determines the cross-validation splitting strategy.
      It specifies the number of folds..
  \default{None}

    \item \xmlNode{alphas}: \xmlDesc{tuple of comma-separated float}, 
      Array of alpha values to try. Regularization strength; must be a positive float.
      Regularization                                                  improves the conditioning of
      the problem and reduces the variance of the estimates.
      Larger values specify stronger regularization. Alpha corresponds to $1 / (2C)$ in other
      linear models such as LogisticRegression or LinearSVC.
  \default{(0.1, 1.0, 10.0)}

    \item \xmlNode{scoring}: \xmlDesc{string}, 
      A string (see model evaluation documentation) or a scorer
      callable object / function with signature.
  \default{None}

    \item \xmlNode{store\_cv\_values}: \xmlDesc{[True, Yes, 1, False, No, 0, t, y, 1, f, n, 0]}, 
      Flag indicating if the cross-validation values corresponding
      to each alpha should be stored in the cv\_values\_ attribute (see below).
      This flag is only compatible with cv=None (i.e. using Leave-One-Out
      Cross-Validation).
  \default{False}
  \end{itemize}


\subsubsection{RidgeClassifier}
  The \xmlNode{RidgeClassifier} is a classifier that uses Ridge regression.
  This classifier first converts the target values into {-1, 1} and then treats
  the problem as a regression task (multi-output regression in the multiclass case).
  \zNormalizationNotPerformed{RidgeClassifier}

  The \xmlNode{RidgeClassifier} node recognizes the following parameters:
    \begin{itemize}
      \item \xmlAttr{name}: \xmlDesc{string, required}, 
        User-defined name to designate this entity in the RAVEN input file.
      \item \xmlAttr{verbosity}: \xmlDesc{[silent, quiet, all, debug], optional}, 
        Desired verbosity of messages coming from this entity
      \item \xmlAttr{subType}: \xmlDesc{string, required}, 
        specify the type of ROM that will be used
  \end{itemize}

  The \xmlNode{RidgeClassifier} node recognizes the following subnodes:
  \begin{itemize}
    \item \xmlNode{Features}: \xmlDesc{comma-separated strings}, 
      specifies the names of the features of this ROM.         \nb These parameters are going to be
      requested for the training of this object         (see Section~\ref{subsec:stepRomTrainer})

    \item \xmlNode{Target}: \xmlDesc{comma-separated strings}, 
      contains a comma separated list of the targets of this ROM. These parameters         are the
      Figures of Merit (FOMs) this ROM is supposed to predict.         \nb These parameters are
      going to be requested for the training of this         object (see Section
      \ref{subsec:stepRomTrainer}).

    \item \xmlNode{pivotParameter}: \xmlDesc{string}, 
      If a time-dependent ROM is requested, please specifies the pivot         variable (e.g. time,
      etc) used in the input HistorySet.
  \default{time}

    \item \xmlNode{CV}: \xmlDesc{string}, 
      The text portion of this node needs to contain the name of the \xmlNode{PostProcessor} with
      \xmlAttr{subType}         ``CrossValidation``.
      The \xmlNode{CV} node recognizes the following parameters:
        \begin{itemize}
          \item \xmlAttr{class}: \xmlDesc{string, optional}, 
            should be set to \xmlString{Model}
          \item \xmlAttr{type}: \xmlDesc{string, optional}, 
            should be set to \xmlString{PostProcessor}
      \end{itemize}

    \item \xmlNode{alias}: \xmlDesc{string}, 
      specifies alias for         any variable of interest in the input or output space. These
      aliases can be used anywhere in the RAVEN input to         refer to the variables. In the body
      of this node the user specifies the name of the variable that the model is going to use
      (during its execution).
      The \xmlNode{alias} node recognizes the following parameters:
        \begin{itemize}
          \item \xmlAttr{variable}: \xmlDesc{string, required}, 
            define the actual alias, usable throughout the RAVEN input
          \item \xmlAttr{type}: \xmlDesc{[input, output], required}, 
            either ``input'' or ``output''.
      \end{itemize}

    \item \xmlNode{alpha}: \xmlDesc{float}, 
      Regularization strength; must be a positive float. Regularization
      improves the conditioning of the problem and reduces the variance of the estimates.
      Larger values specify stronger regularization. Alpha corresponds to $1 / (2C)$ in other
      linear models such as LogisticRegression or LinearSVC.
  \default{1.0}

    \item \xmlNode{fit\_intercept}: \xmlDesc{[True, Yes, 1, False, No, 0, t, y, 1, f, n, 0]}, 
      Whether the intercept should be estimated or not. If False,
      the data is assumed to be already centered.
  \default{True}

    \item \xmlNode{normalize}: \xmlDesc{[True, Yes, 1, False, No, 0, t, y, 1, f, n, 0]}, 
      This parameter is ignored when fit\_intercept is set to False. If True, the
      regressors X will be normalized before regression by subtracting the mean and dividing
      by the l2-norm.
  \default{False}

    \item \xmlNode{max\_iter}: \xmlDesc{integer}, 
      Maximum number of iterations for conjugate gradient solver.
  \default{None}

    \item \xmlNode{tol}: \xmlDesc{float}, 
      Precision of the solution
  \default{0.001}

    \item \xmlNode{solver}: \xmlDesc{[auto, svd, cholesky, lsqr, sparse\_cg, sag, saga]}, 
      Solver to use in the computational routines:
      \begin{itemize}                                                    \item auto, chooses the
      solver automatically based on the type of data.
      \item svd, uses a Singular Value Decomposition of X to compute the Ridge coefficients. More
      stable for singular                                                                matrices
      than ``cholesky''.                                                    \item cholesky, uses the
      standard scipy.linalg.solve function to obtain a closed-form solution.
      \item sparse\_cg, uses the conjugate gradient solver as found in scipy.sparse.linalg.cg. As an
      iterative algorithm,                                                               this solver
      is more appropriate than ‘cholesky’ for large-scale data (possibility to set tol and
      max\_iter).                                                    \item lsqr, uses the dedicated
      regularized least-squares routine scipy.sparse.linalg.lsqr. It is the fastest and uses
      an iterative procedure.                                                    \item sag, uses a
      Stochastic Average Gradient descent, and ``saga'' uses its improved, unbiased version named
      SAGA.                                                               Both methods also use an
      iterative procedure, and are often faster than other solvers when both
      n\_samples and n\_features are large. Note that ``sag'' and ``saga'' fast convergence is only
      guaranteed on                                                               features with
      approximately the same scale. You can preprocess the data with a scaler from
      sklearn.preprocessing.                                                  \end{itemize}
  \default{auto}

    \item \xmlNode{class\_weight}: \xmlDesc{[balanced]}, 
      If not given, all classes are supposed to have weight one.
      The “balanced” mode uses the values of y to automatically adjust weights
      inversely proportional to class frequencies in the input data
  \default{None}

    \item \xmlNode{random\_state}: \xmlDesc{integer}, 
      Used to shuffle the training data, when shuffle is set to
      True. Pass an int for reproducible output across multiple function calls.
  \default{None}
  \end{itemize}


\subsubsection{RidgeClassifierCV}
  The \xmlNode{RidgeClassifierCV} is a classifier that uses Ridge regression.
  This classifier first converts the target values into {-1, 1} and then treats
  the problem as a regression task (multi-output regression in the multiclass case).
  In addition, a cross-validation method is applied to optimize the hyper-parameter.
  By default, it performs Leave-One-Out Cross-Validation.
  \zNormalizationNotPerformed{RidgeClassifierCV}

  The \xmlNode{RidgeClassifierCV} node recognizes the following parameters:
    \begin{itemize}
      \item \xmlAttr{name}: \xmlDesc{string, required}, 
        User-defined name to designate this entity in the RAVEN input file.
      \item \xmlAttr{verbosity}: \xmlDesc{[silent, quiet, all, debug], optional}, 
        Desired verbosity of messages coming from this entity
      \item \xmlAttr{subType}: \xmlDesc{string, required}, 
        specify the type of ROM that will be used
  \end{itemize}

  The \xmlNode{RidgeClassifierCV} node recognizes the following subnodes:
  \begin{itemize}
    \item \xmlNode{Features}: \xmlDesc{comma-separated strings}, 
      specifies the names of the features of this ROM.         \nb These parameters are going to be
      requested for the training of this object         (see Section~\ref{subsec:stepRomTrainer})

    \item \xmlNode{Target}: \xmlDesc{comma-separated strings}, 
      contains a comma separated list of the targets of this ROM. These parameters         are the
      Figures of Merit (FOMs) this ROM is supposed to predict.         \nb These parameters are
      going to be requested for the training of this         object (see Section
      \ref{subsec:stepRomTrainer}).

    \item \xmlNode{pivotParameter}: \xmlDesc{string}, 
      If a time-dependent ROM is requested, please specifies the pivot         variable (e.g. time,
      etc) used in the input HistorySet.
  \default{time}

    \item \xmlNode{CV}: \xmlDesc{string}, 
      The text portion of this node needs to contain the name of the \xmlNode{PostProcessor} with
      \xmlAttr{subType}         ``CrossValidation``.
      The \xmlNode{CV} node recognizes the following parameters:
        \begin{itemize}
          \item \xmlAttr{class}: \xmlDesc{string, optional}, 
            should be set to \xmlString{Model}
          \item \xmlAttr{type}: \xmlDesc{string, optional}, 
            should be set to \xmlString{PostProcessor}
      \end{itemize}

    \item \xmlNode{alias}: \xmlDesc{string}, 
      specifies alias for         any variable of interest in the input or output space. These
      aliases can be used anywhere in the RAVEN input to         refer to the variables. In the body
      of this node the user specifies the name of the variable that the model is going to use
      (during its execution).
      The \xmlNode{alias} node recognizes the following parameters:
        \begin{itemize}
          \item \xmlAttr{variable}: \xmlDesc{string, required}, 
            define the actual alias, usable throughout the RAVEN input
          \item \xmlAttr{type}: \xmlDesc{[input, output], required}, 
            either ``input'' or ``output''.
      \end{itemize}

    \item \xmlNode{fit\_intercept}: \xmlDesc{[True, Yes, 1, False, No, 0, t, y, 1, f, n, 0]}, 
      Whether the intercept should be estimated or not. If False,
      the data is assumed to be already centered.
  \default{True}

    \item \xmlNode{normalize}: \xmlDesc{[True, Yes, 1, False, No, 0, t, y, 1, f, n, 0]}, 
      This parameter is ignored when fit\_intercept is set to False. If True, the
      regressors X will be normalized before regression by subtracting the mean and dividing
      by the l2-norm.
  \default{False}

    \item \xmlNode{cv}: \xmlDesc{integer}, 
      Determines the cross-validation splitting strategy.
      It specifies the number of folds..
  \default{None}

    \item \xmlNode{alphas}: \xmlDesc{comma-separated floats}, 
      Array of alpha values to try. Regularization strength; must be a positive float.
      Regularization                                                  improves the conditioning of
      the problem and reduces the variance of the estimates.
      Larger values specify stronger regularization. Alpha corresponds to $1 / (2C)$ in other
      linear models such as LogisticRegression or LinearSVC.
  \default{[0.1, 1.0, 10.0]}

    \item \xmlNode{scoring}: \xmlDesc{string}, 
      A string (see model evaluation documentation) or a scorer
      callable object / function with signature.
  \default{None}

    \item \xmlNode{class\_weight}: \xmlDesc{[balanced]}, 
      If not given, all classes are supposed to have weight one.
      The “balanced” mode uses the values of y to automatically adjust weights
      inversely proportional to class frequencies in the input data
  \default{None}

    \item \xmlNode{store\_cv\_values}: \xmlDesc{[True, Yes, 1, False, No, 0, t, y, 1, f, n, 0]}, 
      Flag indicating if the cross-validation values corresponding
      to each alpha should be stored in the cv\_values\_ attribute (see below).
      This flag is only compatible with cv=None (i.e. using Leave-One-Out
      Cross-Validation).
  \default{False}
  \end{itemize}


\subsubsection{SGDClassifier}
  The \xmlNode{SGDClassifier} implements regularized linear models with stochastic
  gradient descent (SGD) learning for classification: the gradient of the loss is estimated each
  sample at                         a time and the model is updated along the way with a decreasing
  strength schedule                         (aka learning rate). For best results using the default
  learning rate schedule, the                         data should have zero mean and unit variance.
  This implementation works with data represented as dense or sparse arrays of floating
  point values for the features. The model it fits can be controlled with the loss parameter;
  by default, it fits a linear support vector machine (SVM).                         The regularizer
  is a penalty added to the loss function that shrinks model parameters towards
  the zero vector using either the squared euclidean norm L2 or the absolute norm L1 or a
  combination of both (Elastic Net). If the parameter update crosses the 0.0 value because
  of the regularizer, the update is truncated to $0.0$ to allow for learning sparse models and
  achieve online feature selection.                         \zNormalizationPerformed{SGDClassifier}

  The \xmlNode{SGDClassifier} node recognizes the following parameters:
    \begin{itemize}
      \item \xmlAttr{name}: \xmlDesc{string, required}, 
        User-defined name to designate this entity in the RAVEN input file.
      \item \xmlAttr{verbosity}: \xmlDesc{[silent, quiet, all, debug], optional}, 
        Desired verbosity of messages coming from this entity
      \item \xmlAttr{subType}: \xmlDesc{string, required}, 
        specify the type of ROM that will be used
  \end{itemize}

  The \xmlNode{SGDClassifier} node recognizes the following subnodes:
  \begin{itemize}
    \item \xmlNode{Features}: \xmlDesc{comma-separated strings}, 
      specifies the names of the features of this ROM.         \nb These parameters are going to be
      requested for the training of this object         (see Section~\ref{subsec:stepRomTrainer})

    \item \xmlNode{Target}: \xmlDesc{comma-separated strings}, 
      contains a comma separated list of the targets of this ROM. These parameters         are the
      Figures of Merit (FOMs) this ROM is supposed to predict.         \nb These parameters are
      going to be requested for the training of this         object (see Section
      \ref{subsec:stepRomTrainer}).

    \item \xmlNode{pivotParameter}: \xmlDesc{string}, 
      If a time-dependent ROM is requested, please specifies the pivot         variable (e.g. time,
      etc) used in the input HistorySet.
  \default{time}

    \item \xmlNode{CV}: \xmlDesc{string}, 
      The text portion of this node needs to contain the name of the \xmlNode{PostProcessor} with
      \xmlAttr{subType}         ``CrossValidation``.
      The \xmlNode{CV} node recognizes the following parameters:
        \begin{itemize}
          \item \xmlAttr{class}: \xmlDesc{string, optional}, 
            should be set to \xmlString{Model}
          \item \xmlAttr{type}: \xmlDesc{string, optional}, 
            should be set to \xmlString{PostProcessor}
      \end{itemize}

    \item \xmlNode{alias}: \xmlDesc{string}, 
      specifies alias for         any variable of interest in the input or output space. These
      aliases can be used anywhere in the RAVEN input to         refer to the variables. In the body
      of this node the user specifies the name of the variable that the model is going to use
      (during its execution).
      The \xmlNode{alias} node recognizes the following parameters:
        \begin{itemize}
          \item \xmlAttr{variable}: \xmlDesc{string, required}, 
            define the actual alias, usable throughout the RAVEN input
          \item \xmlAttr{type}: \xmlDesc{[input, output], required}, 
            either ``input'' or ``output''.
      \end{itemize}

    \item \xmlNode{loss}: \xmlDesc{[hinge, log, modified\_huber, squared\_hinge, perceptron, squared\_loss, huber, epsilon\_insensitive, squared\_epsilon\_insensitive]}, 
      The loss function to be used. Defaults to ``hinge'', which gives a linear SVM.The ``log'' loss
      gives logistic regression, a                                                  probabilistic
      classifier. ``modified\_huber'' is another smooth loss that brings tolerance to outliers as
      well as probability estimates.
      ``squared\_hinge'' is like hinge but is quadratically penalized. ``perceptron'' is the linear
      loss used by the perceptron algorithm.                                                  The
      other losses are designed for regression but can be useful in classification as well; see
      SGDRegressor for a description.
  \default{hinge}

    \item \xmlNode{penalty}: \xmlDesc{[l2, l1, elasticnet]}, 
      The penalty (aka regularization term) to be used. Defaults to ``l2'' which is the standard
      regularizer for linear SVM models.                                                  ``l1'' and
      ``elasticnet'' might bring sparsity to the model (feature selection) not achievable with
      ``l2''.
  \default{l2}

    \item \xmlNode{alpha}: \xmlDesc{float}, 
      Constant that multiplies the regularization term. The higher the value, the stronger the
      regularization. Also used to compute                                                  the
      learning rate when set to learning\_rate is set to ``optimal''.
  \default{0.0001}

    \item \xmlNode{l1\_ratio}: \xmlDesc{float}, 
      The Elastic Net mixing parameter, with $0 <= l1\_ratio <= 1$. $l1\_ratio=0$ corresponds to L2
      penalty, $l1\_ratio=1$ to L1.                                                  Only used if
      penalty is ``elasticnet''.
  \default{0.15}

    \item \xmlNode{fit\_intercept}: \xmlDesc{[True, Yes, 1, False, No, 0, t, y, 1, f, n, 0]}, 
      Whether the intercept should be estimated or not. If False,
      the data is assumed to be already centered.
  \default{True}

    \item \xmlNode{max\_iter}: \xmlDesc{integer}, 
      The maximum number of passes over the training data (aka epochs).
  \default{1000}

    \item \xmlNode{tol}: \xmlDesc{float}, 
      The stopping criterion. If it is not None, training will stop when $(loss > best\_loss - tol)$
      for $n\_iter\_no\_change$                                                  consecutive epochs.
  \default{0.001}

    \item \xmlNode{shuffle}: \xmlDesc{[True, Yes, 1, False, No, 0, t, y, 1, f, n, 0]}, 
      TWhether or not the training data should be shuffled after each epoch
  \default{True}

    \item \xmlNode{epsilon}: \xmlDesc{float}, 
      Epsilon in the epsilon-insensitive loss functions; only if loss is ``huber'',
      ``epsilon\_insensitive'', or
      ``squared\_epsilon\_insensitive''. For ``huber'', determines the threshold at which it becomes
      less important to get the                                                  prediction exactly
      right. For epsilon-insensitive, any differences between the current prediction and the correct
      label                                                  are ignored if they are less than this
      threshold.
  \default{0.1}

    \item \xmlNode{learning\_rate}: \xmlDesc{[constant, optimal, invscaling, adaptive]}, 
      The learning rate schedule:                                                  \begin{itemize}
      \item constant: $eta = eta0$                                                   \item optimal:
      $eta = 1.0 / (alpha * (t + t0))$ where t0 is chosen by a heuristic proposed by Leon Bottou.
      \item invscaling: $eta = eta0 / pow(t, power\_t)$
      \item adaptive: $eta = eta0$, as long as the training keeps decreasing. Each time
      n\_iter\_no\_change consecutive epochs fail
      to decrease the training loss by tol or fail to increase validation score by tol if
      early\_stopping is True, the current
      learning rate is divided by 5.                                                  \end{itemize}
  \default{optimal}

    \item \xmlNode{eta0}: \xmlDesc{float}, 
      The initial learning rate for the ``constant'', ``invscaling'' or ``adaptive'' schedules. The
      default value is 0.0                                                  as eta0 is not used by
      the default schedule ``optimal''.
  \default{0.0}

    \item \xmlNode{power\_t}: \xmlDesc{float}, 
      The exponent for inverse scaling learning rate.
  \default{0.5}

    \item \xmlNode{early\_stopping}: \xmlDesc{[True, Yes, 1, False, No, 0, t, y, 1, f, n, 0]}, 
      hether to use early stopping to terminate training when validation score is not
      improving. If set to True, it will automatically set aside a stratified fraction of training
      data as validation and terminate training when validation score is not improving by at least
      tol for n\_iter\_no\_change consecutive epochs.
  \default{False}

    \item \xmlNode{validation\_fraction}: \xmlDesc{float}, 
      The proportion of training data to set aside as validation set for early stopping.
      Must be between 0 and 1. Only used if early\_stopping is True.
  \default{0.1}

    \item \xmlNode{n\_iter\_no\_change}: \xmlDesc{integer}, 
      Number of iterations with no improvement to wait before early stopping.
  \default{5}

    \item \xmlNode{random\_state}: \xmlDesc{integer}, 
      Used to shuffle the training data, when shuffle is set to
      True. Pass an int for reproducible output across multiple function calls.
  \default{None}

    \item \xmlNode{verbose}: \xmlDesc{integer}, 
      The verbosity level
  \default{0}

    \item \xmlNode{class\_weight}: \xmlDesc{[balanced]}, 
      If not given, all classes are supposed to have weight one.
      The “balanced” mode uses the values of y to automatically adjust weights
      inversely proportional to class frequencies in the input data
  \default{None}

    \item \xmlNode{warm\_start}: \xmlDesc{[True, Yes, 1, False, No, 0, t, y, 1, f, n, 0]}, 
      When set to True, reuse the solution of the previous call
      to fit as initialization, otherwise, just erase the previous solution.
  \default{False}

    \item \xmlNode{average}: \xmlDesc{[True, Yes, 1, False, No, 0, t, y, 1, f, n, 0]}, 
      When set to True, computes the averaged SGD weights accross
      all updates and stores the result in the coef\_ attribute.
  \default{False}
  \end{itemize}


\subsubsection{SGDRegressor}
  The \xmlNode{SGDRegressor} implements regularized linear models with stochastic
  gradient descent (SGD) learning for regression: the gradient of the loss is estimated each sample
  at                         a time and the model is updated along the way with a decreasing
  strength schedule                         (aka learning rate). For best results using the default
  learning rate schedule, the                         data should have zero mean and unit variance.
  This implementation works with data represented as dense or sparse arrays of floating
  point values for the features. The model it fits can be controlled with the loss parameter;
  by default, it fits a linear support vector machine (SVM).                         The regularizer
  is a penalty added to the loss function that shrinks model parameters towards
  the zero vector using either the squared euclidean norm L2 or the absolute norm L1 or a
  combination of both (Elastic Net). If the parameter update crosses the 0.0 value because
  of the regularizer, the update is truncated to $0.0$ to allow for learning sparse models and
  achieve online feature selection.                         This implementation works with data
  represented as dense arrays of floating point values for the features.
  \zNormalizationPerformed{SGDRegressor}

  The \xmlNode{SGDRegressor} node recognizes the following parameters:
    \begin{itemize}
      \item \xmlAttr{name}: \xmlDesc{string, required}, 
        User-defined name to designate this entity in the RAVEN input file.
      \item \xmlAttr{verbosity}: \xmlDesc{[silent, quiet, all, debug], optional}, 
        Desired verbosity of messages coming from this entity
      \item \xmlAttr{subType}: \xmlDesc{string, required}, 
        specify the type of ROM that will be used
  \end{itemize}

  The \xmlNode{SGDRegressor} node recognizes the following subnodes:
  \begin{itemize}
    \item \xmlNode{Features}: \xmlDesc{comma-separated strings}, 
      specifies the names of the features of this ROM.         \nb These parameters are going to be
      requested for the training of this object         (see Section~\ref{subsec:stepRomTrainer})

    \item \xmlNode{Target}: \xmlDesc{comma-separated strings}, 
      contains a comma separated list of the targets of this ROM. These parameters         are the
      Figures of Merit (FOMs) this ROM is supposed to predict.         \nb These parameters are
      going to be requested for the training of this         object (see Section
      \ref{subsec:stepRomTrainer}).

    \item \xmlNode{pivotParameter}: \xmlDesc{string}, 
      If a time-dependent ROM is requested, please specifies the pivot         variable (e.g. time,
      etc) used in the input HistorySet.
  \default{time}

    \item \xmlNode{CV}: \xmlDesc{string}, 
      The text portion of this node needs to contain the name of the \xmlNode{PostProcessor} with
      \xmlAttr{subType}         ``CrossValidation``.
      The \xmlNode{CV} node recognizes the following parameters:
        \begin{itemize}
          \item \xmlAttr{class}: \xmlDesc{string, optional}, 
            should be set to \xmlString{Model}
          \item \xmlAttr{type}: \xmlDesc{string, optional}, 
            should be set to \xmlString{PostProcessor}
      \end{itemize}

    \item \xmlNode{alias}: \xmlDesc{string}, 
      specifies alias for         any variable of interest in the input or output space. These
      aliases can be used anywhere in the RAVEN input to         refer to the variables. In the body
      of this node the user specifies the name of the variable that the model is going to use
      (during its execution).
      The \xmlNode{alias} node recognizes the following parameters:
        \begin{itemize}
          \item \xmlAttr{variable}: \xmlDesc{string, required}, 
            define the actual alias, usable throughout the RAVEN input
          \item \xmlAttr{type}: \xmlDesc{[input, output], required}, 
            either ``input'' or ``output''.
      \end{itemize}

    \item \xmlNode{loss}: \xmlDesc{[squared\_loss, huber, epsilon\_insensitive, squared\_epsilon\_insensitive]}, 
      The loss function to be used.                                                  The
      ``squared\_loss'' refers to the ordinary least squares fit. ``huber'' modifies
      ``squared\_loss'' to focus less on getting outliers correct by
      switching from squared to linear loss past a distance of epsilon. ``epsilon\_insensitive''
      ignores errors less than epsilon and is linear past
      that; this is the loss function used in SVR. ``squared\_epsilon\_insensitive'' is the same but
      becomes squared loss past a tolerance of epsilon.
  \default{squared\_loss}

    \item \xmlNode{penalty}: \xmlDesc{[l2, l1, elasticnet]}, 
      The penalty (aka regularization term) to be used. Defaults to ``l2'' which is the standard
      regularizer for linear SVM models.                                                  ``l1'' and
      ``elasticnet'' might bring sparsity to the model (feature selection) not achievable with
      ``l2''.
  \default{l2}

    \item \xmlNode{alpha}: \xmlDesc{float}, 
      Constant that multiplies the regularization term. The higher the value, the stronger the
      regularization. Also used to compute                                                  the
      learning rate when set to learning\_rate is set to ``optimal''.
  \default{0.0001}

    \item \xmlNode{l1\_ratio}: \xmlDesc{float}, 
      The Elastic Net mixing parameter, with $0 <= l1\_ratio <= 1$. $l1\_ratio=0$ corresponds to L2
      penalty, $l1\_ratio=1$ to L1.                                                  Only used if
      penalty is ``elasticnet''.
  \default{0.15}

    \item \xmlNode{fit\_intercept}: \xmlDesc{[True, Yes, 1, False, No, 0, t, y, 1, f, n, 0]}, 
      Whether the intercept should be estimated or not. If False,
      the data is assumed to be already centered.
  \default{True}

    \item \xmlNode{max\_iter}: \xmlDesc{integer}, 
      The maximum number of passes over the training data (aka epochs).
  \default{1000}

    \item \xmlNode{tol}: \xmlDesc{float}, 
      The stopping criterion. If it is not None, training will stop when $(loss > best\_loss - tol)$
      for $n\_iter\_no\_change$                                                  consecutive epochs.
  \default{0.001}

    \item \xmlNode{shuffle}: \xmlDesc{[True, Yes, 1, False, No, 0, t, y, 1, f, n, 0]}, 
      TWhether or not the training data should be shuffled after each epoch
  \default{True}

    \item \xmlNode{epsilon}: \xmlDesc{float}, 
      Epsilon in the epsilon-insensitive loss functions; only if loss is ``huber'',
      ``epsilon\_insensitive'', or
      ``squared\_epsilon\_insensitive''. For ``huber'', determines the threshold at which it becomes
      less important to get the                                                  prediction exactly
      right. For epsilon-insensitive, any differences between the current prediction and the correct
      label                                                  are ignored if they are less than this
      threshold.
  \default{0.1}

    \item \xmlNode{learning\_rate}: \xmlDesc{[constant, optimal, invscaling, adaptive]}, 
      The learning rate schedule:                                                  \begin{itemize}
      \item constant: $eta = eta0$                                                   \item optimal:
      $eta = 1.0 / (alpha * (t + t0))$ where t0 is chosen by a heuristic proposed by Leon Bottou.
      \item invscaling: $eta = eta0 / pow(t, power\_t)$
      \item adaptive: $eta = eta0$, as long as the training keeps decreasing. Each time
      n\_iter\_no\_change consecutive epochs fail
      to decrease the training loss by tol or fail to increase validation score by tol if
      early\_stopping is True, the current
      learning rate is divided by 5.                                                  \end{itemize}
  \default{optimal}

    \item \xmlNode{eta0}: \xmlDesc{float}, 
      The initial learning rate for the ``constant'', ``invscaling'' or ``adaptive'' schedules. The
      default value is 0.0                                                  as eta0 is not used by
      the default schedule ``optimal''.
  \default{0.0}

    \item \xmlNode{power\_t}: \xmlDesc{float}, 
      The exponent for inverse scaling learning rate.
  \default{0.5}

    \item \xmlNode{early\_stopping}: \xmlDesc{[True, Yes, 1, False, No, 0, t, y, 1, f, n, 0]}, 
      hether to use early stopping to terminate training when validation score is not
      improving. If set to True, it will automatically set aside a stratified fraction of training
      data as validation and terminate training when validation score is not improving by at least
      tol for n\_iter\_no\_change consecutive epochs.
  \default{False}

    \item \xmlNode{validation\_fraction}: \xmlDesc{float}, 
      The proportion of training data to set aside as validation set for early stopping.
      Must be between 0 and 1. Only used if early\_stopping is True.
  \default{0.1}

    \item \xmlNode{n\_iter\_no\_change}: \xmlDesc{integer}, 
      Number of iterations with no improvement to wait before early stopping.
  \default{5}

    \item \xmlNode{random\_state}: \xmlDesc{integer}, 
      Used to shuffle the training data, when shuffle is set to
      True. Pass an int for reproducible output across multiple function calls.
  \default{None}

    \item \xmlNode{verbose}: \xmlDesc{integer}, 
      The verbosity level
  \default{0}

    \item \xmlNode{warm\_start}: \xmlDesc{[True, Yes, 1, False, No, 0, t, y, 1, f, n, 0]}, 
      When set to True, reuse the solution of the previous call
      to fit as initialization, otherwise, just erase the previous solution.
  \default{False}

    \item \xmlNode{average}: \xmlDesc{[True, Yes, 1, False, No, 0, t, y, 1, f, n, 0]}, 
      When set to True, computes the averaged SGD weights accross
      all updates and stores the result in the coef\_ attribute.
  \default{False}
  \end{itemize}


\subsubsection{ComplementNB}
  The \\textit{ComplementNB} classifier (Complement Naive Bayes classifier) was designed to correct
  the ``severe assumptions'' made by the standard Multinomial Naive Bayes classifier.
  It is particularly suited for imbalanced data sets (see Rennie et al. (2003))
  \zNormalizationPerformed{ComplementNB}

  The \xmlNode{ComplementNB} node recognizes the following parameters:
    \begin{itemize}
      \item \xmlAttr{name}: \xmlDesc{string, required}, 
        User-defined name to designate this entity in the RAVEN input file.
      \item \xmlAttr{verbosity}: \xmlDesc{[silent, quiet, all, debug], optional}, 
        Desired verbosity of messages coming from this entity
      \item \xmlAttr{subType}: \xmlDesc{string, required}, 
        specify the type of ROM that will be used
  \end{itemize}

  The \xmlNode{ComplementNB} node recognizes the following subnodes:
  \begin{itemize}
    \item \xmlNode{Features}: \xmlDesc{comma-separated strings}, 
      specifies the names of the features of this ROM.         \nb These parameters are going to be
      requested for the training of this object         (see Section~\ref{subsec:stepRomTrainer})

    \item \xmlNode{Target}: \xmlDesc{comma-separated strings}, 
      contains a comma separated list of the targets of this ROM. These parameters         are the
      Figures of Merit (FOMs) this ROM is supposed to predict.         \nb These parameters are
      going to be requested for the training of this         object (see Section
      \ref{subsec:stepRomTrainer}).

    \item \xmlNode{pivotParameter}: \xmlDesc{string}, 
      If a time-dependent ROM is requested, please specifies the pivot         variable (e.g. time,
      etc) used in the input HistorySet.
  \default{time}

    \item \xmlNode{CV}: \xmlDesc{string}, 
      The text portion of this node needs to contain the name of the \xmlNode{PostProcessor} with
      \xmlAttr{subType}         ``CrossValidation``.
      The \xmlNode{CV} node recognizes the following parameters:
        \begin{itemize}
          \item \xmlAttr{class}: \xmlDesc{string, optional}, 
            should be set to \xmlString{Model}
          \item \xmlAttr{type}: \xmlDesc{string, optional}, 
            should be set to \xmlString{PostProcessor}
      \end{itemize}

    \item \xmlNode{alias}: \xmlDesc{string}, 
      specifies alias for         any variable of interest in the input or output space. These
      aliases can be used anywhere in the RAVEN input to         refer to the variables. In the body
      of this node the user specifies the name of the variable that the model is going to use
      (during its execution).
      The \xmlNode{alias} node recognizes the following parameters:
        \begin{itemize}
          \item \xmlAttr{variable}: \xmlDesc{string, required}, 
            define the actual alias, usable throughout the RAVEN input
          \item \xmlAttr{type}: \xmlDesc{[input, output], required}, 
            either ``input'' or ``output''.
      \end{itemize}

    \item \xmlNode{alpha}: \xmlDesc{float}, 
      Additive (Laplace and Lidstone) smoothing parameter (0 for no smoothing).
  \default{1.0}

    \item \xmlNode{norm}: \xmlDesc{[True, Yes, 1, False, No, 0, t, y, 1, f, n, 0]}, 
      Whether or not a second normalization of the weights is performed.
  \default{False}

    \item \xmlNode{class\_prior}: \xmlDesc{comma-separated floats}, 
      Prior probabilities of the classes. If specified the priors are
      not adjusted according to the data. \nb the number of elements inputted here must
      match the number of classes in the data set used in the training stage.
  \default{None}

    \item \xmlNode{fit\_prior}: \xmlDesc{[True, Yes, 1, False, No, 0, t, y, 1, f, n, 0]}, 
      Whether to learn class prior probabilities or not. If false, a uniform
      prior will be used.
  \default{True}
  \end{itemize}


\subsubsection{CategoricalNB}
  The \\textit{CategoricalNB} classifier (Naive Bayes classifier for categorical features)
  is suitable for classification with discrete features that are categorically distributed.
  The categories of each feature are drawn from a categorical distribution.
  \zNormalizationPerformed{CategoricalNB}

  The \xmlNode{CategoricalNB} node recognizes the following parameters:
    \begin{itemize}
      \item \xmlAttr{name}: \xmlDesc{string, required}, 
        User-defined name to designate this entity in the RAVEN input file.
      \item \xmlAttr{verbosity}: \xmlDesc{[silent, quiet, all, debug], optional}, 
        Desired verbosity of messages coming from this entity
      \item \xmlAttr{subType}: \xmlDesc{string, required}, 
        specify the type of ROM that will be used
  \end{itemize}

  The \xmlNode{CategoricalNB} node recognizes the following subnodes:
  \begin{itemize}
    \item \xmlNode{Features}: \xmlDesc{comma-separated strings}, 
      specifies the names of the features of this ROM.         \nb These parameters are going to be
      requested for the training of this object         (see Section~\ref{subsec:stepRomTrainer})

    \item \xmlNode{Target}: \xmlDesc{comma-separated strings}, 
      contains a comma separated list of the targets of this ROM. These parameters         are the
      Figures of Merit (FOMs) this ROM is supposed to predict.         \nb These parameters are
      going to be requested for the training of this         object (see Section
      \ref{subsec:stepRomTrainer}).

    \item \xmlNode{pivotParameter}: \xmlDesc{string}, 
      If a time-dependent ROM is requested, please specifies the pivot         variable (e.g. time,
      etc) used in the input HistorySet.
  \default{time}

    \item \xmlNode{CV}: \xmlDesc{string}, 
      The text portion of this node needs to contain the name of the \xmlNode{PostProcessor} with
      \xmlAttr{subType}         ``CrossValidation``.
      The \xmlNode{CV} node recognizes the following parameters:
        \begin{itemize}
          \item \xmlAttr{class}: \xmlDesc{string, optional}, 
            should be set to \xmlString{Model}
          \item \xmlAttr{type}: \xmlDesc{string, optional}, 
            should be set to \xmlString{PostProcessor}
      \end{itemize}

    \item \xmlNode{alias}: \xmlDesc{string}, 
      specifies alias for         any variable of interest in the input or output space. These
      aliases can be used anywhere in the RAVEN input to         refer to the variables. In the body
      of this node the user specifies the name of the variable that the model is going to use
      (during its execution).
      The \xmlNode{alias} node recognizes the following parameters:
        \begin{itemize}
          \item \xmlAttr{variable}: \xmlDesc{string, required}, 
            define the actual alias, usable throughout the RAVEN input
          \item \xmlAttr{type}: \xmlDesc{[input, output], required}, 
            either ``input'' or ``output''.
      \end{itemize}

    \item \xmlNode{alpha}: \xmlDesc{float}, 
      Additive (Laplace and Lidstone) smoothing parameter (0 for no smoothing).
  \default{1.0}

    \item \xmlNode{class\_prior}: \xmlDesc{comma-separated floats}, 
      Prior probabilities of the classes. If specified the priors are
      not adjusted according to the data. \nb the number of elements inputted here must
      match the number of classes in the data set used in the training stage.
  \default{None}

    \item \xmlNode{fit\_prior}: \xmlDesc{[True, Yes, 1, False, No, 0, t, y, 1, f, n, 0]}, 
      Whether to learn class prior probabilities or not. If false, a uniform
      prior will be used.
  \default{True}
  \end{itemize}


\subsubsection{BernoulliNB}
  The \textit{BernoulliNB} classifier implements the naive Bayes training and
  classification algorithms for data that is distributed according to multivariate
  Bernoulli distributions; i.e., there may be multiple features but each one is
  assumed to be a binary-valued (Bernoulli, boolean) variable.                          Therefore,
  this class requires samples to be represented as binary-valued                          feature
  vectors; if handed any other kind of data, a \textit{Bernoulli Naive
  Bayes} instance may binarize its input (depending on the binarize parameter).
  The decision rule for Bernoulli naive Bayes is based on                          \begin{equation}
  P(x\_i \mid y) = P(i \mid y) x\_i + (1 - P(i \mid y)) (1 - x\_i)
  \end{equation}                          which differs from multinomial NB's rule in that it
  explicitly penalizes the                          non-occurrence of a feature $i$ that is an
  indicator for class $y$, where the                          multinomial variant would simply
  ignore a non-occurring feature.                          In the case of text classification, word
  occurrence vectors (rather than word                          count vectors) may be used to train
  and use this classifier.                          \textit{Bernoulli Naive Bayes} might perform
  better on some datasets, especially                          those with shorter documents.
  \zNormalizationPerformed{BernoulliNB}

  The \xmlNode{BernoulliNB} node recognizes the following parameters:
    \begin{itemize}
      \item \xmlAttr{name}: \xmlDesc{string, required}, 
        User-defined name to designate this entity in the RAVEN input file.
      \item \xmlAttr{verbosity}: \xmlDesc{[silent, quiet, all, debug], optional}, 
        Desired verbosity of messages coming from this entity
      \item \xmlAttr{subType}: \xmlDesc{string, required}, 
        specify the type of ROM that will be used
  \end{itemize}

  The \xmlNode{BernoulliNB} node recognizes the following subnodes:
  \begin{itemize}
    \item \xmlNode{Features}: \xmlDesc{comma-separated strings}, 
      specifies the names of the features of this ROM.         \nb These parameters are going to be
      requested for the training of this object         (see Section~\ref{subsec:stepRomTrainer})

    \item \xmlNode{Target}: \xmlDesc{comma-separated strings}, 
      contains a comma separated list of the targets of this ROM. These parameters         are the
      Figures of Merit (FOMs) this ROM is supposed to predict.         \nb These parameters are
      going to be requested for the training of this         object (see Section
      \ref{subsec:stepRomTrainer}).

    \item \xmlNode{pivotParameter}: \xmlDesc{string}, 
      If a time-dependent ROM is requested, please specifies the pivot         variable (e.g. time,
      etc) used in the input HistorySet.
  \default{time}

    \item \xmlNode{CV}: \xmlDesc{string}, 
      The text portion of this node needs to contain the name of the \xmlNode{PostProcessor} with
      \xmlAttr{subType}         ``CrossValidation``.
      The \xmlNode{CV} node recognizes the following parameters:
        \begin{itemize}
          \item \xmlAttr{class}: \xmlDesc{string, optional}, 
            should be set to \xmlString{Model}
          \item \xmlAttr{type}: \xmlDesc{string, optional}, 
            should be set to \xmlString{PostProcessor}
      \end{itemize}

    \item \xmlNode{alias}: \xmlDesc{string}, 
      specifies alias for         any variable of interest in the input or output space. These
      aliases can be used anywhere in the RAVEN input to         refer to the variables. In the body
      of this node the user specifies the name of the variable that the model is going to use
      (during its execution).
      The \xmlNode{alias} node recognizes the following parameters:
        \begin{itemize}
          \item \xmlAttr{variable}: \xmlDesc{string, required}, 
            define the actual alias, usable throughout the RAVEN input
          \item \xmlAttr{type}: \xmlDesc{[input, output], required}, 
            either ``input'' or ``output''.
      \end{itemize}

    \item \xmlNode{alpha}: \xmlDesc{float}, 
      Additive (Laplace and Lidstone) smoothing parameter (0 for no smoothing).
  \default{1.0}

    \item \xmlNode{binarize}: \xmlDesc{float}, 
      Threshold for binarizing (mapping to booleans) of sample features. If None,
      input is presumed to already consist of binary vectors.
  \default{None}

    \item \xmlNode{class\_prior}: \xmlDesc{comma-separated floats}, 
      Prior probabilities of the classes. If specified the priors are
      not adjusted according to the data. \nb the number of elements inputted here must
      match the number of classes in the data set used in the training stage.
  \default{None}

    \item \xmlNode{fit\_prior}: \xmlDesc{[True, Yes, 1, False, No, 0, t, y, 1, f, n, 0]}, 
      Whether to learn class prior probabilities or not. If false, a uniform
      prior will be used.
  \default{True}
  \end{itemize}


\subsubsection{MultinomialNB}
  The \\textit{MultinomialNB} classifier implements the naive Bayes algorithm for
  multinomially distributed data, and is one of the two classic naive Bayes
  variants used in text classification (where the data is typically represented
  as word vector counts, although tf-idf vectors are also known to work well in
  practice).                         The distribution is parametrized by vectors $\theta\_y =
  (\theta\_{y1},\ldots,\theta\_{yn})$ for each class $y$, where $n$ is the number of
  features (in text classification, the size of the vocabulary) and $\theta\_{yi}$
  is the probability $P(x\_i \mid y)$ of feature $i$ appearing in a sample
  belonging to class $y$.                         The parameters $\theta\_y$ are estimated by a
  smoothed version of maximum                         likelihood, i.e. relative frequency counting:
  \begin{equation}                         \hat{\theta}\_{yi} = \frac{ N\_{yi} + \alpha}{N\_y + \alpha
  n}                         \end{equation}                         where $N\_{yi} = \sum\_{x \in T}
  x\_i$ is the number of times feature $i$ appears                         in a sample of class y in
  the training set T, and                         $N\_{y} = \sum\_{i=1}^{|T|} N\_{yi}$ is the total
  count of all features for class                         $y$.                         The smoothing
  priors $\alpha \ge 0$ account for features not present in the                         learning
  samples and prevents zero probabilities in further computations.                         Setting
  $\alpha = 1$ is called Laplace smoothing, while $\alpha < 1$ is called
  Lidstone smoothing.                         \zNormalizationPerformed{MultinomialNB}

  The \xmlNode{MultinomialNB} node recognizes the following parameters:
    \begin{itemize}
      \item \xmlAttr{name}: \xmlDesc{string, required}, 
        User-defined name to designate this entity in the RAVEN input file.
      \item \xmlAttr{verbosity}: \xmlDesc{[silent, quiet, all, debug], optional}, 
        Desired verbosity of messages coming from this entity
      \item \xmlAttr{subType}: \xmlDesc{string, required}, 
        specify the type of ROM that will be used
  \end{itemize}

  The \xmlNode{MultinomialNB} node recognizes the following subnodes:
  \begin{itemize}
    \item \xmlNode{Features}: \xmlDesc{comma-separated strings}, 
      specifies the names of the features of this ROM.         \nb These parameters are going to be
      requested for the training of this object         (see Section~\ref{subsec:stepRomTrainer})

    \item \xmlNode{Target}: \xmlDesc{comma-separated strings}, 
      contains a comma separated list of the targets of this ROM. These parameters         are the
      Figures of Merit (FOMs) this ROM is supposed to predict.         \nb These parameters are
      going to be requested for the training of this         object (see Section
      \ref{subsec:stepRomTrainer}).

    \item \xmlNode{pivotParameter}: \xmlDesc{string}, 
      If a time-dependent ROM is requested, please specifies the pivot         variable (e.g. time,
      etc) used in the input HistorySet.
  \default{time}

    \item \xmlNode{CV}: \xmlDesc{string}, 
      The text portion of this node needs to contain the name of the \xmlNode{PostProcessor} with
      \xmlAttr{subType}         ``CrossValidation``.
      The \xmlNode{CV} node recognizes the following parameters:
        \begin{itemize}
          \item \xmlAttr{class}: \xmlDesc{string, optional}, 
            should be set to \xmlString{Model}
          \item \xmlAttr{type}: \xmlDesc{string, optional}, 
            should be set to \xmlString{PostProcessor}
      \end{itemize}

    \item \xmlNode{alias}: \xmlDesc{string}, 
      specifies alias for         any variable of interest in the input or output space. These
      aliases can be used anywhere in the RAVEN input to         refer to the variables. In the body
      of this node the user specifies the name of the variable that the model is going to use
      (during its execution).
      The \xmlNode{alias} node recognizes the following parameters:
        \begin{itemize}
          \item \xmlAttr{variable}: \xmlDesc{string, required}, 
            define the actual alias, usable throughout the RAVEN input
          \item \xmlAttr{type}: \xmlDesc{[input, output], required}, 
            either ``input'' or ``output''.
      \end{itemize}

    \item \xmlNode{class\_prior}: \xmlDesc{comma-separated floats}, 
      Prior probabilities of the classes. If specified the priors are
      not adjusted according to the data. \nb the number of elements inputted here must
      match the number of classes in the data set used in the training stage.
  \default{None}

    \item \xmlNode{alpha}: \xmlDesc{float}, 
      Additive (Laplace and Lidstone) smoothing parameter (0 for no smoothing).
  \default{1.0}

    \item \xmlNode{fit\_prior}: \xmlDesc{[True, Yes, 1, False, No, 0, t, y, 1, f, n, 0]}, 
      Whether to learn class prior probabilities or not. If false, a uniform
      prior will be used.
  \default{True}
  \end{itemize}


\subsubsection{GaussianNB}
  The \\textit{GaussianNB} classifier implements the Gaussian Naive Bayes
  algorithm for classification.                          The likelihood of the features is assumed
  to be Gaussian:                          \begin{equation}                              P(x\_i \mid
  y) = \frac{1}{\sqrt{2\pi\sigma^2\_y}} \exp\left(-\frac{(x\_i -
  \mu\_y)^2}{2\sigma^2\_y}\right)                          \end{equation}                          The
  parameters $\sigma\_y$ and $\mu\_y$ are estimated using maximum likelihood.
  \zNormalizationPerformed{GaussianNB}

  The \xmlNode{GaussianNB} node recognizes the following parameters:
    \begin{itemize}
      \item \xmlAttr{name}: \xmlDesc{string, required}, 
        User-defined name to designate this entity in the RAVEN input file.
      \item \xmlAttr{verbosity}: \xmlDesc{[silent, quiet, all, debug], optional}, 
        Desired verbosity of messages coming from this entity
      \item \xmlAttr{subType}: \xmlDesc{string, required}, 
        specify the type of ROM that will be used
  \end{itemize}

  The \xmlNode{GaussianNB} node recognizes the following subnodes:
  \begin{itemize}
    \item \xmlNode{Features}: \xmlDesc{comma-separated strings}, 
      specifies the names of the features of this ROM.         \nb These parameters are going to be
      requested for the training of this object         (see Section~\ref{subsec:stepRomTrainer})

    \item \xmlNode{Target}: \xmlDesc{comma-separated strings}, 
      contains a comma separated list of the targets of this ROM. These parameters         are the
      Figures of Merit (FOMs) this ROM is supposed to predict.         \nb These parameters are
      going to be requested for the training of this         object (see Section
      \ref{subsec:stepRomTrainer}).

    \item \xmlNode{pivotParameter}: \xmlDesc{string}, 
      If a time-dependent ROM is requested, please specifies the pivot         variable (e.g. time,
      etc) used in the input HistorySet.
  \default{time}

    \item \xmlNode{CV}: \xmlDesc{string}, 
      The text portion of this node needs to contain the name of the \xmlNode{PostProcessor} with
      \xmlAttr{subType}         ``CrossValidation``.
      The \xmlNode{CV} node recognizes the following parameters:
        \begin{itemize}
          \item \xmlAttr{class}: \xmlDesc{string, optional}, 
            should be set to \xmlString{Model}
          \item \xmlAttr{type}: \xmlDesc{string, optional}, 
            should be set to \xmlString{PostProcessor}
      \end{itemize}

    \item \xmlNode{alias}: \xmlDesc{string}, 
      specifies alias for         any variable of interest in the input or output space. These
      aliases can be used anywhere in the RAVEN input to         refer to the variables. In the body
      of this node the user specifies the name of the variable that the model is going to use
      (during its execution).
      The \xmlNode{alias} node recognizes the following parameters:
        \begin{itemize}
          \item \xmlAttr{variable}: \xmlDesc{string, required}, 
            define the actual alias, usable throughout the RAVEN input
          \item \xmlAttr{type}: \xmlDesc{[input, output], required}, 
            either ``input'' or ``output''.
      \end{itemize}

    \item \xmlNode{priors}: \xmlDesc{comma-separated floats}, 
      Prior probabilities of the classes. If specified the priors are
      not adjusted according to the data. \nb the number of elements inputted here must
      match the number of classes in the data set used in the training stage.
  \default{None}

    \item \xmlNode{var\_smoothing}: \xmlDesc{float}, 
      Portion of the largest variance of all features that is added to variances for
      calculation stability.
  \default{1e-09}
  \end{itemize}


\subsubsection{MLPClassifier}
  The \xmlNode{MLPClassifier} implements a multi-layer perceptron algorithm that trains using
  \textbf{Backpropagation}                             More precisely, it trains using some form of
  gradient descent and the gradients are calculated using Backpropagation.
  For classification, it minimizes the Cross-Entropy loss function, and it supports multi-class
  classification.                             \zNormalizationPerformed{MLPClassifier}

  The \xmlNode{MLPClassifier} node recognizes the following parameters:
    \begin{itemize}
      \item \xmlAttr{name}: \xmlDesc{string, required}, 
        User-defined name to designate this entity in the RAVEN input file.
      \item \xmlAttr{verbosity}: \xmlDesc{[silent, quiet, all, debug], optional}, 
        Desired verbosity of messages coming from this entity
      \item \xmlAttr{subType}: \xmlDesc{string, required}, 
        specify the type of ROM that will be used
  \end{itemize}

  The \xmlNode{MLPClassifier} node recognizes the following subnodes:
  \begin{itemize}
    \item \xmlNode{Features}: \xmlDesc{comma-separated strings}, 
      specifies the names of the features of this ROM.         \nb These parameters are going to be
      requested for the training of this object         (see Section~\ref{subsec:stepRomTrainer})

    \item \xmlNode{Target}: \xmlDesc{comma-separated strings}, 
      contains a comma separated list of the targets of this ROM. These parameters         are the
      Figures of Merit (FOMs) this ROM is supposed to predict.         \nb These parameters are
      going to be requested for the training of this         object (see Section
      \ref{subsec:stepRomTrainer}).

    \item \xmlNode{pivotParameter}: \xmlDesc{string}, 
      If a time-dependent ROM is requested, please specifies the pivot         variable (e.g. time,
      etc) used in the input HistorySet.
  \default{time}

    \item \xmlNode{CV}: \xmlDesc{string}, 
      The text portion of this node needs to contain the name of the \xmlNode{PostProcessor} with
      \xmlAttr{subType}         ``CrossValidation``.
      The \xmlNode{CV} node recognizes the following parameters:
        \begin{itemize}
          \item \xmlAttr{class}: \xmlDesc{string, optional}, 
            should be set to \xmlString{Model}
          \item \xmlAttr{type}: \xmlDesc{string, optional}, 
            should be set to \xmlString{PostProcessor}
      \end{itemize}

    \item \xmlNode{alias}: \xmlDesc{string}, 
      specifies alias for         any variable of interest in the input or output space. These
      aliases can be used anywhere in the RAVEN input to         refer to the variables. In the body
      of this node the user specifies the name of the variable that the model is going to use
      (during its execution).
      The \xmlNode{alias} node recognizes the following parameters:
        \begin{itemize}
          \item \xmlAttr{variable}: \xmlDesc{string, required}, 
            define the actual alias, usable throughout the RAVEN input
          \item \xmlAttr{type}: \xmlDesc{[input, output], required}, 
            either ``input'' or ``output''.
      \end{itemize}

    \item \xmlNode{hidden\_layer\_sizes}: \xmlDesc{comma-separated integers}, 
      The ith element represents the number of neurons in the ith hidden layer.
      lenght = n\_layers - 2
  \default{(100,)}

    \item \xmlNode{activation}: \xmlDesc{[identity, logistic, tanh, tanh]}, 
      Activation function for the hidden layer:
      \begin{itemize}                                                   \item identity:  no-op
      activation, useful to implement linear bottleneck, returns $f(x) = x$
      \item logistic: the logistic sigmoid function, returns $f(x) = 1 / (1 + exp(-x))$.
      \item tanh: the hyperbolic tan function, returns $f(x) = tanh(x)$.
      \item relu:  the rectified linear unit function, returns $f(x) = max(0, x)$
      \end{itemize}
  \default{relu}

    \item \xmlNode{solver}: \xmlDesc{[lbfgs, sgd, adam]}, 
      The solver for weight optimization:
      \begin{itemize}                                                   \item lbfgs: is an optimizer
      in the family of quasi-Newton methods.                                                   \item
      sgd: refers to stochastic gradient descent.
      \item adam: refers to a stochastic gradient-based optimizer proposed by Kingma, Diederik, and
      Jimmy Ba                                                  \end{itemize}
  \default{adam}

    \item \xmlNode{alpha}: \xmlDesc{float}, 
      L2 penalty (regularization term) parameter.
  \default{0.0001}

    \item \xmlNode{batch\_size}: \xmlDesc{integer or string}, 
      Size of minibatches for stochastic optimizers. If the solver is `lbfgs',
      the classifier will not use minibatch. When set to ``auto", batch\_size=min(200, n\_samples)
  \default{auto}

    \item \xmlNode{learning\_rate}: \xmlDesc{[constant, invscaling, adaptive]}, 
      Learning rate schedule for weight updates.:
      \begin{itemize}                                                   \item constant: is a
      constant learning rate given by `learning\_rate\_init'.
      \item invscaling: gradually decreases the learning rate at each time step `t' using
      an inverse scaling exponent of `power\_t'. effective\_learning\_rate = learning\_rate\_init /
      pow(t, power\_t)                                                   \item adaptive: keeps the
      learning rate constant to `learning\_rate\_init' as long as training
      loss keeps decreasing. Each time two consecutive epochs fail to decrease training loss by at
      least tol, or fail to increase validation score by at least tol if `early\_stopping' is on,
      the current learning rate is divided by 5. Only used when solver=`sgd'.
      \end{itemize}
  \default{constant}

    \item \xmlNode{learning\_rate\_init}: \xmlDesc{float}, 
      The initial learning rate used. It controls the step-size in updating the weights.
      Only used when solver=`sgd' or `adam'.
  \default{0.001}

    \item \xmlNode{power\_t}: \xmlDesc{float}, 
      The exponent for inverse scaling learning rate. It is used in updating effective
      learning rate when the learning\_rate is set to `invscaling'. Only used when solver=`sgd'.
  \default{0.5}

    \item \xmlNode{max\_iter}: \xmlDesc{integer}, 
      Maximum number of iterations. The solver iterates until convergence
      (determined by `tol') or this number of iterations. For stochastic solvers (`sgd', `adam'),
      note that this determines the number of epochs (how many times each data point will be used),
      not the number of gradient steps.
  \default{200}

    \item \xmlNode{shuffle}: \xmlDesc{[True, Yes, 1, False, No, 0, t, y, 1, f, n, 0]}, 
      Whether to shuffle samples in each iteration. Only used when solver=`sgd' or `adam'.
  \default{True}

    \item \xmlNode{random\_state}: \xmlDesc{integer}, 
      Determines random number generation for weights and bias initialization,
      train-test split if early stopping is used, and batch sampling when solver=`sgd' or `adam'.
  \default{None}

    \item \xmlNode{tol}: \xmlDesc{float}, 
      Tolerance for the optimization.
  \default{0.0001}

    \item \xmlNode{verbose}: \xmlDesc{[True, Yes, 1, False, No, 0, t, y, 1, f, n, 0]}, 
      Whether to print progress messages to stdout.
  \default{False}

    \item \xmlNode{warm\_start}: \xmlDesc{[True, Yes, 1, False, No, 0, t, y, 1, f, n, 0]}, 
      When set to True, reuse the solution of the previous call to fit as initialization, otherwise,
      just erase the previous solution.
  \default{False}

    \item \xmlNode{momentum}: \xmlDesc{float}, 
      Momentum for gradient descent update. Should be between 0 and 1. Only used when solver=`sgd'.
  \default{0.9}

    \item \xmlNode{nesterovs\_momentum}: \xmlDesc{[True, Yes, 1, False, No, 0, t, y, 1, f, n, 0]}, 
      Whether to use Nesterov's momentum. Only used when solver=`sgd' and momentum > 0.
  \default{True}

    \item \xmlNode{early\_stopping}: \xmlDesc{[True, Yes, 1, False, No, 0, t, y, 1, f, n, 0]}, 
      Whether to use early stopping to terminate training when validation score is not improving.
      If set to true, it will automatically set aside ten-percent of training data as validation and
      terminate                                                  training when validation score is
      not improving by at least tol for n\_iter\_no\_change consecutive
      epochs. The split is stratified, except in a multilabel setting. If early stopping is False,
      then                                                  the training stops when the training
      loss does not improve by more than tol for n\_iter\_no\_change
      consecutive passes over the training set. Only effective when solver=`sgd' or `adam'.
  \default{False}

    \item \xmlNode{validation\_fraction}: \xmlDesc{float}, 
      The proportion of training data to set aside as validation set for early stopping. Must be
      between 0 and 1.                                                  Only used if early\_stopping
      is True
  \default{0.1}

    \item \xmlNode{beta\_1}: \xmlDesc{float}, 
      Exponential decay rate for estimates of first moment vector in adam, should be in $[0, 1)$.
      Only used when solver=`adam'.
  \default{0.9}

    \item \xmlNode{beta\_2}: \xmlDesc{float}, 
      Exponential decay rate for estimates of second moment vector in adam, should be in $[0, 1)$.
      Only used when solver=`adam'.
  \default{0.999}

    \item \xmlNode{epsilon}: \xmlDesc{float}, 
      Value for numerical stability in adam. Only used when solver=`adam'.
  \default{1e-08}

    \item \xmlNode{n\_iter\_no\_change}: \xmlDesc{integer}, 
      Maximum number of epochs to not meet tol improvement. Only effective when
      solver=`sgd' or `adam'
  \default{10}
  \end{itemize}


\subsubsection{MLPRegressor}
  The \xmlNode{MLPRegressor} implements a multi-layer perceptron algorithm that trains using
  \textbf{Backpropagation}                             More precisely, it trains using some form of
  gradient descent and the gradients are calculated using Backpropagation.
  \zNormalizationPerformed{MLPRegressor}

  The \xmlNode{MLPRegressor} node recognizes the following parameters:
    \begin{itemize}
      \item \xmlAttr{name}: \xmlDesc{string, required}, 
        User-defined name to designate this entity in the RAVEN input file.
      \item \xmlAttr{verbosity}: \xmlDesc{[silent, quiet, all, debug], optional}, 
        Desired verbosity of messages coming from this entity
      \item \xmlAttr{subType}: \xmlDesc{string, required}, 
        specify the type of ROM that will be used
  \end{itemize}

  The \xmlNode{MLPRegressor} node recognizes the following subnodes:
  \begin{itemize}
    \item \xmlNode{Features}: \xmlDesc{comma-separated strings}, 
      specifies the names of the features of this ROM.         \nb These parameters are going to be
      requested for the training of this object         (see Section~\ref{subsec:stepRomTrainer})

    \item \xmlNode{Target}: \xmlDesc{comma-separated strings}, 
      contains a comma separated list of the targets of this ROM. These parameters         are the
      Figures of Merit (FOMs) this ROM is supposed to predict.         \nb These parameters are
      going to be requested for the training of this         object (see Section
      \ref{subsec:stepRomTrainer}).

    \item \xmlNode{pivotParameter}: \xmlDesc{string}, 
      If a time-dependent ROM is requested, please specifies the pivot         variable (e.g. time,
      etc) used in the input HistorySet.
  \default{time}

    \item \xmlNode{CV}: \xmlDesc{string}, 
      The text portion of this node needs to contain the name of the \xmlNode{PostProcessor} with
      \xmlAttr{subType}         ``CrossValidation``.
      The \xmlNode{CV} node recognizes the following parameters:
        \begin{itemize}
          \item \xmlAttr{class}: \xmlDesc{string, optional}, 
            should be set to \xmlString{Model}
          \item \xmlAttr{type}: \xmlDesc{string, optional}, 
            should be set to \xmlString{PostProcessor}
      \end{itemize}

    \item \xmlNode{alias}: \xmlDesc{string}, 
      specifies alias for         any variable of interest in the input or output space. These
      aliases can be used anywhere in the RAVEN input to         refer to the variables. In the body
      of this node the user specifies the name of the variable that the model is going to use
      (during its execution).
      The \xmlNode{alias} node recognizes the following parameters:
        \begin{itemize}
          \item \xmlAttr{variable}: \xmlDesc{string, required}, 
            define the actual alias, usable throughout the RAVEN input
          \item \xmlAttr{type}: \xmlDesc{[input, output], required}, 
            either ``input'' or ``output''.
      \end{itemize}

    \item \xmlNode{hidden\_layer\_sizes}: \xmlDesc{comma-separated integers}, 
      The ith element represents the number of neurons in the ith hidden layer.
      lenght = n\_layers - 2
  \default{(100,)}

    \item \xmlNode{activation}: \xmlDesc{[identity, logistic, tanh, tanh]}, 
      Activation function for the hidden layer:
      \begin{itemize}                                                    \item identity:  no-op
      activation, useful to implement linear bottleneck, returns $f(x) = x$
      \item logistic: the logistic sigmoid function, returns $f(x) = 1 / (1 + exp(-x))$.
      \item tanh: the hyperbolic tan function, returns $f(x) = tanh(x)$.
      \item relu:  the rectified linear unit function, returns $f(x) = max(0, x)$
      \end{itemize}
  \default{relu}

    \item \xmlNode{solver}: \xmlDesc{[lbfgs, sgd, adam]}, 
      The solver for weight optimization:
      \begin{itemize}                                                    \item lbfgs: is an
      optimizer in the family of quasi-Newton methods.
      \item sgd: refers to stochastic gradient descent.
      \item adam: refers to a stochastic gradient-based optimizer proposed by Kingma, Diederik, and
      Jimmy Ba                                                  \end{itemize}
  \default{adam}

    \item \xmlNode{alpha}: \xmlDesc{float}, 
      L2 penalty (regularization term) parameter.
  \default{0.0001}

    \item \xmlNode{batch\_size}: \xmlDesc{integer or string}, 
      Size of minibatches for stochastic optimizers. If the solver is `lbfgs',
      the classifier will not use minibatch. When set to ``auto", batch\_size=min(200, n\_samples)
  \default{auto}

    \item \xmlNode{learning\_rate}: \xmlDesc{[constant, invscaling, adaptive]}, 
      Learning rate schedule for weight updates.:
      \begin{itemize}                                                   \item constant: is a
      constant learning rate given by `learning\_rate\_init'.
      \item invscaling: gradually decreases the learning rate at each time step `t' using
      an inverse scaling exponent of `power\_t'. effective\_learning\_rate = learning\_rate\_init /
      pow(t, power\_t)                                                   \item adaptive: keeps the
      learning rate constant to `learning\_rate\_init' as long as training
      loss keeps decreasing. Each time two consecutive epochs fail to decrease training loss by at
      least tol, or fail to increase validation score by at least tol if `early\_stopping' is on,
      the current learning rate is divided by 5. Only used when solver=`sgd'.
      \end{itemize}
  \default{constant}

    \item \xmlNode{learning\_rate\_init}: \xmlDesc{float}, 
      The initial learning rate used. It controls the step-size in updating the weights.
      Only used when solver=`sgd' or `adam'.
  \default{0.001}

    \item \xmlNode{power\_t}: \xmlDesc{float}, 
      The exponent for inverse scaling learning rate. It is used in updating effective
      learning rate when the learning\_rate is set to `invscaling'. Only used when solver=`sgd'.
  \default{0.5}

    \item \xmlNode{max\_iter}: \xmlDesc{integer}, 
      Maximum number of iterations. The solver iterates until convergence
      (determined by `tol') or this number of iterations. For stochastic solvers (`sgd', `adam'),
      note that this determines the number of epochs (how many times each data point will be used),
      not the number of gradient steps.
  \default{200}

    \item \xmlNode{shuffle}: \xmlDesc{[True, Yes, 1, False, No, 0, t, y, 1, f, n, 0]}, 
      Whether to shuffle samples in each iteration. Only used when solver=`sgd' or `adam'.
  \default{True}

    \item \xmlNode{random\_state}: \xmlDesc{integer}, 
      Determines random number generation for weights and bias initialization,
      train-test split if early stopping is used, and batch sampling when solver=`sgd' or `adam'.
  \default{None}

    \item \xmlNode{tol}: \xmlDesc{float}, 
      Tolerance for the optimization.
  \default{0.0001}

    \item \xmlNode{verbose}: \xmlDesc{[True, Yes, 1, False, No, 0, t, y, 1, f, n, 0]}, 
      Whether to print progress messages to stdout.
  \default{False}

    \item \xmlNode{warm\_start}: \xmlDesc{[True, Yes, 1, False, No, 0, t, y, 1, f, n, 0]}, 
      When set to True, reuse the solution of the previous call to fit as initialization, otherwise,
      just erase the previous solution.
  \default{False}

    \item \xmlNode{momentum}: \xmlDesc{float}, 
      Momentum for gradient descent update. Should be between 0 and 1. Only used when solver=`sgd'.
  \default{0.9}

    \item \xmlNode{nesterovs\_momentum}: \xmlDesc{[True, Yes, 1, False, No, 0, t, y, 1, f, n, 0]}, 
      Whether to use Nesterov's momentum. Only used when solver=`sgd' and momentum > 0.
  \default{True}

    \item \xmlNode{early\_stopping}: \xmlDesc{[True, Yes, 1, False, No, 0, t, y, 1, f, n, 0]}, 
      Whether to use early stopping to terminate training when validation score is not improving.
      If set to true, it will automatically set aside ten-percent of training data as validation and
      terminate                                                  training when validation score is
      not improving by at least tol for n\_iter\_no\_change consecutive
      epochs. The split is stratified, except in a multilabel setting. If early stopping is False,
      then                                                  the training stops when the training
      loss does not improve by more than tol for n\_iter\_no\_change
      consecutive passes over the training set. Only effective when solver=`sgd' or `adam'.
  \default{False}

    \item \xmlNode{validation\_fraction}: \xmlDesc{float}, 
      The proportion of training data to set aside as validation set for early stopping. Must be
      between 0 and 1.                                                  Only used if early\_stopping
      is True
  \default{0.1}

    \item \xmlNode{beta\_1}: \xmlDesc{float}, 
      Exponential decay rate for estimates of first moment vector in adam, should be in $[0, 1)$.
      Only used when solver=`adam'.
  \default{0.9}

    \item \xmlNode{beta\_2}: \xmlDesc{float}, 
      Exponential decay rate for estimates of second moment vector in adam, should be in $[0, 1)$.
      Only used when solver=`adam'.
  \default{0.999}

    \item \xmlNode{epsilon}: \xmlDesc{float}, 
      Value for numerical stability in adam. Only used when solver=`adam'.
  \default{1e-08}

    \item \xmlNode{n\_iter\_no\_change}: \xmlDesc{integer}, 
      Maximum number of epochs to not meet tol improvement. Only effective when
      solver=`sgd' or `adam'
  \default{10}
  \end{itemize}


\subsubsection{GaussianProcessClassifier}
  The \xmlNode{GaussianProcessClassifier} is based on Laplace approximation. The implementation
  is based on Algorithm 3.1, 3.2, and 5.1 of Gaussian Processes for Machine Learning (GPML) by
  Rasmussen and Williams.Internally, the Laplace approximation is used for approximating the
  non-Gaussian posterior by a Gaussian. Currently, the implementation is using the
  logistic link function.The method is a generic supervised learning                          method
  primarily designed to solve classification problems.                          The advantages of
  Gaussian Processes for Machine Learning are:                          \begin{itemize}
  \item The prediction interpolates the observations (at least for regular
  correlation models).                            \item The prediction is probabilistic (Gaussian)
  so that one can compute                            empirical confidence intervals and exceedance
  probabilities that might be used                            to refit (online fitting, adaptive
  fitting) the prediction in some region of                            interest.
  \item Versatile: different linear regression models and correlation models can
  be specified.                            Common models are provided, but it is also possible to
  specify custom models                            provided they are stationary.
  \end{itemize}                          The disadvantages of Gaussian Processes for Machine
  Learning include:                          \begin{itemize}                            \item It is
  not sparse.                            It uses the whole samples/features information to perform
  the prediction.                            \item It loses efficiency in high dimensional spaces –
  namely when the                            number of features exceeds a few dozens.
  It might indeed give poor performance and it loses computational efficiency.
  \item Classification is only a post-processing, meaning that one first needs
  to solve a regression problem by providing the complete scalar float precision
  output $y$ of the experiment one is attempting to model.                          \end{itemize}
  \zNormalizationNotPerformed{GaussianProcessClassifier}

  The \xmlNode{GaussianProcessClassifier} node recognizes the following parameters:
    \begin{itemize}
      \item \xmlAttr{name}: \xmlDesc{string, required}, 
        User-defined name to designate this entity in the RAVEN input file.
      \item \xmlAttr{verbosity}: \xmlDesc{[silent, quiet, all, debug], optional}, 
        Desired verbosity of messages coming from this entity
      \item \xmlAttr{subType}: \xmlDesc{string, required}, 
        specify the type of ROM that will be used
  \end{itemize}

  The \xmlNode{GaussianProcessClassifier} node recognizes the following subnodes:
  \begin{itemize}
    \item \xmlNode{Features}: \xmlDesc{comma-separated strings}, 
      specifies the names of the features of this ROM.         \nb These parameters are going to be
      requested for the training of this object         (see Section~\ref{subsec:stepRomTrainer})

    \item \xmlNode{Target}: \xmlDesc{comma-separated strings}, 
      contains a comma separated list of the targets of this ROM. These parameters         are the
      Figures of Merit (FOMs) this ROM is supposed to predict.         \nb These parameters are
      going to be requested for the training of this         object (see Section
      \ref{subsec:stepRomTrainer}).

    \item \xmlNode{pivotParameter}: \xmlDesc{string}, 
      If a time-dependent ROM is requested, please specifies the pivot         variable (e.g. time,
      etc) used in the input HistorySet.
  \default{time}

    \item \xmlNode{CV}: \xmlDesc{string}, 
      The text portion of this node needs to contain the name of the \xmlNode{PostProcessor} with
      \xmlAttr{subType}         ``CrossValidation``.
      The \xmlNode{CV} node recognizes the following parameters:
        \begin{itemize}
          \item \xmlAttr{class}: \xmlDesc{string, optional}, 
            should be set to \xmlString{Model}
          \item \xmlAttr{type}: \xmlDesc{string, optional}, 
            should be set to \xmlString{PostProcessor}
      \end{itemize}

    \item \xmlNode{alias}: \xmlDesc{string}, 
      specifies alias for         any variable of interest in the input or output space. These
      aliases can be used anywhere in the RAVEN input to         refer to the variables. In the body
      of this node the user specifies the name of the variable that the model is going to use
      (during its execution).
      The \xmlNode{alias} node recognizes the following parameters:
        \begin{itemize}
          \item \xmlAttr{variable}: \xmlDesc{string, required}, 
            define the actual alias, usable throughout the RAVEN input
          \item \xmlAttr{type}: \xmlDesc{[input, output], required}, 
            either ``input'' or ``output''.
      \end{itemize}

    \item \xmlNode{kernel}: \xmlDesc{[Constant, DotProduct, ExpSineSquared, Exponentiation, Matern, Pairwise, RBF, RationalQuadratic]}, 
      The kernel specifying the covariance function of the GP. If None is passed,
      the kernel $RBF$ is used as default. The kernel hyperparameters are optimized during fitting
      and consequentially the hyperparameters are
      not inputable. The following kernels are avaialable:
      \begin{itemize}                                                    \item Constant, Constant
      kernel: $k(x\_1, x\_2) = constant\_value \;\forall\; x\_1, x\_2$.
      \item DotProduct, it is non-stationary and can be obtained from linear regression by putting
      $N(0, 1)$ priors on the coefficients of $x\_d (d = 1, . . . , D)$
      and a prior of $N(0, \sigma\_0^2)$ on the bias. The DotProduct kernel is invariant to a
      rotation of the coordinates about the origin, but not translations.
      It is parameterized by a parameter sigma\_0 $\sigma$ which controls the inhomogenity of the
      kernel.                                                    \item ExpSineSquared, it allows one
      to model functions which repeat themselves exactly. It is parameterized by a length scale
      parameter $l>0$ and a periodicity parameter $p>0$.
      The kernel is given by $k(x\_i, x\_j) = \text{exp}\left(-\frac{ 2\sin^2(\pi d(x\_i, x\_j)/p) }{ l^
      2} \right)$ where $d(\\cdot,\\cdot)$ is the Euclidean distance.
      \item Exponentiation, it takes one base kernel and a scalar parameter $p$ and combines them
      via $k\_{exp}(X, Y) = k(X, Y) ^p$.                                                    \item
      Matern, is a generalization of the RBF. It has an additional parameter $\nu$ which controls
      the smoothness of the resulting function. The smaller $\nu$,
      the less smooth the approximated function is. As $\nu\rightarrow\infty$, the kernel becomes
      equivalent to the RBF kernel. When $\nu = 1/2$, the Matérn kernel becomes
      identical to the absolute exponential kernel. Important intermediate values are $\nu = 1.5$
      (once differentiable functions) and $\nu = 2.5$ (twice differentiable functions).
      The kernel is given by $k(x\_i, x\_j) =  \frac{1}{\Gamma(\nu)2^{\nu-1}}\Bigg(
      \frac{\sqrt{2\nu}}{l} d(x\_i , x\_j ) \Bigg)^\nu K\_\nu\Bigg( \frac{\sqrt{2\nu}}{l} d(x\_i , x\_j
      )\Bigg)$                                                                  where
      $d(\cdot,\cdot)$ is the Euclidean distance, $K\_{\nu}(\cdot)$ is a modified Bessel function and
      $\Gamma(\cdot)$ is the gamma function.
      \item PairwiseLinear, it is a thin wrapper around the functionality of the pairwise kernels.
      It uses the a linear metric to calculate kernel between instances
      in a feature array. Evaluation of the gradient is not analytic but numeric and all kernels
      support only isotropic distances.                                                    \item
      PairwiseAdditiveChi2, it is a thin wrapper around the functionality of the pairwise metrics.
      It uses the an additive chi squared metric to calculate kernel between instances
      in a feature array. Evaluation of the gradient is not analytic but numeric and all kernels
      support only isotropic distances.                                                    \item
      PairwiseChi2, it is a thin wrapper around the functionality of the pairwise metrics. It uses
      the a chi squared metric to calculate kernel between instances
      in a feature array. Evaluation of the gradient is not analytic but numeric and all kernels
      support only isotropic distances.                                                    \item
      PairwisePoly, it is a thin wrapper around the functionality of the pairwise metrics. It uses
      the a poly metric to calculate kernel between instances
      in a feature array. Evaluation of the gradient is not analytic but numeric and all kernels
      support only isotropic distances.                                                    \item
      PairwisePolynomial, it is a thin wrapper around the functionality of the pairwise metrics. It
      uses the a polynomial metric to calculate kernel between instances
      in a feature array. Evaluation of the gradient is not analytic but numeric and all kernels
      support only isotropic distances.                                                    \item
      PairwiseRbf, it is a thin wrapper around the functionality of the pairwise metrics. It uses
      the a rbf metric to calculate kernel between instances
      in a feature array. Evaluation of the gradient is not analytic but numeric and all kernels
      support only isotropic distances.                                                    \item
      PairwiseLaplacian, it is a thin wrapper around the functionality of the pairwise metrics. It
      uses the a laplacian metric to calculate kernel between instances
      in a feature array. Evaluation of the gradient is not analytic but numeric and all kernels
      support only isotropic distances.                                                    \item
      PairwiseSigmoid, it is a thin wrapper around the functionality of the pairwise metrics. It
      uses the a sigmoid metric to calculate kernel between instances
      in a feature array. Evaluation of the gradient is not analytic but numeric and all kernels
      support only isotropic distances.                                                    \item
      PairwiseCosine, it is a thin wrapper around the functionality of the pairwise metrics. It uses
      the a cosine metric to calculate kernel between instances
      in a feature array. Evaluation of the gradient is not analytic but numeric and all kernels
      support only isotropic distances.                                                    \item
      RBF, it is a stationary kernel. It is also known as the ``squared exponential'' kernel. It is
      parameterized by a length scale parameter $l>0$,
      which can either be a scalar (isotropic variant of the kernel) or a vector with the same
      number of dimensions as the inputs $X$ (anisotropic variant of the kernel).
      The kernel is given by $k(x\_i, x\_j) = \exp\left(- \frac{d(x\_i, x\_j)^2}{2l^2} \right)$ where
      $l$ is the length scale of the kernel and $d(\cdot,\cdot)$ is the Euclidean distance.
      \item RationalQuadratic, it can be seen as a scale mixture (an infinite sum) of RBF kernels
      with different characteristic length scales. It is parameterized by a length scale parameter
      $l>0$ and a scale mixture parameter $\alpha>0$ . The kernel is given by $k(x\_i, x\_j) = \left(1
      + \frac{d(x\_i, x\_j)^2 }{ 2\alpha  l^2}\right)^{-\alpha}$ where
      $d(\cdot,\cdot)$ is the Euclidean distance.
      \end{itemize}
  \default{RBF}

    \item \xmlNode{n\_restarts\_optimizer}: \xmlDesc{integer}, 
      The number of restarts of the optimizer for finding the kernel's parameters which maximize the
      log-marginal likelihood. The first run of the optimizer is performed
      from the kernel's initial parameters, the remaining ones (if any) from thetas sampled log-
      uniform randomly from the space of allowed theta-values. If greater than
      0, all bounds must be finite.
  \default{0}

    \item \xmlNode{max\_iter\_predict}: \xmlDesc{integer}, 
      The maximum number of iterations in Newton’s method for approximating the posterior during
      predict. Smaller values will reduce computation time at
      the cost of worse results.
  \default{100}

    \item \xmlNode{multi\_class}: \xmlDesc{[one\_vs\_rest, one\_vs\_one]}, 
      Specifies how multi-class classification problems are handled. Supported are ``one\_vs\_rest''
      and ``one\_vs\_one''.                                                         In
      ``one\_vs\_rest', one binary Gaussian process classifier is fitted for each class, which is
      trained to separate this class                                                         from
      the rest. In ``one\_vs\_one'', one binary Gaussian process classifier is fitted for each pair
      of classes, which is trained to
      separate these two classes. The predictions of these binary predictors are combined into
      multi-class predictions.
  \default{one\_vs\_rest}

    \item \xmlNode{random\_state}: \xmlDesc{integer}, 
      Seed for the internal random number generator
  \default{None}

    \item \xmlNode{optimizer}: \xmlDesc{[fmin\_l\_bfgs\_b]}, 
      Per default, the 'L-BGFS-B' algorithm from
      scipy.optimize.minimize is used. If None is passed, the kernel’s
      parameters are kept fixed.
  \default{L-BGFS-B}
  \end{itemize}


\subsubsection{GaussianProcessRegressor}
  The \xmlNode{GaussianProcessRegressor} is based on Algorithm 2.1 of Gaussian Processes
  for Machine Learning (GPML) by Rasmussen and Williams. The method is a generic supervised learning
  method primarily designed to solve regression problems.                          The advantages of
  Gaussian Processes for Machine Learning are:                          \begin{itemize}
  \item The prediction interpolates the observations (at least for regular
  correlation models).                            \item The prediction is probabilistic (Gaussian)
  so that one can compute                            empirical confidence intervals and exceedance
  probabilities that might be used                            to refit (online fitting, adaptive
  fitting) the prediction in some region of                            interest.
  \item Versatile: different linear regression models and correlation models can
  be specified.                            Common models are provided, but it is also possible to
  specify custom models                            provided they are stationary.
  \end{itemize}                          The disadvantages of Gaussian Processes for Machine
  Learning include:                          \begin{itemize}                            \item It is
  not sparse.                            It uses the whole samples/features information to perform
  the prediction.                            \item It loses efficiency in high dimensional spaces –
  namely when the                            number of features exceeds a few dozens.
  It might indeed give poor performance and it loses computational efficiency.
  \item Classification is only a post-processing, meaning that one first needs
  to solve a regression problem by providing the complete scalar float precision
  output $y$ of the experiment one is attempting to model.                          \end{itemize}
  \zNormalizationNotPerformed{GaussianProcessRegressor}

  The \xmlNode{GaussianProcessRegressor} node recognizes the following parameters:
    \begin{itemize}
      \item \xmlAttr{name}: \xmlDesc{string, required}, 
        User-defined name to designate this entity in the RAVEN input file.
      \item \xmlAttr{verbosity}: \xmlDesc{[silent, quiet, all, debug], optional}, 
        Desired verbosity of messages coming from this entity
      \item \xmlAttr{subType}: \xmlDesc{string, required}, 
        specify the type of ROM that will be used
  \end{itemize}

  The \xmlNode{GaussianProcessRegressor} node recognizes the following subnodes:
  \begin{itemize}
    \item \xmlNode{Features}: \xmlDesc{comma-separated strings}, 
      specifies the names of the features of this ROM.         \nb These parameters are going to be
      requested for the training of this object         (see Section~\ref{subsec:stepRomTrainer})

    \item \xmlNode{Target}: \xmlDesc{comma-separated strings}, 
      contains a comma separated list of the targets of this ROM. These parameters         are the
      Figures of Merit (FOMs) this ROM is supposed to predict.         \nb These parameters are
      going to be requested for the training of this         object (see Section
      \ref{subsec:stepRomTrainer}).

    \item \xmlNode{pivotParameter}: \xmlDesc{string}, 
      If a time-dependent ROM is requested, please specifies the pivot         variable (e.g. time,
      etc) used in the input HistorySet.
  \default{time}

    \item \xmlNode{CV}: \xmlDesc{string}, 
      The text portion of this node needs to contain the name of the \xmlNode{PostProcessor} with
      \xmlAttr{subType}         ``CrossValidation``.
      The \xmlNode{CV} node recognizes the following parameters:
        \begin{itemize}
          \item \xmlAttr{class}: \xmlDesc{string, optional}, 
            should be set to \xmlString{Model}
          \item \xmlAttr{type}: \xmlDesc{string, optional}, 
            should be set to \xmlString{PostProcessor}
      \end{itemize}

    \item \xmlNode{alias}: \xmlDesc{string}, 
      specifies alias for         any variable of interest in the input or output space. These
      aliases can be used anywhere in the RAVEN input to         refer to the variables. In the body
      of this node the user specifies the name of the variable that the model is going to use
      (during its execution).
      The \xmlNode{alias} node recognizes the following parameters:
        \begin{itemize}
          \item \xmlAttr{variable}: \xmlDesc{string, required}, 
            define the actual alias, usable throughout the RAVEN input
          \item \xmlAttr{type}: \xmlDesc{[input, output], required}, 
            either ``input'' or ``output''.
      \end{itemize}

    \item \xmlNode{kernel}: \xmlDesc{[Constant, DotProduct, ExpSineSquared, Exponentiation, Matern, PairwiseLinear, PairwiseAdditiveChi2, PairwiseChi2, PairwisePoly, PairwisePolynomial, PairwiseRBF, PairwiseLaplassian, PairwiseSigmoid, PairwiseCosine, RBF, RationalQuadratic]}, 
      The kernel specifying the covariance function of the GP. If None is passed,
      the kernel $Constant$ is used as default. The kernel hyperparameters are optimized during
      fitting and consequentially the hyperparameters are
      not inputable. The following kernels are avaialable:
      \begin{itemize}                                                    \item Constant, Constant
      kernel: $k(x\_1, x\_2) = constant\_value \;\forall\; x\_1, x\_2$.
      \item DotProduct, it is non-stationary and can be obtained from linear regression by putting
      $N(0, 1)$ priors on the coefficients of $x\_d (d = 1, . . . , D)$
      and a prior of $N(0, \sigma\_0^2)$ on the bias. The DotProduct kernel is invariant to a
      rotation of the coordinates about the origin, but not translations.
      It is parameterized by a parameter sigma\_0 $\sigma$ which controls the inhomogenity of the
      kernel.                                                    \item ExpSineSquared, it allows one
      to model functions which repeat themselves exactly. It is parameterized by a length scale
      parameter $l>0$ and a periodicity parameter $p>0$.
      The kernel is given by $k(x\_i, x\_j) = \text{exp}\left(-\frac{ 2\sin^2(\pi d(x\_i, x\_j)/p) }{ l^
      2} \right)$ where $d(\\cdot,\\cdot)$ is the Euclidean distance.
      \item Exponentiation, it takes one base kernel and a scalar parameter $p$ and combines them
      via $k\_{exp}(X, Y) = k(X, Y) ^p$.                                                    \item
      Matern, is a generalization of the RBF. It has an additional parameter $\nu$ which controls
      the smoothness of the resulting function. The smaller $\nu$,
      the less smooth the approximated function is. As $\nu\rightarrow\infty$, the kernel becomes
      equivalent to the RBF kernel. When $\nu = 1/2$, the Matérn kernel becomes
      identical to the absolute exponential kernel. Important intermediate values are $\nu = 1.5$
      (once differentiable functions) and $\nu = 2.5$ (twice differentiable functions).
      The kernel is given by $k(x\_i, x\_j) =  \frac{1}{\Gamma(\nu)2^{\nu-1}}\Bigg(
      \frac{\sqrt{2\nu}}{l} d(x\_i , x\_j ) \Bigg)^\nu K\_\nu\Bigg( \frac{\sqrt{2\nu}}{l} d(x\_i , x\_j
      )\Bigg)$                                                                  where
      $d(\cdot,\cdot)$ is the Euclidean distance, $K\_{\nu}(\cdot)$ is a modified Bessel function and
      $\Gamma(\cdot)$ is the gamma function.
      \item PairwiseLinear, it is a thin wrapper around the functionality of the pairwise kernels.
      It uses the a linear metric to calculate kernel between instances
      in a feature array. Evaluation of the gradient is not analytic but numeric and all kernels
      support only isotropic distances.                                                    \item
      PairwiseAdditiveChi2, it is a thin wrapper around the functionality of the pairwise metrics.
      It uses the an additive chi squared metric to calculate kernel between instances
      in a feature array. Evaluation of the gradient is not analytic but numeric and all kernels
      support only isotropic distances.                                                    \item
      PairwiseChi2, it is a thin wrapper around the functionality of the pairwise metrics. It uses
      the a chi squared metric to calculate kernel between instances
      in a feature array. Evaluation of the gradient is not analytic but numeric and all kernels
      support only isotropic distances.                                                    \item
      PairwisePoly, it is a thin wrapper around the functionality of the pairwise metrics. It uses
      the a poly metric to calculate kernel between instances
      in a feature array. Evaluation of the gradient is not analytic but numeric and all kernels
      support only isotropic distances.                                                    \item
      PairwisePolynomial, it is a thin wrapper around the functionality of the pairwise metrics. It
      uses the a polynomial metric to calculate kernel between instances
      in a feature array. Evaluation of the gradient is not analytic but numeric and all kernels
      support only isotropic distances.                                                    \item
      PairwiseRbf, it is a thin wrapper around the functionality of the pairwise metrics. It uses
      the a rbf metric to calculate kernel between instances
      in a feature array. Evaluation of the gradient is not analytic but numeric and all kernels
      support only isotropic distances.                                                    \item
      PairwiseLaplacian, it is a thin wrapper around the functionality of the pairwise metrics. It
      uses the a laplacian metric to calculate kernel between instances
      in a feature array. Evaluation of the gradient is not analytic but numeric and all kernels
      support only isotropic distances.                                                    \item
      PairwiseSigmoid, it is a thin wrapper around the functionality of the pairwise metrics. It
      uses the a sigmoid metric to calculate kernel between instances
      in a feature array. Evaluation of the gradient is not analytic but numeric and all kernels
      support only isotropic distances.                                                    \item
      PairwiseCosine, it is a thin wrapper around the functionality of the pairwise metrics. It uses
      the a cosine metric to calculate kernel between instances
      in a feature array. Evaluation of the gradient is not analytic but numeric and all kernels
      support only isotropic distances.                                                    \item
      RBF, it is a stationary kernel. It is also known as the ``squared exponential'' kernel. It is
      parameterized by a length scale parameter $l>0$,
      which can either be a scalar (isotropic variant of the kernel) or a vector with the same
      number of dimensions as the inputs $X$ (anisotropic variant of the kernel).
      The kernel is given by $k(x\_i, x\_j) = \exp\left(- \frac{d(x\_i, x\_j)^2}{2l^2} \right)$ where
      $l$ is the length scale of the kernel and $d(\cdot,\cdot)$ is the Euclidean distance.
      \item RationalQuadratic, it can be seen as a scale mixture (an infinite sum) of RBF kernels
      with different characteristic length scales. It is parameterized by a length scale parameter
      $l>0$ and a scale mixture parameter $\alpha>0$ . The kernel is given by $k(x\_i, x\_j) = \left(1
      + \frac{d(x\_i, x\_j)^2 }{ 2\alpha  l^2}\right)^{-\alpha}$ where
      $d(\cdot,\cdot)$ is the Euclidean distance.
      \end{itemize}.
  \default{None}

    \item \xmlNode{alpha}: \xmlDesc{float}, 
      Value added to the diagonal of the kernel matrix during fitting. This can prevent a potential
      numerical issue during fitting, by ensuring that the calculated
      values form a positive definite matrix. It can also be interpreted as the variance of
      additional Gaussian measurement noise on the training observations.
  \default{1e-10}

    \item \xmlNode{n\_restarts\_optimizer}: \xmlDesc{integer}, 
      The number of restarts of the optimizer for finding the kernel's parameters which maximize the
      log-marginal likelihood. The first run of the optimizer is performed
      from the kernel's initial parameters, the remaining ones (if any) from thetas sampled log-
      uniform randomly from the space of allowed theta-values. If greater than
      0, all bounds must be finite.
  \default{0}

    \item \xmlNode{normalize\_y}: \xmlDesc{[True, Yes, 1, False, No, 0, t, y, 1, f, n, 0]}, 
      Whether the target values y are normalized, the mean and variance of the target values are set
      equal to 0 and 1 respectively. This is recommended for cases where zero-mean,
      unit-variance priors are used.
  \default{False}

    \item \xmlNode{random\_state}: \xmlDesc{integer}, 
      Seed for the internal random number generator.
  \default{None}

    \item \xmlNode{optimizer}: \xmlDesc{[fmin\_l\_bfgs\_b]}, 
      Per default, the 'L-BFGS-B' algorithm from
      scipy.optimize.minimize is used. If None is passed, the kernel’s
      parameters are kept fixed.
  \default{fmin\_l\_bfgs\_b}
  \end{itemize}


\subsubsection{OneVsOneClassifier}
  The \xmlNode{OneVsOneClassifier} (\textit{One-vs-one multiclass strategy})
  This strategy consists in fitting one classifier per class pair. At prediction time, the class
  which received the most votes is selected.
  \zNormalizationNotPerformed{OneVsOneClassifier}

  The \xmlNode{OneVsOneClassifier} node recognizes the following parameters:
    \begin{itemize}
      \item \xmlAttr{name}: \xmlDesc{string, required}, 
        User-defined name to designate this entity in the RAVEN input file.
      \item \xmlAttr{verbosity}: \xmlDesc{[silent, quiet, all, debug], optional}, 
        Desired verbosity of messages coming from this entity
      \item \xmlAttr{subType}: \xmlDesc{string, required}, 
        specify the type of ROM that will be used
  \end{itemize}

  The \xmlNode{OneVsOneClassifier} node recognizes the following subnodes:
  \begin{itemize}
    \item \xmlNode{Features}: \xmlDesc{comma-separated strings}, 
      specifies the names of the features of this ROM.         \nb These parameters are going to be
      requested for the training of this object         (see Section~\ref{subsec:stepRomTrainer})

    \item \xmlNode{Target}: \xmlDesc{comma-separated strings}, 
      contains a comma separated list of the targets of this ROM. These parameters         are the
      Figures of Merit (FOMs) this ROM is supposed to predict.         \nb These parameters are
      going to be requested for the training of this         object (see Section
      \ref{subsec:stepRomTrainer}).

    \item \xmlNode{pivotParameter}: \xmlDesc{string}, 
      If a time-dependent ROM is requested, please specifies the pivot         variable (e.g. time,
      etc) used in the input HistorySet.
  \default{time}

    \item \xmlNode{CV}: \xmlDesc{string}, 
      The text portion of this node needs to contain the name of the \xmlNode{PostProcessor} with
      \xmlAttr{subType}         ``CrossValidation``.
      The \xmlNode{CV} node recognizes the following parameters:
        \begin{itemize}
          \item \xmlAttr{class}: \xmlDesc{string, optional}, 
            should be set to \xmlString{Model}
          \item \xmlAttr{type}: \xmlDesc{string, optional}, 
            should be set to \xmlString{PostProcessor}
      \end{itemize}

    \item \xmlNode{alias}: \xmlDesc{string}, 
      specifies alias for         any variable of interest in the input or output space. These
      aliases can be used anywhere in the RAVEN input to         refer to the variables. In the body
      of this node the user specifies the name of the variable that the model is going to use
      (during its execution).
      The \xmlNode{alias} node recognizes the following parameters:
        \begin{itemize}
          \item \xmlAttr{variable}: \xmlDesc{string, required}, 
            define the actual alias, usable throughout the RAVEN input
          \item \xmlAttr{type}: \xmlDesc{[input, output], required}, 
            either ``input'' or ``output''.
      \end{itemize}

    \item \xmlNode{estimator}: \xmlDesc{string}, 
      name of a ROM that can be used as an estimator
      The \xmlNode{estimator} node recognizes the following parameters:
        \begin{itemize}
          \item \xmlAttr{class}: \xmlDesc{string, required}, 
            RAVEN class for this entity (e.g. Samplers, Models, DataObjects)
          \item \xmlAttr{type}: \xmlDesc{string, required}, 
            RAVEN type for this entity; a subtype of the class (e.g. MonteCarlo, Code, PointSet)
      \end{itemize}

    \item \xmlNode{n\_jobs}: \xmlDesc{integer}, 
      The number of jobs to use for the computation: the n\_classes * ( n\_classes - 1) / 2 OVO
      problems are computed in parallel. None means 1 unless in a joblib.parallel\_backend
      context. -1 means using all processors.
  \default{None}
  \end{itemize}


\subsubsection{OneVsRestClassifier}
  The \xmlNode{OneVsRestClassifier} (\textit{One-vs-the-rest (OvR) multiclass strategy})
  Also known as one-vs-all, this strategy consists in fitting one classifier per class. For each
  classifier, the class is fitted against all the other classes. In addition to its computational
  efficiency (only n\_classes classifiers are needed), one advantage of this approach is its
  interpretability. Since each class is represented by one and one classifier only, it is
  possible to gain knowledge about the class by inspecting its corresponding classifier.
  This is the most commonly used strategy for multiclass classification and is a fair default
  choice.                         \zNormalizationNotPerformed{OneVsRestClassifier}

  The \xmlNode{OneVsRestClassifier} node recognizes the following parameters:
    \begin{itemize}
      \item \xmlAttr{name}: \xmlDesc{string, required}, 
        User-defined name to designate this entity in the RAVEN input file.
      \item \xmlAttr{verbosity}: \xmlDesc{[silent, quiet, all, debug], optional}, 
        Desired verbosity of messages coming from this entity
      \item \xmlAttr{subType}: \xmlDesc{string, required}, 
        specify the type of ROM that will be used
  \end{itemize}

  The \xmlNode{OneVsRestClassifier} node recognizes the following subnodes:
  \begin{itemize}
    \item \xmlNode{Features}: \xmlDesc{comma-separated strings}, 
      specifies the names of the features of this ROM.         \nb These parameters are going to be
      requested for the training of this object         (see Section~\ref{subsec:stepRomTrainer})

    \item \xmlNode{Target}: \xmlDesc{comma-separated strings}, 
      contains a comma separated list of the targets of this ROM. These parameters         are the
      Figures of Merit (FOMs) this ROM is supposed to predict.         \nb These parameters are
      going to be requested for the training of this         object (see Section
      \ref{subsec:stepRomTrainer}).

    \item \xmlNode{pivotParameter}: \xmlDesc{string}, 
      If a time-dependent ROM is requested, please specifies the pivot         variable (e.g. time,
      etc) used in the input HistorySet.
  \default{time}

    \item \xmlNode{CV}: \xmlDesc{string}, 
      The text portion of this node needs to contain the name of the \xmlNode{PostProcessor} with
      \xmlAttr{subType}         ``CrossValidation``.
      The \xmlNode{CV} node recognizes the following parameters:
        \begin{itemize}
          \item \xmlAttr{class}: \xmlDesc{string, optional}, 
            should be set to \xmlString{Model}
          \item \xmlAttr{type}: \xmlDesc{string, optional}, 
            should be set to \xmlString{PostProcessor}
      \end{itemize}

    \item \xmlNode{alias}: \xmlDesc{string}, 
      specifies alias for         any variable of interest in the input or output space. These
      aliases can be used anywhere in the RAVEN input to         refer to the variables. In the body
      of this node the user specifies the name of the variable that the model is going to use
      (during its execution).
      The \xmlNode{alias} node recognizes the following parameters:
        \begin{itemize}
          \item \xmlAttr{variable}: \xmlDesc{string, required}, 
            define the actual alias, usable throughout the RAVEN input
          \item \xmlAttr{type}: \xmlDesc{[input, output], required}, 
            either ``input'' or ``output''.
      \end{itemize}

    \item \xmlNode{estimator}: \xmlDesc{string}, 
      name of a ROM that can be used as an estimator
      The \xmlNode{estimator} node recognizes the following parameters:
        \begin{itemize}
          \item \xmlAttr{class}: \xmlDesc{string, required}, 
            RAVEN class for this entity (e.g. Samplers, Models, DataObjects)
          \item \xmlAttr{type}: \xmlDesc{string, required}, 
            RAVEN type for this entity; a subtype of the class (e.g. MonteCarlo, Code, PointSet)
      \end{itemize}

    \item \xmlNode{n\_jobs}: \xmlDesc{integer}, 
      TThe number of jobs to use for the computation: the n\_classes one-vs-rest
      problems are computed in parallel. None means 1 unless in a joblib.parallel\_backend
      context. -1 means using all processors.
  \default{None}
  \end{itemize}


\subsubsection{OutputCodeClassifier}
  The \xmlNode{OutputCodeClassifier} (\textit{(Error-Correcting) Output-Code multiclass strategy})
  Output-code based strategies consist in representing each class with a binary code (an array of
  0s and 1s). At fitting time, one binary classifier per bit in the code book is fitted. At
  prediction time, the classifiers are used to project new points in the class space and the class
  closest to the points is chosen. The main advantage of these strategies is that the number of
  classifiers used can be controlled by the user, either for compressing the model
  (0 < code\_size < 1) or for making the model more robust to errors (code\_size > 1). See the
  documentation for more details.
  \zNormalizationNotPerformed{OutputCodeClassifier}

  The \xmlNode{OutputCodeClassifier} node recognizes the following parameters:
    \begin{itemize}
      \item \xmlAttr{name}: \xmlDesc{string, required}, 
        User-defined name to designate this entity in the RAVEN input file.
      \item \xmlAttr{verbosity}: \xmlDesc{[silent, quiet, all, debug], optional}, 
        Desired verbosity of messages coming from this entity
      \item \xmlAttr{subType}: \xmlDesc{string, required}, 
        specify the type of ROM that will be used
  \end{itemize}

  The \xmlNode{OutputCodeClassifier} node recognizes the following subnodes:
  \begin{itemize}
    \item \xmlNode{Features}: \xmlDesc{comma-separated strings}, 
      specifies the names of the features of this ROM.         \nb These parameters are going to be
      requested for the training of this object         (see Section~\ref{subsec:stepRomTrainer})

    \item \xmlNode{Target}: \xmlDesc{comma-separated strings}, 
      contains a comma separated list of the targets of this ROM. These parameters         are the
      Figures of Merit (FOMs) this ROM is supposed to predict.         \nb These parameters are
      going to be requested for the training of this         object (see Section
      \ref{subsec:stepRomTrainer}).

    \item \xmlNode{pivotParameter}: \xmlDesc{string}, 
      If a time-dependent ROM is requested, please specifies the pivot         variable (e.g. time,
      etc) used in the input HistorySet.
  \default{time}

    \item \xmlNode{CV}: \xmlDesc{string}, 
      The text portion of this node needs to contain the name of the \xmlNode{PostProcessor} with
      \xmlAttr{subType}         ``CrossValidation``.
      The \xmlNode{CV} node recognizes the following parameters:
        \begin{itemize}
          \item \xmlAttr{class}: \xmlDesc{string, optional}, 
            should be set to \xmlString{Model}
          \item \xmlAttr{type}: \xmlDesc{string, optional}, 
            should be set to \xmlString{PostProcessor}
      \end{itemize}

    \item \xmlNode{alias}: \xmlDesc{string}, 
      specifies alias for         any variable of interest in the input or output space. These
      aliases can be used anywhere in the RAVEN input to         refer to the variables. In the body
      of this node the user specifies the name of the variable that the model is going to use
      (during its execution).
      The \xmlNode{alias} node recognizes the following parameters:
        \begin{itemize}
          \item \xmlAttr{variable}: \xmlDesc{string, required}, 
            define the actual alias, usable throughout the RAVEN input
          \item \xmlAttr{type}: \xmlDesc{[input, output], required}, 
            either ``input'' or ``output''.
      \end{itemize}

    \item \xmlNode{estimator}: \xmlDesc{string}, 
      name of a ROM that can be used as an estimator
      The \xmlNode{estimator} node recognizes the following parameters:
        \begin{itemize}
          \item \xmlAttr{class}: \xmlDesc{string, required}, 
            RAVEN class for this entity (e.g. Samplers, Models, DataObjects)
          \item \xmlAttr{type}: \xmlDesc{string, required}, 
            RAVEN type for this entity; a subtype of the class (e.g. MonteCarlo, Code, PointSet)
      \end{itemize}

    \item \xmlNode{code\_size}: \xmlDesc{float}, 
      Percentage of the number of classes to be used to create
      the code book. A number between 0 and 1 will require fewer classifiers
      than one-vs-the-rest. A number greater than 1 will require more classifiers
      than one-vs-the-rest.
  \default{1.5}

    \item \xmlNode{random\_state}: \xmlDesc{integer}, 
      The generator used to initialize the codebook. Pass an int
      for reproducible output across multiple function calls.
  \default{None}

    \item \xmlNode{n\_jobs}: \xmlDesc{integer}, 
      TThe number of jobs to use for the computation: the n\_classes one-vs-rest
      problems are computed in parallel. None means 1 unless in a joblib.parallel\_backend
      context. -1 means using all processors. See Glossary for more details.
  \default{None}
  \end{itemize}


\subsubsection{KNeighborsClassifier}
  The \xmlNode{KNeighborsClassifier} is a type of instance-based learning or
  non-generalizing learning: it does not attempt to construct a general internal
  model, but simply stores instances of the training data.                          Classification
  is computed from a simple majority vote of the nearest neighbors                          of each
  point: a query point is assigned the data class which has the most
  representatives within the nearest neighbors of the point.                          It implements
  learning based on the $k$ nearest neighbors of each query point,                          where
  $k$ is an integer value specified by the user.
  \zNormalizationPerformed{KNeighborsClassifier}

  The \xmlNode{KNeighborsClassifier} node recognizes the following parameters:
    \begin{itemize}
      \item \xmlAttr{name}: \xmlDesc{string, required}, 
        User-defined name to designate this entity in the RAVEN input file.
      \item \xmlAttr{verbosity}: \xmlDesc{[silent, quiet, all, debug], optional}, 
        Desired verbosity of messages coming from this entity
      \item \xmlAttr{subType}: \xmlDesc{string, required}, 
        specify the type of ROM that will be used
  \end{itemize}

  The \xmlNode{KNeighborsClassifier} node recognizes the following subnodes:
  \begin{itemize}
    \item \xmlNode{Features}: \xmlDesc{comma-separated strings}, 
      specifies the names of the features of this ROM.         \nb These parameters are going to be
      requested for the training of this object         (see Section~\ref{subsec:stepRomTrainer})

    \item \xmlNode{Target}: \xmlDesc{comma-separated strings}, 
      contains a comma separated list of the targets of this ROM. These parameters         are the
      Figures of Merit (FOMs) this ROM is supposed to predict.         \nb These parameters are
      going to be requested for the training of this         object (see Section
      \ref{subsec:stepRomTrainer}).

    \item \xmlNode{pivotParameter}: \xmlDesc{string}, 
      If a time-dependent ROM is requested, please specifies the pivot         variable (e.g. time,
      etc) used in the input HistorySet.
  \default{time}

    \item \xmlNode{CV}: \xmlDesc{string}, 
      The text portion of this node needs to contain the name of the \xmlNode{PostProcessor} with
      \xmlAttr{subType}         ``CrossValidation``.
      The \xmlNode{CV} node recognizes the following parameters:
        \begin{itemize}
          \item \xmlAttr{class}: \xmlDesc{string, optional}, 
            should be set to \xmlString{Model}
          \item \xmlAttr{type}: \xmlDesc{string, optional}, 
            should be set to \xmlString{PostProcessor}
      \end{itemize}

    \item \xmlNode{alias}: \xmlDesc{string}, 
      specifies alias for         any variable of interest in the input or output space. These
      aliases can be used anywhere in the RAVEN input to         refer to the variables. In the body
      of this node the user specifies the name of the variable that the model is going to use
      (during its execution).
      The \xmlNode{alias} node recognizes the following parameters:
        \begin{itemize}
          \item \xmlAttr{variable}: \xmlDesc{string, required}, 
            define the actual alias, usable throughout the RAVEN input
          \item \xmlAttr{type}: \xmlDesc{[input, output], required}, 
            either ``input'' or ``output''.
      \end{itemize}

    \item \xmlNode{n\_neighbors}: \xmlDesc{integer}, 
      Number of neighbors to use by default for kneighbors queries.
  \default{5}

    \item \xmlNode{weights}: \xmlDesc{[uniform, distance]}, 
      weight function used in prediction. If ``uniform'', all points in each neighborhood
      are weighted equally. If ``distance'' weight points by the inverse of their distance. in this
      case,closer neighbors of a query point will have a greater influence than neighbors which are
      further away.
  \default{uniform}

    \item \xmlNode{algorithm}: \xmlDesc{[auto, ball\_tree, kd\_tree, brute]}, 
      Algorithm used to compute the nearest neighbors
  \default{auto}

    \item \xmlNode{leaf\_size}: \xmlDesc{integer}, 
      Leaf size passed to BallTree or KDTree. This can affect the speed of the construction
      and query, as well as the memory required to store the tree. The optimal value depends on the
      nature of the problem.
  \default{30}

    \item \xmlNode{p}: \xmlDesc{integer}, 
      Power parameter for the Minkowski metric. When $p = 1$, this is equivalent to using
      manhattan\_distance (l1), and euclidean\_distance (l2) for $p = 2$. For arbitrary $p$,
      minkowski\_distance                                                  (l\_p) is used.
  \default{2}

    \item \xmlNode{metric}: \xmlDesc{[euclidean, manhattan, minkowski, chebyshev, hamming, braycurtis]}, 
      the distance metric to use for the tree. The default metric is minkowski, and with
      $p=2$ is equivalent to the standard Euclidean metric.
      The available metrics are:                                                  \begin{itemize}
      \item minkowski: $sum(|x - y|^p)^(1/p)$
      \item euclidean: $sqrt(sum((x - y)^2))$
      \item manhattan: $sum(|x - y|)$                                                    \item
      chebyshev: $max(|x - y|)$                                                    \item hamming:
      $N\_unequal(x, y) / N\_tot$                                                    \item canberra:
      $sum(|x - y| / (|x| + |y|))$                                                    \item
      braycurtis: $sum(|x - y|) / (sum(|x|) + sum(|y|))$
      \end{itemize}
  \default{minkowski}
  \end{itemize}


\subsubsection{NearestCentroid}
  The \xmlNode{RadiusNeighborsClassifier} is a type of instance-based learning or
  non-generalizing learning: it does not attempt to construct a general internal
  model, but simply stores instances of the training data.                          Classification
  is computed from a simple majority vote of the nearest neighbors                          of each
  point: a query point is assigned the data class which has the most
  representatives within the nearest neighbors of the point.                          It implements
  learning based on the number of neighbors within a fixed radius                          $r$ of
  each training point, where $r$ is a floating-point value specified by the
  user.                          \zNormalizationPerformed{RadiusNeighborsClassifier}

  The \xmlNode{NearestCentroid} node recognizes the following parameters:
    \begin{itemize}
      \item \xmlAttr{name}: \xmlDesc{string, required}, 
        User-defined name to designate this entity in the RAVEN input file.
      \item \xmlAttr{verbosity}: \xmlDesc{[silent, quiet, all, debug], optional}, 
        Desired verbosity of messages coming from this entity
      \item \xmlAttr{subType}: \xmlDesc{string, required}, 
        specify the type of ROM that will be used
  \end{itemize}

  The \xmlNode{NearestCentroid} node recognizes the following subnodes:
  \begin{itemize}
    \item \xmlNode{Features}: \xmlDesc{comma-separated strings}, 
      specifies the names of the features of this ROM.         \nb These parameters are going to be
      requested for the training of this object         (see Section~\ref{subsec:stepRomTrainer})

    \item \xmlNode{Target}: \xmlDesc{comma-separated strings}, 
      contains a comma separated list of the targets of this ROM. These parameters         are the
      Figures of Merit (FOMs) this ROM is supposed to predict.         \nb These parameters are
      going to be requested for the training of this         object (see Section
      \ref{subsec:stepRomTrainer}).

    \item \xmlNode{pivotParameter}: \xmlDesc{string}, 
      If a time-dependent ROM is requested, please specifies the pivot         variable (e.g. time,
      etc) used in the input HistorySet.
  \default{time}

    \item \xmlNode{CV}: \xmlDesc{string}, 
      The text portion of this node needs to contain the name of the \xmlNode{PostProcessor} with
      \xmlAttr{subType}         ``CrossValidation``.
      The \xmlNode{CV} node recognizes the following parameters:
        \begin{itemize}
          \item \xmlAttr{class}: \xmlDesc{string, optional}, 
            should be set to \xmlString{Model}
          \item \xmlAttr{type}: \xmlDesc{string, optional}, 
            should be set to \xmlString{PostProcessor}
      \end{itemize}

    \item \xmlNode{alias}: \xmlDesc{string}, 
      specifies alias for         any variable of interest in the input or output space. These
      aliases can be used anywhere in the RAVEN input to         refer to the variables. In the body
      of this node the user specifies the name of the variable that the model is going to use
      (during its execution).
      The \xmlNode{alias} node recognizes the following parameters:
        \begin{itemize}
          \item \xmlAttr{variable}: \xmlDesc{string, required}, 
            define the actual alias, usable throughout the RAVEN input
          \item \xmlAttr{type}: \xmlDesc{[input, output], required}, 
            either ``input'' or ``output''.
      \end{itemize}

    \item \xmlNode{shrink\_threshold}: \xmlDesc{float}, 
      Threshold for shrinking centroids to remove features.
  \default{None}

    \item \xmlNode{metric}: \xmlDesc{[uniform, distance]}, 
      The metric to use when calculating distance between instances in a feature array.
      The available metrics are allo the ones explained in the \xmlNode{Metrics} section (pairwise).
      The centroids for the samples corresponding to each class is the point from which the sum of
      the distances (according to the metric) of all samples that belong to that particular class
      are                                                  minimized. If the ``manhattan'' metric is
      provided, this centroid is the median and for all other metrics,
      the centroid is now set to be the mean.
  \default{minkowski}
  \end{itemize}


\subsubsection{RadiusNeighborsRegressor}
  The \xmlNode{RadiusNeighborsRegressor} is a type of instance-based learning or
  non-generalizing learning: it does not attempt to construct a general internal
  model, but simply stores instances of the training data.                          The target is
  predicted by local interpolation of the targets associated of the                          nearest
  neighbors in the training set.                          It implements learning based on the number
  of neighbors within a fixed radius                          $r$ of each training point, where $r$
  is a floating-point value specified by the                          user.
  \zNormalizationPerformed{RadiusNeighborsRegressor}

  The \xmlNode{RadiusNeighborsRegressor} node recognizes the following parameters:
    \begin{itemize}
      \item \xmlAttr{name}: \xmlDesc{string, required}, 
        User-defined name to designate this entity in the RAVEN input file.
      \item \xmlAttr{verbosity}: \xmlDesc{[silent, quiet, all, debug], optional}, 
        Desired verbosity of messages coming from this entity
      \item \xmlAttr{subType}: \xmlDesc{string, required}, 
        specify the type of ROM that will be used
  \end{itemize}

  The \xmlNode{RadiusNeighborsRegressor} node recognizes the following subnodes:
  \begin{itemize}
    \item \xmlNode{Features}: \xmlDesc{comma-separated strings}, 
      specifies the names of the features of this ROM.         \nb These parameters are going to be
      requested for the training of this object         (see Section~\ref{subsec:stepRomTrainer})

    \item \xmlNode{Target}: \xmlDesc{comma-separated strings}, 
      contains a comma separated list of the targets of this ROM. These parameters         are the
      Figures of Merit (FOMs) this ROM is supposed to predict.         \nb These parameters are
      going to be requested for the training of this         object (see Section
      \ref{subsec:stepRomTrainer}).

    \item \xmlNode{pivotParameter}: \xmlDesc{string}, 
      If a time-dependent ROM is requested, please specifies the pivot         variable (e.g. time,
      etc) used in the input HistorySet.
  \default{time}

    \item \xmlNode{CV}: \xmlDesc{string}, 
      The text portion of this node needs to contain the name of the \xmlNode{PostProcessor} with
      \xmlAttr{subType}         ``CrossValidation``.
      The \xmlNode{CV} node recognizes the following parameters:
        \begin{itemize}
          \item \xmlAttr{class}: \xmlDesc{string, optional}, 
            should be set to \xmlString{Model}
          \item \xmlAttr{type}: \xmlDesc{string, optional}, 
            should be set to \xmlString{PostProcessor}
      \end{itemize}

    \item \xmlNode{alias}: \xmlDesc{string}, 
      specifies alias for         any variable of interest in the input or output space. These
      aliases can be used anywhere in the RAVEN input to         refer to the variables. In the body
      of this node the user specifies the name of the variable that the model is going to use
      (during its execution).
      The \xmlNode{alias} node recognizes the following parameters:
        \begin{itemize}
          \item \xmlAttr{variable}: \xmlDesc{string, required}, 
            define the actual alias, usable throughout the RAVEN input
          \item \xmlAttr{type}: \xmlDesc{[input, output], required}, 
            either ``input'' or ``output''.
      \end{itemize}

    \item \xmlNode{radius}: \xmlDesc{float}, 
      Range of parameter space to use by default for radius neighbors queries.
  \default{1.0}

    \item \xmlNode{weights}: \xmlDesc{[uniform, distance]}, 
      weight function used in prediction. If ``uniform'', all points in each neighborhood
      are weighted equally. If ``distance'' weight points by the inverse of their distance. in this
      case,closer neighbors of a query point will have a greater influence than neighbors which are
      further away.
  \default{uniform}

    \item \xmlNode{algorithm}: \xmlDesc{[auto, ball\_tree, kd\_tree, brute]}, 
      Algorithm used to compute the nearest neighbors
  \default{auto}

    \item \xmlNode{leaf\_size}: \xmlDesc{integer}, 
      Leaf size passed to BallTree or KDTree. This can affect the speed of the construction
      and query, as well as the memory required to store the tree. The optimal value depends on the
      nature of the problem.
  \default{30}

    \item \xmlNode{p}: \xmlDesc{integer}, 
      Power parameter for the Minkowski metric. When $p = 1$, this is equivalent to using
      manhattan\_distance (l1), and euclidean\_distance (l2) for $p = 2$. For arbitrary $p$,
      minkowski\_distance                                                  (l\_p) is used.
  \default{2}

    \item \xmlNode{metric}: \xmlDesc{[minkowski, euclidean, manhattan, chebyshev, hamming, canberra, braycurtis]}, 
      the distance metric to use for the tree. The default metric is minkowski, and with
      $p=2$ is equivalent to the standard Euclidean metric.
      The available metrics are:                                                  \begin{itemize}
      \item minkowski: $sum(|x - y|^p)^(1/p)$
      \item euclidean: $sqrt(sum((x - y)^2))$
      \item manhattan: $sum(|x - y|)$                                                    \item
      chebyshev: $max(|x - y|)$                                                    \item hamming:
      $N\_unequal(x, y) / N\_tot$                                                    \item canberra:
      $sum(|x - y| / (|x| + |y|))$                                                    \item
      braycurtis: $sum(|x - y|) / (sum(|x|) + sum(|y|))$
      \end{itemize}
  \default{minkowski}
  \end{itemize}


\subsubsection{KNeighborsRegressor}
  The \xmlNode{KNeighborsRegressor} is a type of instance-based learning or
  non-generalizing learning: it does not attempt to construct a general internal
  model, but simply stores instances of the training data.                          The target is
  predicted by local interpolation of the targets associated                          of the nearest
  neighbors in the training set.                          It implements learning based on the $k$
  nearest neighbors of each query point,                          where $k$ is an integer value
  specified by the user.                          \zNormalizationPerformed{KNeighborsRegressor}

  The \xmlNode{KNeighborsRegressor} node recognizes the following parameters:
    \begin{itemize}
      \item \xmlAttr{name}: \xmlDesc{string, required}, 
        User-defined name to designate this entity in the RAVEN input file.
      \item \xmlAttr{verbosity}: \xmlDesc{[silent, quiet, all, debug], optional}, 
        Desired verbosity of messages coming from this entity
      \item \xmlAttr{subType}: \xmlDesc{string, required}, 
        specify the type of ROM that will be used
  \end{itemize}

  The \xmlNode{KNeighborsRegressor} node recognizes the following subnodes:
  \begin{itemize}
    \item \xmlNode{Features}: \xmlDesc{comma-separated strings}, 
      specifies the names of the features of this ROM.         \nb These parameters are going to be
      requested for the training of this object         (see Section~\ref{subsec:stepRomTrainer})

    \item \xmlNode{Target}: \xmlDesc{comma-separated strings}, 
      contains a comma separated list of the targets of this ROM. These parameters         are the
      Figures of Merit (FOMs) this ROM is supposed to predict.         \nb These parameters are
      going to be requested for the training of this         object (see Section
      \ref{subsec:stepRomTrainer}).

    \item \xmlNode{pivotParameter}: \xmlDesc{string}, 
      If a time-dependent ROM is requested, please specifies the pivot         variable (e.g. time,
      etc) used in the input HistorySet.
  \default{time}

    \item \xmlNode{CV}: \xmlDesc{string}, 
      The text portion of this node needs to contain the name of the \xmlNode{PostProcessor} with
      \xmlAttr{subType}         ``CrossValidation``.
      The \xmlNode{CV} node recognizes the following parameters:
        \begin{itemize}
          \item \xmlAttr{class}: \xmlDesc{string, optional}, 
            should be set to \xmlString{Model}
          \item \xmlAttr{type}: \xmlDesc{string, optional}, 
            should be set to \xmlString{PostProcessor}
      \end{itemize}

    \item \xmlNode{alias}: \xmlDesc{string}, 
      specifies alias for         any variable of interest in the input or output space. These
      aliases can be used anywhere in the RAVEN input to         refer to the variables. In the body
      of this node the user specifies the name of the variable that the model is going to use
      (during its execution).
      The \xmlNode{alias} node recognizes the following parameters:
        \begin{itemize}
          \item \xmlAttr{variable}: \xmlDesc{string, required}, 
            define the actual alias, usable throughout the RAVEN input
          \item \xmlAttr{type}: \xmlDesc{[input, output], required}, 
            either ``input'' or ``output''.
      \end{itemize}

    \item \xmlNode{n\_neighbors}: \xmlDesc{integer}, 
      Number of neighbors to use by default for kneighbors queries.
  \default{5}

    \item \xmlNode{weights}: \xmlDesc{[uniform, distance]}, 
      weight function used in prediction. If ``uniform'', all points in each neighborhood
      are weighted equally. If ``distance'' weight points by the inverse of their distance. in this
      case,closer neighbors of a query point will have a greater influence than neighbors which are
      further away.
  \default{uniform}

    \item \xmlNode{algorithm}: \xmlDesc{[auto, ball\_tree, kd\_tree, brute]}, 
      Algorithm used to compute the nearest neighbors
  \default{auto}

    \item \xmlNode{leaf\_size}: \xmlDesc{integer}, 
      Leaf size passed to BallTree or KDTree. This can affect the speed of the construction
      and query, as well as the memory required to store the tree. The optimal value depends on the
      nature of the problem.
  \default{30}

    \item \xmlNode{p}: \xmlDesc{integer}, 
      Power parameter for the Minkowski metric. When $p = 1$, this is equivalent to using
      manhattan\_distance (l1), and euclidean\_distance (l2) for $p = 2$. For arbitrary $p$,
      minkowski\_distance                                                  (l\_p) is used.
  \default{2}

    \item \xmlNode{metric}: \xmlDesc{[euclidean, manhattan, minkowski, chebyshev, hamming, braycurtis]}, 
      the distance metric to use for the tree. The default metric is minkowski, and with
      $p=2$ is equivalent to the standard Euclidean metric.
      The available metrics are:                                                  \begin{itemize}
      \item minkowski: $sum(|x - y|^p)^(1/p)$
      \item euclidean: $sqrt(sum((x - y)^2))$
      \item manhattan: $sum(|x - y|)$                                                    \item
      chebyshev: $max(|x - y|)$                                                    \item hamming:
      $N\_unequal(x, y) / N\_tot$                                                    \item canberra:
      $sum(|x - y| / (|x| + |y|))$                                                    \item
      braycurtis: $sum(|x - y|) / (sum(|x|) + sum(|y|))$
      \end{itemize}
  \default{minkowski}
  \end{itemize}


\subsubsection{RadiusNeighborsClassifier}
  The \xmlNode{RadiusNeighborsClassifier} is a type of instance-based learning or
  non-generalizing learning: it does not attempt to construct a general internal
  model, but simply stores instances of the training data.                          Classification
  is computed from a simple majority vote of the nearest neighbors                          of each
  point: a query point is assigned the data class which has the most
  representatives within the nearest neighbors of the point.                          It implements
  learning based on the number of neighbors within a fixed radius                          $r$ of
  each training point, where $r$ is a floating-point value specified by the
  user.                          \zNormalizationPerformed{RadiusNeighborsClassifier}

  The \xmlNode{RadiusNeighborsClassifier} node recognizes the following parameters:
    \begin{itemize}
      \item \xmlAttr{name}: \xmlDesc{string, required}, 
        User-defined name to designate this entity in the RAVEN input file.
      \item \xmlAttr{verbosity}: \xmlDesc{[silent, quiet, all, debug], optional}, 
        Desired verbosity of messages coming from this entity
      \item \xmlAttr{subType}: \xmlDesc{string, required}, 
        specify the type of ROM that will be used
  \end{itemize}

  The \xmlNode{RadiusNeighborsClassifier} node recognizes the following subnodes:
  \begin{itemize}
    \item \xmlNode{Features}: \xmlDesc{comma-separated strings}, 
      specifies the names of the features of this ROM.         \nb These parameters are going to be
      requested for the training of this object         (see Section~\ref{subsec:stepRomTrainer})

    \item \xmlNode{Target}: \xmlDesc{comma-separated strings}, 
      contains a comma separated list of the targets of this ROM. These parameters         are the
      Figures of Merit (FOMs) this ROM is supposed to predict.         \nb These parameters are
      going to be requested for the training of this         object (see Section
      \ref{subsec:stepRomTrainer}).

    \item \xmlNode{pivotParameter}: \xmlDesc{string}, 
      If a time-dependent ROM is requested, please specifies the pivot         variable (e.g. time,
      etc) used in the input HistorySet.
  \default{time}

    \item \xmlNode{CV}: \xmlDesc{string}, 
      The text portion of this node needs to contain the name of the \xmlNode{PostProcessor} with
      \xmlAttr{subType}         ``CrossValidation``.
      The \xmlNode{CV} node recognizes the following parameters:
        \begin{itemize}
          \item \xmlAttr{class}: \xmlDesc{string, optional}, 
            should be set to \xmlString{Model}
          \item \xmlAttr{type}: \xmlDesc{string, optional}, 
            should be set to \xmlString{PostProcessor}
      \end{itemize}

    \item \xmlNode{alias}: \xmlDesc{string}, 
      specifies alias for         any variable of interest in the input or output space. These
      aliases can be used anywhere in the RAVEN input to         refer to the variables. In the body
      of this node the user specifies the name of the variable that the model is going to use
      (during its execution).
      The \xmlNode{alias} node recognizes the following parameters:
        \begin{itemize}
          \item \xmlAttr{variable}: \xmlDesc{string, required}, 
            define the actual alias, usable throughout the RAVEN input
          \item \xmlAttr{type}: \xmlDesc{[input, output], required}, 
            either ``input'' or ``output''.
      \end{itemize}

    \item \xmlNode{radius}: \xmlDesc{float}, 
      Range of parameter space to use by default for radius neighbors queries.
  \default{1.0}

    \item \xmlNode{weights}: \xmlDesc{[uniform, distance]}, 
      weight function used in prediction. If ``uniform'', all points in each neighborhood
      are weighted equally. If ``distance'' weight points by the inverse of their distance. in this
      case,closer neighbors of a query point will have a greater influence than neighbors which are
      further away.
  \default{uniform}

    \item \xmlNode{algorithm}: \xmlDesc{[auto, ball\_tree, kd\_tree, brute]}, 
      Algorithm used to compute the nearest neighbors
  \default{auto}

    \item \xmlNode{leaf\_size}: \xmlDesc{integer}, 
      Leaf size passed to BallTree or KDTree. This can affect the speed of the construction
      and query, as well as the memory required to store the tree. The optimal value depends on the
      nature of the problem.
  \default{30}

    \item \xmlNode{p}: \xmlDesc{integer}, 
      Power parameter for the Minkowski metric. When $p = 1$, this is equivalent to using
      manhattan\_distance (l1), and euclidean\_distance (l2) for $p = 2$. For arbitrary $p$,
      minkowski\_distance                                                  (l\_p) is used.
  \default{2}

    \item \xmlNode{metric}: \xmlDesc{[minkowski, euclidean, manhattan, chebyshev, hamming, canberra, braycurtis]}, 
      the distance metric to use for the tree. The default metric is minkowski, and with
      $p=2$ is equivalent to the standard Euclidean metric.
      The available metrics are:                                                  \begin{itemize}
      \item minkowski: $sum(|x - y|^p)^(1/p)$
      \item euclidean: $sqrt(sum((x - y)^2))$
      \item manhattan: $sum(|x - y|)$                                                    \item
      chebyshev: $max(|x - y|)$                                                    \item hamming:
      $N\_unequal(x, y) / N\_tot$                                                    \item canberra:
      $sum(|x - y| / (|x| + |y|))$                                                    \item
      braycurtis: $sum(|x - y|) / (sum(|x|) + sum(|y|))$
      \end{itemize}
  \default{minkowski}

    \item \xmlNode{outlier\_label}: \xmlDesc{comma-separated strings}, 
      label for outlier samples (samples with no neighbors in given radius).
      The available options are:                                                  \begin{itemize}
      \item manual label: strings or int labels. list of manual labels if multi-output is used.
      \item most\_frequent: assign the most frequent label of y to outliers.
      \item None: when any outlier is detected, an error will be raised.
      \end{itemize}
  \default{None}
  \end{itemize}


\subsubsection{LinearSVC}
  The \xmlNode{LinearSVC} \textit{Linear Support Vector Classification} is
  similar to SVC with parameter kernel=’linear’, but implemented in terms of liblinear rather than
  libsvm,                             so it has more flexibility in the choice of penalties and loss
  functions and should scale better to large numbers of samples.                             This
  class supports both dense and sparse input and the multiclass support is handled according to a
  one-vs-the-rest scheme.                             \zNormalizationPerformed{LinearSVC}

  The \xmlNode{LinearSVC} node recognizes the following parameters:
    \begin{itemize}
      \item \xmlAttr{name}: \xmlDesc{string, required}, 
        User-defined name to designate this entity in the RAVEN input file.
      \item \xmlAttr{verbosity}: \xmlDesc{[silent, quiet, all, debug], optional}, 
        Desired verbosity of messages coming from this entity
      \item \xmlAttr{subType}: \xmlDesc{string, required}, 
        specify the type of ROM that will be used
  \end{itemize}

  The \xmlNode{LinearSVC} node recognizes the following subnodes:
  \begin{itemize}
    \item \xmlNode{Features}: \xmlDesc{comma-separated strings}, 
      specifies the names of the features of this ROM.         \nb These parameters are going to be
      requested for the training of this object         (see Section~\ref{subsec:stepRomTrainer})

    \item \xmlNode{Target}: \xmlDesc{comma-separated strings}, 
      contains a comma separated list of the targets of this ROM. These parameters         are the
      Figures of Merit (FOMs) this ROM is supposed to predict.         \nb These parameters are
      going to be requested for the training of this         object (see Section
      \ref{subsec:stepRomTrainer}).

    \item \xmlNode{pivotParameter}: \xmlDesc{string}, 
      If a time-dependent ROM is requested, please specifies the pivot         variable (e.g. time,
      etc) used in the input HistorySet.
  \default{time}

    \item \xmlNode{CV}: \xmlDesc{string}, 
      The text portion of this node needs to contain the name of the \xmlNode{PostProcessor} with
      \xmlAttr{subType}         ``CrossValidation``.
      The \xmlNode{CV} node recognizes the following parameters:
        \begin{itemize}
          \item \xmlAttr{class}: \xmlDesc{string, optional}, 
            should be set to \xmlString{Model}
          \item \xmlAttr{type}: \xmlDesc{string, optional}, 
            should be set to \xmlString{PostProcessor}
      \end{itemize}

    \item \xmlNode{alias}: \xmlDesc{string}, 
      specifies alias for         any variable of interest in the input or output space. These
      aliases can be used anywhere in the RAVEN input to         refer to the variables. In the body
      of this node the user specifies the name of the variable that the model is going to use
      (during its execution).
      The \xmlNode{alias} node recognizes the following parameters:
        \begin{itemize}
          \item \xmlAttr{variable}: \xmlDesc{string, required}, 
            define the actual alias, usable throughout the RAVEN input
          \item \xmlAttr{type}: \xmlDesc{[input, output], required}, 
            either ``input'' or ``output''.
      \end{itemize}

    \item \xmlNode{penalty}: \xmlDesc{[l1, l2]}, 
      Specifies the norm used in the penalization. The ``l2'' penalty is the standard used in SVC.
      The ``l1' leads to coefficients vectors that are sparse.
  \default{l2}

    \item \xmlNode{loss}: \xmlDesc{[hinge, squared\_hinge]}, 
      Specifies the loss function. ``hinge'' is the standard SVM loss (used e.g. by the SVC class)
      while ``squared\_hinge'' is the square of the hinge loss. The combination of penalty=``l1' and
      loss=``hinge''                                                  is not supported.
  \default{squared\_hinge}

    \item \xmlNode{dual}: \xmlDesc{[True, Yes, 1, False, No, 0, t, y, 1, f, n, 0]}, 
      Select the algorithm to either solve the dual or primal optimization problem.
      Prefer dual=False when $n\_samples > n\_features$.
  \default{True}

    \item \xmlNode{C}: \xmlDesc{float}, 
      Regularization parameter. The strength of the regularization is inversely
      proportional to C.                                                            Must be strictly
      positive. The penalty is a squared l2 penalty..
  \default{1.0}

    \item \xmlNode{tol}: \xmlDesc{float}, 
      Tolerance for stopping criterion
  \default{0.0001}

    \item \xmlNode{multi\_class}: \xmlDesc{[crammer\_singer, ovr]}, 
      Determines the multi-class strategy if y contains more than two classes. ``ovr'' trains
      $n\_classes$ one-vs-rest classifiers, while ``crammer\_singer'' optimizes a joint objective over
      all classes.                                                  While crammer\_singer is
      interesting from a theoretical perspective as it is consistent, it is seldom used
      in practice as it rarely leads to better accuracy and is more expensive to compute. If
      ``crammer\_singer''                                                  is chosen, the options
      loss, penalty and dual will be ignored.
  \default{ovr}

    \item \xmlNode{fit\_intercept}: \xmlDesc{[True, Yes, 1, False, No, 0, t, y, 1, f, n, 0]}, 
      Whether to calculate the intercept for this model. If set to false, no
      intercept will be used in calculations (i.e. data is expected to be already centered).
  \default{True}

    \item \xmlNode{intercept\_scaling}: \xmlDesc{float}, 
      When fit\_intercept is True, instance vector x becomes $[x, intercept\_scaling]$,
      i.e. a “synthetic” feature with constant value equals to intercept\_scaling is appended
      to the instance vector. The intercept becomes $intercept\_scaling * synthetic feature weight$
      \nb the synthetic feature weight is subject to $l1/l2$ regularization as all other features.
      To lessen the effect of regularization on synthetic feature weight (and therefore on the
      intercept)                                                  $intercept\_scaling$ has to be
      increased.
  \default{1.0}

    \item \xmlNode{max\_iter}: \xmlDesc{integer}, 
      Hard limit on iterations within solver.``-1'' for no limit
  \default{1000}

    \item \xmlNode{verbose}: \xmlDesc{integer}, 
      Enable verbose output. Note that this setting takes advantage
      of a per-process runtime setting in liblinear that, if enabled, may not
      work properly in a multithreaded context.
  \default{0}

    \item \xmlNode{random\_state}: \xmlDesc{integer}, 
      Controls the pseudo random number generation for shuffling
      the data for the dual coordinate descent (if dual=True). When dual=False
      the underlying implementation of LinearSVC is not random and
      random\_state has no effect on the results. Pass an int for reproducible
      output across multiple function calls.
  \default{None}

    \item \xmlNode{class\_weight}: \xmlDesc{[balanced]}, 
      If not given, all classes are supposed to have weight one.
      The “balanced” mode uses the values of y to automatically adjust weights
      inversely proportional to class frequencies in the input data
  \default{None}
  \end{itemize}


\subsubsection{LinearSVR}
  The \xmlNode{LinearSVR} \textit{Linear Support Vector Regressor} is
  similar to SVR with parameter kernel=’linear’, but implemented in terms of liblinear rather than
  libsvm,                             so it has more flexibility in the choice of penalties and loss
  functions and should scale better to large numbers of samples.                             This
  class supports both dense and sparse input.
  \zNormalizationPerformed{LinearSVR}

  The \xmlNode{LinearSVR} node recognizes the following parameters:
    \begin{itemize}
      \item \xmlAttr{name}: \xmlDesc{string, required}, 
        User-defined name to designate this entity in the RAVEN input file.
      \item \xmlAttr{verbosity}: \xmlDesc{[silent, quiet, all, debug], optional}, 
        Desired verbosity of messages coming from this entity
      \item \xmlAttr{subType}: \xmlDesc{string, required}, 
        specify the type of ROM that will be used
  \end{itemize}

  The \xmlNode{LinearSVR} node recognizes the following subnodes:
  \begin{itemize}
    \item \xmlNode{Features}: \xmlDesc{comma-separated strings}, 
      specifies the names of the features of this ROM.         \nb These parameters are going to be
      requested for the training of this object         (see Section~\ref{subsec:stepRomTrainer})

    \item \xmlNode{Target}: \xmlDesc{comma-separated strings}, 
      contains a comma separated list of the targets of this ROM. These parameters         are the
      Figures of Merit (FOMs) this ROM is supposed to predict.         \nb These parameters are
      going to be requested for the training of this         object (see Section
      \ref{subsec:stepRomTrainer}).

    \item \xmlNode{pivotParameter}: \xmlDesc{string}, 
      If a time-dependent ROM is requested, please specifies the pivot         variable (e.g. time,
      etc) used in the input HistorySet.
  \default{time}

    \item \xmlNode{CV}: \xmlDesc{string}, 
      The text portion of this node needs to contain the name of the \xmlNode{PostProcessor} with
      \xmlAttr{subType}         ``CrossValidation``.
      The \xmlNode{CV} node recognizes the following parameters:
        \begin{itemize}
          \item \xmlAttr{class}: \xmlDesc{string, optional}, 
            should be set to \xmlString{Model}
          \item \xmlAttr{type}: \xmlDesc{string, optional}, 
            should be set to \xmlString{PostProcessor}
      \end{itemize}

    \item \xmlNode{alias}: \xmlDesc{string}, 
      specifies alias for         any variable of interest in the input or output space. These
      aliases can be used anywhere in the RAVEN input to         refer to the variables. In the body
      of this node the user specifies the name of the variable that the model is going to use
      (during its execution).
      The \xmlNode{alias} node recognizes the following parameters:
        \begin{itemize}
          \item \xmlAttr{variable}: \xmlDesc{string, required}, 
            define the actual alias, usable throughout the RAVEN input
          \item \xmlAttr{type}: \xmlDesc{[input, output], required}, 
            either ``input'' or ``output''.
      \end{itemize}

    \item \xmlNode{epsilon}: \xmlDesc{float}, 
      Epsilon parameter in the epsilon-insensitive loss function. The value of
      this parameter depends on the scale of the target variable y. If unsure, set $epsilon=0.$
  \default{0.0}

    \item \xmlNode{loss}: \xmlDesc{[epsilon\_insensitive’, squared\_epsilon\_insensitive]}, 
      Specifies the loss function. The epsilon-insensitive loss (standard SVR)
      is the L1 loss, while the squared epsilon-insensitive loss (``squared\_epsilon\_insensitive'')
      is the L2 loss.
  \default{squared\_epsilon\_insensitive}

    \item \xmlNode{tol}: \xmlDesc{float}, 
      Tolerance for stopping criterion
  \default{0.0001}

    \item \xmlNode{fit\_intercept}: \xmlDesc{[True, Yes, 1, False, No, 0, t, y, 1, f, n, 0]}, 
      Whether to calculate the intercept for this model. If set to false, no
      intercept will be used in calculations (i.e. data is expected to be already centered).
  \default{True}

    \item \xmlNode{intercept\_scaling}: \xmlDesc{float}, 
      When fit\_intercept is True, instance vector x becomes $[x, intercept\_scaling]$,
      i.e. a “synthetic” feature with constant value equals to intercept\_scaling is appended
      to the instance vector. The intercept becomes $intercept\_scaling * synthetic feature weight$
      \nb the synthetic feature weight is subject to $l1/l2$ regularization as all other features.
      To lessen the effect of regularization on synthetic feature weight (and therefore on the
      intercept)                                                  $intercept\_scaling$ has to be
      increased.
  \default{1.0}

    \item \xmlNode{dual}: \xmlDesc{[True, Yes, 1, False, No, 0, t, y, 1, f, n, 0]}, 
      Select the algorithm to either solve the dual or primal optimization problem.
      Prefer dual=False when $n\_samples > n\_features$.
  \default{True}

    \item \xmlNode{max\_iter}: \xmlDesc{integer}, 
      Hard limit on iterations within solver.``-1'' for no limit
  \default{-1}
  \end{itemize}


\subsubsection{NuSVC}
  The \xmlNode{NuSVC} \textit{Nu-Support Vector Classification} is an Nu-Support Vector
  Classification.                             It is very similar to SVC but with the addition of the
  hyper-parameter Nu for controlling the                             number of support vectors. In
  NuSVC nu replaces C of SVC.                             The implementation is based on libsvm.
  \zNormalizationPerformed{NuSVC}

  The \xmlNode{NuSVC} node recognizes the following parameters:
    \begin{itemize}
      \item \xmlAttr{name}: \xmlDesc{string, required}, 
        User-defined name to designate this entity in the RAVEN input file.
      \item \xmlAttr{verbosity}: \xmlDesc{[silent, quiet, all, debug], optional}, 
        Desired verbosity of messages coming from this entity
      \item \xmlAttr{subType}: \xmlDesc{string, required}, 
        specify the type of ROM that will be used
  \end{itemize}

  The \xmlNode{NuSVC} node recognizes the following subnodes:
  \begin{itemize}
    \item \xmlNode{Features}: \xmlDesc{comma-separated strings}, 
      specifies the names of the features of this ROM.         \nb These parameters are going to be
      requested for the training of this object         (see Section~\ref{subsec:stepRomTrainer})

    \item \xmlNode{Target}: \xmlDesc{comma-separated strings}, 
      contains a comma separated list of the targets of this ROM. These parameters         are the
      Figures of Merit (FOMs) this ROM is supposed to predict.         \nb These parameters are
      going to be requested for the training of this         object (see Section
      \ref{subsec:stepRomTrainer}).

    \item \xmlNode{pivotParameter}: \xmlDesc{string}, 
      If a time-dependent ROM is requested, please specifies the pivot         variable (e.g. time,
      etc) used in the input HistorySet.
  \default{time}

    \item \xmlNode{CV}: \xmlDesc{string}, 
      The text portion of this node needs to contain the name of the \xmlNode{PostProcessor} with
      \xmlAttr{subType}         ``CrossValidation``.
      The \xmlNode{CV} node recognizes the following parameters:
        \begin{itemize}
          \item \xmlAttr{class}: \xmlDesc{string, optional}, 
            should be set to \xmlString{Model}
          \item \xmlAttr{type}: \xmlDesc{string, optional}, 
            should be set to \xmlString{PostProcessor}
      \end{itemize}

    \item \xmlNode{alias}: \xmlDesc{string}, 
      specifies alias for         any variable of interest in the input or output space. These
      aliases can be used anywhere in the RAVEN input to         refer to the variables. In the body
      of this node the user specifies the name of the variable that the model is going to use
      (during its execution).
      The \xmlNode{alias} node recognizes the following parameters:
        \begin{itemize}
          \item \xmlAttr{variable}: \xmlDesc{string, required}, 
            define the actual alias, usable throughout the RAVEN input
          \item \xmlAttr{type}: \xmlDesc{[input, output], required}, 
            either ``input'' or ``output''.
      \end{itemize}

    \item \xmlNode{nu}: \xmlDesc{float}, 
      An upper bound on the fraction of margin errors and
      a lower bound of the fraction of support vectors. Should be in the interval $(0, 1]$.
  \default{0.5}

    \item \xmlNode{kernel}: \xmlDesc{[linear, poly, rbf, sigmoid]}, 
      Specifies the kernel type to be used in the algorithm. It must be one of
      ``linear'', ``poly'', ``rbf'' or ``sigmoid''.
  \default{rbf}

    \item \xmlNode{degree}: \xmlDesc{integer}, 
      Degree of the polynomial kernel function ('poly').Ignored by all other kernels.
  \default{3}

    \item \xmlNode{gamma}: \xmlDesc{float}, 
      Kernel coefficient for ``poly'', ``rbf'' or ``sigmoid''. If not input, then it uses
      $1 / (n\_features * X.var())$ as value of gamma
  \default{scale}

    \item \xmlNode{coef0}: \xmlDesc{float}, 
      Independent term in kernel function
  \default{0.0}

    \item \xmlNode{tol}: \xmlDesc{float}, 
      Tolerance for stopping criterion
  \default{0.001}

    \item \xmlNode{cache\_size}: \xmlDesc{float}, 
      Size of the kernel cache (in MB)
  \default{200.0}

    \item \xmlNode{shrinking}: \xmlDesc{[True, Yes, 1, False, No, 0, t, y, 1, f, n, 0]}, 
      Whether to use the shrinking heuristic.
  \default{True}

    \item \xmlNode{max\_iter}: \xmlDesc{integer}, 
      Hard limit on iterations within solver.``-1'' for no limit
  \default{-1}

    \item \xmlNode{decision\_function\_shape}: \xmlDesc{[ovo, ovr]}, 
      Whether to return a one-vs-rest (``ovr'') decision function of shape $(n\_samples, n\_classes)$
      as                                                            all other classifiers, or the
      original one-vs-one (``ovo'') decision function of libsvm which has
      shape $(n\_samples, n\_classes * (n\_classes - 1) / 2)$. However, one-vs-one (``ovo'') is always
      used as                                                            multi-class strategy. The
      parameter is ignored for binary classification.
  \default{ovr}

    \item \xmlNode{verbose}: \xmlDesc{[True, Yes, 1, False, No, 0, t, y, 1, f, n, 0]}, 
      Enable verbose output. Note that this setting takes advantage
      of a per-process runtime setting in libsvm that, if enabled, may not
      work properly in a multithreaded context.
  \default{False}

    \item \xmlNode{probability}: \xmlDesc{[True, Yes, 1, False, No, 0, t, y, 1, f, n, 0]}, 
      Whether to enable probability estimates.
  \default{False}

    \item \xmlNode{class\_weight}: \xmlDesc{[balanced]}, 
      If not given, all classes are supposed to have weight one.
      The “balanced” mode uses the values of y to automatically adjust weights
      inversely proportional to class frequencies in the input data
  \default{None}

    \item \xmlNode{random\_state}: \xmlDesc{integer}, 
      Controls the pseudo random number generation for shuffling
      the data for probability estimates. Ignored when probability is False.
      Pass an int for reproducible output across multiple function calls.
  \default{None}
  \end{itemize}


\subsubsection{NuSVR}
  The \xmlNode{NuSVR} \textit{Nu-Support Vector Regression} is an Nu-Support Vector Regressor.
  It is very similar to SVC but with the addition of the hyper-parameter Nu for controlling the
  number of support vectors. However, unlike NuSVC, where nu replaces C,
  here nu replaces the parameter epsilon of epsilon-SVR.
  \zNormalizationPerformed{NuSVR}

  The \xmlNode{NuSVR} node recognizes the following parameters:
    \begin{itemize}
      \item \xmlAttr{name}: \xmlDesc{string, required}, 
        User-defined name to designate this entity in the RAVEN input file.
      \item \xmlAttr{verbosity}: \xmlDesc{[silent, quiet, all, debug], optional}, 
        Desired verbosity of messages coming from this entity
      \item \xmlAttr{subType}: \xmlDesc{string, required}, 
        specify the type of ROM that will be used
  \end{itemize}

  The \xmlNode{NuSVR} node recognizes the following subnodes:
  \begin{itemize}
    \item \xmlNode{Features}: \xmlDesc{comma-separated strings}, 
      specifies the names of the features of this ROM.         \nb These parameters are going to be
      requested for the training of this object         (see Section~\ref{subsec:stepRomTrainer})

    \item \xmlNode{Target}: \xmlDesc{comma-separated strings}, 
      contains a comma separated list of the targets of this ROM. These parameters         are the
      Figures of Merit (FOMs) this ROM is supposed to predict.         \nb These parameters are
      going to be requested for the training of this         object (see Section
      \ref{subsec:stepRomTrainer}).

    \item \xmlNode{pivotParameter}: \xmlDesc{string}, 
      If a time-dependent ROM is requested, please specifies the pivot         variable (e.g. time,
      etc) used in the input HistorySet.
  \default{time}

    \item \xmlNode{CV}: \xmlDesc{string}, 
      The text portion of this node needs to contain the name of the \xmlNode{PostProcessor} with
      \xmlAttr{subType}         ``CrossValidation``.
      The \xmlNode{CV} node recognizes the following parameters:
        \begin{itemize}
          \item \xmlAttr{class}: \xmlDesc{string, optional}, 
            should be set to \xmlString{Model}
          \item \xmlAttr{type}: \xmlDesc{string, optional}, 
            should be set to \xmlString{PostProcessor}
      \end{itemize}

    \item \xmlNode{alias}: \xmlDesc{string}, 
      specifies alias for         any variable of interest in the input or output space. These
      aliases can be used anywhere in the RAVEN input to         refer to the variables. In the body
      of this node the user specifies the name of the variable that the model is going to use
      (during its execution).
      The \xmlNode{alias} node recognizes the following parameters:
        \begin{itemize}
          \item \xmlAttr{variable}: \xmlDesc{string, required}, 
            define the actual alias, usable throughout the RAVEN input
          \item \xmlAttr{type}: \xmlDesc{[input, output], required}, 
            either ``input'' or ``output''.
      \end{itemize}

    \item \xmlNode{nu}: \xmlDesc{float}, 
      An upper bound on the fraction of margin errors and
      a lower bound of the fraction of support vectors. Should be in the interval $(0, 1]$.
  \default{0.5}

    \item \xmlNode{C}: \xmlDesc{float}, 
      Regularization parameter. The strength of the regularization is inversely
      proportional to C.                                                           Must be strictly
      positive. The penalty is a squared l2 penalty.
  \default{1.0}

    \item \xmlNode{kernel}: \xmlDesc{[linear, poly, rbf, sigmoid]}, 
      Specifies the kernel type to be used in the algorithm. It must be one of
      ``linear'', ``poly'', ``rbf'' or ``sigmoid''.
  \default{rbf}

    \item \xmlNode{degree}: \xmlDesc{integer}, 
      Degree of the polynomial kernel function ('poly').Ignored by all other kernels.
  \default{3}

    \item \xmlNode{gamma}: \xmlDesc{float}, 
      Kernel coefficient for ``poly'', ``rbf'' or ``sigmoid''. If not input, then it uses
      $1 / (n\_features * X.var())$ as value of gamma
  \default{scale}

    \item \xmlNode{coef0}: \xmlDesc{float}, 
      Independent term in kernel function
  \default{0.0}

    \item \xmlNode{tol}: \xmlDesc{float}, 
      Tolerance for stopping criterion
  \default{0.001}

    \item \xmlNode{cache\_size}: \xmlDesc{float}, 
      Size of the kernel cache (in MB)
  \default{200.0}

    \item \xmlNode{shrinking}: \xmlDesc{[True, Yes, 1, False, No, 0, t, y, 1, f, n, 0]}, 
      Whether to use the shrinking heuristic.
  \default{True}

    \item \xmlNode{max\_iter}: \xmlDesc{integer}, 
      Hard limit on iterations within solver.``-1'' for no limit
  \default{-1}
  \end{itemize}


\subsubsection{SVC}
  The \xmlNode{SVC} \textit{C-Support Vector Classification} is an epsilon-Support Vector
  Classification.                             The free parameters in this model are C and epsilon.
  The implementation is based on libsvm. The fit time scales at least
  quadratically with the number of samples and may be impractical                             beyond
  tens of thousands of samples. The multiclass support is handled according to a one-vs-one scheme.
  \zNormalizationPerformed{SVC}

  The \xmlNode{SVC} node recognizes the following parameters:
    \begin{itemize}
      \item \xmlAttr{name}: \xmlDesc{string, required}, 
        User-defined name to designate this entity in the RAVEN input file.
      \item \xmlAttr{verbosity}: \xmlDesc{[silent, quiet, all, debug], optional}, 
        Desired verbosity of messages coming from this entity
      \item \xmlAttr{subType}: \xmlDesc{string, required}, 
        specify the type of ROM that will be used
  \end{itemize}

  The \xmlNode{SVC} node recognizes the following subnodes:
  \begin{itemize}
    \item \xmlNode{Features}: \xmlDesc{comma-separated strings}, 
      specifies the names of the features of this ROM.         \nb These parameters are going to be
      requested for the training of this object         (see Section~\ref{subsec:stepRomTrainer})

    \item \xmlNode{Target}: \xmlDesc{comma-separated strings}, 
      contains a comma separated list of the targets of this ROM. These parameters         are the
      Figures of Merit (FOMs) this ROM is supposed to predict.         \nb These parameters are
      going to be requested for the training of this         object (see Section
      \ref{subsec:stepRomTrainer}).

    \item \xmlNode{pivotParameter}: \xmlDesc{string}, 
      If a time-dependent ROM is requested, please specifies the pivot         variable (e.g. time,
      etc) used in the input HistorySet.
  \default{time}

    \item \xmlNode{CV}: \xmlDesc{string}, 
      The text portion of this node needs to contain the name of the \xmlNode{PostProcessor} with
      \xmlAttr{subType}         ``CrossValidation``.
      The \xmlNode{CV} node recognizes the following parameters:
        \begin{itemize}
          \item \xmlAttr{class}: \xmlDesc{string, optional}, 
            should be set to \xmlString{Model}
          \item \xmlAttr{type}: \xmlDesc{string, optional}, 
            should be set to \xmlString{PostProcessor}
      \end{itemize}

    \item \xmlNode{alias}: \xmlDesc{string}, 
      specifies alias for         any variable of interest in the input or output space. These
      aliases can be used anywhere in the RAVEN input to         refer to the variables. In the body
      of this node the user specifies the name of the variable that the model is going to use
      (during its execution).
      The \xmlNode{alias} node recognizes the following parameters:
        \begin{itemize}
          \item \xmlAttr{variable}: \xmlDesc{string, required}, 
            define the actual alias, usable throughout the RAVEN input
          \item \xmlAttr{type}: \xmlDesc{[input, output], required}, 
            either ``input'' or ``output''.
      \end{itemize}

    \item \xmlNode{C}: \xmlDesc{float}, 
      Regularization parameter. The strength of the regularization is inversely
      proportional to C.                                                            Must be strictly
      positive. The penalty is a squared l2 penalty..
  \default{1.0}

    \item \xmlNode{kernel}: \xmlDesc{[linear, poly, rbf, sigmoid]}, 
      Specifies the kernel type to be used in the algorithm. It must be one of
      ``linear'', ``poly'', ``rbf'' or ``sigmoid''.
  \default{rbf}

    \item \xmlNode{degree}: \xmlDesc{integer}, 
      Degree of the polynomial kernel function ('poly').Ignored by all other kernels.
  \default{3}

    \item \xmlNode{gamma}: \xmlDesc{float}, 
      Kernel coefficient for ``poly'', ``rbf'' or ``sigmoid''. If not input, then it uses
      $1 / (n\_features * X.var())$ as value of gamma
  \default{scale}

    \item \xmlNode{coef0}: \xmlDesc{float}, 
      Independent term in kernel function
  \default{0.0}

    \item \xmlNode{tol}: \xmlDesc{float}, 
      Tolerance for stopping criterion
  \default{0.001}

    \item \xmlNode{cache\_size}: \xmlDesc{float}, 
      Size of the kernel cache (in MB)
  \default{200.0}

    \item \xmlNode{shrinking}: \xmlDesc{[True, Yes, 1, False, No, 0, t, y, 1, f, n, 0]}, 
      Whether to use the shrinking heuristic.
  \default{True}

    \item \xmlNode{max\_iter}: \xmlDesc{integer}, 
      Hard limit on iterations within solver.``-1'' for no limit
  \default{-1}

    \item \xmlNode{decision\_function\_shape}: \xmlDesc{[ovo, ovr]}, 
      Whether to return a one-vs-rest (``ovr'') decision function of shape $(n\_samples, n\_classes)$
      as                                                            all other classifiers, or the
      original one-vs-one (``ovo'') decision function of libsvm which has
      shape $(n\_samples, n\_classes * (n\_classes - 1) / 2)$. However, one-vs-one (``ovo'') is always
      used as                                                            multi-class strategy. The
      parameter is ignored for binary classification.
  \default{ovr}

    \item \xmlNode{verbose}: \xmlDesc{[True, Yes, 1, False, No, 0, t, y, 1, f, n, 0]}, 
      Enable verbose output. Note that this setting takes advantage
      of a per-process runtime setting in libsvm that, if enabled, may not
      work properly in a multithreaded context.
  \default{False}

    \item \xmlNode{probability}: \xmlDesc{[True, Yes, 1, False, No, 0, t, y, 1, f, n, 0]}, 
      Whether to enable probability estimates.
  \default{False}

    \item \xmlNode{class\_weight}: \xmlDesc{[balanced]}, 
      If not given, all classes are supposed to have weight one.
      The “balanced” mode uses the values of y to automatically adjust weights
      inversely proportional to class frequencies in the input data
  \default{None}

    \item \xmlNode{random\_state}: \xmlDesc{integer}, 
      Controls the pseudo random number generation for shuffling
      the data for probability estimates. Ignored when probability is False.
      Pass an int for reproducible output across multiple function calls.
  \default{None}
  \end{itemize}


\subsubsection{SVR}
  The \xmlNode{SVR} \textit{Support Vector Regression} is an epsilon-Support Vector Regression.
  The free parameters in this model are C and epsilon. The implementations is a based on libsvm.
  The implementation is based on libsvm. The fit time complexity                             is more
  than quadratic with the number of samples which makes it hard                             to scale
  to datasets with more than a couple of 10000 samples.
  \zNormalizationPerformed{SVR}

  The \xmlNode{SVR} node recognizes the following parameters:
    \begin{itemize}
      \item \xmlAttr{name}: \xmlDesc{string, required}, 
        User-defined name to designate this entity in the RAVEN input file.
      \item \xmlAttr{verbosity}: \xmlDesc{[silent, quiet, all, debug], optional}, 
        Desired verbosity of messages coming from this entity
      \item \xmlAttr{subType}: \xmlDesc{string, required}, 
        specify the type of ROM that will be used
  \end{itemize}

  The \xmlNode{SVR} node recognizes the following subnodes:
  \begin{itemize}
    \item \xmlNode{Features}: \xmlDesc{comma-separated strings}, 
      specifies the names of the features of this ROM.         \nb These parameters are going to be
      requested for the training of this object         (see Section~\ref{subsec:stepRomTrainer})

    \item \xmlNode{Target}: \xmlDesc{comma-separated strings}, 
      contains a comma separated list of the targets of this ROM. These parameters         are the
      Figures of Merit (FOMs) this ROM is supposed to predict.         \nb These parameters are
      going to be requested for the training of this         object (see Section
      \ref{subsec:stepRomTrainer}).

    \item \xmlNode{pivotParameter}: \xmlDesc{string}, 
      If a time-dependent ROM is requested, please specifies the pivot         variable (e.g. time,
      etc) used in the input HistorySet.
  \default{time}

    \item \xmlNode{CV}: \xmlDesc{string}, 
      The text portion of this node needs to contain the name of the \xmlNode{PostProcessor} with
      \xmlAttr{subType}         ``CrossValidation``.
      The \xmlNode{CV} node recognizes the following parameters:
        \begin{itemize}
          \item \xmlAttr{class}: \xmlDesc{string, optional}, 
            should be set to \xmlString{Model}
          \item \xmlAttr{type}: \xmlDesc{string, optional}, 
            should be set to \xmlString{PostProcessor}
      \end{itemize}

    \item \xmlNode{alias}: \xmlDesc{string}, 
      specifies alias for         any variable of interest in the input or output space. These
      aliases can be used anywhere in the RAVEN input to         refer to the variables. In the body
      of this node the user specifies the name of the variable that the model is going to use
      (during its execution).
      The \xmlNode{alias} node recognizes the following parameters:
        \begin{itemize}
          \item \xmlAttr{variable}: \xmlDesc{string, required}, 
            define the actual alias, usable throughout the RAVEN input
          \item \xmlAttr{type}: \xmlDesc{[input, output], required}, 
            either ``input'' or ``output''.
      \end{itemize}

    \item \xmlNode{C}: \xmlDesc{float}, 
      Regularization parameter. The strength of the regularization is inversely
      proportional to C.                                                            Must be strictly
      positive. The penalty is a squared l2 penalty..
  \default{1.0}

    \item \xmlNode{kernel}: \xmlDesc{[linear, poly, rbf, sigmoid]}, 
      Specifies the kernel type to be used in the algorithm. It must be one of
      ``linear'', ``poly'', ``rbf'' or ``sigmoid''.
  \default{rbf}

    \item \xmlNode{degree}: \xmlDesc{integer}, 
      Degree of the polynomial kernel function ('poly').Ignored by all other kernels.
  \default{3}

    \item \xmlNode{gamma}: \xmlDesc{float}, 
      Kernel coefficient for ``poly'', ``rbf'' or ``sigmoid''. If not input, then it uses
      $1 / (n\_features * X.var())$ as value of gamma
  \default{scale}

    \item \xmlNode{coef0}: \xmlDesc{float}, 
      Independent term in kernel function
  \default{0.0}

    \item \xmlNode{tol}: \xmlDesc{float}, 
      Tolerance for stopping criterion
  \default{0.001}

    \item \xmlNode{cache\_size}: \xmlDesc{float}, 
      Size of the kernel cache (in MB)
  \default{200.0}

    \item \xmlNode{epsilon}: \xmlDesc{float}, 
      Epsilon in the epsilon-SVR model. It specifies the epsilon-tube
      within which no penalty is associated in the training loss function
      with points predicted within a distance epsilon from the actual
      value.
  \default{0.1}

    \item \xmlNode{shrinking}: \xmlDesc{[True, Yes, 1, False, No, 0, t, y, 1, f, n, 0]}, 
      Whether to use the shrinking heuristic.
  \default{True}

    \item \xmlNode{max\_iter}: \xmlDesc{integer}, 
      Hard limit on iterations within solver.``-1'' for no limit
  \default{-1}

    \item \xmlNode{verbose}: \xmlDesc{[True, Yes, 1, False, No, 0, t, y, 1, f, n, 0]}, 
      Enable verbose output. Note that this setting takes advantage
      of a per-process runtime setting in libsvm that, if enabled, may not
      work properly in a multithreaded context.
  \default{False}
  \end{itemize}


\subsubsection{DecisionTreeClassifier}
  The \xmlNode{DecisionTreeClassifier} is a classifier that is based on the
  decision tree logic.                          \zNormalizationPerformed{DecisionTreeClassifier}

  The \xmlNode{DecisionTreeClassifier} node recognizes the following parameters:
    \begin{itemize}
      \item \xmlAttr{name}: \xmlDesc{string, required}, 
        User-defined name to designate this entity in the RAVEN input file.
      \item \xmlAttr{verbosity}: \xmlDesc{[silent, quiet, all, debug], optional}, 
        Desired verbosity of messages coming from this entity
      \item \xmlAttr{subType}: \xmlDesc{string, required}, 
        specify the type of ROM that will be used
  \end{itemize}

  The \xmlNode{DecisionTreeClassifier} node recognizes the following subnodes:
  \begin{itemize}
    \item \xmlNode{Features}: \xmlDesc{comma-separated strings}, 
      specifies the names of the features of this ROM.         \nb These parameters are going to be
      requested for the training of this object         (see Section~\ref{subsec:stepRomTrainer})

    \item \xmlNode{Target}: \xmlDesc{comma-separated strings}, 
      contains a comma separated list of the targets of this ROM. These parameters         are the
      Figures of Merit (FOMs) this ROM is supposed to predict.         \nb These parameters are
      going to be requested for the training of this         object (see Section
      \ref{subsec:stepRomTrainer}).

    \item \xmlNode{pivotParameter}: \xmlDesc{string}, 
      If a time-dependent ROM is requested, please specifies the pivot         variable (e.g. time,
      etc) used in the input HistorySet.
  \default{time}

    \item \xmlNode{CV}: \xmlDesc{string}, 
      The text portion of this node needs to contain the name of the \xmlNode{PostProcessor} with
      \xmlAttr{subType}         ``CrossValidation``.
      The \xmlNode{CV} node recognizes the following parameters:
        \begin{itemize}
          \item \xmlAttr{class}: \xmlDesc{string, optional}, 
            should be set to \xmlString{Model}
          \item \xmlAttr{type}: \xmlDesc{string, optional}, 
            should be set to \xmlString{PostProcessor}
      \end{itemize}

    \item \xmlNode{alias}: \xmlDesc{string}, 
      specifies alias for         any variable of interest in the input or output space. These
      aliases can be used anywhere in the RAVEN input to         refer to the variables. In the body
      of this node the user specifies the name of the variable that the model is going to use
      (during its execution).
      The \xmlNode{alias} node recognizes the following parameters:
        \begin{itemize}
          \item \xmlAttr{variable}: \xmlDesc{string, required}, 
            define the actual alias, usable throughout the RAVEN input
          \item \xmlAttr{type}: \xmlDesc{[input, output], required}, 
            either ``input'' or ``output''.
      \end{itemize}

    \item \xmlNode{criterion}: \xmlDesc{[gini, entropy]}, 
      The function to measure the quality of a split. Supported criteria are ``gini'' for the
      Gini impurity and ``entropy'' for the information gain.
  \default{gini}

    \item \xmlNode{splitter}: \xmlDesc{[best, random]}, 
      The strategy used to choose the split at each node. Supported strategies are ``best''
      to choose the best split and ``random'' to choose the best random split.
  \default{best}

    \item \xmlNode{max\_depth}: \xmlDesc{integer}, 
      The maximum depth of the tree. If None, then nodes are expanded until all leaves are pure
      or until all leaves contain less than min\_samples\_split samples.
  \default{None}

    \item \xmlNode{min\_samples\_split}: \xmlDesc{integer}, 
      The minimum number of samples required to split an internal node
  \default{2}

    \item \xmlNode{min\_samples\_leaf}: \xmlDesc{integer}, 
      The minimum number of samples required to be at a leaf node. A split point at any
      depth will only be considered if it leaves at least min\_samples\_leaf training samples in
      each                                                  of the left and right branches. This may
      have the effect of smoothing the model, especially
      in regression.
  \default{1}

    \item \xmlNode{min\_weight\_fraction\_leaf}: \xmlDesc{float}, 
      The minimum weighted fraction of the sum total of weights (of all the input samples)
      required to be at a leaf node. Samples have equal weight when sample\_weight is not provided.
  \default{0.0}

    \item \xmlNode{max\_features}: \xmlDesc{[auto, sqrt, log2]}, 
      The strategy to compute the number of features to consider when looking for the best split:
      \begin{itemize}                                                     \item sqrt:
      $max\_features=sqrt(n\_features)$                                                     \item
      log2: $max\_features=log2(n\_features)$
      \item auto: automatic selection
      \end{itemize}                                                   \nb the search for a split
      does not stop until at least one valid partition of the node
      samples is found, even if it requires to effectively inspect more than max\_features features.
  \default{None}

    \item \xmlNode{max\_leaf\_nodes}: \xmlDesc{integer}, 
      Grow a tree with max\_leaf\_nodes in best-first fashion. Best nodes are defined as relative
      reduction                                                  in impurity. If None then unlimited
      number of leaf nodes.
  \default{None}

    \item \xmlNode{min\_impurity\_decrease}: \xmlDesc{float}, 
      A node will be split if this split induces a decrease of the impurity greater than or equal to
      this value.                                                  The weighted impurity decrease
      equation is the following:                                                  $N\_t / N *
      (impurity - N\_t\_R / N\_t * right\_impurity - N\_t\_L / N\_t * left\_impurity)$
      where $N$ is the total number of samples, $N\_t$ is the number of samples at the current node,
      $N\_t\_L$ is the number                                                  of samples in the
      left child, and $N\_t\_R$ is the number of samples in the right child.
      $N$, $N\_t$, $N\_t]\_R$ and $N\_t\_L$ all refer to the weighted sum, if sample\_weight is
      passed.
  \default{0.0}

    \item \xmlNode{random\_state}: \xmlDesc{integer}, 
      Controls the randomness of the estimator. The features are
      always randomly permuted at each split, even if splitter is set to
      "best". When max\_features < n\_features, the algorithm will select
      max\_features at random at each split before finding the best split
      among them. But the best found split may vary across different runs,
      even if max\_features=n\_features. That is the case, if the improvement
      of the criterion is identical for several splits and one split has to
      be selected at random. To obtain a deterministic behaviour during
      fitting, random\_state has to be fixed to an integer.
  \default{None}
  \end{itemize}


\subsubsection{DecisionTreeRegressor}
  The \xmlNode{DecisionTreeRegressor} is a regressor that is based on the
  decision tree logic.                          \zNormalizationPerformed{DecisionTreeRegressor}

  The \xmlNode{DecisionTreeRegressor} node recognizes the following parameters:
    \begin{itemize}
      \item \xmlAttr{name}: \xmlDesc{string, required}, 
        User-defined name to designate this entity in the RAVEN input file.
      \item \xmlAttr{verbosity}: \xmlDesc{[silent, quiet, all, debug], optional}, 
        Desired verbosity of messages coming from this entity
      \item \xmlAttr{subType}: \xmlDesc{string, required}, 
        specify the type of ROM that will be used
  \end{itemize}

  The \xmlNode{DecisionTreeRegressor} node recognizes the following subnodes:
  \begin{itemize}
    \item \xmlNode{Features}: \xmlDesc{comma-separated strings}, 
      specifies the names of the features of this ROM.         \nb These parameters are going to be
      requested for the training of this object         (see Section~\ref{subsec:stepRomTrainer})

    \item \xmlNode{Target}: \xmlDesc{comma-separated strings}, 
      contains a comma separated list of the targets of this ROM. These parameters         are the
      Figures of Merit (FOMs) this ROM is supposed to predict.         \nb These parameters are
      going to be requested for the training of this         object (see Section
      \ref{subsec:stepRomTrainer}).

    \item \xmlNode{pivotParameter}: \xmlDesc{string}, 
      If a time-dependent ROM is requested, please specifies the pivot         variable (e.g. time,
      etc) used in the input HistorySet.
  \default{time}

    \item \xmlNode{CV}: \xmlDesc{string}, 
      The text portion of this node needs to contain the name of the \xmlNode{PostProcessor} with
      \xmlAttr{subType}         ``CrossValidation``.
      The \xmlNode{CV} node recognizes the following parameters:
        \begin{itemize}
          \item \xmlAttr{class}: \xmlDesc{string, optional}, 
            should be set to \xmlString{Model}
          \item \xmlAttr{type}: \xmlDesc{string, optional}, 
            should be set to \xmlString{PostProcessor}
      \end{itemize}

    \item \xmlNode{alias}: \xmlDesc{string}, 
      specifies alias for         any variable of interest in the input or output space. These
      aliases can be used anywhere in the RAVEN input to         refer to the variables. In the body
      of this node the user specifies the name of the variable that the model is going to use
      (during its execution).
      The \xmlNode{alias} node recognizes the following parameters:
        \begin{itemize}
          \item \xmlAttr{variable}: \xmlDesc{string, required}, 
            define the actual alias, usable throughout the RAVEN input
          \item \xmlAttr{type}: \xmlDesc{[input, output], required}, 
            either ``input'' or ``output''.
      \end{itemize}

    \item \xmlNode{criterion}: \xmlDesc{[mse, friedman\_mse, mae, poisson]}, 
      The function to measure the quality of a split. Supported criteria are ``mse'' for the mean
      squared error,                                                  which is equal to variance
      reduction as feature selection criterion and minimizes the L2 loss using the mean of each
      terminal node, ``friedman\_mse'', which uses mean squared error with Friedman's improvement
      score for potential splits,                                                  ``mae'' for the
      mean absolute error, which minimizes the L1 loss using the median of each terminal node, and
      ``poisson''                                                  which uses reduction in Poisson
      deviance to find splits.
  \default{mse}

    \item \xmlNode{splitter}: \xmlDesc{[best, random]}, 
      The strategy used to choose the split at each node. Supported strategies are ``best''
      to choose the best split and ``random'' to choose the best random split.
  \default{best}

    \item \xmlNode{max\_depth}: \xmlDesc{integer}, 
      The maximum depth of the tree. If None, then nodes are expanded until all leaves are pure
      or until all leaves contain less than min\_samples\_split samples.
  \default{None}

    \item \xmlNode{min\_samples\_split}: \xmlDesc{integer}, 
      The minimum number of samples required to split an internal node
  \default{2}

    \item \xmlNode{min\_samples\_leaf}: \xmlDesc{integer}, 
      The minimum number of samples required to be at a leaf node. A split point at any
      depth will only be considered if it leaves at least min\_samples\_leaf training samples in
      each                                                  of the left and right branches. This may
      have the effect of smoothing the model, especially
      in regression.
  \default{1}

    \item \xmlNode{min\_weight\_fraction\_leaf}: \xmlDesc{float}, 
      The minimum weighted fraction of the sum total of weights (of all the input samples)
      required to be at a leaf node. Samples have equal weight when sample\_weight is not provided.
  \default{0.0}

    \item \xmlNode{max\_features}: \xmlDesc{[auto, sqrt, log2]}, 
      The strategy to compute the number of features to consider when looking for the best split:
      \begin{itemize}                                                     \item sqrt:
      $max\_features=sqrt(n\_features)$                                                     \item
      log2: $max\_features=log2(n\_features)$
      \item None: $max\_features=n\_features$
      \item auto: automatic selection
      \end{itemize}                                                   \nb the search for a split
      does not stop until at least one valid partition of the node
      samples is found, even if it requires to effectively inspect more than max\_features features.
  \default{None}

    \item \xmlNode{max\_leaf\_nodes}: \xmlDesc{integer}, 
      Grow a tree with max\_leaf\_nodes in best-first fashion. Best nodes are defined as relative
      reduction                                                  in impurity. If None then unlimited
      number of leaf nodes.
  \default{None}

    \item \xmlNode{min\_impurity\_decrease}: \xmlDesc{float}, 
      A node will be split if this split induces a decrease of the impurity greater than or equal to
      this value.                                                  The weighted impurity decrease
      equation is the following:                                                  $N\_t / N *
      (impurity - N\_t\_R / N\_t * right\_impurity - N\_t\_L / N\_t * left\_impurity)$
      where $N$ is the total number of samples, $N\_t$ is the number of samples at the current node,
      $N\_t\_L$ is the number                                                  of samples in the
      left child, and $N\_t\_R$ is the number of samples in the right child.
      $N$, $N\_t$, $N\_t]\_R$ and $N\_t\_L$ all refer to the weighted sum, if sample\_weight is
      passed.
  \default{0.0}

    \item \xmlNode{random\_state}: \xmlDesc{integer}, 
      Controls the randomness of the estimator. The features are
      always randomly permuted at each split, even if splitter is set to
      "best". When max\_features < n\_features, the algorithm will select
      max\_features at random at each split before finding the best split
      among them. But the best found split may vary across different runs,
      even if max\_features=n\_features. That is the case, if the improvement
      of the criterion is identical for several splits and one split has to
      be selected at random. To obtain a deterministic behaviour during
      fitting, random\_state has to be fixed to an integer.
  \default{None}
  \end{itemize}


\subsubsection{ExtraTreeClassifier}
  The \xmlNode{ExtraTreeClassifier} is an ``extremely randomized tree classifier''.
  Extra-trees differ from classic decision trees in the way they are built. When looking for the
  best                          split to separate the samples of a node into two groups, random
  splits are drawn for each of the                          max\_features randomly selected features
  and the best split among those is chosen. When max\_features                          is set 1,
  this amounts to building a totally random decision tree.
  \zNormalizationPerformed{ExtraTreeClassifier}

  The \xmlNode{ExtraTreeClassifier} node recognizes the following parameters:
    \begin{itemize}
      \item \xmlAttr{name}: \xmlDesc{string, required}, 
        User-defined name to designate this entity in the RAVEN input file.
      \item \xmlAttr{verbosity}: \xmlDesc{[silent, quiet, all, debug], optional}, 
        Desired verbosity of messages coming from this entity
      \item \xmlAttr{subType}: \xmlDesc{string, required}, 
        specify the type of ROM that will be used
  \end{itemize}

  The \xmlNode{ExtraTreeClassifier} node recognizes the following subnodes:
  \begin{itemize}
    \item \xmlNode{Features}: \xmlDesc{comma-separated strings}, 
      specifies the names of the features of this ROM.         \nb These parameters are going to be
      requested for the training of this object         (see Section~\ref{subsec:stepRomTrainer})

    \item \xmlNode{Target}: \xmlDesc{comma-separated strings}, 
      contains a comma separated list of the targets of this ROM. These parameters         are the
      Figures of Merit (FOMs) this ROM is supposed to predict.         \nb These parameters are
      going to be requested for the training of this         object (see Section
      \ref{subsec:stepRomTrainer}).

    \item \xmlNode{pivotParameter}: \xmlDesc{string}, 
      If a time-dependent ROM is requested, please specifies the pivot         variable (e.g. time,
      etc) used in the input HistorySet.
  \default{time}

    \item \xmlNode{CV}: \xmlDesc{string}, 
      The text portion of this node needs to contain the name of the \xmlNode{PostProcessor} with
      \xmlAttr{subType}         ``CrossValidation``.
      The \xmlNode{CV} node recognizes the following parameters:
        \begin{itemize}
          \item \xmlAttr{class}: \xmlDesc{string, optional}, 
            should be set to \xmlString{Model}
          \item \xmlAttr{type}: \xmlDesc{string, optional}, 
            should be set to \xmlString{PostProcessor}
      \end{itemize}

    \item \xmlNode{alias}: \xmlDesc{string}, 
      specifies alias for         any variable of interest in the input or output space. These
      aliases can be used anywhere in the RAVEN input to         refer to the variables. In the body
      of this node the user specifies the name of the variable that the model is going to use
      (during its execution).
      The \xmlNode{alias} node recognizes the following parameters:
        \begin{itemize}
          \item \xmlAttr{variable}: \xmlDesc{string, required}, 
            define the actual alias, usable throughout the RAVEN input
          \item \xmlAttr{type}: \xmlDesc{[input, output], required}, 
            either ``input'' or ``output''.
      \end{itemize}

    \item \xmlNode{criterion}: \xmlDesc{[gini, entropy]}, 
      The function to measure the quality of a split. Supported criteria are ``gini'' for the
      Gini impurity and ``entropy'' for the information gain.
  \default{gini}

    \item \xmlNode{splitter}: \xmlDesc{[best, random]}, 
      The strategy used to choose the split at each node. Supported strategies are ``best''
      to choose the best split and ``random'' to choose the best random split.
  \default{best}

    \item \xmlNode{max\_depth}: \xmlDesc{integer}, 
      The maximum depth of the tree. If None, then nodes are expanded until all leaves are pure
      or until all leaves contain less than min\_samples\_split samples.
  \default{None}

    \item \xmlNode{min\_samples\_split}: \xmlDesc{integer}, 
      The minimum number of samples required to split an internal node
  \default{2}

    \item \xmlNode{min\_samples\_leaf}: \xmlDesc{integer}, 
      The minimum number of samples required to be at a leaf node. A split point at any
      depth will only be considered if it leaves at least min\_samples\_leaf training samples in
      each                                                  of the left and right branches. This may
      have the effect of smoothing the model, especially
      in regression.
  \default{1}

    \item \xmlNode{min\_weight\_fraction\_leaf}: \xmlDesc{float}, 
      The minimum weighted fraction of the sum total of weights (of all the input samples)
      required to be at a leaf node. Samples have equal weight when sample\_weight is not provided.
  \default{0.0}

    \item \xmlNode{max\_features}: \xmlDesc{[auto, sqrt, log2]}, 
      The strategy to compute the number of features to consider when looking for the best split:
      \begin{itemize}                                                     \item sqrt:
      $max\_features=sqrt(n\_features)$                                                     \item
      log2: $max\_features=log2(n\_features)$
      \item None: $max\_features=n\_features$
      \item auto: automatic selection
      \end{itemize}                                                   \nb the search for a split
      does not stop until at least one valid partition of the node
      samples is found, even if it requires to effectively inspect more than max\_features features.
  \default{None}

    \item \xmlNode{max\_leaf\_nodes}: \xmlDesc{integer}, 
      Grow a tree with max\_leaf\_nodes in best-first fashion. Best nodes are defined as relative
      reduction                                                  in impurity. If None then unlimited
      number of leaf nodes.
  \default{None}

    \item \xmlNode{min\_impurity\_decrease}: \xmlDesc{float}, 
      A node will be split if this split induces a decrease of the impurity greater than or equal to
      this value.                                                  The weighted impurity decrease
      equation is the following:                                                  $N\_t / N *
      (impurity - N\_t\_R / N\_t * right\_impurity - N\_t\_L / N\_t * left\_impurity)$
      where $N$ is the total number of samples, $N\_t$ is the number of samples at the current node,
      $N\_t\_L$ is the number                                                  of samples in the
      left child, and $N\_t\_R$ is the number of samples in the right child.
      $N$, $N\_t$, $N\_t]\_R$ and $N\_t\_L$ all refer to the weighted sum, if sample\_weight is
      passed.
  \default{0.0}

    \item \xmlNode{random\_state}: \xmlDesc{integer}, 
      Used to pick randomly the max\_features used at each split.
  \default{None}
  \end{itemize}


\subsubsection{ExtraTreeRegressor}
  The \xmlNode{ExtraTreeRegressor} is extremely randomized tree regressor.
  Extra-trees differ from classic decision trees in the way they are built. When
  looking for the best split to separate the samples of a node into two groups,
  random splits are drawn for each of the max\_features randomly selected features
  and the best split among those is chosen. When max\_features is set 1, this amounts
  to building a totally random decision tree.
  \zNormalizationPerformed{ExtraTreeRegressor}

  The \xmlNode{ExtraTreeRegressor} node recognizes the following parameters:
    \begin{itemize}
      \item \xmlAttr{name}: \xmlDesc{string, required}, 
        User-defined name to designate this entity in the RAVEN input file.
      \item \xmlAttr{verbosity}: \xmlDesc{[silent, quiet, all, debug], optional}, 
        Desired verbosity of messages coming from this entity
      \item \xmlAttr{subType}: \xmlDesc{string, required}, 
        specify the type of ROM that will be used
  \end{itemize}

  The \xmlNode{ExtraTreeRegressor} node recognizes the following subnodes:
  \begin{itemize}
    \item \xmlNode{Features}: \xmlDesc{comma-separated strings}, 
      specifies the names of the features of this ROM.         \nb These parameters are going to be
      requested for the training of this object         (see Section~\ref{subsec:stepRomTrainer})

    \item \xmlNode{Target}: \xmlDesc{comma-separated strings}, 
      contains a comma separated list of the targets of this ROM. These parameters         are the
      Figures of Merit (FOMs) this ROM is supposed to predict.         \nb These parameters are
      going to be requested for the training of this         object (see Section
      \ref{subsec:stepRomTrainer}).

    \item \xmlNode{pivotParameter}: \xmlDesc{string}, 
      If a time-dependent ROM is requested, please specifies the pivot         variable (e.g. time,
      etc) used in the input HistorySet.
  \default{time}

    \item \xmlNode{CV}: \xmlDesc{string}, 
      The text portion of this node needs to contain the name of the \xmlNode{PostProcessor} with
      \xmlAttr{subType}         ``CrossValidation``.
      The \xmlNode{CV} node recognizes the following parameters:
        \begin{itemize}
          \item \xmlAttr{class}: \xmlDesc{string, optional}, 
            should be set to \xmlString{Model}
          \item \xmlAttr{type}: \xmlDesc{string, optional}, 
            should be set to \xmlString{PostProcessor}
      \end{itemize}

    \item \xmlNode{alias}: \xmlDesc{string}, 
      specifies alias for         any variable of interest in the input or output space. These
      aliases can be used anywhere in the RAVEN input to         refer to the variables. In the body
      of this node the user specifies the name of the variable that the model is going to use
      (during its execution).
      The \xmlNode{alias} node recognizes the following parameters:
        \begin{itemize}
          \item \xmlAttr{variable}: \xmlDesc{string, required}, 
            define the actual alias, usable throughout the RAVEN input
          \item \xmlAttr{type}: \xmlDesc{[input, output], required}, 
            either ``input'' or ``output''.
      \end{itemize}

    \item \xmlNode{criterion}: \xmlDesc{[mse, friedman\_mse, mae]}, 
      The function to measure the quality of a split. Supported criteria are ``mse'' for the mean
      squared error,                                                  which is equal to variance
      reduction as feature selection criterion and minimizes the L2 loss using the mean of each
      terminal node, ``friedman\_mse'', which uses mean squared error with Friedman's improvement
      score for potential splits,                                                  ``mae'' for the
      mean absolute error, which minimizes the L1 loss using the median of each terminal node.
  \default{mse}

    \item \xmlNode{splitter}: \xmlDesc{[best, random]}, 
      The strategy used to choose the split at each node. Supported strategies are ``best''
      to choose the best split and ``random'' to choose the best random split.
  \default{best}

    \item \xmlNode{max\_depth}: \xmlDesc{integer}, 
      The maximum depth of the tree. If None, then nodes are expanded until all leaves are pure
      or until all leaves contain less than min\_samples\_split samples.
  \default{None}

    \item \xmlNode{min\_samples\_split}: \xmlDesc{integer}, 
      The minimum number of samples required to split an internal node
  \default{2}

    \item \xmlNode{min\_samples\_leaf}: \xmlDesc{integer}, 
      The minimum number of samples required to be at a leaf node. A split point at any
      depth will only be considered if it leaves at least min\_samples\_leaf training samples in
      each                                                  of the left and right branches. This may
      have the effect of smoothing the model, especially
      in regression.
  \default{1}

    \item \xmlNode{min\_weight\_fraction\_leaf}: \xmlDesc{float}, 
      The minimum weighted fraction of the sum total of weights (of all the input samples)
      required to be at a leaf node. Samples have equal weight when sample\_weight is not provided.
  \default{0.0}

    \item \xmlNode{max\_features}: \xmlDesc{[auto, sqrt, log2]}, 
      The strategy to compute the number of features to consider when looking for the best split:
      \begin{itemize}                                                     \item sqrt:
      $max\_features=sqrt(n\_features)$                                                     \item
      log2: $max\_features=log2(n\_features)$
      \item auto: automatic selection
      \end{itemize}                                                   \nb the search for a split
      does not stop until at least one valid partition of the node
      samples is found, even if it requires to effectively inspect more than max\_features features.
  \default{None}

    \item \xmlNode{max\_leaf\_nodes}: \xmlDesc{integer}, 
      Grow a tree with max\_leaf\_nodes in best-first fashion. Best nodes are defined as relative
      reduction                                                  in impurity. If None then unlimited
      number of leaf nodes.
  \default{None}

    \item \xmlNode{min\_impurity\_decrease}: \xmlDesc{float}, 
      A node will be split if this split induces a decrease of the impurity greater than or equal to
      this value.                                                  The weighted impurity decrease
      equation is the following:                                                  $N\_t / N *
      (impurity - N\_t\_R / N\_t * right\_impurity - N\_t\_L / N\_t * left\_impurity)$
      where $N$ is the total number of samples, $N\_t$ is the number of samples at the current node,
      $N\_t\_L$ is the number                                                  of samples in the
      left child, and $N\_t\_R$ is the number of samples in the right child.
      $N$, $N\_t$, $N\_t]\_R$ and $N\_t\_L$ all refer to the weighted sum, if sample\_weight is
      passed.
  \default{0.0}

    \item \xmlNode{ccp\_alpha}: \xmlDesc{float}, 
      Complexity parameter used for Minimal Cost-Complexity Pruning. The subtree with the largest
      cost                                                  complexity that is smaller than
      ccp\_alpha will be chosen. By default, no pruning is performed.
  \default{0.0}

    \item \xmlNode{random\_state}: \xmlDesc{integer}, 
      Used to pick randomly the max\_features used at each split.
  \default{None}
  \end{itemize}


\subsubsection{VotingRegressor}
  The \xmlNode{VotingRegressor} is an ensemble meta-estimator that fits several base
  regressors, each on the whole dataset. Then it averages the individual predictions to form
  a final prediction.

  The \xmlNode{VotingRegressor} node recognizes the following parameters:
    \begin{itemize}
      \item \xmlAttr{name}: \xmlDesc{string, required}, 
        User-defined name to designate this entity in the RAVEN input file.
      \item \xmlAttr{verbosity}: \xmlDesc{[silent, quiet, all, debug], optional}, 
        Desired verbosity of messages coming from this entity
      \item \xmlAttr{subType}: \xmlDesc{string, required}, 
        specify the type of ROM that will be used
  \end{itemize}

  The \xmlNode{VotingRegressor} node recognizes the following subnodes:
  \begin{itemize}
    \item \xmlNode{Features}: \xmlDesc{comma-separated strings}, 
      specifies the names of the features of this ROM.         \nb These parameters are going to be
      requested for the training of this object         (see Section~\ref{subsec:stepRomTrainer})

    \item \xmlNode{Target}: \xmlDesc{comma-separated strings}, 
      contains a comma separated list of the targets of this ROM. These parameters         are the
      Figures of Merit (FOMs) this ROM is supposed to predict.         \nb These parameters are
      going to be requested for the training of this         object (see Section
      \ref{subsec:stepRomTrainer}).

    \item \xmlNode{pivotParameter}: \xmlDesc{string}, 
      If a time-dependent ROM is requested, please specifies the pivot         variable (e.g. time,
      etc) used in the input HistorySet.
  \default{time}

    \item \xmlNode{CV}: \xmlDesc{string}, 
      The text portion of this node needs to contain the name of the \xmlNode{PostProcessor} with
      \xmlAttr{subType}         ``CrossValidation``.
      The \xmlNode{CV} node recognizes the following parameters:
        \begin{itemize}
          \item \xmlAttr{class}: \xmlDesc{string, optional}, 
            should be set to \xmlString{Model}
          \item \xmlAttr{type}: \xmlDesc{string, optional}, 
            should be set to \xmlString{PostProcessor}
      \end{itemize}

    \item \xmlNode{alias}: \xmlDesc{string}, 
      specifies alias for         any variable of interest in the input or output space. These
      aliases can be used anywhere in the RAVEN input to         refer to the variables. In the body
      of this node the user specifies the name of the variable that the model is going to use
      (during its execution).
      The \xmlNode{alias} node recognizes the following parameters:
        \begin{itemize}
          \item \xmlAttr{variable}: \xmlDesc{string, required}, 
            define the actual alias, usable throughout the RAVEN input
          \item \xmlAttr{type}: \xmlDesc{[input, output], required}, 
            either ``input'' or ``output''.
      \end{itemize}

    \item \xmlNode{estimator}: \xmlDesc{string}, 
      name of a ROM that can be used as an estimator
      The \xmlNode{estimator} node recognizes the following parameters:
        \begin{itemize}
          \item \xmlAttr{class}: \xmlDesc{string, required}, 
            RAVEN class for this entity (e.g. Samplers, Models, DataObjects)
          \item \xmlAttr{type}: \xmlDesc{string, required}, 
            RAVEN type for this entity; a subtype of the class (e.g. MonteCarlo, Code, PointSet)
      \end{itemize}

    \item \xmlNode{weights}: \xmlDesc{comma-separated floats}, 
      Sequence of weights (float or int) to weight the occurrences of predicted
      values before averaging. Uses uniform weights if None.
  \default{None}
  \end{itemize}


\subsubsection{BaggingRegressor}
  The \xmlNode{BaggingRegressor} is an ensemble meta-estimator that fits base regressors each on
  random subsets of the original                             dataset and then aggregate their
  individual predictions (either by voting or by averaging) to form a final
  prediction. Such a meta-estimator can typically be used as a way to reduce the variance of a
  black-box estimator                             (e.g., a decision tree), by introducing
  randomization into its construction procedure and then making an ensemble
  out of it.

  The \xmlNode{BaggingRegressor} node recognizes the following parameters:
    \begin{itemize}
      \item \xmlAttr{name}: \xmlDesc{string, required}, 
        User-defined name to designate this entity in the RAVEN input file.
      \item \xmlAttr{verbosity}: \xmlDesc{[silent, quiet, all, debug], optional}, 
        Desired verbosity of messages coming from this entity
      \item \xmlAttr{subType}: \xmlDesc{string, required}, 
        specify the type of ROM that will be used
  \end{itemize}

  The \xmlNode{BaggingRegressor} node recognizes the following subnodes:
  \begin{itemize}
    \item \xmlNode{Features}: \xmlDesc{comma-separated strings}, 
      specifies the names of the features of this ROM.         \nb These parameters are going to be
      requested for the training of this object         (see Section~\ref{subsec:stepRomTrainer})

    \item \xmlNode{Target}: \xmlDesc{comma-separated strings}, 
      contains a comma separated list of the targets of this ROM. These parameters         are the
      Figures of Merit (FOMs) this ROM is supposed to predict.         \nb These parameters are
      going to be requested for the training of this         object (see Section
      \ref{subsec:stepRomTrainer}).

    \item \xmlNode{pivotParameter}: \xmlDesc{string}, 
      If a time-dependent ROM is requested, please specifies the pivot         variable (e.g. time,
      etc) used in the input HistorySet.
  \default{time}

    \item \xmlNode{CV}: \xmlDesc{string}, 
      The text portion of this node needs to contain the name of the \xmlNode{PostProcessor} with
      \xmlAttr{subType}         ``CrossValidation``.
      The \xmlNode{CV} node recognizes the following parameters:
        \begin{itemize}
          \item \xmlAttr{class}: \xmlDesc{string, optional}, 
            should be set to \xmlString{Model}
          \item \xmlAttr{type}: \xmlDesc{string, optional}, 
            should be set to \xmlString{PostProcessor}
      \end{itemize}

    \item \xmlNode{alias}: \xmlDesc{string}, 
      specifies alias for         any variable of interest in the input or output space. These
      aliases can be used anywhere in the RAVEN input to         refer to the variables. In the body
      of this node the user specifies the name of the variable that the model is going to use
      (during its execution).
      The \xmlNode{alias} node recognizes the following parameters:
        \begin{itemize}
          \item \xmlAttr{variable}: \xmlDesc{string, required}, 
            define the actual alias, usable throughout the RAVEN input
          \item \xmlAttr{type}: \xmlDesc{[input, output], required}, 
            either ``input'' or ``output''.
      \end{itemize}

    \item \xmlNode{estimator}: \xmlDesc{string}, 
      name of a ROM that can be used as an estimator
      The \xmlNode{estimator} node recognizes the following parameters:
        \begin{itemize}
          \item \xmlAttr{class}: \xmlDesc{string, required}, 
            RAVEN class for this entity (e.g. Samplers, Models, DataObjects)
          \item \xmlAttr{type}: \xmlDesc{string, required}, 
            RAVEN type for this entity; a subtype of the class (e.g. MonteCarlo, Code, PointSet)
      \end{itemize}

    \item \xmlNode{n\_estimators}: \xmlDesc{integer}, 
      The number of base estimators in the ensemble.
  \default{10}

    \item \xmlNode{max\_samples}: \xmlDesc{float}, 
      The number of samples to draw from X to train each base estimator
  \default{1.0}

    \item \xmlNode{max\_features}: \xmlDesc{float}, 
      The number of features to draw from X to train each base estimator
  \default{1.0}

    \item \xmlNode{bootstrap}: \xmlDesc{[True, Yes, 1, False, No, 0, t, y, 1, f, n, 0]}, 
      Whether samples are drawn with replacement. If False, sampling without
      replacement is performed.
  \default{True}

    \item \xmlNode{bootstrap\_features}: \xmlDesc{[True, Yes, 1, False, No, 0, t, y, 1, f, n, 0]}, 
      Whether features are drawn with replacement.
  \default{False}

    \item \xmlNode{oob\_score}: \xmlDesc{[True, Yes, 1, False, No, 0, t, y, 1, f, n, 0]}, 
      Whether to use out-of-bag samples to estimate the generalization error.
      Only available if bootstrap=True.
  \default{False}

    \item \xmlNode{warm\_start}: \xmlDesc{[True, Yes, 1, False, No, 0, t, y, 1, f, n, 0]}, 
      When set to True, reuse the solution of the previous call to fit and add more
      estimators to the ensemble, otherwise, just fit a whole new ensemble.
  \default{False}

    \item \xmlNode{random\_state}: \xmlDesc{integer}, 
      Controls the random resampling of the original dataset (sample wise and feature wise).
  \default{None}
  \end{itemize}


\subsubsection{AdaBoostRegressor}
  The \xmlNode{AdaBoostRegressor} is a meta-estimator that begins by fitting a regressor on
  the original dataset and then fits additional copies of the regressor on the same dataset
  but where the weights of instances are adjusted according to the error of the current
  prediction. As such, subsequent regressors focus more on difficult cases.

  The \xmlNode{AdaBoostRegressor} node recognizes the following parameters:
    \begin{itemize}
      \item \xmlAttr{name}: \xmlDesc{string, required}, 
        User-defined name to designate this entity in the RAVEN input file.
      \item \xmlAttr{verbosity}: \xmlDesc{[silent, quiet, all, debug], optional}, 
        Desired verbosity of messages coming from this entity
      \item \xmlAttr{subType}: \xmlDesc{string, required}, 
        specify the type of ROM that will be used
  \end{itemize}

  The \xmlNode{AdaBoostRegressor} node recognizes the following subnodes:
  \begin{itemize}
    \item \xmlNode{Features}: \xmlDesc{comma-separated strings}, 
      specifies the names of the features of this ROM.         \nb These parameters are going to be
      requested for the training of this object         (see Section~\ref{subsec:stepRomTrainer})

    \item \xmlNode{Target}: \xmlDesc{comma-separated strings}, 
      contains a comma separated list of the targets of this ROM. These parameters         are the
      Figures of Merit (FOMs) this ROM is supposed to predict.         \nb These parameters are
      going to be requested for the training of this         object (see Section
      \ref{subsec:stepRomTrainer}).

    \item \xmlNode{pivotParameter}: \xmlDesc{string}, 
      If a time-dependent ROM is requested, please specifies the pivot         variable (e.g. time,
      etc) used in the input HistorySet.
  \default{time}

    \item \xmlNode{CV}: \xmlDesc{string}, 
      The text portion of this node needs to contain the name of the \xmlNode{PostProcessor} with
      \xmlAttr{subType}         ``CrossValidation``.
      The \xmlNode{CV} node recognizes the following parameters:
        \begin{itemize}
          \item \xmlAttr{class}: \xmlDesc{string, optional}, 
            should be set to \xmlString{Model}
          \item \xmlAttr{type}: \xmlDesc{string, optional}, 
            should be set to \xmlString{PostProcessor}
      \end{itemize}

    \item \xmlNode{alias}: \xmlDesc{string}, 
      specifies alias for         any variable of interest in the input or output space. These
      aliases can be used anywhere in the RAVEN input to         refer to the variables. In the body
      of this node the user specifies the name of the variable that the model is going to use
      (during its execution).
      The \xmlNode{alias} node recognizes the following parameters:
        \begin{itemize}
          \item \xmlAttr{variable}: \xmlDesc{string, required}, 
            define the actual alias, usable throughout the RAVEN input
          \item \xmlAttr{type}: \xmlDesc{[input, output], required}, 
            either ``input'' or ``output''.
      \end{itemize}

    \item \xmlNode{estimator}: \xmlDesc{string}, 
      name of a ROM that can be used as an estimator
      The \xmlNode{estimator} node recognizes the following parameters:
        \begin{itemize}
          \item \xmlAttr{class}: \xmlDesc{string, required}, 
            RAVEN class for this entity (e.g. Samplers, Models, DataObjects)
          \item \xmlAttr{type}: \xmlDesc{string, required}, 
            RAVEN type for this entity; a subtype of the class (e.g. MonteCarlo, Code, PointSet)
      \end{itemize}

    \item \xmlNode{n\_estimators}: \xmlDesc{integer}, 
      The maximum number of estimators at which boosting is
      terminated. In case of perfect fit, the learning procedure is
      stopped early.
  \default{50}

    \item \xmlNode{learning\_rate}: \xmlDesc{float}, 
      Weight applied to each regressor at each boosting iteration.
      A higher learning rate increases the contribution of each regressor.
      There is a trade-off between the learning\_rate and n\_estimators
      parameters.
  \default{1.0}

    \item \xmlNode{loss}: \xmlDesc{[linear, square, exponential]}, 
      The loss function to use when updating the weights after each
      boosting iteration.
  \default{linear}

    \item \xmlNode{random\_state}: \xmlDesc{integer}, 
      Controls the random seed given at each estimator at each
      boosting iteration.
  \default{None}
  \end{itemize}


\subsubsection{StackingRegressor}
  The \xmlNode{StackingRegressor} consists in stacking the output of individual estimator and
  use a regressor to compute the final prediction. Stacking allows to use the strength of each
  individual estimator by using their output as input of a final estimator.

  The \xmlNode{StackingRegressor} node recognizes the following parameters:
    \begin{itemize}
      \item \xmlAttr{name}: \xmlDesc{string, required}, 
        User-defined name to designate this entity in the RAVEN input file.
      \item \xmlAttr{verbosity}: \xmlDesc{[silent, quiet, all, debug], optional}, 
        Desired verbosity of messages coming from this entity
      \item \xmlAttr{subType}: \xmlDesc{string, required}, 
        specify the type of ROM that will be used
  \end{itemize}

  The \xmlNode{StackingRegressor} node recognizes the following subnodes:
  \begin{itemize}
    \item \xmlNode{Features}: \xmlDesc{comma-separated strings}, 
      specifies the names of the features of this ROM.         \nb These parameters are going to be
      requested for the training of this object         (see Section~\ref{subsec:stepRomTrainer})

    \item \xmlNode{Target}: \xmlDesc{comma-separated strings}, 
      contains a comma separated list of the targets of this ROM. These parameters         are the
      Figures of Merit (FOMs) this ROM is supposed to predict.         \nb These parameters are
      going to be requested for the training of this         object (see Section
      \ref{subsec:stepRomTrainer}).

    \item \xmlNode{pivotParameter}: \xmlDesc{string}, 
      If a time-dependent ROM is requested, please specifies the pivot         variable (e.g. time,
      etc) used in the input HistorySet.
  \default{time}

    \item \xmlNode{CV}: \xmlDesc{string}, 
      The text portion of this node needs to contain the name of the \xmlNode{PostProcessor} with
      \xmlAttr{subType}         ``CrossValidation``.
      The \xmlNode{CV} node recognizes the following parameters:
        \begin{itemize}
          \item \xmlAttr{class}: \xmlDesc{string, optional}, 
            should be set to \xmlString{Model}
          \item \xmlAttr{type}: \xmlDesc{string, optional}, 
            should be set to \xmlString{PostProcessor}
      \end{itemize}

    \item \xmlNode{alias}: \xmlDesc{string}, 
      specifies alias for         any variable of interest in the input or output space. These
      aliases can be used anywhere in the RAVEN input to         refer to the variables. In the body
      of this node the user specifies the name of the variable that the model is going to use
      (during its execution).
      The \xmlNode{alias} node recognizes the following parameters:
        \begin{itemize}
          \item \xmlAttr{variable}: \xmlDesc{string, required}, 
            define the actual alias, usable throughout the RAVEN input
          \item \xmlAttr{type}: \xmlDesc{[input, output], required}, 
            either ``input'' or ``output''.
      \end{itemize}

    \item \xmlNode{estimator}: \xmlDesc{string}, 
      name of a ROM that can be used as an estimator
      The \xmlNode{estimator} node recognizes the following parameters:
        \begin{itemize}
          \item \xmlAttr{class}: \xmlDesc{string, required}, 
            RAVEN class for this entity (e.g. Samplers, Models, DataObjects)
          \item \xmlAttr{type}: \xmlDesc{string, required}, 
            RAVEN type for this entity; a subtype of the class (e.g. MonteCarlo, Code, PointSet)
      \end{itemize}

    \item \xmlNode{final\_estimator}: \xmlDesc{string}, 
      The name of estimator which will be used to combine the base estimators.

    \item \xmlNode{cv}: \xmlDesc{integer}, 
      specify the number of folds in a (Stratified) KFold,
  \default{5}

    \item \xmlNode{passthrough}: \xmlDesc{[True, Yes, 1, False, No, 0, t, y, 1, f, n, 0]}, 
      When False, only the predictions of estimators will be used as training
      data for final\_estimator. When True, the final\_estimator is trained on the predictions
      as well as the original training data.
  \default{False}
  \end{itemize}


%%%%%%%%%%%%%%%%%%%%%%%%%%%%%%%%%%%%%%%%%%%%%%%%%%%%%
%%%%% ROM Model - TensorFlow-Keras Interface  %%%%%%%
%%%%%%%%%%%%%%%%%%%%%%%%%%%%%%%%%%%%%%%%%%%%%%%%%%%%%
%%%%%% command used for TensorFlow-Keras Deep neural networks %%%%%%%%%%%%%%%%%
\newcommand{\layerNameAttr}[0]
{
    This node require the following attribute:
    \begin{itemize}
      \item \xmlAttr{name}, \xmlDesc{string, required field}, name of this layer. The value will be
        used in \xmlNode{layer\_layout} to construct the fully connected neural network.
    \end{itemize}
}
%%% Arguments for Dense Layer %%%
\newcommand{\activation}[0]
{
      \item \xmlNode{activation}, \xmlDesc{string, optional field}, including
        {`relu', `tanh', `elu', `selu', `softplus', `softsign', `sigmoid', `hard\_sigmoid', `linear', `softmax'}.
        (see~\ref{activationsDNN})
        \default{linear}
}
\newcommand{\dimOut}[0]
{
      \item \xmlNode{dim\_out}, \xmlDesc{positive integer, required except if this layer is used as the last output layer},
        dimensionality of the output space of this layer
}
\newcommand{\useBias}[0]
{
  \item \xmlNode{use\_bias}, \xmlDesc{boolean, optional field}, whether the layer uses a bias vector.
    \default{True}
}
\newcommand{\kernelInitializer}[0]
{
  \item \xmlNode{kernel\_initializer}, \xmlDesc{string, optional field}, initializer for the kernel weights matrix
    (see~\ref{initializersDNN}).
    \default{glorot\_uniform}
}
\newcommand{\biasInitializer}[0]
{
  \item \xmlNode{bias\_initializer}, \xmlDesc{string, optional field}, initializer for the bias vector
    (see ~\ref{initializersDNN}).
    \default{zeros}
}
\newcommand{\kernelRegularizer}[0]
{
  \item \xmlNode{kernel\_regularizer}, \xmlDesc{string, optional field}, regularizer function applied to
    the kernel weights matrix (see ~\ref{regularizersDNN}).
    \default{None}
}
\newcommand{\biasRegularizer}[0]
{
  \item \xmlNode{bias\_regularizer}, \xmlDesc{string, optional field}, regularizer function applied to the bias vector
    (see~\ref{regularizersDNN}).
    \default{None}
}
\newcommand{\activityRegularizer}[0]
{
  \item \xmlNode{activity\_regularizer}, \xmlDesc{string, optional field}, regularizer function applied to the output
    of the layer (its "activation"). (see~\ref{regularizersDNN})
    \default{None}
}
\newcommand{\kernelConstraint}[0]
{
  \item \xmlNode{kernel\_constraint}, \xmlDesc{string, optional field}, constraint function applied to the kernel weights
    matrix (see~\ref{constraintsDNN}).
    \default{None}
}
\newcommand{\biasConstraint}[0]
{
  \item \xmlNode{bias\_constraint}, \xmlDesc{string, optional field}, constraint function applied to the bias vector
    (see ~\ref{constraintsDNN})
    \default{None}
}
%%% Arguments for Dropout Layer %%%
\newcommand{\rate}[0]
{
      \item \xmlNode{rate}, \xmlDesc{float between 0 and 1, optional field}, fraction of the input units to drop.
        \default{0}
}
\newcommand{\noiseShape}[0]
{
      \item \xmlNode{noise\_shape}, \xmlDesc{list of integers, optional field}, 1D integer tensor representing the shape
        of the binary dropout mask that will be multiplied with the input.
        \default{None}
}
\newcommand{\seed}[0]
{
      \item \xmlNode{seed}, \xmlDesc{integer, optional field}, a integer to use as random seed.
        \default{None}
}
%%% Arguments for LSTM Layer %%%
\newcommand{\recurrentActivation}[0]
{
      \item \xmlNode{recurrent\_activation}, \xmlDesc{string, optional field}, activation function to use for the recurrent
        step, including {`relu', `tanh', `elu', `selu', `softplus', `softsign', `sigmoid', `hard\_sigmoid', `linear', `softmax'}.
        \default{hard\_sigmoid}
}
\newcommand{\recurrentInitializer}[0]
{
      \item \xmlNode{recurrent\_initializer}, \xmlDesc{string, optional field}, used for the linear transformation of
        the recurrent state (see ~\ref{initializersDNN}).
        \default{orthogonal}
}
\newcommand{\unitForgetBias}[0]
{
      \item \xmlNode{unit\_forget\_bias}, \xmlDesc{boolean, optional field}, add 1 to the bias of the forget gate at
        initialization if True.
        \default{True}
}
\newcommand{\recurrentRegularizer}[0]
{
      \item \xmlNode{recurrent\_regularizer}, \xmlDesc{string, optional field}, regularizer function applied to the
        \textit{recurrent\_kernel} weights matrix (see ~\ref{regularizersDNN}).
        \default{None}
}
\newcommand{\recurrentConstraint}[0]
{
      \item \xmlNode{recurrent\_constraint}, \xmlDesc{string, optional field}, constraint function applied to the
        \textit{recurrent\_kernel} weights matrix (see ~\ref{constraints}).
        \default{None}
}
\newcommand{\dropout}[0]
{
      \item \xmlNode{dropout}, \xmlDesc{float between 0 and 1, optional field}, fraction of the units to drop for the linear
        transformation of the inputs
        \default{0}
}
\newcommand{\recurrentDropout}[0]
{
      \item \xmlNode{recurrent\_dropout}, \xmlDesc{float between 0 and 1, optional field}, fraction of the units to drop for the linear
        transformation of the recurrent state.
        \default{0}
}
\newcommand{\returnSequence}[0]
{
      \item \xmlNode{return\_sequence}, \xmlDesc{boolean, optional field}, whether to return the last output in the output sequence, or
        full sequence.
        \default{False}
}
\newcommand{\implementation}[0]
{
      \item \xmlNode{implementation}, \xmlDesc{integer, optional field},
        implementation mode, either 1 or 2. Mode 1 will structure its operations as a larger number of smaller dot products and additions,
        whereas mode 2 will batch them into fewer, larger operations. These modes will have different performance profiles on different
        hardware and for different applications.
        \default{1}
}
\newcommand{\returnState}[0]
{
      \item \xmlNode{return\_state}, \xmlDesc{boolean, optional field},
        whether to return the last output in the output sequence, or the full sequence.
        \default{False}
}
\newcommand{\goBackwards}[0]
{
      \item \xmlNode{go\_backwards}, \xmlDesc{boolean, optional field},
        if True, process the input sequence backwards and return the reversed sequence.
        \default{False}
}
\newcommand{\stateful}[0]
{
      \item \xmlNode{stateful}, \xmlDesc{boolean, optional field},
        if True, the last state for each sample at index i in a batch will be used as initial state for the sample
        of index i in the following batch.
        \default{False}
}
\newcommand{\unroll}[0]
{
      \item \xmlNode{unroll}, \xmlDesc{boolean, optional field},
        if True, the network will be unrolled, else a symbolic loop will be used. Unrolling can speed-up a RNN,
        although it tends to be more memory-intensive. Unrolling is only suitable for short sequences.
        \default{False}
}

%%% Arguments for Conv1D Layer %%%

\newcommand{\kernelSize}[0]
{
      \item \xmlNode{kernel\_size}, \xmlDesc{integer or list of integers, required field}, specifying the length
        of the 1D convolution window.
}
\newcommand{\strides}[0]
{
      \item \xmlNode{strides}, \xmlDesc{integer or list of integers, optional field}, pecifying the stride
        length of the convolution. Specifying any stride value not equal 1 is incompatible with specifying any
        dilation\_rate value not equal 1.
        \default{1}
}
\newcommand{\padding}[0]
{
      \item \xmlNode{padding}, \xmlDesc{string, optional field},
        one of "valid", "causal" or "same" (case-insensitive).  "valid" means "no padding".
        "same" results in padding the input such that the output has the same length as the original input.
        "causal" results in causal (dilated) convolutions, e.g. output[t] does not depend on input[t + 1:].
        A zero padding is used such that the output has the same length as the original input. Useful when
        modeling temporal data where the model should not violate the temporal order.
        \default{valid}
}
\newcommand{\dataFormat}[0]
{
      \item \xmlNode{data\_format}, \xmlDesc{string, optional field},
        A string, one of "channels\_last" (default) or "channels\_first". The ordering of the dimensions in the inputs.
        "channels\_last" corresponds to inputs with shape  (batch, steps, channels) (default format for temporal data
        in Keras) while "channels\_first" corresponds to inputs with shape (batch, channels, steps).
        \default{channels\_last}
}
\newcommand{\dilationRate}[0]
{
      \item \xmlNode{dilation\_rate}, \xmlDesc{integer or list of integers, optional field},
        specifying the dilation rate to use for dilated convolution. Currently, specifying any dilation\_rate value
        not equal 1 is incompatible with specifying any strides value not equal 1.
        \default{1}
}

%%% Arguments for Pooling Layer %%%

\newcommand{\poolSize}[0]
{
      \item \xmlNode{pool\_size}, \xmlDesc{integer, required field}, size of the max pooling windows.
        \default{2}
}

\newcommand{\DenseLayer}[0]
{
  \item \xmlNode{Dense}, \xmlDesc{required field}, regular densely-connected neural network layer.
    \layerNameAttr
    In addition, this node also accepts the following subnodes
    \begin{itemize}
        \activation
        \dimOut
        \useBias
        \kernelInitializer
        \biasInitializer
        \kernelRegularizer
        \biasRegularizer
        \activityRegularizer
        \kernelConstraint
        \biasConstraint
    \end{itemize}
}
\newcommand{\DropoutLayer}[0]
{
  \item \xmlNode{Dropout}, \xmlDesc{optional field}, applies Dropout to the input. Dropout consists in
    randomly setting a fraction \xmlNode{rate} of input units to 0 at each update during training time,
    which helps prevent overfitting.
    \layerNameAttr
    In addition, this node also accepts the following subnodes
    \begin{itemize}
        \rate
        \noiseShape
        \seed
    \end{itemize}
}

\newcommand{\PoolingLayer}[1]
{

  \item \xmlNode{#1}, \xmlDesc{optional field},
    In addition, this node also accepts the following subnodes
    \begin{itemize}
        \poolSize
        \strides
        \padding
        \dataFormat
    \end{itemize}
}
\newcommand{\GlobalPoolingLayer}[1]
{

  \item \xmlNode{#1}, \xmlDesc{optional field},
    In addition, this node also accepts the following subnodes
    \begin{itemize}
        \dataFormat
    \end{itemize}
}

%%%%%%%%%%%%%%%%%%%%%%%%%%%%%%%%%%%%%%%%%%%%%%%%%%%%%%%%%%%%%%%%%%%%%%%%%%%%%%%

%%%%% ROM Model - TensorFlow-Keras Interface  %%%%%%%
\subsubsection{TensorFlow-Keras Deep Neural Networks}
\label{subsubsec:TFK_DNNs}

\textcolor{red}{\\It is important to NOTE that Python3 is required in order to use these deep neural networks.
If python2 is installed, these ROMs will not be imported by RAVEN, and an error will be raised if the user tries
to use these capabilities.}

\textbf{TensorFlow} is an open source software library for high performance numerical computation. Its flexible architecture
allows easy deployment of computation across a variety of platforms (CPUs, GPUs, TPUs), and from desktops to clusters
of servers to mobile and edge devices. Originally developed by researchers and engineers from the Google Brain team
within Google’s AI organization, it comes with strong support for machine learning and deep learning and the flexible
numerical computation core is used across many other scientific domains.

\textbf{Keras} is a high-level API to build and train deep learning models. It's used for fast prototyping, advanced research,
and production, with three key advantages:
\begin{itemize}
  \item \textit{User friendly}: Keras has a simple, consistent interface optimized for common use cases.
    It provides clear and actionable feedback for user errors.
  \item \textit{Modular and composable}: Keras models are made by connecting configurable building blocks together,
    with few restrictions.
  \item \textit{Easy to extend}: Write custom building blocks to express new ideas for research. Create new layers,
    loss functions, and develop state-of-the-art models.
\end{itemize}

\textbf{tf.keras} is TensorFlow's implementation of the Keras API specification. This is a high-level API to build and train
models that include first-class support for TensorFlow-specific functionality, such as eager \textit{execution},
\textit{tf.data} pipelines, and \textit{Estimators}. \textbf{tf.keras} makes TensorFlow easier to use without sacrificing
flexibility and performance. RAVEN will utilize this high-level API to build and train deep neural networks (DNNs) as ROMs, and
these ROMs can be employed by other RAVEN entities to perform uncertainty quantification, model optimization and data analysis.

Before analyzing each classifier in detail, it is important to mention that each type has a similar syntax. In the
example below, the subnodes that can be included in the main XML node \xmlNode{ROM} are reported:
\textbf{Example:}
\begin{lstlisting}[style=XML,morekeywords={name,subType}]
<Simulation>
  ...
  <Models>
    ...
    <ROM name='aUserDefinedName' subType='whatever'>
      <Features>X,Y</Features>
      <Target>Z</Target>
      <loss>mean_squared_error</loss>
      <metrics>accuracy</metrics>
      <batch_size>4</batch_size>
      <epochs>4</epochs>
      <num_classes>2</num_classes>
      <validation_split>0.25</validation_split>
      <optimizerSetting>
        <optimizer>Adam</optimizer>
        ...
      </optimizerSetting>
      <WhateverLayer1 name="layerName1">
        ...
      </WhateverLayer1>
      ...
      <WhateverLayerN name="layerNameN">
        ...
      </WhateverLayerN>
      <layer_layout>layerName1, ..., layerNameN</layer_layout>
    </ROM>
    ...
  </Models>
  ...
</Simulation>
\end{lstlisting}

As shown in above example, in addition to the common subnodes \xmlNode{Target} and \xmlNode{Features}, the \xmlNode{ROM} of DNNs
can be initialized with the following children:
\begin{itemize}
  \item \xmlNode{loss}, \xmlDesc{string or comma separated string, optional field}, if the model has multiple outputs, you can use a different
    loss metric on each output by passing a list of loss metrics. The value that will be minimized by the model will then
    be the sum of all individual value from each loss metric. Available loss functions include \textit{mean\_squared\_error},
    \textit{mean\_absolute\_error}, \textit{mean\_absolute\_percentage\_error}, \textit{mean\_squared\_logarithmic\_error},
    \textit{squared\_hinge}, \textit{hinge}, \textit{categorical\_hinge}, \textit{logcosh}, \textit{categorical\_crossentropy},
    \textit{sparse\_categorical\_crossentropy}, \textit{binary\_crossentropy}, \textit{kullback\_leibler\_divergence},
    \textit{poisson}, \textit{cosine\_proximity}.
  \default{mean\_squared\_error}
  \item \xmlNode{metrics}, \xmlDesc{string or comma separated string, optional field}, list of metrics to be evaluated by
    the model during training and testing. available metrics include
    \textit{binary\_accuracy}, \textit{categorical\_accuracy}, \textit{sparse\_categorical\_accuracy},
    \textit{top\_k\_categorical\_accuracy}, \textit{sparse\_top\_k\_categorical\_accuracy}.
  \default{accuracy}
  \item \xmlNode{batch\_size}, \xmlDesc{integer, optional field}, number of samples per gradient update.
  \default{20}
  \item \xmlNode{epochs}, \xmlDesc{integer, optional field}, number of epochs to train the model. An epoch
    is an iteration over the entire training data.
  \default{20}
  \item \xmlNode{num\_classes}, \xmlDesc{positive integer, optional field}, dimensionality of the output space of given classifier.
  \default{1}
  \item \xmlNode{validation\_split}, \xmlDesc{float between 0 and 1, optional field}, fraction of the training data to
    be used as validation data.
  \default{0.25}
  \item \xmlNode{plot\_model}, \xmlDesc{boolean, optional field}, if true the DNN model constructed by RAVEN will be
    plotted and stored in the working directory. The file name will be \textit{"ROM name" + "\_" + "model.png"}.
    \nb This capability requires the following libraries, i.e. pydot-ng and graphviz to be installed.
  \default{False}
  \item \xmlNode{optimizerSetting}, \xmlDesc{optional field}, including several subnode depending on the type of
    optimizers.
    \begin{itemize}
      \item \xmlNode{optimizer}, \xmlDesc{string, optional field}, name of optimizer.
    \end{itemize}
    \default{Adam}
    \nb The users can also choose different optimizers to train the ROM. The default algorithm is \textit{Adam}.
    Other available optimizers include:
    \textit{SGD}, \textit{RMSprop}, \textit{Adagrad}, \textit{Adadelta}, \textit{Adamx}, \textit{Nadam}.
    For the detailed information, i.e. the parameters for each optimization, the user can refer to
    \url{https://keras.io/optimizers/}. In raven, the user can use \xmlNode{optimizerSetting} to set the
    parameters of the above optimizer as follows:
    \begin{itemize}
      \item \textbf{Adam}, adam optimizer
        \begin{itemize}
          \item \xmlNode{beta\_1}, \xmlDesc{float, optional field}, $0 < beta < 1$. Generally close to 1.
          \default{0.9}
          \item \xmlNode{beta\_2}, \xmlDesc{float, optional field}, $0 < beta < 1$. Generally close to 1.
          \default{0.999}
          \item \xmlNode{epsilon}, \xmlDesc{float, optional field}, fuzz factor.
          \default{None}
          \item \xmlNode{decay}, \xmlDesc{float, optional field}, learning rate decay over each update.
          \default{0.0}
          \item \xmlNode{lr}, \xmlDesc{float, optional field}, learning rate.
          \default{0.001}
        \end{itemize}
    %
      \item \textbf{SGD}, stochastic gradient descent optimizer.
        \begin{itemize}
          \item \xmlNode{momentum}, \xmlDesc{float, optional field}, $> 0$. Parameter that accelerates SGD in
            the relevant direction and dampens oscillations.
          \default{0.0}
          \item \xmlNode{nesterov}, \xmlDesc{boolean, optional field}, whether to apply Nesterov momentum
          \default{False}
          \item \xmlNode{decay}, \xmlDesc{float, optional field}, learning rate decay over each update.
          \default{0.0}
          \item \xmlNode{lr}, \xmlDesc{float, optional field}, learning rate.
          \default{0.001}
        \end{itemize}
    %
      \item \textbf{RMSprop}, RMSProp optimizer.
        \begin{itemize}
          \item \xmlNode{rho}, \xmlDesc{float, optional field}, $> 0$.
          \default{0.9}
          \item \xmlNode{decay}, \xmlDesc{float, optional field}, learning rate decay over each update.
          \default{0.0}
          \item \xmlNode{lr}, \xmlDesc{float, optional field}, learning rate.
          \default{0.001}
          \item \xmlNode{epsilon}, \xmlDesc{float, optional field}, fuzz factor.
          \default{None}
        \end{itemize}
    %
      \item \textbf{Adagrad}, Adagrad optimizer.
        \begin{itemize}
          \item \xmlNode{decay}, \xmlDesc{float, optional field}, learning rate decay over each update.
          \default{0.0}
          \item \xmlNode{lr}, \xmlDesc{float, optional field}, learning rate.
          \default{0.01}
          \item \xmlNode{epsilon}, \xmlDesc{float, optional field}, fuzz factor.
          \default{None}
        \end{itemize}
    %
      \item \textbf{Adadelta}, Adadelta optimizer.
        \begin{itemize}
          \item \xmlNode{decay}, \xmlDesc{float, optional field}, learning rate decay over each update.
          \default{0.0}
          \item \xmlNode{lr}, \xmlDesc{float, optional field}, learning rate.
          \default{1.0}
          \item \xmlNode{epsilon}, \xmlDesc{float, optional field}, fuzz factor.
          \default{None}
          \item \xmlNode{rho}, \xmlDesc{float, optional field}, $> 0$.
          \default{0.95}
        \end{itemize}
    %
      \item \textbf{Adamax}, Adamax optimizer
        \begin{itemize}
          \item \xmlNode{beta\_1}, \xmlDesc{float, optional field}, $0 < beta < 1$. Generally close to 1.
          \default{0.9}
          \item \xmlNode{beta\_2}, \xmlDesc{float, optional field}, $0 < beta < 1$. Generally close to 1.
          \default{0.999}
          \item \xmlNode{epsilon}, \xmlDesc{float, optional field}, fuzz factor.
          \default{None}
          \item \xmlNode{decay}, \xmlDesc{float, optional field}, learning rate decay over each update.
          \default{0.0}
          \item \xmlNode{lr}, \xmlDesc{float, optional field}, learning rate.
          \default{0.002}
        \end{itemize}
    %
      \item \textbf{Nadam},
        \begin{itemize}
          \item \xmlNode{beta\_1}, \xmlDesc{float, optional field}, $0 < beta < 1$. Generally close to 1.
          \default{0.9}
          \item \xmlNode{beta\_2}, \xmlDesc{float, optional field}, $0 < beta < 1$. Generally close to 1.
          \default{0.999}
          \item \xmlNode{epsilon}, \xmlDesc{float, optional field}, fuzz factor.
          \default{None}
          \item \xmlNode{lr}, \xmlDesc{float, optional field}, learning rate.
          \default{0.002}
        \end{itemize}
  \end{itemize}
  \item \xmlNode{layer\_layout}, \xmlDesc{comma separated string, required}, the layout/order of layers in the
    deep neural networks. The values in the subnode should be the name of layers defined in layer node, such as
    \xmlNode{Dense}, \xmlNode{Dropout}, and \xmlNode{Conv1D}.
\end{itemize}

\nb The descriptions regarding the \xmlNode{WhateverLayer} node will be introduced in following subsections.
Basically, different classifiers will require different layers.
In addition, most core layers will accept the \xmlNode{activation} subnode (see ~\ref{activationsDNN}).

%%%%% Activation Functions  %%%%%%%
\paragraph{Activation Functions}
\label{activationsDNN}
Activations can either be used through an \xmlNode{Activation} layer, or through the
\xmlNode{activation} argument supported by all forward layers.
Available activations include:
\begin{itemize}
  \item \textit{relu}, the rectified linear unit function, returns $f(x) = max(0, x)$.
  \item \textit{tanh}, the hyperbolic tan function, returns $f(x) = tanh(x)$.
  \item \textit{elu}, exponential linear units try to make the mean activations closer to zero which speeds
    up learning. $f(x) = x$ if $x \ge 0$, otherwise $(exp(x) - 1.)$.
  \item \textit{selu}, scaled exponential linear unit, i.e. $scale * elu(x, alpha)$, where $scale, alpha$
    are pre-defined constants.
  \item \textit{softplus}, a smooth approximation to the rectifier linear unit function, return
    $f(x) = log(1. + exp(x))$.
  \item \textit{softsign}, return $f(x) = \frac{x}{1. + |x|}$.
  \item \textit{sigmoid},return $f(x) = \frac{1.}{1. + exp(-x)}$.
  \item \textit{hard\_sigmoid}, hard sigmoid activation function.
  \item \textit{linear}, i.e. identity.
  \item \textit{softmax}, softmax activation function, return $f(x) = \frac{exp(x_i)}{\sum_i{exp(x_i)}}$
\end{itemize}

%%%%% Initializer Functions  %%%%%%%
\paragraph{Initializer Functions}
\label{initializersDNN}
Initializations define the way to set the initial random weights of TensorFlow-Keras layers. The keyword
arguments used to passing initializers to layers will depend on the layer. Usually it is simply
\xmlNode{kernel\_initializer} and \xmlNode{bias\_initializer}.
Available initializers include:
\begin{itemize}
  \item \textit{Zeros}, generates tensors initialized to 0.
  \item \textit{Ones}, generates tensors initialized to 1.
  \item \textit{Constant}, generates tensors initialized to a constant value.
  \item \textit{RandomNormal}, generates tensors with a normal distribution.
  \item \textit{RandomUniform}, generates tensors with a uniform distribution.
  \item \textit{TruncatedNormal}, generates a truncated normal distribution.
  \item \textit{VarianceScaling}, initializer capable of adapting its scale to the shape of weights.
  \item \textit{Orthogonal}, generates a random orthogonal matrix.
  \item \textit{Identity}, generates the identity matrix.
  \item \textit{lecun\_uniform}, LeCun uniform initializer.
    It draws samples from a uniform distribution within
    $[-limit, limit]$ where \textit{limit} is $sqrt(3/fanIn)$ where \textit{fanIn} is the number of input dimensions
    in the weight tensor.
  \item \textit{glorot\_normal}, Glorot normal initializer.
    It draws samples from a truncated normal distribution
    centered on 0 with $stddev = sqrt(2/(fanIn + fanOut))$ where \textit{fanIn} is the number of input dimensions
    in the weight tensor and \textit{fanOut} is the number of output dimensions in the weight tensors.
  \item \textit{glorot\_uniform}, Glorot uniform initializer.
    It draws samples from a uniform distribution within
    $[-limit, limit]$ where \textit{limit} is $sqrt(6/(fanIn+fanOut))$.
  \item \textit{he\_normal}, He normal initializer.
    It draws samples from a truncated normal distribution
    centered on 0 with $stddev = sqrt(2/fanIn)$.
  \item \textit{lecun\_normal}, LeCun normal initializer.
    It draws samples from a truncated normal distribution
    centered on 0 with $stddev = sqrt(1/fanIn)$.
  \item \textit{he\_uniform}, He uniform variance scaling initializer.
    It draws samples from a uniform distribution within
    $[-limit, limit]$ where \textit{limit} is $sqrt(6/fanIn)$ where \textit{fanIn} is the number of input dimensions
    in the weight tensor.
\end{itemize}

%%%%% Rgularizer Functions  %%%%%%%
\paragraph{Regularizer Functions}
\label{regularizersDNN}
Regularizers allow to apply penalties on layer parameters or layer activity during optimization.
These penalties are incorporated in the loss function that the network optimizes. The exact API
will depend on the layer, but the layers \xmlNode{Dense, Conv1D, Conv2D, and Conv3D} have a
unified API.
Available regularizers include:
\begin{itemize}
  \item \textit{l1}, l1 regularization
  \item \textit{l2}, l2 regularization
  \item \textit{l1\_l2}, l1 and l2 regularization
\end{itemize}

%%%%% Constraint Functions  %%%%%%%
\paragraph{Constraint Functions}
\label{constraintsDNN}
Functions from the \textit{constraint} module allow setting constraints on network parameters during optimization.
Available constraints include:
\begin{itemize}
  \item \textit{MaxNorm}, constrains the weights incident to each hidden unit to have a norm less than or equal to
    a desired value.
  \item \textit{NonNeg}, constrains the weights to be non-negative
  \item \textit{UnitNorm}, constrains the weights incident to each hidden unit to have unit norm.
  \item \textit{MinMaxNorm}, constrains the weights incident to each hidden unit to have the norm between a lower bound
    and an upper bound.
\end{itemize}

%%%%% ROM Model - KerasMLPClassifier and KerasMLPRegression  %%%%%%%
\paragraph{KerasMLPClassifier and KerasMLPRegression}
\label{KerasMLPClassifier}
\label{KerasMLPRegression}

Multi-Layer Perceptron (MLP) (or Artificial Neural Network - ANN), a class of feedforward
ANN, can be viewed as a logistic regression classifier where input is first transformed
using a non-linear transformation. This transformation projects the input data into a
space where it becomes linearly separable. This intermediate layer is referred to as a
\textbf{hidden layer}. An MLP consists of at least three layers of nodes. Except for the
input nodes, each node is a neuron that uses a nonlinear \textbf{activation function}. MLP
utilizes a supervised learning technique called \textbf{Backpropagation} for training.
Generally, a single hidden layer is sufficient to make MLPs a universal approximator.
However, many hidden layers, i.e. deep learning, can be used to model more complex nonlinear
relationships. The extra layers enable composition of features from lower layers, potentially
modeling complex data with fewer units than a similarly performing shallow network.

\zNormalizationPerformed{KerasMLPClassifier \textup{and} KerasMLPRegression}

In order to use this ROM, the \xmlNode{ROM} attribute \xmlAttr{subType} needs to
be \xmlString{KerasMLPClassifier} or \xmlString{KerasMLPRegression} (see the examples below). This model can be initialized with
the following layers:

\begin{itemize}
  \DenseLayer
  \DropoutLayer
\end{itemize}

\textbf{KerasMLPClassifier Example:}
\begin{lstlisting}[style=XML,morekeywords={name,subType}]
<Simulation>
  ...
  <Models>
    ...
    <ROM name='aUserDefinedName' subType='KerasMLPClassifier'>
      <Features>X,Y</Features>
      <Target>Z</Target>
      <loss>mean_squared_error</loss>
      <metrics>accuracy</metrics>
      <batch_size>4</batch_size>
      <epochs>4</epochs>
      <optimizerSetting>
        <beta_1>0.9</beta_1>
        <optimizer>Adam</optimizer>
        <beta_2>0.999</beta_2>
        <epsilon>1e-8</epsilon>
        <decay>0.0</decay>
        <lr>0.001</lr>
      </optimizerSetting>
      <Dense name="layer1">
          <activation>relu</activation>
          <dim_out>15</dim_out>
      </Dense>
      <Dropout name="dropout1">
          <rate>0.2</rate>
      </Dropout>
      <Dense name="layer2">
          <activation>tanh</activation>
          <dim_out>8</dim_out>
      </Dense>
      <Dropout name="dropout2">
          <rate>0.2</rate>
      </Dropout>
      <Dense name="outLayer">
          <activation>sigmoid</activation>
      </Dense>
      <layer_layout>layer1, dropout1, layer2, dropout2, outLayer</layer_layout>
    </ROM>
    ...
  </Models>
  ...
</Simulation>
\end{lstlisting}

\textbf{KerasMLPRegression Example:}
\begin{lstlisting}[style=XML,morekeywords={name,subType}]
<Simulation>
  ...
  <Models>
    <ROM name="modelUnderTest" subType="KerasMLPRegression">
      <Features>x1,x2,x3,x4,x5,x6,x7,x8</Features>
      <Target>y</Target>
      <loss>mean_squared_error</loss>
      <batch_size>10</batch_size>
      <epochs>60</epochs>
      <plot_model>False</plot_model>
      <validation_split>0.25</validation_split>
      <random_seed>1986</random_seed>
      <Dense name="layer1">
          <dim_out>30</dim_out>
      </Dense>
      <Dense name="layer2">
          <dim_out>12</dim_out>
      </Dense>
      <Dense name="outLayer">
      </Dense>
      <layer_layout>layer1, layer2, outLayer</layer_layout>
    </ROM>
  </Models>
  ...
</Simulation>
\end{lstlisting}

%%%%% ROM Model - KerasConvNetClassifier  %%%%%%%
\paragraph{KerasConvNetClassifier}
\label{KerasClassifier}

Convolutional Neural Network (CNN) is a deep learning algorithm which can take in an input image, assign
importance to various objects in the image and be able to differentiate one from the other. The
architecture of a CNN is analogous to that of the connectivity pattern of Neurons in the Human Brain
and was inspired by the organization of the Visual Cortex. Individual neurons respond to stimuli only
in a restricted region of the visual field known as the Receptive Field. A collection of such fields
overlap to cover the entire visual area. CNN is able to successfully capture the spatial and temporal
dependencies in an image through the application of relevant filters. The architecture performs
a better fitting to the image dataset due to the reduction in the number of parameters involved
and reusability of weights. In other words, the network can be trained to understand the sophistication
of the image better.

\zNormalizationPerformed{KerasConvNetClassifier}

In order to use this ROM, the \xmlNode{ROM} attribute \xmlAttr{subType} needs to
be \xmlString{KerasConvNetClassifier} (see the example below). This model can be initialized with
the following layers:

\begin{itemize}
  \DenseLayer
  \DropoutLayer
  \item \xmlNode{Conv1D}, \xmlDesc{optional field},
    \layerNameAttr
    In addition, this node also accepts the following subnodes
    \begin{itemize}
        \activation
        \dimOut
        \useBias
        \kernelSize
        \strides
        \padding
        \dataFormat
        \dilationRate
        \kernelInitializer
        \biasInitializer
        \kernelRegularizer
        \biasRegularizer
        \activityRegularizer
        \kernelConstraint
        \biasConstraint
    \end{itemize}

  \item \xmlNode{Conv2D}, \xmlDesc{optional field},
    In addition, this node also accepts the following subnodes
    \begin{itemize}
        \activation
        \dimOut
        \useBias
        \kernelSize
        \strides
        \padding
        \dataFormat
        \dilationRate
        \kernelInitializer
        \biasInitializer
        \kernelRegularizer
        \biasRegularizer
        \activityRegularizer
        \kernelConstraint
        \biasConstraint
    \end{itemize}

  \item \xmlNode{Conv3D}, \xmlDesc{optional field},
    In addition, this node also accepts the following subnodes
    \begin{itemize}
        \activation
        \dimOut
        \useBias
        \kernelSize
        \strides
        \padding
        \dataFormat
        \dilationRate
        \kernelInitializer
        \biasInitializer
        \kernelRegularizer
        \biasRegularizer
        \activityRegularizer
        \kernelConstraint
        \biasConstraint
    \end{itemize}

  \item \xmlNode{Flatten}, \xmlDesc{optional field},
    In addition, this node also accepts the following subnodes
    \begin{itemize}
        \dataFormat
    \end{itemize}

  \PoolingLayer{MaxPooling1D}
  \PoolingLayer{MaxPooling2D}
  \PoolingLayer{MaxPooling3D}
  \PoolingLayer{AveragePooling1D}
  \PoolingLayer{AveragePooling2D}
  \PoolingLayer{AveragePooling3D}
  \GlobalPoolingLayer{GlobalMaxPooling1D}
  \GlobalPoolingLayer{GlobalMaxPooling2D}
  \GlobalPoolingLayer{GlobalMaxPooling3D}
  \GlobalPoolingLayer{GlobalAveragePooling1D}
  \GlobalPoolingLayer{GlobalAveragePooling2D}
  \GlobalPoolingLayer{GlobalAveragePooling3D}
\end{itemize}

\textbf{Example:}
\begin{lstlisting}[style=XML,morekeywords={name,subType}]
<Simulation>
  ...
  <Models>
    ...
    <ROM name='aUserDefinedName' subType='KerasConvNetClassifier'>
      <Features>x1,x2</Features>
      <Target>labels</Target>
      <loss>mean_squared_error</loss>
      <metrics>accuracy</metrics>
      <batch_size>1</batch_size>
      <epochs>2</epochs>
      <plot_model>True</plot_model>
      <validation_split>0.25</validation_split>
      <num_classes>1</num_classes>
      <optimizerSetting>
        <beta_1>0.9</beta_1>
        <optimizer>Adam</optimizer>
        <beta_2>0.999</beta_2>
        <epsilon>1e-8</epsilon>
        <decay>0.0</decay>
        <lr>0.001</lr>
      </optimizerSetting>
      <Conv1D name="firstConv1D">
          <activation>relu</activation>
          <strides>1</strides>
          <kernel_size>2</kernel_size>
          <padding>valid</padding>
          <dim_out>32</dim_out>
      </Conv1D>
      <MaxPooling1D name="pooling1">
          <strides>2</strides>
          <pool_size>2</pool_size>
      </MaxPooling1D>
      <Conv1D name="SecondConv1D">
          <activation>relu</activation>
          <strides>1</strides>
          <kernel_size>2</kernel_size>
          <padding>valid</padding>
          <dim_out>32</dim_out>
      </Conv1D>
      <MaxPooling1D name="pooling2">
          <strides>2</strides>
          <pool_size>2</pool_size>
      </MaxPooling1D>
      <Flatten name="flatten">
      </Flatten>
      <Dense name="dense1">
          <activation>relu</activation>
          <dim_out>10</dim_out>
      </Dense>
      <Dropout name="dropout1">
          <rate>0.25</rate>
      </Dropout>
      <Dropout name="dropout2">
          <rate>0.25</rate>
      </Dropout>
      <Dense name="dense2">
          <activation>softmax</activation>
      </Dense>
      <layer_layout>firstConv1D, pooling1, dropout1, SecondConv1D, pooling2, dropout2, flatten, dense1, dense2</layer_layout>
    </ROM>
    ...
  </Models>
  ...
</Simulation>
\end{lstlisting}

%%%%% ROM Model - KerasLSTMClassifier  %%%%%%%
\paragraph{KerasLSTMClassifier and KerasLSTMRegression}
\label{KerasClassifier}

Long Short Term Memory networks (LSTM) are a special kind of recurrent neural network, capable
of learning long-term dependencies. They work tremendously well on a large variety of problems, and
are now widely used. LSTMs are explicitly designed to avoid the long-term dependency problem. Remembering
information for long periods of time is practically their default behavior, not something that they
struggle to learn.

LSTM's can be used for either classification (with
\xmlString{KerasLSTMClassifier}) or prediction of values (with
\xmlString{KerasLSTMRegression}).

\zNormalizationPerformed{KerasLSTMClassifier \textup{and} KerasLSTMRegression}

In order to use this ROM, the \xmlNode{ROM} attribute \xmlAttr{subType} needs to
be \xmlString{KerasLSTMClassifier} or \xmlString{KerasLSTMRegression} (see the examples below). This model can be initialized with
the following layers:

\begin{itemize}
  \DenseLayer
  \DropoutLayer
  \item \xmlNode{LSTM}, \xmlDesc{required field}, long short-term memory layer.
    \layerNameAttr
    In addition, this node also accepts the following subnodes
    \begin{itemize}
        \activation
        \dimOut
        \recurrentActivation
        \dropout
        \recurrentDropout
        \returnSequence
        \useBias
        \kernelInitializer
        \recurrentInitializer
        \biasInitializer
        \unitForgetBias
        \kernelRegularizer
        \recurrentRegularizer
        \biasRegularizer
        \activityRegularizer
        \kernelConstraint
        \recurrentConstraint
        \biasConstraint
        \implementation
        \returnState
        \goBackwards
        \stateful
        \unroll
    \end{itemize}
\end{itemize}

\textbf{KerasLSTMClassifier Example:}
\begin{lstlisting}[style=XML,morekeywords={name,subType}]
<Simulation>
  ...
  <Models>
    ...
    <ROM name='aUserDefinedName' subType='KerasLSTMClassifier'>
      <Features>x</Features>
      <Target>y</Target>
      <loss>categorical_crossentropy</loss>
      <metrics>accuracy</metrics>
      <batch_size>1</batch_size>
      <epochs>10</epochs>
      <validation_split>0.25</validation_split>
      <num_classes>26</num_classes>
      <optimizerSetting>
        <beta_1>0.9</beta_1>
        <optimizer>Adam</optimizer>
        <beta_2>0.999</beta_2>
        <epsilon>1e-8</epsilon>
        <decay>0.0</decay>
        <lr>0.001</lr>
      </optimizerSetting>
      <LSTM name="lstm1">
          <activation>tanh</activation>
          <dim_out>32</dim_out>
      </LSTM>
      <LSTM name="lstm2">
          <activation>tanh</activation>
          <dim_out>16</dim_out>
      </LSTM>
      <Dropout name="dropout">
          <rate>0.25</rate>
      </Dropout>
      <Dense name="dense">
          <activation>softmax</activation>
      </Dense>
      <layer_layout>lstm1,lstm2,dropout,dense</layer_layout>
    </ROM>
    ...
  </Models>
  ...
</Simulation>
\end{lstlisting}

\textbf{KerasLSTMRegression Example:}
\begin{lstlisting}[style=XML,morekeywords={name,subType}]
<Simulation>
  ...
  <Models>
    ...
    <ROM name="lstmROM" subType="KerasLSTMRegression">
      <Features>prev_sum, prev_square, prev_square_sum</Features>
      <Target>sum, square</Target>
      <pivotParameter>index</pivotParameter>
      <loss>mean_squared_error</loss>
      <LSTM name="lstm1">
        <dim_out>32</dim_out>
      </LSTM>
      <LSTM name="lstm2">
        <dim_out>16</dim_out>
      </LSTM>
      <Dense name="dense">
      </Dense>
      <layer_layout>lstm1, lstm2, dense</layer_layout>

    </ROM>
    ...
  </Models>
  ...
</Simulation>
\end{lstlisting}


\subsubsection{SerializePyomo}
\label{subsubsec:serializepyomo}
For the purpose of exporting RAVEN trained ROMs in an universal-accessible format for Pyomo to solve, the platform to prepare
the python syntax including the process to setup concrete models and constraints is created. The python syntax allows to
to retrieve pickled (serialized) ROMs and solve the defined concrete model via Pynumero's GreyBoxModel, an extension of Pyomo.
Details on how GreyBoxModel works can be found in the following site:

\begin{lstlisting}
  https://pyomo.readthedocs.io/en/stable/contributed_packages/pynumero/index.html
\end{lstlisting}

Two files are required to run the optimization: the pickled rom and the generated python file. The workflow to create both files is
provided in the example below located at the end of this subsection. The following command line is the defined format to interface both files:

\begin{lstlisting}
  python3 printed_python_file.py -r rom_pickled_file.pk -f location_of_raven_framework -order order_of_derivative
\end{lstlisting}

The file names `printed\_python\_file', `rom\_pickled\_file', and `order\_of\_derivative' are arbitrary names, and are not requirements.
The `location\_of\_raven\_framework' is the relative path to where `raven\_framework' is located. The created python file includes modules from
RAVEN and is required dealing with derivatives for GreyBoxModel to solve. The rest of the command line syntax `-r', `-f', `-order' are place holders for
the python file to locate where the pickled ROMs and the python file itself are. Caution, make sure the `order\_of\_derivative' is an integer. If not provided,
the default value is 1. When running the command a text file named `GreyModelOutput\_cyipopt.txt' will be created showing the results of the optimization.

Since the code developed in ROMs is not a subtype, the code activation for python interface file print is triggered by adding the input subnode with type `Pyomo':

\begin{lstlisting}[style=XML,morekeywords={name,subType}]
  <Input name="rom_out.py" type="Pyomo">rom_out.py</Input>
\end{lstlisting}

To verify the output files in the given test file in:

\begin{lstlisting}[style=XML,morekeywords={name,subType}]
  raven/tests/framework/Models/External/serialize_pyomo.xml
\end{lstlisting}

and the associated example optimization case:

\begin{lstlisting}[style=XML,morekeywords={name,subType}]
  python3 rom_out.py -r rom_pickle.pk -f ../../../..
\end{lstlisting}

are runnable with GreyBoxModel, several optional RAVEN dependencies are required. The following are the required
optional libraries: `pyomo 6.4' or over, `cmake', `glpk', `ipopt', `cyipopt', and `pyomo-extensions'. Although almost all required
libraries are standard packages (available on conda or pipy install), the `pyomo-extensions' is a custom library created by RAVEN developers to install `Pynumero'.
It functions to run the following commands if `pyomo-extensions' is part of the RAVEN library installation:

\begin{lstlisting}
  pyomo download-extensions
  pyomo build-extensions
\end{lstlisting}

For this reason, libraries `Pyomo' and `cmake' are required for successful `Pynumero' installation. Note, although all RAVEN dependency installs will
be placed in the RAVEN library directory, the current `Pyomo' version saves extensions on the local python local directory.
Other libraries are optimization solvers for `Pyomo' and `Pynumero'. Make sure the RAVEN conda library is activated when testing the example cases.
The method to include optional RAVEN libraries is by adding the `optional' flag to the RAVEN library installation command:

\begin{lstlisting}
  cd raven
  ./scripts/establish_conda_env.sh --install --optional='pyomo cmake ipopt cyipopt pyomo-extensions'
\end{lstlisting}

If the user plans to run the optimization case outside of RAVEN libraries and the pickled ROM has already been generated, instead install `pyomo', `cmake',
`glpk', `ipopt', and `cyipopt' to your system. After this is done, on a terminal, run the `Pyomo' download and extension command introduced earlier.
This will install `Pynumero' to your system.

If the user plans to use the RAVEN library, it is recommended to activate RAVEN libraries using the following command
on the terminal:

\begin{lstlisting}
  source scripts/establish_conda_env.sh --load
\end{lstlisting}

Make sure that the optional libraries are installed before running the pickled ROM and printed python file. Regardless of whether `Pynumero' is
installed inside or outside RAVEN libraries, depending on the OS, C++ libraries may be outdated for `Pynumero' to run. It has been reported that
the following OS has failed to run `Pynumero':

\begin{itemize}
  \item CentOS 7, 8
  \item Ubuntu 16, 18
\end{itemize}

For further guidance of editing printed Python file, please refer to the following documentation: 

\begin{lstlisting}
  raven/doc/misc/Optimization Problem Solving in PyNumero Framework_vf.docx
\end{lstlisting}

\textbf{Example:} For this example the external model is trained and further loaded as `out\_rom'. The loaded trained model is separately
processed to `rom\_out.py' and `rom\_pickle.pk'.
\begin{lstlisting}[style=XML,morekeywords={name,subType}]
  <Simulation>
    ...
    <Files>
      <Input name="rom_out.py" type="Pyomo">rom_out.py</Input>
      <Input name="rom_pickle.pk" type="">rom_pickle.pk</Input>
    </Files>
    ...
    <Steps>
      ...
      <MultiRun name="sample">
        ...
        <Model class="Models" type="ExternalModel">attenuate</Model>
        ...
        <Output class="DataObjects" type="PointSet">samples</Output>
      </MultiRun>
      <RomTrainer name="train">
        <Input class="DataObjects" type="PointSet">samples</Input>
        <Output class="Models" type="ROM">out_rom</Output>
      </RomTrainer>
      <IOStep name="serialize">
        <Input class="Models" type="ROM">out_rom</Input>
        <Output class="Files" type="">out_rom.py</Output>
      </IOStep>
      <IOStep name="pickle">
        <Input class="Models" type="ROM">out_rom</Input>
        <Output class="Files" type="">out_pickle.pk</Output>
      </IOStep>
      ...
    </Steps>
    ...
  </Simulation>
\end{lstlisting}
